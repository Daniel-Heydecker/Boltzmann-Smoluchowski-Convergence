\section{Analysis of Smoluchowski Equations}We will now analyse the relationship between various notions of solution for various Smoluchowski equations.  The remainder of this chapter is dedicated to a partial proof of these results, based on a previous analysis of Smolouchowski-type equations \cite{N00}. The remainder of the proof uses the convergence of the stochastic coalescent, and a coupling to an \emph{inhomogenous random graph process}, which will be introduced in later chapters.  
\begin{lemma} [Existence and Uniquness of Solutions]\label{lemma: E and U} For any $\mu_0$, the equation with gel (\ref{eq: E+G}) has a unique global solution $(\mu_t)_{t\geq 0}$ starting at $\mu_0$.  \end{lemma} \begin{corollary}\label{cor: maximal conservative solutions} Suppose $(\mu'_t)_{t<T}$ is a conservative local solution to the equation without gel, (\ref{eq: E}), starting at $\mu_0$. Then $\mu_t=\mu'_t$ for all $t<T$, and hence \ref{eq: E} has a unique maximal conservative solution, given by $(\mu_t)_{t<t_\text{gel}}$.\end{corollary}  We consider a further modified equation, resulting from a related, and more easily analysed kernel. Let \begin{equation}
    \label{eq: modified K} \begin{split} K^\text{m}(x,y,dz)& \\  = & \gamma(\pi_n(x)\pi_e(y)-\pi_p(x)\cdot \pi_p(y)+ \pi_e(x)\pi_n(y))\delta_{x+y}(dz) \\ & + \gamma(\pi_n(x)\pi_e(y)+\pi_p(x)\cdot \pi_p(y)+ \pi_e(x)\pi_n(y))\left(\frac{\delta_{x+Ry}+\delta_{Rx+y}}{2}\right)(dz) \\ & =\frac{1}{4}K(Rx, y, dz)+\frac{1}{2}K(x,y,dz)+\frac{1}{4}K(x,Ry, dz). \end{split} 
\end{equation} Let $L^\text{m}$ be the drift operator for the modified kernel $K^\text{m}$, and consider the modified equation \begin{equation} \label{eq: mE}
    \mu_t=\mu_0+\int_0^t L^\text{m}(\mu_s)ds.
\end{equation}The total mass of the modified kernel is \begin{equation}
    \label{eq: modified Kbar} \overline{K^\text{m}}(x,y)=2\gamma(\pi_n(x)\pi_e(y)+\pi_e(x)\pi_n(y)).
\end{equation}
Consider the modified state space, for $\epsilon>0$, \begin{equation*}
    S^\epsilon= \{x\in S: \epsilon \pi_n(x) \leq \pi_e(x) \leq \epsilon^{-1} \pi_n(x)\}.
\end{equation*} Note that this state space is preserved under both kernels $K, K^\text{m}$. Moreover, on the reduced state space $S^\epsilon$, the modified kernel $K^\text{m}$ is approximately multiplicative: for some $\delta_\epsilon>0$ and $\Delta_\epsilon<\infty$, we have \begin{equation*}
    \delta_\epsilon\hspace{0.1cm} (\pi_n(x)+\pi_e(x))\hspace{0.1cm}(\pi_n(y)+\pi_e(y)) \leq \overline{K^\text{m}}(x,y) \leq  \Delta_\epsilon\hspace{0.1cm} (\pi_n(x)+\pi_e(x))\hspace{0.1cm}(\pi_n(y)+\pi_e(y))
\end{equation*}for all $x,y \in S^\epsilon$. For any $\epsilon$, let $\mu_0^\epsilon$ denote the restriction $\mu_0^\epsilon(dx)=1[x\in S^\epsilon]\mu_0(dx).$ Applying a result from  \cite{N00}, we obtain the following:
\begin{lemma}\label{lemma: solution to modified equation}
    For all $\epsilon>0$, there is a unique maximal conservative solution  $(\nu^\epsilon_t)_{t< T^\epsilon}$ in $S^\epsilon$ to the modified equation (\ref{eq: mE}). Moreover, the map $t\mapsto \langle (\pi_n+\pi_e)^2, \nu^\epsilon_t\rangle$ is finite and increasing on $[0,T_\epsilon)$, and increases to $\infty$ as $t\uparrow T_\epsilon$. 
\end{lemma} Similarly, we can also apply Corollary \ref{cor: maximal conservative solutions} to deduce the existence of maximal conservative solutions $(\mu^\epsilon_t)_{t<t_\text{gel}^\epsilon}$ to (\ref{eq: E}) starting at $\mu^\epsilon_0$, which are again given by initial segments of a global solution $(\mu^\epsilon_t)_{t\geq 0}$ to (\ref{eq: E+G}). Repeatedly exploiting uniqueness, we show that these coincide: \begin{lemma}[Relationship of equations] Let $\mu^\epsilon_0$ be as above, for initial data $\mu_0$ satisfying (\textbf{R1}). Then the maximal conservative solutions $(\mu^\epsilon_t)_{t<t_\text{gel}^\epsilon}$ and $(\nu^\epsilon_t)_{t<T^\epsilon}$, to (\ref{eq: E}) and (\ref{eq: mE}) respectively, coincide. In particular, the map \begin{equation*}
    t\mapsto \langle (\pi_n+\pi_e)^2, \mu^\epsilon_t\rangle
\end{equation*} is finite and increasing on $[0, t_\text{gel}^\epsilon)$, and increases to $\infty$ as $t\uparrow t_\text{gel}^\epsilon.$ \end{lemma}\begin{proof} Firstly, we note that the proccess $(R_\star \mu^\epsilon_t)_{t<t_\text{gel}}$ also solves (\ref{eq: E}), starts at $\mu^\epsilon_0$ and is conservative. Therefore, by uniqueness in Corollary \ref{cor: maximal conservative solutions}, we must have $R_\star \mu^\epsilon_t=\mu^\epsilon_t$ for all $t<t_\text{gel}.$ Therefore, for any function $f\in \mathcal{T},$ and $t<t^\epsilon_\text{gel},$ \begin{equation} \begin{split}
        \int_{S\times S\times S} f(z)K(x,y,dz)\mu^\epsilon_t(dx)\mu^\epsilon_t(dy)& = \int_{S\times S} f(x+y)\overline{K}(x,y)\mu^\epsilon_t(dx)\mu^\epsilon_t(dy) \\ &=\int_{S\times S} f(x+y)\overline{K}(x,y)(R_\star\mu^\epsilon_t)(dx)\mu^\epsilon_t(dy) \\& = \int_{S\times S} f(Rx+y)\overline{K}(Rx,y)\mu^\epsilon_t(dx)\mu^\epsilon_t(dy)\\ = &\int_{S\times S\times S} f(z)K(Rx,y,dz)\mu^\epsilon_t(dx)\mu^\epsilon_t(dy).
    \end{split} \end{equation} Similarly, \begin{equation} \begin{split}
        \int_{S\times S\times S} f(x)K(x,y,dz)\mu^\epsilon_t(dx)\mu^\epsilon_t(dy)& = \int_{S\times S} f(x)\overline{K}(x,y)\mu^\epsilon_t(dx)\mu^\epsilon_t(dy) \\ &=\int_{S\times S} f(x)\overline{K}(x,y)\mu^\epsilon_t(dx)(R_\star\mu^\epsilon_t)(dy) \\& = \int_{S\times S} f(x)\overline{K}(x,Ry)\mu^\epsilon_t(dx)\mu^\epsilon_t(dy)\\& = \int_{S\times S} f(x)\overline{K}(Rx,y)\mu^\epsilon_t(dx)\mu^\epsilon_t(dy)\\ = &\int_{S\times S\times S} f(x)K(Rx,y,dz)\mu^\epsilon_t(dx)\mu^\epsilon_t(dy)
    \end{split} \end{equation} and \begin{equation}
        \int_{S\times S \times S} f(y)K(x,y,dz)\mu^\epsilon(dx)\mu^\epsilon(dy)=\int_{S\times S \times S} f(y)K(Rx,y,dz)\mu^\epsilon_t(dx)\mu^\epsilon_t(dy). 
    \end{equation} Hence, it follows that \begin{equation}\label{eq: symmetry under R1}
        \int_{S\times S\times S} (f(z)-f(x)-f(y))K(x,y,dz)\mu^\epsilon_t(dx)\mu^\epsilon_t(dy) =  \int_{S\times S\times S} (f(z)-f(x)-f(y))K(Rx,y,dz)\mu^\epsilon_t(dx)\mu^\epsilon_t(dy)
    \end{equation} and by an identical argument \begin{equation}\label{eq: symmetry under R2}
        \int_{S\times S\times S} (f(z)-f(x)-f(y))K(x,y,dz)\mu^\epsilon_t(dx)\mu^\epsilon_t(dy) =  \int_{S\times S\times S} (f(z)-f(x)-f(y))K(x,Ry,dz)\mu^\epsilon_t(dx)\mu^\epsilon_t(dy).
    \end{equation} Combining these, we see that $(\mu^\epsilon_t)_{t<t^\epsilon_\text{gel}}$ solves the modified equation (\ref{eq: mE}), and so, by uniqueness of the maximal conservative solution $(\nu^\epsilon_t)_{t<T^\epsilon}$ in Lemma \ref{lemma: solution to modified equation}, we have \begin{equation}
        t^\epsilon_\text{gel} \leq T^\epsilon; \hspace{1cm} \mu^\epsilon_t=\nu^\epsilon_t \hspace{0.5cm}\forall t<t^\epsilon_\text{gel}.
    \end{equation} The other implication is identical, using the uniqueness of the maximal conservative solution $(\nu^\epsilon_t)_{t<T^\epsilon}$ in Lemma \ref{lemma: solution to modified equation} to deduce that $\nu^\epsilon_t=R_\star \nu^\epsilon_t$ for $t<T^\epsilon$. Hence, the equations (\ref{eq: symmetry under R1}, \ref{eq: symmetry under R2}) hold with $\nu^\epsilon_t$ in place of $\mu^\epsilon_t$, for any $f\in \mathcal{T}$ and $t<T^\epsilon.$ Therefore, $(\nu^\epsilon_t)_{t<T^\epsilon}$ is a conservative solution to the unmodified equation (\ref{eq: E}), and so by Corollary \ref{cor: maximal conservative solutions}, \begin{equation}
        T^\epsilon \leq t^\epsilon_\text{gel}; \hspace{1cm} \nu^\epsilon_t=\mu^\epsilon_t \hspace{0.5cm}\forall t<T^\epsilon.
    \end{equation} \end{proof}  Repeating the calculation we have already done, we can calculate the gelation times explicitly: \begin{lemma}[Identification of Gelation Times]\label{lemma: calculation of gelation} For $\epsilon>0$, the gelation time $t_\text{gel}^\epsilon$ is given by \begin{equation} t_\text{gel}^\epsilon = \frac{1}{4\gamma}\sqrt{\frac{\langle \pi_n^2, \mu_0^\epsilon\rangle}{\langle \pi_e^2, \mu_0^\epsilon\rangle}}\left(\langle \pi_n^2, \mu_0^\epsilon\rangle+\sqrt{\frac{\langle \pi_n^2, \mu_0^\epsilon\rangle}{\langle \pi_e^2, \mu_0^\epsilon\rangle}}\langle \pi_n\pi_e, \mu_0^\epsilon\rangle \right)^{-1}. \end{equation} As $\epsilon \downarrow 0$, we have \begin{equation}
    t_\text{gel}^\epsilon \rightarrow t_\text{gel} = \frac{1}{2\gamma \sigma^2(d+\sqrt{d^2+2d})}.
\end{equation}  \end{lemma} In order to deduce the conservation properties of the solution $(\mu_t)_{t\geq 0}$ to (\ref{eq: E+G}), we use the following monotonicity argument, which follows from a result of \cite{N00}: \begin{lemma}
    For $\epsilon>0$, let $(\mu^\epsilon_t)_{t\geq 0}$ be the global solution to (\ref{eq: E+G}), starting at $\mu_0^\epsilon.$ Let $M^\epsilon_t$ and $E^\epsilon_t$ be the mass and energy of the corresponding gel. Then we have the relation \begin{equation}
        \epsilon M^\epsilon_t \leq E^\epsilon_t \leq \epsilon^{-1}M^\epsilon_t.
    \end{equation}We have the monotonicity principle\begin{equation}
        \mu^\epsilon_t \leq \mu_t;
    \end{equation}\begin{equation}
        M^\epsilon_t \leq M_t;
    \end{equation}
    \begin{equation}
        E^\epsilon_t \leq E_t.
    \end{equation} In particular, \begin{enumerate}[label=\roman{*}).]
        \item If $t<t_\text{gel}$, then $M_t=E_t=0$.
        \item If $t>t_\text{gel}$, then $M_t >0$ and $E_t>0$.
    \end{enumerate}
\end{lemma} \dhcomment{Haven't yet proven monotonicity principle required, but should follow from \cite{N00}}. Finally, we turn to the behaviour of the second moment $\mathcal{E}(t)$ in the subcritical phase \begin{lemma} The second moment $\mathcal{E}(t)=\langle \pi_n^2+\pi_e^2, \mu_t\rangle$ is finite and increasing on $[0, t_\text{gel})$, and increases to infinity as $t\uparrow t_\text{gel}$. \end{lemma} \begin{proof} We apply the same argument as in the identification of the gelation time to the full solution $(\mu_t)_{t\geq 0}$ to deduce integral equations for $\langle \pi_n^2, \mu_t\rangle, \langle \pi_e^2, \mu_t\rangle$ and $\langle \pi_n\pi_e, \mu_t\rangle$, potentially as an equality of infinite integrals, which imply the relation \begin{equation}
    \langle \pi_e^2, \mu_t\rangle = \frac{\langle \pi_e^2, \mu_0\rangle}{\langle \pi_n^2, \mu_0\rangle}\langle \pi_n^2, \mu_t\rangle
\end{equation}in the same sense. To avoid pathologies (for instance, all equal to $\infty$ for any $t>0$), we use the following small-time existence result, which follows by applying \cite[Theorem 2.1]{N00}:\begin{lemma} Let $(\mu_t)_{t\geq 0}$ be the unique global solution to \ref{eq: E+G} with initial data $(\mu_0)$. Then there exists $T>0$ such that the map $t\mapsto\langle (\pi_n+\pi_e)^2, \mu_t\rangle$ is finite and increasing on $[0,T)$, and increases to $\infty$ as $t\uparrow T$. \end{lemma} We define the map \begin{equation}
    f(t) := \langle \pi_n^2, \mu_t\rangle
           + \sqrt{\frac{\langle \pi_n^2, \mu_0\rangle}
                        {\langle \pi_e^2, \mu_0\rangle}}
             \langle \pi_n \pi_e, \mu_t\rangle.
\end{equation} Then $f$ is finite on $[0, T)$, and so we find the differential equation on $[0,T)$ \begin{equation}
    \frac{d}{dt}f(t)=4\gamma\sqrt{\frac{\langle \pi_e^2, \mu_t\rangle}{\langle \pi_n^2, \mu_0\rangle}}f(t)^2
\end{equation} has a finite solution on $[0,T)$. By the previous calculation, $f$ blows up precisely at $t_\text{gel}$, and so $T\leq t_\text{gel}.$ On the other hand, we observe that there is some $C(\mu_0)$ such that, for all times $t<T$, \begin{equation}
    \langle (\pi_n+\pi_e)^2, \mu_t \rangle \leq Cf(t).
\end{equation} Therefore, if we assume that $T<t_\text{gel}$, we would conclude that $\sup_{t<T} \langle (\pi_n+\pi_e)^2, \mu_t\rangle <\infty$, which is a contradiction. Hence, $T=t_\text{gel}$, which proves the claimed result. \end{proof} 
