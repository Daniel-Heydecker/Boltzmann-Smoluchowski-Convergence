\section{Convergence of the Gel \textbf{\textcolor{red}{OLD!!!}}}
We also prove the following supplementary result about the convergence of the gel:
  To do so, we fix a scale $\xi(N)\ll N$, and consider the `anomalous' clusters, which have mass larger $\frac{\xi_N}{N}$, but are not the macroscopic cluster of size $N$. The following result guarantees that such clusters do not asymptotically contribute to mass, momentum or energy. \begin{lemma}\label{lemma: anomalous clusters} Let $G^N_t$ be the random graph process constructed in \ref{sec:rg}, and fix a sequence $\xi_N\rightarrow \infty$ with $\xi_N\ll N$. Let $K\subset [t_\text{gel}, \infty)$ be a compact set of the form \begin{equation}\label{eq: form of K}
    K=\{t_\text{gel}\}\cup K_2; \hspace{1cm} \inf K_2 >t_\text{gel}.
\end{equation}  Then \begin{enumerate}
[label=\roman{*}).]
    \item We have the convergence \begin{equation}
       \sup_{t\in K}\left[\frac{1}{N}\sum_{j\geq 2: \#C_j(G^N_t)\geq \xi_N} \#C_j(G^N_t)\right] \rightarrow 0 \hspace{1cm}\text{in probability.}\end{equation}
        \item The momentum and energy of the anomalous clusters satisfy
        \begin{equation}
            \sup_{t\in K}\left[\frac{1}{N}\sum_{j\geq 2: \#C_j(G^N_t)}\hspace{0.2cm} \sum_{v\in C_j(G^N_t)}v\right] \rightarrow 0  \hspace{1cm} 
    \text{in probability}
        \end{equation} and  \begin{equation}
            \sup_{t\in K}\left[\frac{1}{N}\sum_{j\geq 2: C_j(G^N_t)}\hspace{0.2cm} \sum_{v\in C_j(G^N_t)}\frac{1}{2}\|v\|^2\right] \rightarrow 0  \hspace{1cm} 
    \text{in probability}.
        \end{equation}
\end{enumerate} \end{lemma}  \begin{proof} We first prove item i)., dealing with the critical and supercritical cases separately. We then show that this implies items ii-iii).
\paragraph{1. Critical Case.} We first deal with item i). for the critical case $t=t_\text{gel}$. This follows immediately from Theorem \ref{thrm: RG2}.
\paragraph{2. Supercritical Case.} To complete the proof of i)., we prove the result for the supercritical case. By the choice of $K_2$, it is sufficient to prove that, for all $t>t_\text{gel}$, there exists an open interval $t \in (t_-, t_+) \subset (t_\text{gel}, \infty)$ such that
\begin{equation}
    \sup_{t_-<s<t_+}\left[\frac{1}{N}\sum_{j\geq 2: \#C_j(G^N_s)\geq \xi_N} \#C_j(G^N_s)\right] \rightarrow 0 \hspace{1cm}\text{in probability.}
\end{equation} Recall from Theorem \ref{lemma: second moment finite after tgel} that, if we let (\textcolor{blue}{\textbf{....}}), then $\widehat{t_\text{gel}}(t)$ is a continuous map, and that $\widehat{t_\text{gel}}(t)>t$ for any $t>t_\text{gel}.$ Therefore, we can choose $t_\pm$ such that
\begin{equation}
    t_\text{gel}<t_-<t<t_+<\widehat{t_\text{gel}}(t_-).
\end{equation} We form $\widetilde{G}^N_{t_-}$ from $G^N_{t-}$ by deleting all vertexes of the giant component of $C_1(G^N_{t_-})$. We now form a new graph, $\widetilde{G}^N_{t_-,t_+}$ by including all edges between vertexes of $\widetilde{G}^N_{t_-}$ which are present in the graph $G^N_{t_+}$. \medskip \\ From Theorem \ref{thrm: coupling supercritical and subcritical}, we can construct a generalised vertex space $\widehat{\mathcal{V}}$ and graphs $\widehat{G}^N_{t_-}\sim \mathcal{G}^{\widehat{\mathcal{V}}}(N,t_-K)$, with the same vertex data as $\widetilde{G}^N_{t_-}$, such that \begin{equation}
    \mathbb{P}\left(\widehat{G}^N_{t_-}=\widetilde{G}^N_{t_-}\right)\rightarrow 1.
\end{equation} We now form $G^N_{t_-,t_+}$ by adding those edges present in $G^N_{t_+}$. By the Markov property of the graph process $(G^N_s)_{t\geq 0}$, these edges are independent of the construction of $\widehat{G}^N_{t_-}$, and so $G^N_{t_-,t_+}\sim \mathcal{G}^{\widehat{\mathcal{V}}}(N,t_+K)$. By the choices of $t_\pm$, $\widehat{G}^N_{t_-,t_+}$ is still subcritical, and by construction, \begin{equation}
    \mathbb{P}\left(\widehat{G}^N_{t_-, t_+}=\widetilde{G}^N_{t_-, t_+}\right)\rightarrow 1.
\end{equation} Moreover, we also see that \begin{equation}
    \sup_{s\in [t_-, t_+]} \left[\sum_{j\geq 2: \#C_j(G^N_t)\geq \xi_N} \#C_j(G^N_t)\right] \leq \sum_{j\geq 1: \#C_j(\widehat{G}^N_{t_-,t_+})\geq \xi_N} \#C_j(\widehat{G}^N_{t_-,t_+})
\end{equation} and that the right-hand side converges to $0$ in probability, as for the subcritical case.
\paragraph{3. Deduction of ii-iii).} Finally, it remains to show how items ii-iii. may be deduced from the first item. Using Cauchy-Schwarz, we have \begin{equation}
    \sup_{t\in I}\left[\left\|\frac{1}{N}\sum_{j\geq 2: \#C_j(G^N_t)\geq \xi_N}\hspace{0.2cm} \sum_{v\in C_j(G^N_t)}v\right\|\right] \leq \left(\sup_{t\in I}\left[\frac{1}{N} \sum_{j\geq 2: \#C_j(G^N_t)}\hspace{0.1cm} C_j(G^N_t)\right]\right)^{1/2}\left(\frac{1}{N}\sum_{v\in G^N_t} \|v\|^2\right)^{1/2}.
\end{equation} 
As $N\rightarrow \infty$, the first term converges in probability to $0$, and the second term converges in probability to a finite constant, by the law of large numbers. The claimed convergence in probability follows, and the case for the energy is identical. \end{proof} 
We next prove a preliminary version of the convergence result Theorem \ref{thrm: convergence of gel}, for a special class of compact sets. We can analyse the subcritical regime by relating this case to the critical case $t=t_\text{gel}$ and using monotonicity; however, both this, and the compactness technique used above, break down on neighbourhoods to the the right of $t_\text{gel}$. We therefore deal with compact sets of the form \begin{equation}
    \label{eq: form of I}
I=I_1\cup I_2; \hspace{0.5cm} I_1\subset [0, t_\text{gel}]; \hspace{0.5cm}\inf I_2 >t_\text{gel}. \end{equation} Our result is as follows.
\begin{lemma}\label{lemma: preliminary CoG} Let $M^N_t, P^N_t, E^N_t$ be the mass, energy, and momentum of the stochastic gel, and let $I\subset [0,\infty)$ be a compact set of the form (\ref{eq: form of I}). Then \begin{equation}
    \sup_{t\in I}\left|M^N_t - M_t\right|\rightarrow 0;
\end{equation} \begin{equation}
    \sup_{t\in I}\left\|P^N_t - P_t\right\|\rightarrow 0;
\end{equation} \begin{equation}
    \sup_{t\in I}\left|E^N_t - E_t\right|\rightarrow 0
\end{equation} in probability. \end{lemma}
\begin{proof} For a set $I$ of the form given, we decompose $I\subset J\cup K$, with $J\subset[0, t_\text{gel})$, and $K\subset [t_\text{gel}, \infty)$ is of the form (\ref{eq: form of K}); these correspond to the subcritical, and critical/supercritical cases, respectively. Let $G^N_t$ be the random graph process (\textcolor{red}{ref}), and let $\mu^N_t=\mu^N(G^N_t)$ be the associated stochastic coagulent. By uniqueness in law, it is sufficient to prove the claimed convergence for this particular realisation. For this realisation, we have \begin{equation}
    M^N_t = \frac{1}{N} C_1(G^N_t).
\end{equation} We now deal with the subcritical, and critical/supercritical cases separately. \paragraph{1. Subcritical Case.} For the subcritical regime $t\in J$, we know that $M_t=E_t=0$, and that $P_t=0$. Now, we observe that \begin{equation}
    \sup_{t\in J}\hspace{0.1cm} M^N_t =\frac{1}{N}\hspace{0.1cm} \sup_{t\in J} \#C_1(G^N_t) \leq \frac{1}{N}\#C_1(G^N_{t_\text{gel}}) \rightarrow 0
\end{equation} in probability. This proves the claimed result for the mass $M^N_t$; the cases for momentum $P^N_t$ and energy $E^N_t$ are similar, using Cauchy-Schwarz as in Lemma \ref{lemma: anomalous clusters}
\paragraph{2. Critical and Supercritical Cases.} We construct a sequence $\xi(N)\uparrow \infty$ such that, taking $
    B_N=\{x \in S: \pi_n(x)<\xi_N\},$ we have  \begin{equation}
   \sup_{t\in K}\hspace{0.1cm} \langle \pi_n 1_{B_N}, \mu^N_t-\mu_t\rangle \rightarrow 0;
\end{equation} \begin{equation}
   \sup_{t\in K}\hspace{0.1cm} \langle \pi_p 1_{B_N}, \mu^N_t-\mu_t\rangle \rightarrow 0;
\end{equation}\begin{equation}
   \sup_{t\in K}\hspace{0.1cm} \langle \pi_e 1_{B_N}, \mu^N_t-\mu_t\rangle \rightarrow 0
\end{equation} in probability. By monotone convergence, $\langle \pi_n1_{B_N},\mu_t\rangle \uparrow \langle \pi_n, \mu_t\rangle=1-M_t$, and by Lemma (...), the limit is continuous on $K$. Therefore, by Dini's theorem, the convergence is uniform, and so  \begin{equation} \langle \pi_n1_{B_N}, \mu^N_t\rangle \rightarrow \langle \pi_n, \mu_t\rangle =1-M_t \hspace{1cm}\text{uniformly on $K$, in probability.} \end{equation} Similarly, \begin{equation} \langle \pi_e1_{B_N}, \mu_t^N\rangle \rightarrow \langle \pi_e, \mu_t\rangle =\langle \pi_e, \mu_0\rangle -E_t \hspace{1cm}\text{uniformly on $K$, in probability.} \end{equation} For the momentum, the symmetry (\textbf{A1}) implies that $\langle \pi_p 1_{B_N}, \mu_t\rangle =0=\langle \pi_p, \mu_t\rangle$, and so \begin{equation} 0=\langle \pi_p1_{B_N}, \mu^N_t\rangle \rightarrow 0=-P_t \hspace{1cm}\text{uniformly on $K$, in probability.} \end{equation} Moreover, we can also construct $\xi_N$ such that $\xi_N \ll N$. By considering the node of the graph process with the largest velocity, it is straightforward to see that $C_1(G^N_{t_\text{gel}})\geq \alpha \log N$ with high probability, for some constant $\alpha>0$. Hence, we can also arrange that $\xi_N \ll C_1(G^N_{t_\text{gel}})$ with high probability. 
and so, with high probability, \begin{equation}
    1-\langle \pi_n1_{B_N},\mu^N_t\rangle = M^N_t + \sum_{j\geq 2: C_j(G^N_t)\geq \xi(N)} C_j(G^N_t).
\end{equation} By Lemma \ref{lemma: anomalous clusters}, the second term on the right hand side converges to $0$ in probability, uniformly on $K$. Taking limits, we conclude that \begin{equation}
    M^N_t \rightarrow M_t \hspace{1cm} \text{in probability}.
\end{equation} The cases for momentum and energy are identical.\end{proof} We now have the following corollaries: \begin{corollary} \label{cor: m and e at tgel}At the gelation time $t_\text{gel}$, we have \begin{equation}
    M_{t_\text{gel}}=E_{t_\text{gel}}=0.
\end{equation} In particular, the representations (\ref{eq: formula for M, E}) hold for all $t\geq 0$, and $M_t, E_t$ are continuous on $[0, \infty)$. \end{corollary} \begin{corollary} At the critical point, \begin{equation}
    \mathcal{E}(t_\text{gel})=\langle \pi_n^2+\pi_e^2, \mu_{t_\text{gel}}\rangle =\infty.
\end{equation} \end{corollary}
\begin{proof}[Proof of Corollary \ref{cor: m and e at tgel}] The first assertion follows from taking $K=\{t_\text{gel}\}$ in Lemma \ref{lemma: preliminary CoG}. Therefore, $M_t=E_t=0$ for all $t\leq t_\text{gel}$; this implies that the representations (\ref{eq: formula for M, E}) hold, since $\rho_t=0$ for $t\leq t_\text{gel}$, by Lemma \ref{lemma: survival function}. Continuity follows by the continuity property Theorem \ref{thrm: continuity of rho} of $\rho_t$, and using dominated convergence.   \end{proof}
\begin{proof}[Proof of Theorem \ref{thrm: convergence of gel}] It suffices to prove that, for $t>t_\text{gel}$, we have the convergence \begin{equation}
    \sup_{0\leq s \leq t}\left|M^N_s - M_s\right|\rightarrow 0;
\end{equation} \begin{equation}
    \sup_{0\leq s \leq t}\left\|P^N_s - P_s\right\|\rightarrow 0;
\end{equation} \begin{equation}
    \sup_{0\leq s \leq t}\left|E^N_s - E_s\right|\rightarrow 0
\end{equation} in probability. Fix $\epsilon>0.$ By Corollary \ref{cor: m and e at tgel}, we can choose $t_+\in(t_\text{gel},t)$ such that \begin{equation} \label{eq: choice of tplus}
    M_{t_+}<\frac{\epsilon}{2}; \hspace{1cm} E_{t_+}<\frac{\epsilon}{2}.
\end{equation} Let $I$ be given by \begin{equation}
    I=[0,t]\setminus (t_\text{gel}, t_+).
\end{equation}Note that this is in the form (\ref{eq: form of I}). Consider the events
\begin{equation}
    A_N=\left\{\sup_{s\in I} \left|M^N_s - M_s\right|<\frac{\epsilon}{2};\hspace{0.2cm} \sup_{s\in I} \left\|P^N_s\right\|<\epsilon;  \hspace{0.2cm}\sup_{s\in I} \left|E^N_s - E_s\right|<\frac{\epsilon}{2} \right\};
\end{equation}
\begin{equation}
    B_N=\left\{\sup_{s\leq t} \left|M^N_s - M_s\right|<\epsilon;\hspace{0.2cm} \sup_{s\leq t} \left\|P^N_s \right\|< \epsilon;  \hspace{0.2cm}\sup_{s\leq t} \left|E^N_s - E_s\right|< \epsilon \right\}.
\end{equation} We \emph{claim} that $A_N\subset B_N$. By Lemma \ref{lemma: preliminary CoG}, $\mathbb{P}(A_N)\rightarrow 1$, and so $\mathbb{P}(B_N)\rightarrow 1$, proving the desired result. \medskip \\ We now prove the claimed inclusion. On the event $A_N$, evaluating at $t_+\in I$ shows that \begin{equation}
    M^N_{t+}<\epsilon;\hspace{1cm} E^N_{t_+}<\epsilon.
\end{equation} Therefore, on $A_N$, for all $s\in (t_\text{gel}, t_+)$, we have \begin{equation}
    0\leq M^N_s\leq M^N_{t_+}<\epsilon; \hspace{0.5cm} 0<M_s<\frac{\epsilon}{2};
\end{equation}
\begin{equation}
    0\leq E^N_s\leq E^N_{t_+}<\epsilon; \hspace{0.5cm} 0<E_s<\frac{\epsilon}{2}
\end{equation} which imply that, on $A_N$, \begin{equation}
    \sup_{s\in (t_\text{gel}, t_+)} \hspace{0.1cm}\left|M^N_s-M_s\right|< \epsilon;\hspace{0.5cm}\sup_{s\in (t_\text{gel}, t_+)} \hspace{0.1cm}\left|E^N_s-E_s\right|< \epsilon.
\end{equation} Combining these with the definition of $A_N$, we see that \begin{equation}
    A_N\subset\left\{\sup_{s\leq t} \left|M^N_s - M_s\right|<\epsilon; \hspace{0.2cm}\sup_{s\leq t} \left|E^N_s - E_s\right|<\epsilon \right\}.
\end{equation}
Finally, for $s\in (t_\text{gel},t_+)$, we have by Cauchy-Schwarz that \begin{equation}
    \|P^N_s\|\leq \sqrt{2M^N_sE^N_s} \leq \sqrt{2 M^N_{t_+}E^N_{t_+}}<\sqrt{\frac{\epsilon^2}{2}}<\epsilon.
\end{equation} Therefore, on $A_N$, \begin{equation}
    \sup_{s \in (t_\text{gel}, t_+)} \left\|P^N_t\right\| <\epsilon.
\end{equation} Combining with the definition of $A_N$ shows that \begin{equation}
    A_N\subset\left\{\sup_{s\leq t} \left\|P^N_s\right\|<\epsilon \right\}
\end{equation} which concludes the proof.
\end{proof} 

\begin{lemma} As $t\downarrow t_\text{gel}$, the second moment $\langle \pi_n^2+\pi_e^2, \mu_t\rangle \rightarrow \infty$.  \end{lemma} \begin{proof} Fix $C<\infty$. For $R<\infty$, let $B_R=\{x\in S: \pi_n(x)\leq R, \pi_e(x) \leq R\}$. Let $\mu^R_t$ be the measures constructed in the proof of Lemma \ref{lemma: E and U}. Since $\langle \pi_n^2 + \pi_e^2, \mu_{t_\text{gel}}\rangle =\infty$, and $\mu^R_t\uparrow \mu_t$, we can choose $R<\infty$ such that \begin{equation}
    \langle \pi_n^2+\pi_e^2, \mu_{t_\text{gel}}1_{B_R}\rangle >C+1.
\end{equation} However, from the construction of the solution $(\mu_t)_{t\geq 0}$ in Lemma \ref{lemma: E and U}, the restricted measure $\mu_t1_{B_R}$ satisfies an integral equation (...), and so is continuous in total variation norm. In particular, the map \begin{equation}
    t\mapsto \langle \pi_n^2+\pi_e^2, 1_{B_R}\mu_t\rangle
\end{equation} is continuous, and hence, for some $\delta>0$ and all $t\in(t_\text{gel}, t_\text{gel}+\delta)$, we have \begin{equation}
    \langle \pi_n^2+\pi_e^2, \mu_t\rangle \geq \langle \pi_n^2+\pi_e^2, \mu_t1_{B_R}\rangle \geq \langle \pi_n^2+\pi_e^2, \mu_{t_\text{gel}}1_{B_R}\rangle -1 >C.
\end{equation}Since $C$ was arbitrary, this shows that $\langle \pi_n^2+\pi_e^2, \mu_t\rangle \rightarrow \infty$ as $t\downarrow t_\text{gel}$, as claimed.\end{proof}