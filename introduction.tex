\section{Introduction and Main Result}

Kac \cite{K56} introduced a model for the velocities of molecules undergoing collisions in which deterministic collisions based on trajectory intersection were replaced by a stochastic collision rule and position could be ignored.
In the large particle number limit (since particles do not have positions the hydrodynamic and thermodynamic limits are the same) the single particle marginal converges to the solution of the Boltzmann equation ?Wagner.
Interaction clusters for the Kac model (equivalence classes of particles that have collided by a given time) were introduced in \cite{GKSZ08} ...

In \cite{PSW17} it was formally shown that the sizes, of these interaction clusters can be understood as the cluster sizes of a stochastic coagulation process---the Marcus--Lushnikov process \cite{L78}---until a critical time at which a macroscopic cluster forms.
In the standard case the clusters in a Marcus--Lushnikov clusters are completely described by their size (or mass), however for Kac interaction clusters kinetic energy and momentum have to be included.
The resulting coagulation rates are a generalisation of the multiplicative coagulation kernel, which is the classic example of a coagulation rate definition that leads to macroscopic cluster formation \cite{L78,A99}.

Norris

Random graphs

Kac process with parameter $N$.
$N$ molecules each with mass 1 and a momentum in $\RR^d$.
State $\left(V_i^N(t)\right)_{i=1}^N$.
Elastic collisions.
Boltzmann limit.
Maxwell, hard sphere, very hard sphere.
\dhcomment{O Lanford III's derivation of Boltzmann equation - tree expansion}
\dhcomment{Omit, for obvious reasons, $\kappa=\gamma=0$}
For $i,j\in \{1,2,\dotsc, N \}$ write $i \sim_t j$ if the molecules $i$ and $j$ have collided prior to time $t$.
The (Kac-$N$) interaction clusters at time $t$ are the equivalence classes of $\sim_t$.
If $I$ and $J$ are two distinct, unordered (Kac-$N$) interaction clusters at time $t$ then their instantaneous merger rate is
\begin{multline}\label{e:cluster-merge-rate}
    \frac2N\sum_{i\in I}\sum_{j\in J}
     \left(\kappa + \gamma \norm{V_i^N(t) - V_j^N(t)}^2 \right)\\
    =
    \frac2N\kappa \abs{I}\abs{J}
    +\frac2N\gamma \abs{I} \sum_{j\in J}\norm{V_j^N(t)}^2
    +\frac2N\gamma \abs{J} \sum_{i\in I}\norm{V_i^N(t)}^2
    -\frac4N\gamma \left(\sum_{i\in I}V_i^N(t)\right)\cdot
    \left(\sum_{j\in J}V_j^N(t)\right).
\end{multline}