\section{Relation to Random Graph Processes}\label{sec:rg}

We study the following random graph processes:

\subsection{True Process}

Consider the complete graph $K_N$, and for each vertex $x$, define an independent Gaussian random variable $v_x\sim N(0,I).$ For an edge $e=(xy)$, \rpcomment{I think we understand an edge as an unordered pair, so I agree with this formula.} we define the edge rate \begin{equation*} W_e=2(\kappa+\gamma|v_x-v_y|^2) \sim 2\kappa+2\gamma \chi^2_d. \end{equation*} Independently, we let $(T_e)_{e\in E}$ be independent, $\text{Exponential}(1)$ random variables. We define a random graph process $(G^N_t)_{t\geq 0}$ by including the edge $e$ if $
    t \geq \frac{NT_e}{W_e}$.  \medskip \\ We note that, if $e, e'$ share a vertex, then the weights $W_e, W_{e'}$ are positively correlated, and so \begin{equation*} \mathbb{P}(\text{Both $e, e'$ included at time $t$})>p_N^2.\end{equation*} If $e, e'$ do not share a vertex, then the edges $e,e'$ are independent. This generalises to any number of edges, in a style similar to the FKG inequality. \medskip \\  For any time $t$, the probability that any given edge $e$ is included in $G^N_t$ is given by \begin{equation*}
        p^N_t=\mathbb{E}\left[1-\exp\left(-\frac{tW_e}{N}\right)\right].
    \end{equation*} We observe the following: \begin{lemma}\label{lemma:cutoff time} Fix $t\geq 0$. Then \begin{equation*}
        Np^N_t \rightarrow t\mathbb{E}W_e. \hspace{1cm} \text{as } N \rightarrow \infty.
    \end{equation*} In particular, \begin{itemize}
        \item if $t=\frac{1-\epsilon}{\mathbb{E}W_e}$, then $\limsup_N Np^N_t<1$;
        \item if $t=\frac{1+\epsilon}{\mathbb{E}W_e}$, then $\liminf_N Np^N_t>1$.
    \end{itemize}\end{lemma}
\subsection{Recent development}
   This is an example of the `inhomogenous random graphs' studied by \cite{BJR07}, with `Ground Space' $S=\mathbb{R}^d, \mu=N(0,\sigma^2I)$, and constructing a vertex space by taking $\mathbf{x}_n=(x_1,...,x_n)$ as an i.i.d. sample from $\mu.$ Using the alternative formulation \cite{BJR07}[Equation 2.6], we consider the time-dependent family of kernels
   \begin{equation}
       K_t(v,w)=2t(\kappa+\gamma|v-w|^2)
   \end{equation}
   ($K_t/t$ is the rate at which an unordered pair of molecules, one with velocity $v$ and one with velocity $w$ collide.)
   Consider the convolution operator \begin{equation}\label{eq: T original}
       T: L^2(\mu)\rightarrow L^2(\mu); \hspace{1cm} (T f)(v)=\int_{\mathbb{R}^d} K_1(v,w)f(w)\mu(dw). 
   \end{equation} Write $\|T\|$ for the operator norm of $T$. Then the critical time is $t_G=\|T\|^{-1}.$ We have the following phase transition, which follows from \cite{BJR07}[Theorem 3.1] on verifying that the kernels $K_t$ satisfy some regularity conditions.  \begin{theorem} \begin{itemize}
       \item Suppose $t\leq t_G$. Then, with high probability, the largest component of $G^N_t$ is $o(n)$.
       \item Suppose $t>t_G.$ Then, with high probability, the largest component of $G^N_t$ is $\Theta(N)$.
   \end{itemize} \end{theorem} 
   \begin{lemma}\label{lemma: the one I haven't done yet}
   \dhcomment{Need a result to the effect of: `The empirical measure corresponding to a graph $G^N \sim \mathcal{G}^\mathcal{V}(N,K)$ converges vaugely to $\mu_t$, where $\mu_t$ is the solution to the Smoluchowski equation starting at the pushforward of $m$}  \end{lemma}