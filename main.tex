\documentclass[11pt, notitlepage]{article}
\makeatletter\@addtoreset{section}{part}\makeatother%
\usepackage{amssymb}
\usepackage[margin=1.5cm]{geometry}
\usepackage{amsmath}
\usepackage{amsthm}
\usepackage{bbm}
\usepackage{amsfonts}
\usepackage{mathtools}
\usepackage{graphicx} %for graphics
\usepackage{subcaption} %for subfigures
\usepackage{relsize}
\usepackage{titling}
\usepackage[]{algorithm2e}
\usepackage{enumitem}
\usepackage{hyperref}
\newtheorem{example}{Example}[section]
\usepackage{framed}
\newcommand{\subtitle}[1]{%
  \posttitle{%
    \par\end{center}
    \begin{center}\large#1\end{center}
    \vskip0.5em}%
}

%Cambridge studies in advanced mathematics
\usepackage[dvipsnames]{xcolor}
\newcommand\myworries[1]{\textcolor{red}{#1}}
\usepackage{hyperref}
\hypersetup{
colorlinks=true,
citecolor=OliveGreen, %or blue or ...
linkcolor=blue %blue if you prefer ...
%pagebackref=true,
}


\newcommand\dhcomment[1]{\textcolor{blue}{\textbf{#1}}}

\newcommand\rpcomment[1]{\textcolor{Plum}{\textbf{#1}}}
%\title{Gelation Behaviour for Kac Interaction Clusters}
\title{Kac Interaction Clusters: A Bilinear Coagulation Equation and Phase Transition}
\author{Daniel Heydecker\thanks{University of Cambridge, dh489@cam.ac.uk. This work was supported by the UK Engineering and Physical Sciences Research Council (EPSRC) grant EP/L016516/1 for the University of Cambridge Centre for Doctoral Training, the Cambridge Centre for Analysis}, Robert I. A. Patterson\thanks{Robert.i.a.patterson@wias-berlin.de. This research was supported by the Deutsche Forschungsgemeinschaft (DFG) grant CRC 1114 ``Scaling Cascades in Complex Systems'', Project C08.}}
\date{\today}
\newcommand{\EE}{\ensuremath{\mathbb{E}}}
\newcommand{\NN}{\ensuremath{\mathbb{N}}}
\newcommand{\PP}{\ensuremath{\mathbb{P}}}
\newcommand{\dd}{\ensuremath{\mathrm{d}}}
\newcommand{\XX}{\ensuremath{\mathbb{X}}\xspace}
\newcommand{\sX}{\ensuremath{\mathcal{X}}\xspace}
\newcommand{\abs}[1]{\left\lvert{#1}\right\rvert}
\newcommand{\norm}[1]{|{#1}|}
\newcommand{\supnorm}[1]{\left\lVert{#1}\right\rVert_\infty}
\newcommand{\var}[1]{\mathrm{var}\left({#1}\right)}
\newcommand{\epvar}[1]{\mathrm{epvar}{\left({#1}\right)}}
\newcommand{\lip}[1]{\ensuremath{\mathrm{Lip}\left({#1}\right)}\xspace}
\newcommand{\rca}[1]{\mathrm{rca}{\left({#1}\right)}}
\newcommand{\tmu}{\ensuremath{\widetilde{\mu}}\xspace}
\newcommand{\tl}{\ensuremath{\widetilde{\lambda}}\xspace}

\newtheorem{thm}{Theorem}[section]
\newtheorem{lem}[thm]{Lemma}
\newtheorem{cor}[thm]{Corollary}
\newtheorem{rmk}[thm]{Remark}
\newtheorem{defn}{Definition}[section]

%\newtheorem{remark}[theorem]{Remark}
%\newtheorem{remark}[theorem]{Remark}

\setlength{\parindent}{0mm}
\setlength{\parskip}{2mm}

\begin{document}

\maketitle
\begin{abstract} We consider the interaction clusters for Kac's model of a gas with quadratic interaction rates, and show that they behave as coagulating particles with a bilinear coagulation kernel.
In the large particle number limit the distribution of the interaction cluster sizes is shown to follow an equation of Smoluchowski type.
Using a coupling to random graphs, we analyse the limiting equation, showing well-posedness, and a closed form for the time of the gelation phase transition $t_\mathrm{g}$ when a macroscopic cluster suddenly emerges.
We further prove that the second moment of the cluster size distribution diverges exactly at $t_\mathrm{g}.$
Our methods apply immediately to coagulating particle systems with other bilinear coagulation kernels.
\end{abstract}

%\begin{abstract} We consider the interaction clusters for Kac's model of a gas with quadratic interaction rates, and show that in the large particle limit, the interaction clusters follow an equation of Smoluchowski type with a bilinear coagulation kernel.
%Using a coupling to random graphs, we analyse the limiting equation, showing well-posedness, a closed form for the gelation time $t_\mathrm{g}$, and that the second moment of the cluster size distribution diverges exactly at $t_\mathrm{g}.$
%\end{abstract}

\section{Introduction and Main Results}

Boltzmann \cite{B96} pictured gases as collections of  billiard balls, moving in straight lines except when they ``randomly'' come close to one another or to the wall of a container. 
Under the intuition that particles should be `unordered at the molecular level', he then derived the Boltzmann equation for the molecular velocity distribution; justifying this intuition, which became known as `molecular chaos', has been an active area of research since. 
Grad helped clarify in exactly which limit a result should be possible, the Boltzmann--Grad limit, and the first result was due to Lanford \cite{L75} three quarters of a century after Boltzmann.
Lanford's result is restricted to a fraction of the mean free time, so that a positive fraction of the molecules have not collided with any other molecule; a convergence result for arbitrary finite time intervals and a wide range of initial conditions is still not available. For more detailed discussion and references to more recent results see \cite{PS17}.

Three quarters of a century is a long time to wait, and during this time Kac \cite{K56} introduced a simplified model for the velocities of gas molecules.
In this model, position is ignored and deterministic collisions based on trajectory intersections are replaced by a stochastic collision rule, an idea also introduced by Leontovich \cite{L35}.
For this simplified model, a wide body of literature \cite{McKean,Grunbaum,Sznitman,MischlerMouhot,N16} has shown propagation of chaos, and that the single particle marginal converges to the solution of the spatially homogeneous Boltzmann equation.

The principal challenge in analysing the full billiard model are the dependencies that arise between the molecular velocities as a result of pairs of particles either having collided or not collided.
These dependencies are suppressed in the Kac model.
The Lanford proof looks backward in time from a pair of molecules to the most recent collision involving either molecule, and then recursively builds a tree-like structure. For this argument, a restriction to a short, finite time is necessary, in order to guarantee that the branching process is subcritical and no infinite trees occur. 

The trees appearing in the Lanford proof may be extended to a partition of the molecules into \emph{interaction clusters} \cite{PSW16}, such that any two molecules which have collided belong to the same cluster.  
The concept of interaction clusters was introduced by Gabrielov et al.~\cite{GKSZ08}, who show interesting properties of the interaction cluster size distribution by molecular dynamics simulation and in particular the formation of a giant cluster in a phase transition. We refer to this phenomenon as \emph{gelation}. 



The distributions of the sizes of the interaction clusters are formally derived in \cite{PSW16} in terms of the solution of the Boltzmann equation.
Reducing to the case of cutoff Maxwell molecules for the spatially homogeneous Boltzmann equation, the phase transition observed in \cite{GKSZ08} can be identified precisely and the cluster size distributions observed to match those arising from the Smoluchowski coagulation equation with product kernel \cite{L78,A99,PSW16}. Heuristically, when a collision occurs, the corresponding clusters merge, which may be represented as a coagulation event at the level of interaction clusters.

In \cite{PSW17} the clusters were studied in the Kac setting and the restriction to Maxwell molecules was lifted.
This allowed a general collision rate including the hard sphere case and it was formally shown in a large particle number limit that the distribution of the cluster sizes converges to a version of the Smoluchowski coagulation equation with a time-dependent product kernel. 

In this work, we will consider a class of Kac processes for which the kernel is sufficiently tractable to allow a detailed analysis of the convergence and the limiting process, including the effects of a `gel' formed when interaction clusters become macroscopic beyond the critical time. Our techniques rely on the properties \emph{bilinear} coagulation kernels, generalising the notion of the product kernel \cite{N00}, and on exploiting a random graph representation of the particle system, in the form considered by \cite{BJR07}.

The study of gelation as the formation of a very large connected structure by joining basic building blocks goes back at least to Flory \cite{Flo41} whose motivation was hydrocarbon polymerisation in the manufacture of plastics.
Flory understood polymerisation as the formation of a random graph, rather than in terms of coagulation, and was aware of a sharp phase transition at the emergence of a giant connected structure, which he termed `gel'.
A rigorous proof of the random graph phase transition was provided by Erd\H{o}s and R\'enyi \cite{ER60}.
The existence of a phase transition corresponding to the formation of a giant particle in a coagulation model is first discussed by Lushnikov \cite{L78}, who uses this to explain the explosion of the second moment and the failure of mass conservation for the solution of the Smoluchowski equation with the product coagulation kernel.
The first connection between random graph and particle approaches appears in \cite{BP91}, where the phase transition is proved for the particle coagulation process and an interpretation as a new proof for a phase transition in the Erd\H{o}s-R\'enyi random graph is noted; this is also discussed in the survey article \cite{A99}.
We extend this connection, and show that the bilinear form of the merger rate allows us to couple the stochastic coagulant process to \emph{inhomogenous} random graphs as considered by \cite{BJR07}.

For particles with integer masses coagulating according to a kernel bounded above by the product kernel and below by its square root Jeon \cite{J98} proved the existence of a gelation phase transition and provided an upper bound on the gelation time and presents a number of different definitions of the gelation time for coagulation--fragmentation models.
The product coagulation kernel is the product of the masses of the coagulation partners, which have no properties other than mass.
Norris \cite{N99,N00} replaced mass with a general function growing no more than linearly in particle mass and suitable for use in models where particles have internal structure, a step that is important for the present work.
A lower bound for the gelation time was proved in \cite{N99} and an upper bound was added under appropriate assumptions in \cite{N00}; however, these bounds do not coincide in general. Normand \cite{Nm09} obtained explicit results concerning the blowup of a second moment for a sexed model which gives a lower bound on the gelation time, and in a later work \cite{Nm11} finds explicit expressions for the gelation time for a selection of models with arms.  Consequently, ours is one of the first models for which the gelation time can be found exactly; moreover, several aspects of our analysis extend what was previously known about the Smoluchowski equation, using the connection to random graphs \cite{BJR07}. We also propose a notion of \emph{bilinear} coagulation kernels, which would also be amenable to our analysis.

%For the Kac model with Maxwell molecules the interaction cluster size process is a realisation of the Markus--Lushnikov process \cite{L78} for particle coagulation.
%In the standard Marcus--Lushnikov process clusters are completely described by their size (or mass), however to generalise beyond Maxwell molecules the cluster description has to be augmented by information about the velocities of the molecules in the cluster.
%Generalising to hard spheres seems to require the cluster description to be the list of velocities of all its component molecules and so the state space is no simpler than for the full Kac model.
%However, in \cite{PSW17} a velocity dependent collision rate was introduced, which is simply the square of the hard sphere collision rate and for which the interaction clusters can be described by mass, kinetic energy and momentum, the latter two simply as sums over the constituent molecules, which do not have to be tracked individually.
%This is a very hard sphere model; it will be referred to as the quadratic model and it yields a quantitatively tractable cluster coagulation model, which is numerically seen to exhibit the qualitative features of the hard sphere model \cite{PSW17}.

%The quadratic model from \cite{PSW17} is tractable because it preserves most of the structure of the Marcus--Lushnikov process with the product coagulation kernel.
%Macroscopic cluster formation (also known as gelation) was studied rigorously for coagulation rates given by related generalisations of the product kernel in \cite{N00}.
%The present work builds on \cite{N00} in order to prove the rigorously the convergence of the interaction cluster distribution derived formally in \cite{PSW17} for linear combinations of the Maxwell and quadratic models for molecular collision rates.
%Results from the theory of random graphs play an important role in the analysis presented here and in particular are used to derive the main results, namely the exact time of the emergence of giant cluster, confirming the dimension dependence observed numerically in \cite{PSW17}.

%\dhcomment{Perhaps a more complete literature review, talking about (say) what is already known for Smoluchowski equations. I believe we are the first to rigorously compute a gelation time other than the (trivial) case of the multiplicative kernel}

\subsection{\textbf{Definitions}}
\subsubsection{\textbf{Markov jump processes}}\label{s:jump_procs}
The Kac process with parameter $N$ defines Markovian dynamics for $N$ molecules, each with mass $N^{-1}$ and a velocity $v_i(t)\in\mathbb{R}^d$, undergoing pairwise collisions that conserve momentum and energy. We write $B$ for the kernel on $\mathbb{R}^d \times \mathbb{S}^{d-1}$ underlying the dynamics of the Kac process, and $\left(v_i^N(t)\right)_{i=1}^N$ for the state of the Kac process at time $t\ge 0$.
At time $t$ the instantaneous collision rate between an unordered pair of molecules $i$ and $j$ is given by $\frac2NB(v_i^N(t) - v_j^N(t), \mathbb{S}^{d-1})$, and the post-collision velocities are given by 
\begin{align}
    \widetilde{v}_i^N(t)=\frac{v_i^N(t) + v_j^N(t)}{2} + \sigma \frac{|v_i^N(t) - v_j^N(t)|}{2};
\\   \widetilde{v}_j^N(t)=\frac{v_i^N(t) + v_j^N(t)}{2} - \sigma \frac{|v_i^N(t) - v_j^N(t)|}{2},
\end{align}
with $\sigma$ distributed according to $B(v_i^N(t) - v_j^N(t), \cdot)$.
For the purposes of this work we consider kernels whose total mass is of the form
\begin{equation} \label{eq: total collision rate}
    B(v, \mathbb{S}^{d-1}) =
  \left[\kappa + \gamma |v|^2 \right];
    \hspace{1cm} \kappa,\gamma \geq 0.
\end{equation}
The case $\gamma = 0$ is that of cutoff Maxwell molecules, which arises in $d$ dimensions when molecules are modelled as repelling each other with a force inversely proportional to the $(2d-1)^\mathrm{th}$ power of their separation.
Replacing \ref{eq: total collision rate} with $B(v, \mathbb{S}^{d-1}) =  |v|$ would give the \emph{hard spheres} model, which can be seen as the large exponent limit of an inverse power law force model, and arises when molecules are modelled as non-overlapping `billiards'.
The contribution from $\gamma |v|^2$ is a `very hard spheres' term, which we will refer to as quadratic.

The quadratic model has appeared previously in the literature \cite{Lu,PSW17} and is mathematically interesting, although it does not arise from a physical interaction model \cite{Villani}.
Since Maxwell molecules and the quadratic model (with $\kappa = 0$) are the cases  $B(v, \mathbb{S}^{d-1}) \propto |v|^n$ for $n=0,2$ respectively it is reasonable to regard the hard spheres ($n=1$) as an intermediate case and to suppose that much qualitative behaviour shared by the Maxwell and quadratic cases will also be seen for hard spheres.
This supposition is supported by numerical observations in \cite{PSW17}.
For our analysis, the Maxwell and quadratic cases have the crucial property that the collision rates can be expressed entirely in terms of mass, momentum and energy, which are all conserved during collisions, leading to closed form expressions for the gelation time and other quantities of interest, and we restrict ourselves to this setting.
%Our analysis is restricted to this case, although we will also discuss the implications for hard sphere case.
To avoid triviality, we assume throughout that $\kappa + \gamma > 0$.


Formally, we write $\sim_t$ for the equivalence relation on $\{1,2,\dots, N\}$ generated by the relation containing pairs $(i, j)$ such that molecules $i$ and $j$ have collided at, or before, time $t$. The (Kac-$N$) interaction clusters at time $t$ are the equivalence classes of $\sim_t$.
If $I$ and $J$ are two distinct, unordered interaction clusters, the instantaneous merger rate is
\begin{equation}\begin{split}\label{e:cluster-merge-rate}
  &  \frac{2}{N}\sum_{i\in I}\sum_{j\in J}
     \left(\kappa + \gamma \norm{v_i^N(t) - v_j^N(t)}^2 \right)\\
  \\& \hspace{1cm}   =
    \frac{2}{N}\kappa (\#I)(\#J)
    +\frac{2}{N}\gamma (\#{I})\left( \sum_{j\in J}\norm{v_j^N(t)}^2\right)
    \\ & \hspace{1.5cm} +\frac{2}{N}\gamma (\#{J})\left( \sum_{i\in I}\norm{v_i^N(t)}^2\right)
    -\frac{4}{N}\gamma \left(\sum_{i\in I}v_i^N(t)\right)\cdot
    \left(\sum_{j\in J}v_j^N(t)\right).\end{split}\end{equation}

From \eqref{e:cluster-merge-rate} one sees that Kac interaction cluster dynamics do not require a full knowledge of the Kac process, but only the number of molecules, combined momentum and combined kinetic energy of the molecules indexed by each interaction cluster; for $t\ge 0$, let $I^N_j(t): 1\le j \le k^N(t)$ be an enumeration of the equivalence classes of $\sim_t$, and $x^N_j(t)$ be the associated data of the clusters \begin{equation}
    x^N_j(t)=\sum_{i \in I^N_j(t)}\left(1,v^N_i(t),\frac{1}{2}\left|v^N_i(t)\right|^2\right).
\end{equation} 
Thanks to the conservation properties, these quantities are added when two particles belonging to distinct clusters collide, and no change occurs when if the two particles belong to the same cluster. Therefore, the interaction cluster size dynamics can be replicated by a system of coagulating particles $(x^N_j(t): j\le k^N(t))$ in the state space
\begin{equation}
    S=\{x=(n, p, e) \in \mathbb{N}\times \mathbb{R}^d\times [0,\infty): |p|\leq \sqrt{2ne} \}. 
\end{equation} We study the associated empirical measure, given by \begin{equation}\label{eq: sc1}\mu^N_t = \frac{1}{N}\sum_{j=1}^{k^N(t)} \delta_{x^N_j(t)}. \end{equation} We equip the the state space $S$ with the maps $\pi_n, \pi_p, \pi_e$ for the projection onto the respective factors, and $R:S\rightarrow S$ for the parity map $(n,p,e)\mapsto (n,-p,e)$. Rewriting the calculation (\ref{e:cluster-merge-rate}) in this notation, the rate at which unordered pairs of particles $\left\{ x,y \right\}$ in $S$ merge to form a new particle in $A \subset S$ is $2K(x,y, A)/N$ with
\begin{equation}\label{eq: smoluchowski kernel}
    K(x,y,dz)=\overline{K}(x,y)\delta_{x+y}(dz);
    \end{equation} \begin{equation} \label{eq: overline K}  \overline{K}(x,y)=\kappa \pi_n(x)\pi_n(y)+2\gamma\left(\pi_n(x)\pi_e(y)-\pi_p(x)\cdot \pi_p(y)+\pi_e(x)\pi_n(y)\right).
\end{equation} 
This is a Marcus--Lushnikov coagulation process \cite{L78} on $S$, which we will refer to as the \emph{stochastic coagulant}.
Note that a $1/N$ scaling of the pair interaction rate is used, which ensures that each molecule has a total collision rate that is of order 1; this is to be expected when modelling a gas where the mean free time is a positive real number.
Dividing jump rates by $N$ is equivalent to accelerating time by the same factor and this alternative formulation means that the jump rates in the definition of the ``stochastic coalescent'' in  \cite{A99} as well as of the ``stochastic $K$-coagulant'' in \cite{N00} omit the $1/N$ from the rates and rescale time when taking the $N\rightarrow \infty$ limit. \medskip \\ Finally, we remark  that the definitions (\ref{eq: sc1}, \ref{eq: smoluchowski kernel}, \ref{eq: overline K}) make sense in the more general case when the number of particles in the underlying Kac process is replaced by a random number $l_N$ of the same order as $N$; in this case, the scaling factors $N, N^{-1}$ in (\ref{eq: sc1}, \ref{eq: smoluchowski kernel}) remain fixed. This mild generalisation will be helpful for our arguments.

\subsubsection{\textbf{Limiting kinetic equations}}
We now consider various forms of the limiting Smoluchowski equation.  Define a drift operator $L$, by specifying for all bounded measurable $f\colon S \rightarrow \mathbb{R}$, 
 \begin{equation} \label{eq: drift wo gel}
    \langle f,L(\mu)\rangle=\frac{1}{2}\int_{S^2}\{f(x+y)-f(x)-f(y)\}\overline{K}(x,y)\mu(dx)\mu(dy).
\end{equation}
The associated evolution equation for $(\mu_t)_{t<T}$ is that, for all $t<T$,
\begin{equation}
    \tag{E-G}\label{eq: E} \mu_t =  \mu_0 +\int_0^t L(\mu_s) ds.
\end{equation} Following \cite{N00}, we say that a family $(\mu_t)_{t<T}$ of positive measures is a solution to (\ref{eq: E}) if the following hold: \begin{enumerate}[label=\roman{*}).] \item For all Borel sets $A\subset S$, the map $t\mapsto \mu_t(A)$ is measurable; \item For all bounded, measurable functions $f:S\rightarrow \mathbb{R}_+$ of compact support, $\langle f, \mu_0\rangle<\infty$; \item For all compact subsets $S'\subset S$ and all $t<T$, \begin{equation} \int_0^t ds \int_{S'\times S}\overline{K}(x,y)\mu_s(dx)\mu_s(dy)<\infty; \end{equation}  \item For all bounded, compactly supported functions $f:S\rightarrow \mathbb{R}$, and $t<T$, \begin{equation}
    \langle f, \mu_t \rangle =  \langle f,\mu_0\rangle +\int_0^t \langle f, L(\mu_s)\rangle ds
\end{equation} \end{enumerate} 
 This captures the effects of coagulations between finite clusters. However, as discussed above, we wish to include the possibility of a macroscopic component, which we term \emph{gel}. To include this effect, we define a modified drift operator $L_\mathrm{g}(\mu_t)$ by specifying, for bounded, measurable $f:S\rightarrow \mathbb{R}$, \begin{equation} \langle f,L_\mathrm{g}(\mu_t)\rangle =\langle f, L(\mu_t)\rangle -\int_{S}f(x)\overline{K}(x,y)\mu_t(dx)(\mu_0-\mu_t)(dy). \end{equation} A \emph{coagulant} is then a solution $(\mu_t)_{t<T}$ to \begin{equation} \tag{E+G} \label{eq: E+G}
    \mu_t= \mu_0 + \int_0^t L_\mathrm{g}(\mu_s)ds.
\end{equation} Here, the additional term comes into play only after $\mu_t$ ceases to conserve the quantities $\langle \pi_n, \mu_t\rangle, \langle \pi_p, \mu_t\rangle ,\langle \pi_e, \mu_t\rangle$, and the extra term represents the interaction with the gel.
This may be interpreted concretely in a similar sense to  (\ref{eq: E}) above.
This generalises the Smoluchowski coagulation equations \cite{vS16} analagous to that of Flory \cite{ZS80}.
The solution to this deterministic evolution problem is a `$K$-coagulant' in the language of \cite{N00}. \medskip \\ We write \begin{equation}\label{eq: gel data} \begin{split}
   g_t&= (M_t, P_t, E_t)=\langle x, \mu_0-\mu_t\rangle \\ &= \left(\langle \pi_n, \mu_0-\mu_t\rangle,\langle \pi_p, \mu_0-\mu_t\rangle,\langle \pi_e, \mu_0-\mu_t\rangle\right) \end{split}
\end{equation} for the mass, energy and momentum of the gel. Following remarks in \cite{N00}, one may show that if $\mu_t$ is a solution to (\ref{eq: E+G}), then the maps $t\mapsto \langle \pi_n, \mu_t\rangle, \langle \pi_e, \mu_t\rangle$ are non-increasing, which guarantees that $M_t, E_t\ge 0$. We write  $S_\mathrm{g}$ for the continuum analogue of the state space $S$, given by \begin{equation}
    S_\mathrm{g}=[0,\infty)\times \mathbb{R}^d\times [0,\infty)
\end{equation} and use the same notation $\pi_n, \pi_p, \pi_e$ for the projections onto the factors, as for $S$. When $x\in S$ and $g\in S_\mathrm{g}$, we use $\overline{K}(x,g)$ for the rate of absorption, given by (\ref{eq: overline K}) with the new meanings of $\pi_n(g), \pi_p(g),\pi_e(g).$ We will also write $\varphi$ for the linear combination $\varphi=\pi_n+\pi_e$, defined on both $S$ and $ S_\mathrm{g}$.

\begin{defn}[Conservative Solutions] As noted above, the functions $t\mapsto \langle \pi_n, \mu_t\rangle$ and $t\mapsto \langle \pi_e, \mu_t\rangle$ are non-increasing, whenever $(\mu_t)_{t<T}$ is a local solution to either \eqref{eq: E} or \eqref{eq: E+G}. We say that a solution $(\mu_t)_{t<T}$ is \emph{conservative} if both are constant on $[0,T)$, or equivalently, if $\langle \varphi, \mu_t\rangle$ is constant on $[0,T)$. \end{defn}
 Thus, any solution to (\ref{eq: E+G}) is conservative up to some time $0\le t_\mathrm{g}\leq \infty$, and non-conservative thereafter.
 
\begin{defn}[Metrisation of Convergence] Let $\mathcal{M}=\mathcal{M}_{\le 1}(S)$ be the space of measures on $S$ with total mass at most 1. We equip $\mathcal{M}$ with the \emph{vague} topology $\mathcal{F}(\mathcal{M}, C_c(S))$ induced by continuous, compactly supported functions on $S$, and fix a complete metric $d_0$ compatible with this topology. Let $\mathcal{M}^\star$ be the space $\mathcal{M}\times S_\mathrm{g}$, and define the complete, separable metric \begin{equation} d^\star\left((\mu, g), (\mu', g')\right):=d_0(\mu, \mu')+|g-g'| \end{equation} where $|\cdot|$ is the Euclidean distance on $S_\mathrm{g}\subset \mathbb{R}^{d+2}.$ \end{defn}

\subsection{\textbf{Statement of Results}}\label{sec: results}

We make the following hypotheses on the initial data $\mu_0$.
\\A1. The initial data $\mu_0 \in \mathcal{M}$ is equal to its pushforward under $R$, that is, $\mu_0 = \mu_0 \circ R^{-1}$.
\\ A2. The initial data is given in terms of a sub-probability measure $m$ of particle velocities, by pushforward under the map
\begin{equation}
    \iota: \mathbb{R}^d \rightarrow S; \hspace{1cm} v\mapsto (1, v, |v|^2).
\end{equation}
\\ {A3.} For $0\leq k\leq 6$, we have
\begin{equation}
    \sigma_k(m)=\langle |v|^k, m\rangle <\infty.
\end{equation} 
\\ {A4.} The underlying measure $m$ has $m(\{0\})=0.$
\\We remark that much of our analysis can be done without the assumption (A2.), which corresponds to \emph{monodisperse} initial conditions where each cluster is initially a single particle. However, this assumption is natural in the context of Kac interaction clusters, as it guarantees that the number of particles $N$ initially in the stochastic coagulant corresponds to the number of particles in the underlying Kac process. 
\\ We summarise our results on the analysis of the Smoluchowski equation (\ref{eq: E+G}) as follows.
\begin{thm}\label{thrm: Smoluchowski equation}
Let $\mu_0$ be an initial measure on $S$ satisfying ({A1-4}.), and assume that $m$ is a probability measure. Then the equation (\ref{eq: E+G}) has a unique solution $(\mu_t)_{t\geq 0}$ starting at $\mu_0$; we write $g_t=(M_t, P_t, E_t)$ for the gel data defined in (\ref{eq: gel data}). This solution has the following properties.
\paragraph{\textbf{1. Phase Transition.}} Let $t_\mathrm{g}$ be the first time at which the solution $\mu_t$ fails to be conservative, that is:
\begin{equation} t_\mathrm{g}:=\inf\{t\ge 0: \langle \varphi, \mu_t\rangle < \langle \varphi, \mu_0\rangle \}.
\end{equation}
Then
$t_\mathrm{g}\in (0,\infty)$, and can be given explicitly in terms of the moments of $m$ as \begin{equation}\label{eq: closed form for tg}
       t_\mathrm{g}= \frac{1}{\kappa +2\gamma\sigma_2(m) + \sqrt{(\kappa+2\gamma\sigma_2(m))^2+4\gamma^2(\sigma_4(m)-\sigma_2^2(m))}}.
   \end{equation}

\paragraph{\textbf{2. Behaviour of the Second Moment.}} Consider the second moment
\begin{equation} \mathcal{E}(t)=\langle \varphi^2, \mu_t\rangle.
\end{equation}
Then \begin{enumerate}[label=\roman{*}).]
    \item $\mathcal{E}(t)$ is finite and continuous, and so locally bounded, on $[0, \infty)\setminus\{t_\mathrm{g}\}.$ 

    \item On $[0, t_\mathrm{g})$, $\mathcal{E}$ is monotonically increasing.
    
    \item At the gelation time, $\mathcal{E}(t_\mathrm{g})=\infty$, and $\mathcal{E}(t)\rightarrow \infty$ as $t\rightarrow t_\mathrm{g}.$ 
\end{enumerate}


\paragraph{\textbf{3. Representation of Gel Data.}} Let $m$ be the underlying distribution of initial velocities, as in ({A2}.) For each $t\ge 0$, there exists a unique maximal pair $c_t=(a_t, b_t) \ge 0$ such that, for all $v\in \mathbb{R}^d$, \begin{equation}\label{eq: NLFP 1} a_t+b_t|v|^2=2t \int_{\mathbb{R}^d} (1-e^{-a_t-b_t|w|^2})(\kappa+\gamma|v-w|^2)m(dw). \end{equation} $c_t$ undergoes a phase transition at time $t_\mathrm{g}$: if $t\le t_\mathrm{g}$, then $c_t=0$, and if $t>t_\mathrm{g}$ then $a_t>0$. If, in addition, $\gamma>0$, then if $t>t_\mathrm{g}$ then both components $a_t, b_t>0$. Moreover, the map $t\mapsto c_t$ is continuous. \medskip \\  The gel data are given in terms of $c_t$ by \begin{equation}\label{eq: formula for M, E_0}
    g_t =(M_t, P_t, E_t)= \int_{\mathbb{R}^d} \left(1,0, \frac{1}{2}|v|^2\right)(1-e^{-a_t-b_t|v|^2})m(dv).
\end{equation} Therefore, if $t>t_\mathrm{g}$ then both $M_t, E_t>0$. Moreover, the map $t\mapsto g_t$ is continuous, and $g_{t_\mathrm{g}}=0$.
\paragraph{\textbf{4. Gel Dynamics.}} The map $t\mapsto g_t$ is differentiable on $t\in(t_\mathrm{g}, \infty)$, and
\begin{equation}
    \frac{d}{dt}M_t=\kappa\left<\pi_n^2,\mu_t\right>M_t +2\gamma 
        \left<\pi_n \pi_e,\mu_t \right>M_t +
        \left<\pi_n^2,\mu_t \right>E_t ;
\end{equation}
\begin{equation}
    \frac{d}{dt}E_t=
      \kappa \left<\pi_n \pi_e,\mu_t\right>M_t +
      2\gamma 
        \left<\pi_e^2,\mu_t \right>M_t +
        \left<\pi_n \pi_e,\mu_t \right>E_t.
\end{equation}
\paragraph{\textbf{5. Order of the Phase Transition, and the Size-Biasing Effect.}} The map $t\mapsto c_t=(a_t,b_t)$ is right-differentiable at $t_\mathrm{g}$, and $a'_{t_\mathrm{g}^+}>0$. The ratio of the components is \begin{equation} \frac{b'_{t_\mathrm{g}^+}}{a'_{t_\mathrm{g}^+}}=\lambda =\frac{\sqrt{\kappa^2+4\gamma(\kappa\sigma_2(m)+\gamma \sigma_4(m))}-\kappa}{2(\kappa\sigma_2(m)+\gamma\sigma_4(m))}.\end{equation} As a consequence, the phase transition is first order; that is, the right-derivative of the gel data $M_t, E_t$ exists and is positive at $t_\mathrm{g}$: \begin{equation}M'_{t_\mathrm{g}^+}>0; \hspace{1cm} E'_{t_\mathrm{g}^+}>0 \end{equation} and we have the following \emph{size-biasing} effect, which further quantifies the way in which gelation is driven by the fast particles:  \begin{equation}
    \lim_{t\downarrow t_\mathrm{g}}\frac{E_t}{M_t}=\frac{1}{2}\frac{\sigma_2(m)+\lambda\sigma_4(m)}{1+\sigma_2(m)\lambda}.
\end{equation} In particular, unless $\gamma=0$ or $|v|$ is constant $m$-almost everywhere, we have a positive bias \begin{equation}
    \lim_{t\downarrow t_\mathrm{g}}\frac{E_t}{M_t}>\frac{1}{2}\int_{\mathbb{R}^d}|v|^2m(dv).
\end{equation}

A notable case of physical interest is when we take $m$ to be a Gaussian density $N_d(0, \sigma^2 I)$, which certainly satisfies (A1-4.). In this case, the gelation time evaluates to
\begin{equation}
    \label{eq: formula of tgel} t_\mathrm{g} = \frac{1}{\kappa+2\gamma d \sigma^2+
    \sqrt{(\kappa+2\gamma d \sigma^2)^2+8\gamma^2 d \sigma^2}}.
\end{equation} \end{thm}
We also prove the following convergence theorem, relating the stochastic coagulant to the solution of the limit equation. Firstly, following ideas of \cite{N00}, we show that the empirical measure $\mu^N_t$ converges to the limiting solution $(\mu_t)_{t\ge 0}$ in the vague topology, uniformly in time. \medskip \\ One might also ask about the connection between the stochastic coagulants and the limiting gel $(g_t)_{t\ge 0}$; since each stochastic coagulant preserves $\langle \varphi, \mu^N_t\rangle$, there is no natural analogue of the representation (\ref{eq: gel data}). However, we will show that the limiting gel $(g_t)_{t\ge 0}$ is closely related to the data of the largest (by particle number) cluster. In particular, the convergence of this \emph{stochastic gel} may be viewed as a phase transition at time $t_\mathrm{g}$.
\begin{thm} \label{thrm: convergence of stochastic coagulent} Let $\mu_0$ be a probability measure on $S$ satisfying ({A1-4}.), for some probability measure $m$, and let $(\mu_t)_{t\ge 0}, (g_t)_{t\ge 0}$ be the associated solution to (\ref{eq: E+G}) and corresponding gel. For $N\ge 1$, let $\mu^N_t$ be the stochastic coagulant, where the initial velocities of particles are sampled independently from $m$. Define \begin{equation} g^N_t=(M^N_t, P^N_t, E^N_t)\end{equation}  as the data of the largest cluster in the stochastic system, normalised by $N^{-1}$. Then we have the convergence \begin{equation} \label{eq: convergence of stochastic system}
    \sup_{t\ge 0}\hspace{0.1cm}d^\star\left((\mu^N_t, g^N_t),(\mu_t, g_t)\right) \rightarrow 0
\end{equation} in probability. In particular, we have the following phase transition: \begin{enumerate}[label=\roman{*}).]
    \item If $t\le t_\mathrm{g}$, then the largest cluster has size of the order $o_\mathrm{p}(N)$;
    \item If $t>t_\mathrm{g}$, the largest cluster has size of the order $\Theta_\mathrm{p}(N)$.
\end{enumerate} 

Moreover, if $\xi_N$ is any sequence with $\xi_N\rightarrow \infty$ and $\frac{\xi_N}{N}\rightarrow 0$, then we may define $\widetilde{g}^N_t=(\widetilde{M}^N_t, \widetilde{P}^N_t, \widetilde{E}^N_t)$ by summing the data of all clusters with mass at least $\xi_N$, and normalising by $N$. Then the same result holds when we replace $g^N_t$ by $\widetilde{g}^N_t$ in (\ref{eq: convergence of stochastic system}).
\end{thm} 

Here, and throughout, we use the notation $o_\mathrm{p}(\cdot), \mathcal{O}_\mathrm{p}(\cdot), \Theta_\mathrm{p}(\cdot)$ for the probabilistic equivalents of $o(\cdot), \mathcal{O}(\cdot), \Theta(\cdot)$. Precise definitions can be found in \cite{JLR}.

\subsection{\textbf{Connection of results to the Literature}}
We briefly discuss Theorems \ref{thrm: Smoluchowski equation} and \ref{thrm: convergence of stochastic coagulent} in relation to some other results in the literature. \medskip \\ The starting point for this project was the conjecture raised in \cite{PSW17} that gelation occurs before or at the mean free time $t_\mathrm{mf}$. In \cite{PSW17} this is supported by numerical evidence for a variety of collision kernels, including the hard spheres and quadratic cases. For the case of the quadratic kernel (\ref{eq: total collision rate}) considered above, the mean free time is \begin{equation} t_\mathrm{mf}=\frac{1}{2\kappa+4\gamma\sigma_2(m)}. \end{equation} The first item Theorem \ref{thrm: Smoluchowski equation} therefore gives a positive, analytical solution to this conjecture. Moreover, except in the special cases where $\gamma=0$ or $|v|$ is constant $m$- almost everywhere, we have the strict inequality $t_\mathrm{g}<t_\mathrm{mf}$. This potentially counterintuitive result may be understood as saying that large velocity tails of $m$ reduce the gelation time more than they reduce the mean-free time; heuristically, gelation is driven by the fastest particles.

Together with Theorem \ref{thrm: convergence of stochastic coagulent}, item 2 of Theorem \ref{thrm: Smoluchowski equation} follows exactly the idea of Lushnikov: the formation of a giant particle at $t_\mathrm{g}$ corresponds exactly to blowup of the second moment $\mathcal{E}(t)$ and breakdown of conservation. We go further, and show that the second moment is finite after $t_\mathrm{g}$, since the giant particle does not correspond to anything in the limit measure. Indeed, the only time when the second moment diverges is at the critical point $t_\mathrm{g}$ when a giant particle is about to form; following the percolation literature, this may be thought of as an `incipient giant component' \cite{Ald16}. \medskip \\ Theorem \ref{thrm: convergence of stochastic coagulent}, concerning the convergence of the stochastic coagulant and the giant particle, follows ideas of \cite[Theorem 4.1]{N00}. However, in our case, we can use the uniqueness statement from Theorem \ref{thrm: Smoluchowski equation} to conclude a local uniform convergence result in Lemma \ref{lemma: local uniform convergence of stochastic coagulent}. The remainder of this theorem follows from a careful analysis of the gel through an associated random graphs process in Sections \ref{sec: uniform convergence}, \ref{sec: COG}. \medskip \\ Following well-known results on Erd\H{o}s-Renyi graphs (see, for example, \cite{B01}), we could further ask about the size of the largest component at and below criticality, and of the small components above criticality.  The results of \cite{BJR07} address both of these results for certain classes of graph processes, but unfortunately the results do not cover our kernel in general.   
\subsection{\textbf{A Note on Generalisations.}} As commented above, our analysis rests on the \emph{bilinear} form of the total rate $\overline{K}$, which allows us to connect the Smoluchowski equation to random graphs in Sections \ref{sec: IRG}, \ref{sec: coupling_to_random_graph}. This motivates the following definition.
\begin{defn} For any measurable space $S$, we say that a kernel $K$ on $S\times S\times S$ is \emph{bilinear} if there exists a finite set of measurable mappings $(\pi_i)_{i\le n+m}$ and a symmetric real matrix $(a_{ij})_{i, j\le n+m}$ satisfying the following. 
\begin{enumerate}[label=\roman{*}).]
\item For all $i\le n+m$ and all $x,y\in S$, \begin{equation} \pi_i=\pi_i(x)+\pi_i(y) \hspace{1cm}K(x,y,\cdot)\text{- almost everywhere}. \end{equation} \item For $i\le n$, the map $\pi_i:S\rightarrow \mathbb{R}$ takes only nonnegative values.
\item There exists a measurable map $R: S\rightarrow S$ such that $R\circ R$ is the identity on $S$, and  \begin{equation} \pi_i \circ R=\begin{cases} \pi_i & 1\le i\le n; \\ -\pi_i, & n+1\le i \le n+m \end{cases}.  \end{equation}
\item There exists a constant $C$ such that, for all $x\in S$, \begin{equation} \sum_{i> n} \pi_i(x)^2 \le C \sum_{i\le n} \pi_i(x)^2. \end{equation} \item For all $i, j\le n$, $a_{ij}\ge 0$, and there is at least one pair $i, j\le n$ such that $a_{ij}>0.$
\item For all $x,y\in S$, the total rate $\overline{K}(x,y)=K(x,y,S)$ may be expressed as \begin{equation} \overline{K}(x,y)=\sum_{i,j\le n+m}a_{ij}\pi_i(x)\pi_j(y). \end{equation}  
\end{enumerate}  \end{defn}
In this setting, we define $\varphi:=\sum_{i\le n} \pi_i$, and seek processes of measures $(\mu_t)_{t\ge 0}$ on $S$ solving the equation analagous to (\ref{eq: E+G}). Similarly, one can consider a stochastic coagulant $(\mu^N_t)_{t\ge 0}$ defined analagously to (\ref{eq: smoluchowski kernel}, \ref{eq: overline K}), with $\lceil N \mu_0(S)\rceil$ particles initially sampled independely from $\mu_0(S)^{-1}\mu_0(\cdot)$. In this context, we would require the following conditions on the initial data $\mu_0$: 
\\ (A1'.) The measure $\mu_0$ is invariant under the transformation $R$ in point iii).: $\mu_0\circ R^{-1}=\mu_0.$
\\ (A3'.) For all $i\le n$, we have $\langle \pi_i^3, \mu_0\rangle <\infty.$
\\ (A4'). For all $i \le n$, \begin{equation} \mu_0(x\in S: \pi_i(x)=0)=0.\end{equation} In this setting, no assumption analagous to (A2.) is necessary. As remarked above this assumption is included in Theorems \ref{thrm: Smoluchowski equation}, \ref{thrm: convergence of stochastic coagulent} in order to guarantee that the number of particles initially in the stochastic coagulant coincides with the number of particles in the underlying Kac process. \medskip \\ Under these hypotheses, all of the arguments used in this paper may be adapted to prove results analagous to Theorems \ref{thrm: Smoluchowski equation}, \ref{thrm: convergence of stochastic coagulent}. In this context, a closed form expression for the gelation time $t_\mathrm{g}$, right-derivative $\lim_{t\downarrow t_\mathrm{g}} \frac{g_t}{t-t_\mathrm{g}}$, or quantities analagous to the bias $\lim_{t\downarrow t_\mathrm{g}} \frac{E_t}{M_t}$ may not be available, but these will instead be characterised in terms of finite-dimensional, explicit eigenvalue problems, and can therefore be found numerically if an explicit value is desired.

\subsection{\textbf{Plan of the Paper.}} Our programme will be as follows. \begin{enumerate} \item In Section \ref{sec:SE}, we will prove that the limiting equation (\ref{eq: E+G}) has unique, globally defined solutions, based on a truncation argument from \cite{N99,N00}.
\item In Section \ref{sec: csc}, we prove an initial result, Lemma \ref{lemma: local uniform convergence of stochastic coagulent}, on the convergence of the stochastic coagulant, using the ideas of \cite[Theorem 4.1]{N00}. This will later be used to prove later points of Theorem \ref{thrm: Smoluchowski equation} based on probabilistic arguments.
\item In Sections \ref{sec: IRG}, \ref{sec: coupling_to_random_graph}, we introduce the theory of inhomogenous random graphs set out in \cite{BJR07}, and show how a particlar example of these graphs may be coupled to the stochastic coagulant. The critical time $t_\mathrm{c}$ for these graphs may be found exactly, leading to the explicit expression in Theorem \ref{thrm: Smoluchowski equation}. \item A weakness of the preceding sections is that, a priori, the critical time $t_\mathrm{c}$ for the graph processes may differ from the gelation time $t_\mathrm{g}$; in Section \ref{sec: ECT}, we show that this cannot happen. This is based on a preliminary version of Theorem \ref{thrm: convergence of stochastic coagulent}, which shows convergence of $(\mu^N_t, g^N_t)$ at a single fixed time $t\ge 0$. 
\item Section \ref{sec: finiteness of second moment} is dedicated to a proof of item 2 of Theorem \ref{thrm: Smoluchowski equation}. The statements about the subcritical regime follow general ideas in \cite{N99,N00}, while statements about the critical and supercritical cases use additional ideas from the theory of random graphs.
\item Section \ref{sec: gel dynamics} uses the ideas of previous sections to prove items 3 and 4 of Theorem \ref{thrm: Smoluchowski equation}, concerning the gel data $g_t$ beyond the critical point. \item Section \ref{sec: uniform convergence} uses the analysis of the gel to extend Lemma \ref{lemma: local uniform convergence of stochastic coagulent} to show that convergence is uniform in time.
\item Section \ref{sec: BNCP} proves item 5 of Theorem \ref{thrm: Smoluchowski equation}, concerning the behaviour near the critical point. This completes the proof of this theorem. \item To finish the proof of Theorem \ref{thrm: convergence of stochastic coagulent}, we revisit the ideas of Section \ref{sec: ECT} to prove convergence of the stochastic gel $g^N_t, \widetilde{g}^N_t$, uniformly in time. This is the focus of Section \ref{sec: COG}, and builds further on ideas of previous sections. \end{enumerate}
\section{\textbf{Well-Posedness of the Limiting Equation}}\label{sec:SE}  This chapter is dedicated to a first analysis of the Smoluchowski equations (\ref{eq: E}, \ref{eq: E+G}), following Norris \cite{N99,N00}. Our goal in this section is to prove the following lemma on the well-posedness of (\ref{eq: E+G}).

\begin{lem}\label{lemma: E and U} For any measure $\mu_0 \in \mathcal{M}$ satisfying (A1.), the equation with gel (\ref{eq: E+G}) has a unique global solution $(\mu_t)_{t\geq 0}$ starting at $\mu_0$. Moreover, the momentum $P_t=0$ for all time $t\ge 0$. \end{lem}
\begin{cor}\label{cor: maximal conservative solutions} Suppose $(\mu'_t)_{t<T}$ is a conservative local solution to the equation without gel, (\ref{eq: E}), starting at $\mu_0$. Then $\mu_t=\mu'_t$ for all $t<T$, and $T<t_\mathrm{g}$. Hence, (\ref{eq: E}) has a unique maximal conservative solution, given by $(\mu_t)_{t<t_\mathrm{g}}$.
\end{cor}

Our proof of Lemma~\ref{lemma: E and U} is an adaptation of the arguments in \cite[Section 2]{N99} and \cite[Section 2]{N00} and is based on a truncation argument. Recalling that $\varphi=\pi_n+\pi_e$, we see that  $\overline{K}(x,y)\leq \Delta\varphi(x)\varphi(y)$ for some $\Delta=\Delta(\kappa, \gamma).$ For all $\xi>0$, we define the truncated particle space \begin{equation} \label{d:trunc_space} S_\xi=\{x\in S: \varphi(x) \le \xi\}. \end{equation}

We consider the following `truncation at level $\xi$': in the empirical measure, we track only those particles inside $S_\xi$, and consider all other particles to belong to a `truncated gel'. Although the particles in the truncated gel affect the dynamics in $S_\xi$, these contributions depend only on the total mass, momentum and energy $g^\xi$ of the truncated gel, due to the bilinear form of the kernel. This leads to an ordinary differential equation with Lipschitz coefficients in an infinite dimensional space. \medskip \\ We formalise this intuition as follows.  For a measure $\mu^\xi$ supported on $S_\xi$ and $g^\xi\in S_\mathrm{g}$, we define a signed measure $L^\xi_\mathrm{g}(\mu^\xi, g^\xi)$ on $S_\xi$ by specifying, for all $f\in C_c(S)$, \begin{equation}\begin{split}\label{eq: truncated drift 1} &\left\langle f, L^\xi_\mathrm{g}(\mu^\xi, g^\xi)\right\rangle \\[1ex] &\hspace{1cm}=\frac{1}{2}\int_{S_\xi^2}[f(x+y)1[\varphi(x+y)\le \xi]-f(x)-f(y)]\overline{K}(x,y)\mu^\xi(dx)\mu^\xi(dy)\\& \hspace{2.5cm}-\int_{S_\xi}f(x)\overline{K}(x,g^\xi)\mu^\xi(dx). \end{split}\end{equation} This corresponds to the dynamics of particles inside $S_\xi$. The rate of change of the truncated gel data is given by \begin{equation} \begin{split}\label{eq: truncated drift 2} \widehat{L}^\xi_\mathrm{g}(\mu^\xi,g^\xi)&=\frac{1}{2}\int_{S_\xi^2} (x+y)1[\varphi(x+y)> \xi]\overline{K}(x,y)\mu^\xi(dx)\mu^\xi(dy) \\ &\hspace{1cm}+\int_{S_\xi}x\overline{K}(x,g^\xi)\mu^\xi(dx).\end{split}\end{equation}We now seek measures $\mu^\xi_t$ supported on $S_\xi$ and gel data $g^\xi_t=(M^\xi_t, P^\xi_t, E^\xi_t)\in S_\mathrm{g}$ such that, for all bounded measurable $f$ on $S_\xi$, \begin{equation} \tag{E$|^1_\mathrm{\xi}$} \label{eq:rE1}  \langle f, \mu^\xi_t\rangle = \langle f, \mu^\xi_0\rangle +\int_0^t\left\langle f, L^\xi_\mathrm{g}(\mu^\xi_s, g^\xi_s)\right\rangle ds;\end{equation}
\begin{equation} \label{eq: rE2} \tag{E$|^2_\mathrm{\xi}$}
g^\xi_t=g^\xi_0+\int_0^t\widehat{L}^\xi_\mathrm{g}(\mu^{\xi}_s,g^\xi_s)ds.
\end{equation} 
We will use the following existence and uniqueness result for the restricted dynamics (\ref{eq:rE1}, \ref{eq: rE2}).
\begin{lem}\label{lemma: E and U of Restricted}[Existence and Uniqueness of Restricted Dynamics]\label{lemma: restricted dynamics} Suppose $\mu^\xi_0$ is a finite measure on $S_\xi$ which satisfies ({A1}.), and $g^\xi_0 = (M^\xi_0, 0, E^\xi_0)\in S_\mathrm{g}$. Then there exists a unique map $(\mu^\xi_t, g^\xi_t)$ on $[0, \infty)$, which solves the restricted dynamics (\ref{eq:rE1}, \ref{eq: rE2}). Moreover, for all $t\le T$, $\mu^\xi_t$ is a positive, finite measure on $S_\xi$, $P^\xi_t=0$ and $g^\xi_t \in S_\mathrm{g}$. 
\end{lem}
\begin{proof}[Sketch Proof of Lemma \ref{lemma: restricted dynamics}] This may be proved by a trivial modification of the arguments in \cite[Proposition 2.2]{N99}. We define Picard iterates $(\mu^{(\xi,n)}_t, g^{(\xi,n)}_t: n\ge 0, t\ge 0)$ by \begin{align} (\mu^{(\xi,0)}_t, g^{(\xi,0)}_t)&=(\mu^\xi_0, g^\xi_0);\\ \left(\mu^{(\xi,n+1)}_t, g^{(\xi,n+1)}_t\right)&=(\mu^\xi_0,g^\xi_0)+\int_0^t (L^\xi_\mathrm{g}, \widehat{L}^\xi_\mathrm{g})\left(\mu^{(n,\xi)}_s, g^{(n,\xi)}_s\right) ds. \end{align} One then uses bilinear continuity arguments in total variation norm $\|\cdot\|$ to show that, given a bound $\langle \varphi, \mu^\xi_0\rangle +M^\xi_0+E^\xi_0\le C$, there is a positive time $T=T(\xi,C)>0$ such that the Picard iterates $(\mu^{(\xi,n)}_t)_{t\le T}$ converge uniformly in total variation on $[0,T]$, and that the limit $\mu^\xi_t$ solves (\ref{eq:rE1}, \ref{eq: rE2}), possibly allowing $\mu^\xi_t$ to be a signed measure. This argument also implies that the solution is unique on this interval. Now, we note that the quantity $\langle \varphi, \mu^\xi_t\rangle +M^\xi_t+E^\xi_t$ is constant in time, and therefore this construction can be repeated on $[T, 2T]$, $[2T, 3T]$, etc, which proves global existence and uniqueness. Finally, an integrating factor is introduced to argue that $\mu_t$ is a positive measure.
In our case, it is also straightforward to see that the gel data $M^\xi_t, E^\xi_t \ge 0$, and that $P^\xi_t=0$ from the symmetry ({A1}.)
\end{proof}

\begin{proof}[Proof of Lemma \ref{lemma: E and U}] 
We first show existence. For all $\xi<\infty$, we let $(\mu^\xi_t, g^\xi_t)$ be the solution to the dynamics (\ref{eq:rE1}, \ref{eq: rE2}) restricted to $S_\xi$, with initial data \begin{equation} \mu^\xi_0(dx)=1[x\in S_\xi]\hspace{0.1cm}\mu_0(dx);\hspace{1cm} g^\xi_0 = \int_{x\not\in S_\xi} x\mu_0(dx). \end{equation}  Observe that, if $\xi<\xi'$, then $\widetilde{\mu}^\xi_t, \widetilde{g}^\xi_t$ given by \begin{equation} \widetilde{\mu}^\xi_t(dx)=1_{x\in S_\xi}\hspace{0.1cm}\mu^{\xi'}_t(dx);\hspace{1cm} \widetilde{g}^\xi_t=g^{\xi'}_t+\int_{x\in S_{\xi'}\setminus S_\xi} x \mu^{\xi'}_t(dx)\end{equation} solve the dymanics (\ref{eq:rE1},\ref{eq: rE2}) with the same initial data $\mu^\xi_0, g^\xi_0$. From uniqueness in Lemma \ref{lemma: restricted dynamics}, it follows that $\widetilde{\mu}^\xi_t=\mu^\xi_t; \widetilde{g}^\xi_t=g^\xi_t$. This shows that the measures $\mu^\xi_t$ are increasing in $\xi$, while the gel data $M^\xi_t, E^\xi_t$ are decreasing, and $P^\xi_t$ is identically $0$, by symmetry ({A1}.). Therefore, the limits \begin{equation} \mu_t=\lim_{\xi\uparrow \infty} \mu^\xi_t; \hspace{1cm} M_t=\lim_{\xi\rightarrow \infty} M^\xi_t; \hspace{1cm} E_t=\lim_{\xi\rightarrow \infty} E^\xi_t \end{equation} exist in the sense of monotone limits; one can then check that $\mu_t$ and $g_t=(M_t,0,E_t)$ satisfy the full equation (\ref{eq: E+G}), with initial values $\mu_0$ and $g_0=0.$

To see uniqueness let $\mu_t$ be the solution constructed above, and write $g_t=(M_t,P_t,E_t)$ for the data of the gel. Let $\widetilde{\mu}_t$ be any solution to (\ref{eq: E+G}) starting at $\mu_0$, and let $\widetilde{g}_t=(M'_t, P'_t,E'_t)$ be the associated data of the gel. For all $\xi<\infty$, it is simple to verify that \begin{equation} \widetilde{\mu}^\xi_t(dx)=1_{x\in S_\xi}\hspace{0.1cm} \mu'_t(dx);\hspace{1cm} \widetilde{g}^\xi_t=\widetilde{g}_t+\int_{S^\mathrm{c}_\xi} x \widetilde{\mu}_t(dx) \end{equation} is a solution to the dynamics (\ref{eq:rE1}, \ref{eq: rE2}) on $S_\xi$. By uniqueness in Lemma \ref{lemma: restricted dynamics}, it follows that $\widetilde{\mu}^\xi_t=\mu^\xi_t$, and taking monotone limits, we see that $\mu'_t=\lim_{\xi\rightarrow \infty} \widetilde{\mu}^\xi_t=\lim_{\xi\rightarrow \infty} {\mu}^\xi_t=\mu_t$. The argument for $\widetilde{g}$ is identical.
\end{proof}
\iffalse 
\subsection{\textbf{Gelation and Explosion of Truncated Dynamics}}
 For a further analysis of the equation (\ref{eq: E+G}), we consider a modified equation, resulting from a more tractable kernel, which is symmetrised with respect to momentum. Let
 \begin{equation}\label{eq: modified K} 
 \begin{split}
 K^\mathrm{m}(x,y,dz)& \\  = & \left[\kappa \pi_n(x) \pi_n(y) +\gamma\left(\pi_n(x)\pi_e(y)-\pi_p(x)\cdot \pi_p(y)+ \pi_e(x)\pi_n(y)\right)\right]\delta_{x+y}(dz) \\ & + \gamma(\pi_n(x)\pi_e(y)+\pi_p(x)\cdot \pi_p(y)+ \pi_e(x)\pi_n(y))\left(\frac{\delta_{x+Ry}+\delta_{Rx+y}}{2}\right)(dz) \\ & 
 =\frac{1}{4}K(Rx, y, dz)+\frac{1}{2}K(x,y,dz)+\frac{1}{4}K(x,Ry, dz). \end{split} 
\end{equation}
Let $L^\mathrm{m}$ be the drift operator for the modified kernel $K^\mathrm{m}$, and consider the modified equation \begin{equation} \tag{mE-G}\label{eq: mE}
    \mu_t=\mu_0+\int_0^t L^\mathrm{m}(\mu_s)ds.
\end{equation}The total mass of the modified kernel is \begin{equation}
    \label{eq: modified Kbar} 
    \overline{K^\mathrm{m}}(x,y)=\kappa \pi_n(x) \pi_n(y) + 2\gamma(\pi_n(x)\pi_e(y)+\pi_e(x)\pi_n(y)).
\end{equation}
Consider a modified state space, which truncates the velocity distribution by excluding clusters with extreme kinetic energies: for $\epsilon>0$, let \begin{equation}
    S^\epsilon= \{x\in S: \epsilon \pi_n(x) \leq \pi_e(x) \leq \epsilon^{-1} \pi_n(x)\}.
\end{equation} Note that this state space is preserved under both kernels $K, K^\mathrm{m}$. Moreover, on the reduced state space $S^\epsilon$, the modified kernel $K^\mathrm{m}$ is \emph{approximately multiplicative} \cite{N00} in the sense that, for some $\delta_\epsilon>0$ and $\Delta_\epsilon<\infty$, we have \begin{equation}
    \delta_\epsilon\hspace{0.1cm}\varphi(x)\varphi(y) \leq \overline{K^\mathrm{m}}(x,y) \leq  \Delta_\epsilon\hspace{0.1cm}\varphi(x)\varphi(y)
\end{equation}for all $x,y \in S^\epsilon$. For any $\epsilon$, let $\mu_0^\epsilon$ denote the restriction $\mu_0^\epsilon(dx)=1_{x\in S^\epsilon}\hspace{0.1cm}\mu_0(dx).$ We now appeal to \cite[Theorem 2.2]{N00}, on existence and uniqueness for approximately multiplicative kernels, to obtain the following, which provides the connection between gelation and explosion of a second moment.
\begin{lem}\label{lemma: solution to modified equation}
    For all $\epsilon>0$, there is a unique maximal conservative solution  $(\nu^\epsilon_t)_{t< t_\mathrm{e}^\epsilon}$ in $S^\epsilon$ to the modified equation (\ref{eq: mE}). Moreover, the map $t\mapsto \langle \varphi^2, \nu^\epsilon_t\rangle$ is finite and increasing on $[0,t_\mathrm{e}^\epsilon)$, and increases to $\infty$ as $t\uparrow t_\mathrm{e}^\epsilon$. 
\end{lem}

Similarly, we can also apply Corollary \ref{cor: maximal conservative solutions} to see that there exist maximal conservative solutions $(\mu^\epsilon_t)_{t<t_\mathrm{g}^\epsilon}$ to (\ref{eq: E}) starting at $\mu^\epsilon_0$, which are given by initial segments of a global solution $(\mu^\epsilon_t)_{t\geq 0}$ to (\ref{eq: E+G}). Repeatedly exploiting uniqueness, we show that these coincide with the solution to (\ref{eq: mE}):

\begin{lem}[Relationship of equations]\label{lemma: Relationship}
Let $\mu^\epsilon_0$ be as above, for initial data $\mu_0$ satisfying ({A1}.).  Then the maximal conservative solutions $(\mu^\epsilon_t)_{t<t_\mathrm{g}^\epsilon}$ and $(\nu^\epsilon_t)_{t<t_\mathrm{e}^\epsilon}$, to (\ref{eq: E}) and (\ref{eq: mE}) respectively, coincide. In particular, $t_\mathrm{e}^\epsilon = t^\epsilon_{gel}$, and the map \begin{equation}
    t\mapsto \langle \varphi^2, \mu^\epsilon_t\rangle
\end{equation} is finite and increasing on $[0, t_\mathrm{g}^\epsilon)$, and increases to $\infty$ as $t\uparrow t_\mathrm{g}^\epsilon.$ \end{lem}

\begin{proof} Firstly, we note that $( \mu^\epsilon_t \circ R^{-1})_{t<t_\mathrm{g}^\epsilon}$ also solves (\ref{eq: E}), starts at $\mu^\epsilon_0$ and is conservative. Therefore, by uniqueness in Corollary \ref{cor: maximal conservative solutions}, we must have $\mu^\epsilon_t \circ R^{-1}=\mu^\epsilon_t$ for all $t<t_\mathrm{g}^\epsilon.$ Therefore, for any bounded, measurable function $f$, and $t<t^\epsilon_\mathrm{g},$ it follows from elementary manipulations that \begin{equation}\label{eq: symmetry under R1}
        \int_{S\times S\times S} (f(z)-f(x)-f(y))K(x,y,dz)\mu^\epsilon_t(dx)\mu^\epsilon_t(dy) =  \int_{S\times S\times S} (f(z)-f(x)-f(y))K(Rx,y,dz)\mu^\epsilon_t(dx)\mu^\epsilon_t(dy)
    \end{equation} and
\begin{equation}\label{eq: symmetry under R2}
        \int_{S\times S\times S} (f(z)-f(x)-f(y))K(x,y,dz)\mu^\epsilon_t(dx)\mu^\epsilon_t(dy) =  \int_{S\times S\times S} (f(z)-f(x)-f(y))K(x,Ry,dz)\mu^\epsilon_t(dx)\mu^\epsilon_t(dy).
    \end{equation} Combining these, we see that $(\mu^\epsilon_t)_{t<t^\epsilon_\mathrm{g}}$ solves the modified equation (\ref{eq: mE}), and so, by uniqueness of the maximal conservative solution $(\nu^\epsilon_t)_{t<t_\mathrm{e}^\epsilon}$ in Lemma \ref{lemma: solution to modified equation}, we have \begin{equation}
        t^\epsilon_\mathrm{g} \leq t_\mathrm{e}^\epsilon; \hspace{1cm} \mu^\epsilon_t=\nu^\epsilon_t \hspace{0.5cm}\forall t<t^\epsilon_\mathrm{g}.
    \end{equation} The other implication is identical, using the uniqueness of the maximal conservative solution $(\nu^\epsilon_t)_{t<t_\mathrm{e}^\epsilon}$ in Lemma \ref{lemma: solution to modified equation} to deduce that $\nu^\epsilon_t= \nu^\epsilon_t\circ R^{-1}$ for $t<t_\mathrm{e}^\epsilon$. Hence, the equations (\ref{eq: symmetry under R1}, \ref{eq: symmetry under R2}) hold with $\nu^\epsilon_t$ in place of $\mu^\epsilon_t$, for any bounded, measurable $f$ and $t<t_\mathrm{e}^\epsilon.$ Therefore, $(\nu^\epsilon_t)_{t<t_\mathrm{e}^\epsilon}$ is a conservative solution to the unmodified equation (\ref{eq: E}), and so by Corollary \ref{cor: maximal conservative solutions}, \begin{equation}
        t_\mathrm{e}^\epsilon \leq t^\epsilon_\mathrm{g}; \hspace{1cm} \nu^\epsilon_t=\mu^\epsilon_t \hspace{0.5cm}\forall t<t_\mathrm{e}^\epsilon.
    \end{equation} \end{proof}  

%\iffalse Repeating the calculation we have already done, we can calculate the gelation times explicitly: \begin{lem}[Identification of Gelation Times]\label{lemma: calculation of gelation} For $\epsilon>0$, the gelation time $t_\mathrm{g}^\epsilon$ is given by \begin{equation} t_\mathrm{g}^\epsilon = \frac{1}{4\gamma}\sqrt{\frac{\langle \pi_n^2, \mu_0^\epsilon\rangle}{\langle \pi_e^2, \mu_0^\epsilon\rangle}}\left(\langle \pi_n^2, \mu_0^\epsilon\rangle+\sqrt{\frac{\langle \pi_n^2, \mu_0^\epsilon\rangle}{\langle \pi_e^2, \mu_0^\epsilon\rangle}}\langle \pi_n\pi_e, \mu_0^\epsilon\rangle \right)^{-1}. \end{equation} As $\epsilon \downarrow 0$, we have \begin{equation}
%    t_\mathrm{g}^\epsilon \rightarrow t_\mathrm{g} =\frac{1}{4\gamma}\sqrt{\frac{\langle \pi_n^2, \mu_0\rangle}{\langle \pi_e^2, \mu_0\rangle}}\left(\langle \pi_n^2, \mu_0\rangle+\sqrt{\frac{\langle \pi_n^2, \mu_0\rangle}{\langle \pi_e^2, \mu_0\rangle}}\langle \pi_n\pi_e, \mu_0\rangle \right)^{-1}.
%\end{equation}  \end{lem} \begin{proof} \textcolor{red}{{Here's what we had before:}} From Lemma \ref{lemma: integral equation}, and standard regularity arguments, we have 
%\begin{align}
%    \frac{\dd}{\dd t}\langle \pi_n^2, \mu_t\rangle &=
%    2\kappa \langle \pi_n^2, \mu_t\rangle^2
%    + 8 \gamma \langle \pi_n \pi_e, \mu_t\rangle\langle \pi_n^2, \mu_t\rangle\\
%    \frac{\dd}{\dd t}\langle \pi_n \pi_e, \mu_t\rangle &=
%    2\kappa \langle \pi_n^2, \mu_t\rangle \langle \pi_n \pi_e, \mu_t\rangle
%    + 4 \gamma \langle \pi_n \pi_e, \mu_t\rangle^2
%    + 4 \gamma \langle \pi_n^2, \mu_t\rangle\langle \pi_e^2, \mu_t\rangle\\
%    \frac{\dd}{\dd t}\langle \pi_e^2, \mu_t\rangle &=
%    2\kappa \langle \pi_n \pi_e, \mu_t\rangle^2
%    + 8 \gamma \langle \pi_n \pi_e, \mu_t\rangle\langle \pi_e^2, \mu_t\rangle.
%\end{align}
%In the case $\kappa = 0$ note that
%\begin{equation}\label{eq:proportional}
%    \langle \pi_e^2, \mu_t\rangle = \langle \pi_e^2, \mu_0\rangle
%    \frac{\langle \pi_n^2, \mu_t\rangle}{\langle \pi_n^2, \mu_0\rangle}
%\end{equation}
%and then one easily checks that
%\begin{equation}
%    f(t) := \langle \pi_n^2, \mu_t\rangle
%           + \sqrt{\frac{\langle \pi_n^2, \mu_0\rangle}
%                        {\langle \pi_e^2, \mu_0\rangle}}
%             \langle \pi_n \pi_e, \mu_t\rangle
%\end{equation}
%satisfies
%\begin{equation}
%    \frac{\dd }{\dd t}f(t) = 4\gamma 
%    \sqrt{\frac{\langle \pi_e^2, \mu_0\rangle}
%                        {\langle \pi_n^2, \mu_0\rangle}}f(t)^2,
%\end{equation}
%which has the solution
%\begin{equation}\label{eq:quadmoment-ode}
%    f(t) = \left(\frac{1}{f(0)} - 4\gamma 
%    \sqrt{\frac{\langle \pi_e^2, \mu_0\rangle}
%                        {\langle \pi_n^2, \mu_0\rangle}}t \right)^{-1}.
%\end{equation}
%The solution to \eqref{eq:quadmoment-ode} blows up when
%\begin{equation}\label{eq:quad-tgel}
%    t = \frac{1}{4\gamma f(0)}
%             \sqrt{\frac{\langle \pi_n^2, \mu_0\rangle}
%                        {\langle \pi_e^2, \mu_0\rangle}}
%\end{equation}

%Assuming the initial condition is independent Maxwell-Boltzmann velocities in $d$ dimensions and all clusters of size 1 one has
%\begin{align*}
%    \langle \pi_n^2, \mu_0\rangle &= 1\\
%    \langle \pi_n \pi_e, \mu_0\rangle &= d\frac{\sigma^2}{2}\\
%    \langle \pi_e^2, \mu_0\rangle &= \frac{\sigma^4}{4}\left(2d + d^2\right)
%\end{align*}
%and thus $f(0) = 1 + d / \sqrt{2d + d^2}$.
%In this case the critical time given by \eqref{eq:quad-tgel} is
%\begin{equation}
%    \frac{1}{2\gamma \sigma^2 \left(d + \sqrt{2d +d^2}\right)}.
%\end{equation}
%This should be compared to the expression for the mean free time \cite[equation 4.38]{PSW17}, which is $1/4\gamma \sigma^2 d$. \end{proof} \fi 


\subsection{\textbf{Convergence of the Truncated Dynamics}}
We have just shown that for $\epsilon > 0$ the time at which conservation breaks down, $t_\mathrm{g}^\epsilon$, is equal to the time $t_\mathrm{e}^\epsilon$ at which the second moment explodes.
We now show that $t_\mathrm{g}^\epsilon \rightarrow t_\mathrm{g}$ and $t_\mathrm{e}^\epsilon \rightarrow t_\mathrm{e}$, the time at which $\mathcal{E}(t)=\langle \varphi^2, \mu_t\rangle$ explodes in order to see that the explosion coincides with $t_\mathrm{g}$.
\begin{lem}\label{lemma: connecting mu-epsilon and mu}
    For $\epsilon>0$, let $(\mu^\epsilon_t)_{t\geq 0}$ be the global solution to (\ref{eq: E+G}), starting at $\mu_0^\epsilon.$ Let $M^\epsilon_t$ and $E^\epsilon_t$ be the mass and energy of the corresponding gel. Then we have  the monotonicity \begin{equation}\label{eq: monontonicity for muepsilont}
        \mu^\epsilon_t \uparrow \mu_t \text{ as }\epsilon\downarrow 0
    \end{equation} in the sense of monotone limits, and\begin{equation} \label{eq: monotonicity for gel data}
        M^\epsilon_t \uparrow M_t; \hspace{1cm} E^\epsilon_t \uparrow E_t \hspace{1cm} \text{as }\epsilon\downarrow 0.
    \end{equation}
It therefore follows that $t^\epsilon_\mathrm{g}\downarrow t_\mathrm{g}$. Noting the relation \begin{equation}\label{eq: Mepsilon and Eepsilon}
        \epsilon M^\epsilon_t \leq E^\epsilon_t \leq \epsilon^{-1}M^\epsilon_t
    \end{equation}it follows that, if $t>t_\mathrm{g}$, then $M_t >0$ and $E_t>0$.
\end{lem}
The proof of the monotonicity principle closely follows \cite[Propositions 2.4,  2.7]{N00}. We indicate below how these arguments should be modified to prove our result.
\begin{proof} Let $0<\epsilon'<\epsilon$, so that $\mu_0^\epsilon\le\mu_0^{\epsilon'}.$ For $\xi<\infty$, let $\mu^{\epsilon, \xi}_t, g^{\epsilon, \xi}_t=(M^{\epsilon,\xi}_t,P^{\epsilon,\xi}_t,E^{\epsilon,\xi}_t)$ be the approximations used in the proof of Lemma \ref{lemma: E and U} by restricting to $S_\xi$, given as the solution to (\ref{eq:rE1}, \ref{eq: rE2}) with initial data
\begin{equation} \mu^{\epsilon,\xi}_0(dx)=1_{x\in S_\xi \cap S^\epsilon}\hspace{0.1cm}\mu_0(dx);\hspace{1cm} g^{\epsilon,\xi}_0=\int_{S^\epsilon\setminus S_\xi}x\mu_0(dx).\end{equation}
and define $\mu^{\epsilon',\xi}_t, g^{\epsilon',\xi}_t$ similarly, with $\epsilon'$ in place of $\epsilon$. We observe that $\mu^{\epsilon,\xi}_0\le \mu^{\epsilon',\xi}_0$ and similarly $M^{\epsilon,\xi}_0\le M^{\epsilon',\xi}_0$, $E^{\epsilon,\xi}_0\le E^{\epsilon',\xi}_0$. \medskip \\ 
We introduce an integrating factor as follows. For $x\in S_\xi$, we define $\theta_t(x)$ by 
\begin{equation} \theta_t(x)=\exp\left( \int_0^t\left[\int_{S_\xi}\overline{K}(x,y) (\mu^{\epsilon',\xi}_s+\mu^{\epsilon,\xi}_s)(d y)+\overline{K}\left(x,g^{\epsilon',\xi}_s\right)\right]ds\right)
\end{equation}
and define the signed measure on $S_\xi$ \begin{equation}
    \Lambda_t(\dd x) = \theta_t(x) \left(\mu^{\epsilon',\xi}_t - \mu^{\epsilon,\xi}_t\right)(\dd x).
\end{equation} Using a Leibniz rule analogous to \cite[Proposition 2.5]{N00}, one can check that, for any bounded, measurable function $f$ on $S_\xi$, \begin{equation} \label{eq: ODE for xit} \begin{split} &\frac{d}{dt}\langle f, \Lambda_t\rangle = \left\langle f \theta_t, \frac{d}{dt}(\mu^{\epsilon',\xi}_t-\mu^{\epsilon, \xi}_t)\right\rangle +\left\langle f\frac{\partial \theta_t}{\partial t},\mu^{\epsilon',\xi}_t-\mu^{\epsilon, \xi}_t  \right\rangle \\[1ex]& = %\frac{1}{2}\int_{S_\xi\times S_\xi}\left\{(f\theta_t)(x+y)1_{x+y\in S_\xi}-(f\theta_t)(x)-(f\theta_t)(y)\right\}\overline{K}(x,y)\left(\mu^{\epsilon',\xi}_t(dx)\mu^{\epsilon',\xi}_t(dy)-\mu^{\epsilon,\xi}_t(dx)\mu^{\epsilon,\xi}_t(dy)\right) \\ &\hspace{2cm} -\int_{S_\xi}(f\theta_t)(x)\left(\overline{K}\left(x, g^{\epsilon',\xi}_t\right)\mu^{\epsilon',\xi}_t(dx)-\overline{K}\left(x,g^{\epsilon,\xi}_t\right)\mu^{\epsilon,\xi}_t(dx)\right) \\  & \hspace{2cm}+ \int_{S_\xi\times S_\xi} (f\theta_t)(x)\overline{K}(x,y)\left(\mu^{\epsilon',\xi}_t(dx)-\mu^{\epsilon,\xi}_t(dx)\right)\left(\mu^{\epsilon',\xi}_t(dy)+\mu^{\epsilon,\xi}_t(dy)\right) \\&\hspace{2cm}  + \int_{S_\xi} (f\theta_t)(x)\overline{K}\left(x, g^{\epsilon',\xi}_t\right)\mu^{\epsilon',\xi}_t(dx). 
\langle f, H_t(\Lambda_t) +\alpha_t\rangle. \end{split} \end{equation} 
%We observe that this simplifies to a differential equation \begin{equation} \frac{d}{dt}\Lambda_t=H_t(\Lambda_t)+\alpha_t;\hspace{1cm} \xi_0=\mu^{\epsilon',\xi}_0-\mu^{\epsilon,\xi}_0\end{equation}%
where we define signed measures $H_t(\lambda), \alpha_t$ for signed measures $\lambda$ on $S_\xi$, by specifying, for all bounded measurable functions $f$, \begin{equation} \langle f,H_t(\lambda)\rangle = \frac{1}{2}\int_{S_\xi\times S_\xi}(f\theta_t)(x+y)1_{x+y\in S_\xi}\overline{K}(x,y)(\mu^{\epsilon',\xi}_t(dx)+\mu^{\epsilon,\xi}_t(dx))(\theta_t^{-1}\lambda)(dy) \end{equation} and  \begin{equation} \langle f, \alpha_t\rangle = \int_{S_\xi} (f\theta_t)(x)\overline{K}\left(x,g^{\epsilon,\xi}_t\right)\mu^{\epsilon,\xi}_t(dx). \end{equation} We now write $\mathcal{M}(S_\xi)$ for the space of signed measures on $S_\xi$, equipped with the total variation norm, and write $\mathcal{M}^+(S_\xi)$ for the positive measures. It is straightforward to see that that $H_t: \mathcal{M}(S_\xi)\rightarrow \mathcal{M}(S_\xi)$ is a bounded linear map, and that both the norm $\|\alpha_t\|$ and the operator norm $\|H_t\|$ are bounded on compact time intervals. It therefore follows, by standard Gr\"onwall-style techniques, that, for any initial data $\Lambda_0$, the solution $\Lambda_t$ to (\ref{eq: ODE for xit}) is unique, and that the Picard iterates \begin{equation} \label{eq: Picard iterates} \Lambda^{0}_t=\Lambda_0; \hspace{1cm} \Lambda^{n+1}_t=\xi_0+\int_0^t \left(H_s\left(\Lambda^n_s\right)+\alpha_s\right) ds \end{equation} converge to $\Lambda_t$ in total variation, uniformly on compact time intervals. \medskip \\ We now note that $\alpha_t \in \mathcal{M}^+(S_\xi)$ is a positive measure for all $t\ge 0$, and that if $\lambda\in \mathcal{M}^+(S_\xi)$, then $H_t(\lambda) \in \mathcal{M}^+$ for all $t\ge 0$. Therefore, by induction, each iterate $\Lambda^n_t$ is a positive measure: $\Lambda^n_t\in \mathcal{M}^+(S_\xi)$ for all $t\ge 0$. Since $\Lambda^n_t\rightarrow \Lambda_t=\theta_t(\mu^{\epsilon',\xi}_t-\mu^{\epsilon,\xi}_t)$ in total variation for all $t\ge 0$, this implies that $\mu^{\epsilon,\xi}_t\le \mu^{\epsilon',\xi}_t$ for all $t\ge 0.$ Taking a limit $\xi\rightarrow \infty,$ $\mu^{\epsilon,\xi}_t\uparrow \mu^{\epsilon}_t$ in the sense of monotone limits, and similarly for $\mu^{\epsilon'}_t$. Together, these imply that $\mu^{\epsilon}_t \le \mu^{\epsilon'}_t$, and therefore the map $\epsilon\mapsto \mu^{\epsilon}_t$ is increasing as $\epsilon \downarrow 0.$ If we define \begin{equation} \widetilde{\mu}_t = \lim_{\epsilon \downarrow 0} \mu^{\epsilon}_t \end{equation} then it is straightforward to verify that $\widetilde{\mu}_t$ satisfies the Smoluchowski dynamics (\ref{eq: E+G}), with $\widetilde{\mu}_0=\mu_0$; by uniqueness, this implies that $\widetilde{\mu}_t=\mu_t$. \medskip \\  This proves the monotonicity principle (\ref{eq: monontonicity for muepsilont}) for $\mu^\epsilon_t$ and $\mu_t$. For the equivalent (\ref{eq: monotonicity for gel data}) for the gel data, the argument is similar. We note first that that, for all $\xi>0$ and $0<\epsilon'<\epsilon$, the initial data satisfy \begin{equation} M^{\epsilon,\xi}_0 \le M^{\epsilon',\xi}_0; \hspace{1cm} E^{\epsilon,\xi}_0 \le E^{\epsilon',\xi}_0.\end{equation} Therefore, from the monotonicity $\mu^{\epsilon,\xi}_t \le \mu^{\epsilon',\xi}_t$ and the evolution equation (\ref{eq: rE2}), it immediately follows that \begin{equation} M^{\epsilon,\xi}_t \le M^{\epsilon',\xi}_t; \hspace{1cm} E^{\epsilon,\xi}_t \le E^{\epsilon',\xi}_t\end{equation} and taking $\xi\rightarrow \infty$, \begin{equation} M^{\epsilon}_t \le M^{\epsilon'}_t; \hspace{1cm} E^{\epsilon}_t \le E^{\epsilon'}_t.\end{equation} Therefore, the map $\epsilon\mapsto M^{\epsilon}_t$ is increasing as $\epsilon\downarrow 0$. Now, using monotone convergence twice as $\epsilon\downarrow 0$, \begin{equation} \begin{split} M^\epsilon_t &= \langle \pi_n, \mu^\epsilon_0\rangle -\langle \pi_n, \mu^\epsilon_t\rangle \\& \rightarrow\langle \pi_n, \mu_0\rangle -\langle \pi_n, \mu_t\rangle = M_t \end{split}\end{equation} which implies that $M^\epsilon_t\uparrow M_t$ as $\epsilon\downarrow 0$. The case for the energy is identical. \medskip \\  For the relation (\ref{eq: Mepsilon and Eepsilon}), we use a similar limiting argument. By definition, $\mu^{\epsilon, \xi}_t$ is supported on $S^\epsilon$, and the initial data satisfy \begin{equation} \epsilon M^{\epsilon,\xi}_0\le E^{\epsilon,\xi}_0\le \epsilon^{-1}M^{\epsilon,\xi}_0. \end{equation} Therefore, immediately from (\ref{eq: rE2}), \begin{equation} \epsilon M^{\epsilon,\xi}_t\le E^{\epsilon,\xi}_t\le \epsilon^{-1}M^{\epsilon,\xi}_t. \end{equation} The claimed bound (\ref{eq: Mepsilon and Eepsilon}) follows on taking the limit $\xi\rightarrow \infty$. \medskip \\ For the final point, let $t>t_\mathrm{g}$. Then, for some $\epsilon>0$ small enough, $t>t^\epsilon_\mathrm{g}$. From the bound (\ref{eq: Mepsilon and Eepsilon}), if one of $M^\epsilon_t$ and $E^\epsilon_t$ is strictly positive, then they both are, and so both $M^\epsilon_t, E^\epsilon_t>0$. Therefore \begin{equation} M_t\ge M^\epsilon_t >0;\hspace{1cm} E_t\ge E^\epsilon_t >0 \end{equation} as claimed \end{proof}
\subsection{\textbf{Gelation and Explosion of non-Truncated Dynamics}} We now turn to the behaviour of the second moment $\mathcal{E}(t)=\langle \varphi^2, \mu_t\rangle$ in the subcritical phase $t<t_\mathrm{g}$. Using the previous results, we show that the gelation coincides exactly with the blowup of the second moment. 
\begin{lem}\label{lemma: second moment before tgel} The second moment $\mathcal{E}(t)=\langle \varphi^2, \mu_t\rangle$ is finite and increasing on $[0, t_\mathrm{g})$, and increases to infinity as $t\uparrow t_\mathrm{g}$. \end{lem}  We say that a local solution $(\nu_t)_{t<T}$ to (\ref{eq: E}) is \emph{strong} if, for all times $t<T$, 
\begin{equation}
    \int_0^t  \hspace{0.1cm} \langle\varphi^2, \nu_s\rangle  \hspace{0.1cm} ds<\infty.
\end{equation} We use the following result from \cite{N00} on the existence and uniqueness of strong solutions. 
\begin{lem}\label{lemma: strong solutions} Any strong solution to (\ref{eq: E}) is conservative. For any finite measure $\mu_0$ with $\langle \varphi^2, \mu_0\rangle <\infty$, there is a unique maximal strong solution $(\mu'_t)_{t<t_\mathrm{e}(\mu_0)}$ to (\ref{eq: E}), starting at $\mu_0$, and with $t_\mathrm{e}(\mu_0)>0$, such that $ \langle \varphi^2, \mu'_t\rangle$ is increasing on $[0,t_\mathrm{e}(\mu_0))$. If $t_\mathrm{e}(\mu_0)<\infty$, then $\langle \varphi^2, \mu'_t\rangle$ increases to $\infty$  as $t\uparrow t_\mathrm{e}(\mu_0)$.  \end{lem} When the measure $\mu_0$ is clear, we will omit the argument of $t_\mathrm{e}.$ 
\begin{proof}
    This is almost a special case of \cite[Thrm. 2.1]{N00}. From \cite[Theorem 2.1]{N00} it follows that, for any finite measure $\mu_0$ with $\langle \varphi^2, \mu_0\rangle <\infty$, there exists a maximal strong solution $(\mu'_t)_{t<t_\mathrm{e}(\mu_0)}$. Moreover, there exists a constant $C=C(\kappa, \gamma)>0$ such that, for all such $\mu_0$,  $t_\mathrm{e}(\mu_0) \ge C \langle \varphi^2, \mu_0\rangle^{-1}$. By applying this bound to $\mu'_t$, if $t_\mathrm{e}(\mu_0)<\infty,$ then $\langle \varphi^2, \mu'_t\rangle \ge (C(t_\mathrm{e}(\mu_0)-t))^{-1}$  which implies the claimed divergence.
\end{proof}

By Corollary \ref{cor: maximal conservative solutions}, it follows that $t_\mathrm{e} \le t_\mathrm{g}$, and $\mu'_t=\mu_t$ for all $t<t_\mathrm{e}$. It remains to show that $t_\mathrm{e}\ge t_\mathrm{g}$. \medskip \\ Following the ideas of \cite[Proposition 2.7]{N00}, we obtain the integral relations, for all $t<t_\mathrm{e}$, \begin{equation} \label{eq: ODE1}
    \langle \pi_n^2, \mu_t\rangle =
    \langle \pi_n^2, \mu_0\rangle + \int_0^t \left[\kappa\langle \pi_n^2, \mu_s\rangle^2+4\gamma\langle \pi_n\pi_e, \mu_t\rangle\langle\pi_n^2, \mu_s\rangle \right] ds;
\end{equation} 

\begin{equation}\label{eq: ODE2}
    \langle \pi_n \pi_e, \mu_t\rangle =
    \langle \pi_n\pi_e, \mu_0\rangle + \int_0^t \left[\kappa\langle \pi_n^2, \mu_s\rangle\langle \pi_n\pi_e, \mu_s\rangle+2\gamma\langle \pi_n\pi_e, \mu_t\rangle^2+2\gamma\langle\pi_n^2, \mu_s\rangle\langle \pi_e^2, \mu_s \rangle \right] ds;
\end{equation}

\begin{equation} \label{eq: ODE3}
    \langle \pi_e^2, \mu_t\rangle =
    \langle \pi_e^2, \mu_0\rangle + \int_0^t \left[\kappa\langle \pi_n\pi_e, \mu_s\rangle^2+4\gamma\langle \pi_n\pi_e, \mu_t\rangle\langle\pi_e^2, \mu_s \rangle \right] ds.
\end{equation} First of all, these imply that $\mathcal{E}(t)=\langle \varphi^2, \mu_t\rangle$ is bounded on compact subsets of $[0, t_\mathrm{e})$, and in particular cannot diverge before $t_\mathrm{e}$. Combining this with Lemma \ref{lemma: strong solutions}, the maximal time $t_\mathrm{e}$ of existence of a strong solution is precisely the first time at which the second moment $\mathcal{E}(t)$ diverges, or $\infty$ if there is no divergence. \medskip \\ We also remark on the relationship of this result to the solutions $(\mu^\epsilon)_{t<t_\mathrm{g}^\epsilon}=(\nu^\epsilon_t)_{t<t_\mathrm{e}^\epsilon}$ discussed in Lemmas \ref{lemma: solution to modified equation}, \ref{lemma: Relationship}. It is clear, from Lemma \ref{lemma: solution to modified equation}, that $(\nu^\epsilon_t)_{t<t_\mathrm{e}^\epsilon}$ is a strong solution. Moreover, in view of the comments above, and since $\langle \varphi^2, \nu^\epsilon_t\rangle \uparrow \infty$ as $t\uparrow t_\mathrm{e}^\epsilon$, it follows that that $(\nu^{\epsilon}_t)_{t<t_\mathrm{e}^\epsilon}$ is the maximal strong solution with initial data $\mu^\epsilon_0.$ This justifies the use of the notation $t^\epsilon_\mathrm{e}$ in Lemma \ref{lemma: solution to modified equation}. \medskip \\  Using standard regularity arguments, we may view (\ref{eq: ODE1}- \ref{eq: ODE3}) as a differential equation for the three moments $q_t=(\langle \pi_n^2, \mu_t\rangle, \langle \pi_n \pi_e, \mu_t\rangle, \langle \pi_e^2, \mu_t\rangle)$ and, from the discussion above, the blow-up time to the ODE system is exactly $t_\mathrm{e}$. An identical argument holds for $\mu^\epsilon_t$, which blows up at $t^\epsilon_\mathrm{e}$, which coincides with  $t^\epsilon_\mathrm{g}$ by Lemma \ref{lemma: Relationship}. To connect the explosion times for truncated and non-truncated data, we now analyse the system of ODEs (\ref{eq: ODE1} - \ref{eq: ODE3}). \dhcomment{This is clear and correct, but could be a bit punchier. I'd really like to get the message across that we prove $t_\mathrm{g}=t_\mathrm{e}$ by taking a limit of the truncated system, where it's true by results of \cite{N00}.}
\begin{lem}\label{lemma: ODE considerations} Consider the ordinary differential equation $\dot{q}_t=b(q_t)$ in $\mathbb{R}^3$, where $b$ is the locally Lipschitz field given by \begin{equation} \label{eq: system of ODEs} b(q_1,q_2,q_3)=\begin{pmatrix}\kappa q_1^2+4\gamma q_1q_2 \\ \kappa q_1q_2+2\gamma q_2^2+2\gamma q_1q_3 \\ \kappa q_2^2 + 4\gamma q_2q_3 \end{pmatrix}. \end{equation} Then, for all $q_0\in \mathbb{R}^3$, there exists a unique maximal solution $\psi(q_0, t)$ starting at $q_0$, defined until time $\zeta(q_0)\in (0, \infty]$. Consider the sets \begin{equation} E=(0, \infty)^3; \hspace{1cm} E_\delta=[\delta,\infty)^3.\end{equation} Then, if $q_0 \in E_\delta$ for some $\delta>0$, then the solution $(\phi(q_0,t))_{t<\zeta(q_0)} \subset E_\delta$. We have the following properties: \begin{enumerate}[label=\roman{*}).] \item Let $J_\epsilon$ be the set \begin{equation} J_\epsilon  =\{q \in E: \hspace{0.2cm} \zeta(q)\ge\epsilon\}.\end{equation} If $\gamma>0$, then for all $\epsilon, \delta>0$, the set $E_\delta \cap J_\epsilon $ is bounded. Moreover, $\zeta<\infty$ everywhere. \item Suppose $q^n_0 \in E$ and $q^n_0 \rightarrow q_0 \in E$. Then $\zeta(q^n_0)\rightarrow \zeta(q_0).$ \item Suppose $I\subset \mathbb{R}_+$ is an open interval, and the map $q_0: I\rightarrow E$ is continuous, and such that $t<\zeta(q_0(t))$ for all $t\ge 0.$ Then the map $I\rightarrow E, t\mapsto \psi(q_0(t), t)$ is continuous. 
 \end{enumerate} \end{lem} 

\begin{proof} \begin{enumerate}[label=\roman{*}).]
    \item For the claimed boundedness, we let $\zeta_0$ denote the blowup time for the dynamics (\ref{eq: system of ODEs}) with $\kappa=0$. It is straightforward to see that $\zeta(q)\le\zeta_0(q)$ for all $q\in E$, and so it is sufficient to show that $E_\delta\cap \{q: \zeta_0(q)\ge \epsilon\}$ is bounded. We argue using the following explicit computation. \medskip \\ Let $q(0)=(q_1(0),q_2(0),q_3(0))\in E$, and let $q(t)=(q_1(t),q_2(t),q_3(t))$ be the solution to (\ref{eq: system of ODEs}) starting at $q(0)$. It is then straightforward to see that \begin{equation}
        \frac{1}{q_1(t)}\frac{d}{dt}q_1(t)=\frac{1}{q_3(t)}\frac{d}{dt}q_3(t)
    \end{equation} which implies that $q_3(t)=q_1(t)q_3(0)/q_1(0)$ for all $t\ge 0$. Now, the linear combination $\widetilde{q}(t)$ given by \begin{equation}
        \widetilde{q}(t)=q_1(t)+\sqrt{\frac{q_1(0)}{q_3(0)}}q_2(t)
    \end{equation} has the same blowup time as $q(t)$, and satisfies the ODE \begin{equation} \frac{d}{dt}\widetilde{q}(t)=4\gamma\sqrt{\frac{q_3(0)}{q_1(0)}}\hspace{0.1cm}\widetilde{q}(t)^2 \end{equation} which has the unique solution \begin{equation} \widetilde{q}(t)=\left(\frac{1}{\widetilde{q}(0)}-4\gamma\sqrt{\frac{q_3(0)}{q_1(0)}}t\right)^{-1}; \hspace{1cm} t< \frac{1}{4\gamma\widetilde{q}(0)}\sqrt{\frac{q_1(0)}{q_3(0)}}. \end{equation} In terms of the initial data $q(0)$, this gives the blowup time as \begin{equation} \zeta(q(0))=\frac{1}{4\gamma}\left(\sqrt{q_1(0)q_3(0)}+q_2(0)\right)^{-1} \end{equation} which converges to $0$ as $q(0)\rightarrow \infty$ in $E_\delta.$ This shows that $E_\delta \cap \{q: \zeta_0(q)\ge \epsilon\}$ is bounded, as claimed. \medskip \\ For the case where $\gamma>0$, the same computation also shows that $\zeta(q)<\infty$ for all $q\in E$ . If $\gamma=0, \kappa>0$, this is an elementary computation.
    \item The lower semicontinuity of explosion times is standard, and follows from the continuous dependence on the initial data. Therefore, it is sufficient to prove that $\limsup_{n\rightarrow \infty} \zeta(q^n)\le \zeta(q).$ In the case $\gamma=0, \kappa>0$, this is an elementary explicit calculation; for the rest of this item, we exclude this case and consider only $\gamma>0.$ \medskip \\ Suppose, for a contradiction, that for some $\epsilon>0$, we have $\limsup_{n\rightarrow \infty} \zeta(q^n)>\zeta(q)+\epsilon$; write $\tau=\zeta(q)$. By passing to a subsequence, we may assume that $\zeta(q^n)>\tau+\epsilon$ for all $n$. Moreover, since $q^n\rightarrow q \in E$, we may assume that $q^n, q \in E_\delta$ for all $n$, for some $\delta>0$, which implies that $\phi(q^n,t)\in E_\delta$ for all $t<\zeta(q^n)$ and all $n\in \mathbb{N}$.\medskip\\  Now, if $t\le \tau$, we have $\zeta(\psi_t(q^n))=\zeta(q^n)-t \ge \epsilon$, which implies the containment \begin{equation} \{\psi_t(q^n): t\le \tau, n\ge 1\} \subset E_\delta\cap J_\epsilon \end{equation} which we know, from item i)., to be bounded: for some $C<\infty$, \begin{equation}
        \{\psi_t(q^n): t\le \tau, n\ge 1\} \subset [0,C]^3.
    \end{equation} By the lemma of leaving compact sets, there exists $s<\tau$ such that, for all $t\in (s,\tau)$, $\psi_t(q)\not \in [0,C]^3.$ However, if we pick $t\in (s,\tau)$, we have $\psi_t(q^n) \rightarrow \psi_t(q)$, by the continuity of the dependence in the initial conditions, which is a contradiction. Therefore, $\limsup_{n\rightarrow \infty} \zeta(q^n)\le \zeta(q)$, which proves the claimed convergence.      
    \item As in item ii)., the case $\gamma=0, \kappa>0$ can be checked by an explicit, elementary computation. For the remainder of this point, we consider only $\gamma>0$. \medskip \\ Firstly, we note that by ii)., the map $t\mapsto \zeta(q_0(t))$ is continuous on $I$. Therefore, fixing $t\in I$, we may choose choose  $\epsilon, \delta > 0$ such that, if $\abs{t-s} \le \delta$, then $s\in I$ and $s < \min \left(\zeta\left(q_0(s)\right), \zeta\left(q_0(t)\right)\right)-\epsilon$. Now, we observe that, for $s\in [t-\delta, t+\delta],$\begin{equation}
    |\psi(q_0(t),t)-\psi(q_0(s), s)|\le|\psi(q_0(t),t)-\psi(q_0(s), t)|+|\psi(q_0(s),t)-\psi(q_0(s), s)|.
\end{equation} As $s\rightarrow t$, the first term converges to $0$ by continuity of the ODE solution $s\mapsto \psi(x_0(t),s)$; it is therefore sufficient to control the second term. We observe that, for all $s\in[t-\delta, t+\delta],$ we have $\zeta(\psi(q_0(s),s))=\zeta(q_0(s))-s>\epsilon$. Moreover, by compactness, there exists some $\eta>0$ such that $q_0(s) \in E_\eta$ for all $s\in [t-\delta, t+\delta]$, and so $\psi(q_0(s),u)\in E_\eta$ for all $u\ge 0.$ However, we showed in point i). above that the region $E_\eta \cap J_\epsilon=\{q \in E_\eta: \zeta(q)\geq\epsilon\}$ is compact  and so there exists a constant $M=M(\epsilon)$: for all $s\in[t-\delta, t+\delta]$, and for all $u \le t+\delta$, \begin{equation} |b(\psi(q_0(s), u)| \le M. \end{equation}
This implies the bound, for all $s\in[t-\delta,t+\delta]$, \begin{equation} |\psi(q_0(s), t)-\psi(q_0(s), s)| \le M|t-s|\end{equation} which implies the claimed continuity.
\end{enumerate}  \end{proof}



We can now use this to prove our main result Lemma \ref{lemma: second moment before tgel} on the second moment $\mathcal{E}(t)$ in the subcritical phase.





\begin{proof}[Proof of Lemma \ref{lemma: second moment before tgel}]
Let $\mu_0$ be any measure on $S$ satisfying ({A1}-{3}.), and let $(\mu_t)_{t\ge 0}$ be the associated solution to (\ref{eq: E+G}). From the discussion following Lemma \ref{lemma: strong solutions}, it is sufficient to show that $t_\mathrm{g}=t_\mathrm{e}<\infty$. We also recall that $t_\mathrm{e}$ is characterised as the explosion time $\zeta$ of the ODE system (\ref{eq: ODE1}-\ref{eq: ODE3}). \medskip \\ Firstly, sinConsider ce the base measure $m$ of $\mu_0$ is not a multiple of the point mass $\delta_0$, all the quadratic moments $q$ of $\mu_0$ are strictly positive, and so $q \in E$. Therefore, by the second point of Lemma \ref{lemma: ODE considerations}, the explosion time $t_\mathrm{e}=\zeta(q)<\infty.$\medskip \\ 
As $\epsilon \downarrow 0$, the quadratic moments $q^\epsilon$ of $\mu^\epsilon_0$ converge to the quadratic moments $q\in E$ of $\mu_0$ by dominated convergence.  Therefore, by Lemma~\ref{lemma: ODE considerations},  $t^\epsilon_\mathrm{e}=\zeta(q^\epsilon)\rightarrow \zeta(q)= t_\mathrm{e}$.
By Lemmas \ref{lemma: Relationship}, ~\ref{lemma: connecting mu-epsilon and mu}, we know that $t_\mathrm{e}^\epsilon=t_\mathrm{g}^\epsilon$ and that $t_\mathrm{g}^\epsilon \rightarrow t_\mathrm{g}$. Together, these imply that $t_\mathrm{e} = t_\mathrm{g}$.

\end{proof} 

%Reference for lowersemicontinuity: https://mathoverflow.net/questions/114747/dependence-of-the-blow-up-time-of-existence-of-an-ode-with-respect-to-initial-co 

\fi


\iffalse \section{\textbf{Convergence of the Stochastic Coagulant}}
\label{sec: csc}

We now turn to a preliminary version of Theorem \ref{thrm: convergence of stochastic coagulent}.
In this section, we will prove \emph{local uniform} convergence of the stochastic coagulant $\mu^N_t$ to a solution $\mu_t$ of (\ref{eq: E+G}); throughout, we fix $\mu_0$ satisfying (A1-4.). As mentioned in the introduction, we will consider a mild generalisation of the stochastic coagulant, which will be helpful for future reference: we allow initial data of the form 
\begin{equation}
    \mu^N_0 := \frac1N \sum_{i=1}^{l_N} \delta_{\left(1,v_i, \frac12 \abs{v_i}^2\right)}
\end{equation}
where the $(v_i)_{i=1}^{l_N}$ are the initial velocities for the underlying Kac process, $l_N\le N$, and we ask that the following conditions hold.
\\B1. As $N\rightarrow \infty$, the initial measures $\mu^N_0=\frac1N\sum_{i\le N}\delta_{(1,v_i,\frac{1}{2}|v_i|^2)}$ converge in probability to $\mu_0$ in the vague topology in probability: \begin{equation}d_0(\mu^N_0, \mu_0)\rightarrow 0 \hspace{1cm}\text{in probability}. \end{equation} 
\\B2. We also have the convergence \begin{equation} \langle x, \mu^N_0\rangle \rightarrow \langle x, \mu_0\rangle \hspace{1cm} \text{in probability}\end{equation} where we recall that $x$ is the identity function on $S$, and \begin{equation} \sup_{N\ge 1} \EE\langle \varphi^2, \mu^N_0\rangle <\infty. \end{equation}
We have the following corollary of the second point of B2. For $\eta>0$, let $X^N_\eta$ be the number of particles with kinetic energy exceeding $\eta:$ \begin{equation}
    X^N_\eta= \#\left\{i\le l_N: \frac{1}{2}|v_i|^2>\eta\right\}.
\end{equation}Then, as $\eta\rightarrow \infty$, we have the convergence \begin{equation}\label{eq: b3} \sup_{N\ge 1} \mathbb{E}\left[\frac{1}{N}X^N_\eta\right]\rightarrow 0. \end{equation} \medskip \\ For example, it is straightforward to verify that these hold in the case where $l_N=N$, and $v_1,....,v_N$ are independent samples from the distribution $m$ described in Section \ref{sec: results}. However, the more general hypotheses will be useful for a `duality' argument in Section \ref{sec: finiteness of second moment}. Our result is as follows:
\begin{lem}\label{lemma: local uniform convergence of stochastic coagulent}
Suppose $\mu_0$ satisfies ({A1}-{4.}), and let $(\mu_t)_{t\ge 0}$ be the solution to (\ref{eq: E+G}) starting at $\mu_0$. Let $(\mu^N_t)_{t\ge 0}$ be stochastic coalescents with initial data $\mu^N_0$ satisfying (B1-2.). Then we have the local uniform convergence
\begin{equation}
\forall t_\mathrm{f}\ge 0 \hspace{0.5cm} \sup_{t\le t_\mathrm{f}} d_0(\mu^N_t, \mu_t)\rightarrow 0 \text{ in probability} 
\end{equation} where recall that $d_0$ is a complete metric inducing the vague topology.
\end{lem} \begin{rmk}
We will later upgrade the \emph{local} uniform convergence to \emph{full} uniform convergence in Lemma \ref{lemma: uniform convergence of coagulant}. We also remark that this does not immediately imply the convergence of the gel terms in Theorem \ref{thrm: convergence of stochastic coagulent}, as the test functions involved are  neither compactly supported nor even bounded. This will be dealt with in Sections \ref{sec: ECT}, \ref{sec: COG}, where the proofs build on this result.
\end{rmk}
The proof proceeds as follows, based on the well known method of proving tightness and identifying possible limit paths. \medskip \\  Firstly, the jump rates can bounded, uniformly in time, in terms of the initial second moment $\langle \varphi^2, \mu^N_0\rangle$ and, thanks to (B2.), these are stochastically bounded $\langle \varphi^2, \mu^N_0\rangle \in \mathcal{O}_\mathrm{p}(1)$ as $N\rightarrow \infty$. As a result, it follows that for all  $t_\mathrm{f}\ge 0$, the processes $(\mu^N_t)_{0\le t\le t_\mathrm{f}}$ are tight in the Skorohod topology of $\mathbb{D}([0,t_\mathrm{f}],(\mathcal{M},d_0))$.  \medskip \\ Next, we wish to argue that if $\overline{\mu}$ is any subsequential limit point, then $\overline{\mu}$ coincides with the solution $\mu_t$ to (\ref{eq: E+G}). For this stage, we show that for certain, well-chosen $\xi>0$, the pair \begin{equation} \mu^{N,\xi}_t=\mu^N_t1_{S_\xi}; \hspace{1cm} g^{N,\xi}_t=\langle x, \mu^N_t-\mu^{N,\xi}_t\rangle  \end{equation} converge to a pair $\overline{\mu}^\xi_t=\overline{\mu}_t1_{S_\xi}, \overline{g}^\xi_t$ which solve the restricted evolution equations (\ref{eq:rE1},  \ref{eq: rE2}), started at \begin{equation} \overline{\mu}^\xi_0=\mu_01_{S_\xi}; \hspace{1cm} \overline{g}^\xi_0=\int_{x\not \in S_\xi} x\mu_0(dx). \end{equation} In order to prove this convergence, we will need a pair of regularity conditions (C1-C2.) which will be displayed below. These allow us to obtain vague convergence of $\mu^{N,\xi}_t$, despite the discontinuity of the cutoff $S_\xi.$ Moreover, one can show that these conditions are satisfied for almost all $\xi>0$.  Thanks to the construction of solutions to the global equation (\ref{eq: E+G}) in Lemma \ref{lemma: E and U}, we know that for all such $\xi$, $\overline{\mu}_t1_{S_\xi}$ coincides with $\mu_t1_{S_\xi}$. Finally, if we may take a limit of such $\xi \uparrow \infty$, we can conclude the equality $\overline{\mu}_t=\mu_t, t\le t_\mathrm{f}$.  Since the limit process $(\mu_t)_{0\le t\le t_\mathrm{f}}$ is continuous in the vague topology $(\mathcal{M},d_0)$, it follows that we may upgrade to uniform convergence: \begin{equation} \sup_{0\le t\le t_\mathrm{f}}\hspace{0.1cm}d_0\left(\mu^N_t, \mu_t\right) \rightarrow 0 \hspace{0.5cm} \text{in probability}  \end{equation} as claimed. \medskip \\ We first prove some tightness results for the processes $(\mu^N_t)_{t\le t_\mathrm{f}}.$ 

\iffalse
\begin{lem}\label{lemma: cpct contain}
Suppose conditions (B1-2.) hold. Then there exists a compact set $\mathcal{A} \subset \mathcal{M}$ \iffalse and $\widetilde{B} \subset \mathcal{M}^\ast$ \fi such that
\begin{equation}
    \PP\left(\exists t \in [0,\infty) \colon \mu^N_t \notin \mathcal{A}  \right) \rightarrow 0
\end{equation} as $N\rightarrow \infty.$
\fi
\iffalse and
\begin{equation}
    \PP\left(\exists (t,\xi) \in [0,\infty)\times (0,\infty) \colon \left({\mu}^{N,\xi}_t, g^{N,\xi}_t\right) \notin \widetilde{B} \right) \rightarrow 0
\end{equation}
\end{lem}
\textcolor{red}{\textbf{Isn't $\mathcal{M}$ already compact?}}
\begin{proof}
Define compact subsets of $S$ by $A_r = \left\{x\in S \colon \pi_n(x) \leq 2r, \pi_e(x) \leq r \sigma_2(m)\right\}$ so that
\begin{equation}
    \sup_{t\ge 0}\hspace{0.1cm} \mu_t^N(A_r^\mathrm{c}) \leq \frac{1}{2r} + \frac{\left<\pi_e, \mu_0^N\right>}{2r \sigma_2(m)}.
\end{equation}
It is straightforward to verify that $\mathcal{A}:= \left\{\nu\in \mathcal{M} \colon \nu\left(A_r^\mathrm{c}\right) < 1/r\ \forall r\in\NN \right\}$ is a relatively compact subset of $\mathcal{M}$ in the vague topology.
By ({A3.}), we have $\mathbb{P}(\left<\pi_e, \mu_0^N\right> \ge \sigma_2(m))\rightarrow 0$, which implies that $ \PP\left(\exists t \colon \mu_t^N \notin \mathcal{A}\right) \rightarrow 0$.
\iffalse The same argument applies to $\mu^N _t1_{\left\{\pi_n \leq \xi\right\}}$, and noting that \begin{equation} \pi_n(g^{N,\xi}_t) \leq 1; \hspace{1cm}\pi_e(g^{N,\xi}_t) \leq \left<\pi_e, \mu_0^N \right> \rightarrow \frac12 \sigma_2(m)\end{equation} a similar argument applies for $g^{N,\xi}_t$. \fi 
\end{proof}
\fi
\begin{lem}\label{lemma: bdd jump rate}
Suppose conditions ({B1}-{3.}) hold. Then
\begin{enumerate}[label=\roman{*}).]
    \item There process $
    \mu^N$ has at most  $N-1$ jumps in $[0,\infty)$;
    \item the jump rates of $\mu^N$ are bounded by
    \begin{equation}
        \lambda^N := N\left\{\kappa  \left<\pi_n, \mu_0^N\right>^2
        +8 \gamma  \left<\pi_n, \mu_0^N\right> \left<\pi_e, \mu_0^N\right>
        \right\},
    \end{equation}
    which is constant in time, and $\lambda^N/N \in \mathcal{O}_\mathrm{p}(1)$.
\end{enumerate}
\end{lem}
\begin{proof}
For the first claim, we note that after $l_N-1 \le N-1$ jumps, precisely one cluster remains and so the system undergoes no further changes.  The second claim is a simple calculation using the conservation laws for $\pi_n$ and $\pi_e$. The claim that $\lambda^N/N \in \mathcal{O}_\mathrm{p}(1)$ follows from (B2.).
\end{proof}

\begin{lem}\label{lemma: tight processes}
Suppose (B1-2.) hold, and let $t_\mathrm{f}>0$. Then
 the distributions of $\mu^N$ are tight in the space of probability measures on $\mathbb{D}\left([0,t_\mathrm{f}]);  \left(\mathcal{M}, d_0\right)\right)$.
\end{lem}
\begin{proof}
First, we consider the real valued processes $\langle f, \mu^N_t\rangle$ for $f\in C_c(S).$ Compact containment is immediate, since $\abs{\left<f,\mu_t^N\right>} \leq \supnorm{f}$. For $\delta>0$, write $(I_i: i=1,...,\lceil t_\mathrm{f}/\delta\rceil)$ for a partition of $[0,t_\mathrm{f}]$ into intervals of length at most $\delta$. Recalling from Lemma \ref{lemma: bdd jump rate} that the rescaled jump rate $\lambda^N/N$ is $O_\mathrm{p}(1)$, it follows that, for all $\epsilon>0$, \begin{equation} \sup_{N\ge 1}\hspace{0.1cm} \mathbb{P}\left(\mu^N \text{ makes at least }\frac{\epsilon N}{3} \text{ jumps in any  } I_i: i\le \lceil t_\mathrm{f}/\delta\rceil\right) \rightarrow 0 \end{equation} as $\delta\rightarrow 0.$ Now, since the jumps are bounded by $\frac{3}{2}\supnorm{f}/N$, for any $\epsilon>0$ it follows that \begin{equation} \sup_{N\ge 1} \hspace{0.1cm} \mathbb{P}\left(\exists s, t \le t_\mathrm{f}: |s-t|<\delta, |\langle f, \mu^N_t-\mu^N_s\rangle|>\epsilon\right) \rightarrow 0 \end{equation} as $\delta \rightarrow 0.$ Therefore, for any fixed $f \in C_c(S),$ we may apply \cite[Chapter 3, Theorem 8.6c)]{EK86} to see that the processes $(\langle f, \mu^N_t\rangle)_{0\le t\le t_\mathrm{f}}$ are tight. \medskip \\ 
We now use a general result \cite[Theorem 4.6]{J86} on tightness in Skorohod spaces. Compact containment of $\mu^N_t$ in $(\mathcal{M}, d_0)$ is immediate, since $\mathcal{M}$ is itself compact, and it is straightforward that the maps $(\mu \mapsto \langle f, \mu\rangle: f\in C_c(S))$ on $(\mathcal{M},d_0)$ satisfy the requirements of the cited theorem. Therefore, the result of the previous paragraph shows that the processes $\mu^N$ are tight in $\mathbb{D}([0,t_\mathrm{f}],(\mathcal{M},d_0))$, as claimed.
\end{proof}

Having proven tightness, we now turn to the identification of possible limit paths. For $N\in\NN$ and $\xi \in \mathbb{R}_+,$ we consider the processes $\mu^N, \mu^{N,\xi} \in \mathbb{D}\left([0,\infty); \left(\mathcal{M}, d_0\right)\right)$, where
$\mu^{N,\xi}_t = \mu^N_t 1_{S_\xi}$. We write $g^{N,\xi} \in \mathbb{D}\left([0,\infty); S_\mathrm{g}\right)$ for the data of the gel cutoff at this level:
\begin{equation}\label{eq: cutoff gel} g^{N,\xi}_t = \left<x, \mu_0^N -\mu^{N,\xi}_t\right>\end{equation} where we recall that $x$ denotes the identity function on $S$, so that
$\left(\mu^{N,\xi}, g^{N,\xi}\right) \in \mathbb{D}\left([0,\infty); \left(\mathcal{M}^\ast, d^\ast\right)\right). $  
We first identify the martingale parts of  the processes $\langle f, \mu^{N,\xi}_t\rangle$ and $g^{N,\xi}_t$.
For $\xi >0$ and $f\in C_c(S)$, let
\begin{equation}\label{eq:mg1} \begin{split}
&\mathfrak{M}_t^{f,N,\xi}:=  \left<f, \mu_t^{N,\xi}\right> - \left<f, \mu_0^{N,\xi}\right> 
- \int_0^t \int_{S_\xi}f(x) \overline{K}\left(x,g^{N,\xi}_t\right)\mu^{N,\xi}_s(d x)d s\\
&\hspace{-0.6cm}- \frac12 \int_0^t \int_{S_\xi^2}\left[f(x+y)1[\varphi(x+y) \leq \xi ] - f(x) - f(y)\right]
      \overline{K}(x,y)\mu^{N,\xi}_s(d x)\mu^{N,\xi}_s(d y) ds \\
     &\hspace{-0.6cm}  + \frac{1}{2N}\int_0^t \int_{S_\xi} \left[f(2x)1[\varphi(2x) \leq \xi ] - 2f(x)\right]\overline{K}(x,x)\mu^{N,\xi}_s(d x) d s
\end{split}\end{equation}
and
\begin{equation}\label{eq:mg2} \begin{split}
\widehat{\mathfrak{M}}_t^{N,\xi} &:= g_t^{N,\xi} - g_0^{N,\xi} 
- \int_0^t \int_{S_\xi}x \overline{K}\left(x,g^{N,\xi}_t\right)\mu^{N,\xi}_s(d x)d s\\& \hspace{1cm}
- \frac12 \int_0^t \int_{S_\xi^2}(x+y)1[\varphi(x+y) \leq \xi ]
      \overline{K}(x,y)\mu^{N,\xi}_s(d x)\mu^{N,\xi}_s(d y) ds \\ &\hspace{1cm}
      + \frac{1}{2N}\int_0^t \int_{S_\xi} 2x 1[\varphi(2x) \leq \xi ]\overline{K}(x,x)\mu^{N,\xi}_s(d x) d s.
\end{split}\end{equation}


\begin{lem}\label{lemma: qvar}
Suppose conditions (B1-2.) hold, and write $(\mathcal{F}^{N}_t)_{t\ge 0}$ for the natural filtration of $(\mu^N_t)_{t\ge 0}$. Then, for all $\xi \in (0, \infty]$ and $f\in C_c(S)$, the processes $\mathfrak{M}_t^{f,N,\xi}, \widehat{\mathfrak{M}}_t^{N,\xi}$ are martingales with respect to $(\mathcal{F}^N_t)_{t\ge 0}.$
The quadratic variations are bounded by
\begin{equation}
    \left[\left<f, \mu^{N,\xi}\right>\right]_t
    = \left[\mathfrak{M}^{f,N,\xi}\right]_t
    \leq \frac{9 \|f\|_\infty^2}{N}; \qquad
    \left[g^{N,\xi}\right]_t
    = \left[\widehat{\mathfrak{M}}_t^{N,\xi}\right]_t
    \leq \frac{2\xi}{N}\left<\varphi, \mu^N_0\right>
\end{equation} uniformly in $ t \in [0,\infty)$ and $ \xi \in [0,\infty).$
\end{lem}
\begin{proof}
The martingale property is a standard; see, for example, \cite[Chapter 4, Proposition 1.7]{EK86} or \cite[Lemma 19.21]{K02}.
Since $\mathfrak{M}^{f,N,\xi}$ and $\widehat{\mathfrak{M}}^{N,\xi}$ are pure jump processes, the quadratic variation is the sum of the squares of the jumps, which can be bounded using Lemma~\ref{lemma: bdd jump rate} for the first claim and by straightforward maximisation arguments for the second.
\end{proof}

\iffalse Having now proven the tightness, we will argue that the only possible limit point is the deterministic path $(\mu_t)_{t\ge 0}$ given as the solution to (\ref{eq: E+G}). \fi

Next, we discuss a regularity condition for potential cutoff values $\xi$. This will allow us to deduce vauge convergence of the truncated measures $\mu^{N,\xi}_t=\mu^N_t1_{S_\xi}$, despite the discontinuity of the cutoff $1_{S_\xi}$.
\begin{lem}[Regularity Condition]\label{lemma: regularity condition} Suppose that $\mu^N$ are stochastic coagulants satisfying (B1-2.), and that some subsequence $(\mu^{N_r})_{r\ge 1}$ of $(\mu^N)_{N\ge 1}$ converges in distribution to a random variable $\overline{\mu}$ in $\mathbb{D}([0,t_\mathrm{f}],(\mathcal{M},d_0))$. Then, for almost all $\xi>0$, the following hold: \begin{enumerate}[label=\roman{*}).]\item Almost surely, for  almost all $t\le t_\mathrm{f}$, \begin{equation} \label{eq: reg} \tag{{C1.}}
   \overline{\mu}_t\left(\left\{x\colon \varphi(x) = \xi\right\}\right) + 
    \overline{\mu}_t \otimes \overline{\mu}_t \left(\left\{(x,y) \colon \varphi(x+y) = \xi\right\}\right) = 0;
\end{equation} \item This also holds for $t=0$. That is, almost surely, \begin{equation} \label{eq: reg2} \tag{{C2.}}
   \overline{\mu}_0\left(\left\{x\colon \varphi(x) = \xi\right\}\right) + 
    \overline{\mu}_0 \otimes \overline{\mu}_0 \left(\left\{(x,y) \colon \varphi(x+y) = \xi\right\}\right) = 0. 
\end{equation} \end{enumerate} \end{lem} \begin{proof} Since $\mathbb{D}([0,t_\mathrm{f}],(\mathcal{M},d_0))$ is separable, we may use the Skorohod representation theorem to realise all $\mu^N, \overline{\mu}$ on a common probability space, such that $\mu^{N_r}\rightarrow \overline{\mu}$ almost surely in the Skorohod topology of $\mathbb{D}([0,t_\mathrm{f}],(\mathcal{M},d_0))$, and such that $\langle \varphi, \mu^N_0\rangle \rightarrow \langle \varphi, \mu_0\rangle$ almost surely. In particular, $\langle \varphi, \mu^N_0\rangle$ are almost surely bounded. \medskip \\ We first consider the case of nonrandom $\nu^{N_r} \rightarrow \nu$ in $\mathbb{D}([0,t_\mathrm{f}],(\mathcal{M},d_0))$, and such that, for some constant $C$, \begin{equation} \sup_{N \ge 0}\hspace{0.1cm}\sup_{t\le t_\mathrm{f}} \langle \varphi, \nu^N_t\rangle \le C <\infty. \end{equation}  Let $\Psi$ be the measure on $[0,\infty)$ given by \begin{equation}\begin{split}\label{eq: Psi} \Psi(A)&=\nu_0(x: \varphi(x)\in A)+\nu_0\otimes\nu_0((x,y):\varphi(x+y)\in A)\\&\hspace{1cm}+\int_0^{t_\mathrm{f}} \left(\nu_t(x: \varphi(x)\in A)+\nu_t\otimes\nu_t((x,y): \varphi(x+y)\in A)\right) dt.\end{split} \end{equation} Since $\nu_t$ is a subprobability measure for all $t\le t_\mathrm{f}$, $\Psi$ has total mass at most $2(1+t_\mathrm{f})$, and so, for all but countably many $\xi\ge 0$, $\Psi(\{\xi\})=0.$ For such $\xi$, we have the desired properties that \begin{equation} \nu_0(x: \varphi(x)=\xi)+\nu_0\otimes\nu_0((x,y):\varphi(x+y)=\xi)=0; \end{equation}\begin{equation} \nu_t(x: \varphi(x)=\xi)+\nu_t\otimes\nu_t((x,y):\varphi(x+y)=\xi)=0 \text{ for almost all }t\le t_\mathrm{f}. \end{equation} We now show how this may be extended to the case of random $\mu^{N}, \overline{\mu}$ such that $\mu^{N_r}\rightarrow \overline{\mu}$ almost surely in $\mathbb{D}([0,t_\mathrm{f}],(\mathcal{M},d_0)).$ Let $\Psi$ be the random measure in (\ref{eq: Psi}) corresponding to $\overline{\mu}$, and let $A(\overline{\mu})$ be the random set \begin{equation} A(\overline{\mu})=\left\{\xi \ge 1: \hspace{0.1cm}\Psi(\{\xi\})>0\right\}. \end{equation} The argument above shows that, almost surely, $ \int_{A(\overline{\mu})} d\xi=0$; by Fubini, this implies that \begin{equation} \int_1^\infty \mathbb{P}(\xi \in A(\overline{\mu}))d\xi=0. \end{equation} Therefore, for almost all $\xi \ge 1$, $\mathbb{P}(\xi \in A(\overline{\mu}))=0$. For all such $\xi$, the following hold almost surely:  \begin{equation} \overline{\mu}_0(x: \varphi(x)=\xi)+\overline{\mu}_0\otimes\overline{\mu}_0((x,y):\varphi(x+y)=\xi)=0; \end{equation}\begin{equation} \overline{\mu}_t(x: \varphi(x)=\xi)+\overline{\mu}_t\otimes\overline{\mu}_t((x,y):\varphi(x+y)=\xi)=0 \text{ for almost all }t\le t_\mathrm{f}. \end{equation}  This corresponds precisely to the desired conditions (C1-2.). \end{proof} 

This regularity condition now allows us to deduce weak convergence of the cutoff measures $\mu^{N_r,\xi}_t$.

\begin{lem}\label{lemma: convergences in distribution}
Suppose that $\mu^N$ are stochastic coagulants satisfying (B1-2.), and that a subsequence  $\mu^{N_r}$ converges in distribution to a random variable $\overline{\mu}$ in $\mathbb{D}([0,t_\mathrm{f}],(\mathcal{M},d_0))$. If $\xi$ is such that (C1-2.) hold for $\overline{\mu}$, then writing $\overline{\mu}^\xi_t$ for the cutoff measures $\overline{\mu}^\xi_t=\overline{\mu}_t1_{S_\xi},$ we have the following convergences in distribution as $r\rightarrow \infty$. 
\begin{enumerate}[label=\roman{*}).] \item
For any $f\in C_c(S)$ and $0\le t \le t_\mathrm{f}$,
\begin{multline}
    \hspace{-0.6cm}\int_0^t \int_{S_\xi^2}\left[f(x+y)1[\varphi(x+y) \leq \xi ] - f(x) - f(y)\right]
      \overline{K}(x,y)\mu^{N_r,\xi}_s(d x)\mu^{N_r,\xi}_s(d y)d s
    \\\rightarrow
    \int_0^t \int_{S_\xi^2}\left[f(x+y)1[\varphi(x+y) \leq \xi ] - f(x) - f(y)\right]
      \overline{K}(x,y)\overline{\mu}^\xi_s(d x)\overline{\mu}^\xi_s(d y)d s;
\end{multline}
and
\begin{multline}
    \int_0^t \int_{S_\xi^2}(x+y)1[\varphi(x+y) > \xi ]
      \overline{K}(x,y)\mu^{N_r,\xi}_s(d x)\mu^{N_r,\xi}_s(d y)d s
    \\\rightarrow
    \int_0^t \int_{S_\xi^2}(x+y)1[\varphi(x+y) > \xi ]
      \overline{K}(x,y)\overline{\mu}^\xi_s(d x)\overline{\mu}^\xi_s(d y)d s.
\end{multline}
\item We write $\overline{g}^\xi$ for the data of the truncated gel corresponding to $\overline{\mu}$: \begin{equation}\label{eq: cutoff gel overline g xi} \overline{g}^{\xi}_t = \left<x, \mu_0 -\overline{\mu}^\xi_t\right>.\end{equation}   Then, for all $f\in C_c(S)$ and $t\le t_\mathrm{f}$, \begin{equation}
    \int_0^t \int_{S_\xi}f(x) \overline{K}(x,g^{N_r,\xi}_s)\mu^{N_r,\xi}_s(d x)d s
    \rightarrow
    \int_0^t \int_{S_\xi}f(x) \overline{K}(x,\overline{g}^\xi_s)\overline{\mu}^\xi_s(d x)d s;
\end{equation}
\begin{equation}
    \int_0^t \int_{S_\xi}x \overline{K}(x,g^{N_r,\xi}_s)\mu^{N_r,\xi}_s(d x)d s \rightarrow \int_0^t \int_{S_\xi}x \overline{K}(x,\overline{g}^\xi_s)\overline{\mu}^\xi_s(d x)d s.
\end{equation}
\item Let $L^\xi_\mathrm{g}, \widehat{L}^\xi_\mathrm{g}$ be the truncated drift operators in (\ref{eq: truncated drift 1}, \ref{eq: truncated drift 2}), and write $\mu^\xi_0, g^\xi_0$ for the cutoff measure and gel corresponding to the base measure $\mu_0$. Then, for all $t\le t_\mathrm{f}$ and $f\in C_c(S)$, \begin{equation}\begin{split} &\int_0^t \left[\langle f, \mu^{N_r,\xi}_s\rangle-\langle f,\mu^{N_r,\xi}_0\rangle -\int_0^s\left\langle f, L^\xi_\mathrm{g}\left(\mu^{N_r,\xi}_u,g^{N_r,\xi}_u\right)\right\rangle du\right]ds \\& \hspace{2cm} \rightarrow \int_0^t \left[\langle f, \overline{\mu}^{\xi}_s\rangle-\langle f,\mu^{\xi}_0\rangle -\int_0^s\left\langle f, L^\xi\left(\overline{\mu}^{\xi}_u,\overline{g}^\xi_u\right)\right\rangle du\right]ds \end{split}\end{equation} and \begin{equation}\begin{split} & \int_0^t \left[g^{N_r,\xi}_s-g^{N_r,\xi}_0-\int_0^s \widehat{L}^\xi_\mathrm{g}(\mu^{N_r,\xi}_u,g^{N_r,\xi}_u)du\right]ds \\  & \hspace{2.5cm}\rightarrow \int_0^t \left[\overline{g}^{\xi}_s-g^{\xi}_0-\int_0^s \widehat{L}^\xi_\mathrm{g}(\overline{\mu}^{\xi}_u,\overline{g}^{\xi}_u)du\right]ds. \end{split}\end{equation}
\end{enumerate}
\end{lem}
\begin{proof} As in the previous lemma, the Skorohod representation theorem shows that we may realise $\mu^N, \overline{\mu}$ on a common probability space, such that $\mu^{N_r}\rightarrow \mu$ almost surely in the Skorohod topology of $\mathbb{D}([0,t_\mathrm{f}],(\mathcal{M},d_0))$. Moreover, using (B1-2.) as above, $\langle \varphi, \mu^{N_r}_t\rangle$ is almost surely bounded, uniformly in $r\ge 1$ and $t\le t_\mathrm{f}$, and  \begin{equation}  d_0(\mu^{N_r}_0,\mu_0)\rightarrow 0; \hspace{1cm} \langle x, \mu^{N_r}_0\rangle \rightarrow \langle x, \mu_0\rangle \end{equation} almost surely. Since the map $(\mu_t)_{t\le t_\mathrm{f}}\mapsto \mu_0$ is continuous in the Skorohod topology of $\mathbb{D}([0,t_\mathrm{f}],(\mathcal{M},d_0))$, the first convergence displayed above implies that $\overline{\mu}_0=\mu_0$ almost surely. It is therefore sufficient to prove the results for the special case of nonrandom $\nu^N$, such that  $\nu^{N_r}\rightarrow \nu$ almost surely, and such that the preceding results hold with $\nu^N, \nu$ in place of $\mu^N, \overline{\mu}$. We will write $\nu_t^\xi, \nu^{N,\xi}_t$ for the cutoff measures, and $\overline{g}^\xi_t, g^{N,\xi}_t$ for the corresponding cutoff gel, defined as above with $\nu^N, \nu$ in place of $\mu^N, \overline{\mu}$.  \medskip \\ 
For part i)., Fix $f\in C_c(S)$ and $\xi\in [0, \infty)$. The functions \begin{equation}(x,y) \mapsto f(x)\overline{K}(x,y)1[\varphi(x) \le \xi,\varphi(y) \leq \xi]\end{equation} \begin{equation} (x,y)\mapsto f(x+y)\overline{K}(x,y)1[\varphi(x+y)\le \xi] \end{equation} are compactly supported, and continuous away from the exceptional set \begin{equation} E_\xi=\{(x,y)\in S^2: \varphi(x)=\xi \text{ or } \varphi(y)=\xi \text{ or } \varphi(x+y)=\xi\}. \end{equation} 

From condition (\ref{eq: reg}) it follows that, for almost all $t\le t_\mathrm{f}$, $(\nu_t\otimes \nu_t)(E_\xi)=0$. Since $d_0\left(\nu^N_t, \nu_t\right) \rightarrow 0$ for all but at most  countably many $t\le t_\mathrm{f}$, it follows that
\begin{equation}\begin{split}
    &\int_{S_\xi^2}\left[f(x) + f(y)\right]\overline{K}(x,y)\nu^{N_r,\xi}_t(d x)\nu^{N_r,\xi}_t(d y)\\& \hspace{1cm}\rightarrow
\int_{S_\xi^2}\left[f(x) + f(y)\right]\overline{K}(x,y)\nu^\xi_t(d x)\nu^\xi_t(d y); \end{split}
\end{equation}
\begin{equation}\begin{split}&
     \int_{S_\xi^2}f(x+y)1[\varphi(x+y) \le \xi]\overline{K}(x,y)\nu^{N_r,\xi}_t(d x)\nu^{N_r,\xi}_t(d y)
 \\& \hspace{1cm}   \rightarrow \int_{S^2} f(x+y)1[\varphi(x+y) \le \xi]\overline{K}(x,y)\nu^\xi_t(d x)\nu^\xi_t(d y) \end{split}
\end{equation}
for almost all $t\le t_\mathrm{f}$.
Recalling that $\overline{K}(x,y)\le \Delta \varphi(x)\varphi(y)$ for some constant $\Delta=\Delta(\kappa,\gamma)$, we bound \begin{equation}\label{eq: bound on Lxigel 1}\begin{split}
    &\left|\int_{S_\xi^2} \left[f(x+y)1[\varphi(x+y)\le \xi]-f(x)-f(y)\right] \overline{K}(x,y)\nu^{N_r,\xi}_t(dx)\nu^{N_r,\xi}_t(dy)\right| \\& \hspace{3cm}  \le 3\Delta\|f\|_\infty\langle \varphi, \nu^{N_r}_0\rangle^2 \end{split}
\end{equation} and so, by bounded convergence, for all $t\le t_\mathrm{f}$, \begin{equation} \begin{split}
   & \hspace{-0.2cm} \int_0^t \int_{S_\xi^2}\left[f(x+y)1[\varphi(x+y) \leq \xi] - f(x) - f(y)\right]
      \overline{K}(x,y)\nu^{N_r,\xi}_s(d x)\nu^{N_r,\xi}_s(d y)d s
    \\&\rightarrow
    \int_0^t \int_{S_\xi^2}\left[f(x+y)1[\varphi(x+y) \leq \xi] - f(x) - f(y)\right]
      \overline{K}(x,y)\nu^\xi_s(d x)\nu^\xi_s(d y)d s
\end{split} \end{equation}
which proves the first claim.
For the second claim, we note that $\nu^{N,\xi}_t, \nu^\xi_t$ are supported on $S_\xi$, and for all $x, y\in S_\xi$ we have the bound $\abs{x},\abs{y} \leq C\xi$, for some constant $C$; the second claim now follows. \medskip \\ For part ii), we first claim that for $t=0$ and almost all $t\le t_\mathrm{f}$, $g^{N_r,\xi}_t \rightarrow \overline{g}^\xi_t$. To see this, we recall that $\langle x, \nu^{N_r}_0\rangle \rightarrow \langle x, \nu_0\rangle = \langle x, \mu_0\rangle$ by construction, and using conditions (C1-2.), for $t=0$ and almost all $t\le t_\mathrm{f}$, \begin{equation} \langle x, \nu^{N_r,\xi}_t\rangle =\langle x1_{S_\xi},\nu^{N_r}_t\rangle \rightarrow \langle x1_{S_\xi},\nu_t\rangle =\langle x, \nu^\xi_t\rangle.  \end{equation} This follows because $x1_{S_\xi}$ is compactly supported, and except for exceptional times $t$, $x1_{S_\xi}$ is continuous except on a set of $\nu_t$-measure 0. Since $\nu_0=\mu_0$, this also implies that \begin{equation} g^{N_r,\xi}_0\rightarrow \langle x, \mu_0-\mu^\xi_0\rangle =g^\xi_0\end{equation} and we may therefore use $g^\xi_0$ in place of $\overline{g}^\xi_0$. \medskip \\  We also observe that $\varphi(g^{N_r,\xi}_t) \le \langle \varphi, \nu^{N_r}_0\rangle$ is bounded uniformly in $t\le t_\mathrm{f}$ and in $r$. The argument is now essentially identical to point i). above.  \medskip \\ For the first claim of item iii)., we first note that the the two parts above show that, for all $t\le t_\mathrm{f}$, \begin{equation}\int_0^t \left\langle f, L^\xi_\mathrm{g}\left(\nu^{N_r,\xi}_s,g^{N_r,\xi}_s\right)\right\rangle ds \rightarrow \int_0^t \left\langle f, L^\xi_\mathrm{g}\left(\overline{\mu}^\xi_s,\overline{g}^{\xi}_s\right)\right\rangle ds. \end{equation} Moreover, following (\ref{eq: bound on Lxigel 1}), we have the uniform bound, for all $u\le t_\mathrm{f},$ \begin{equation} \left|\left\langle f, L^\xi_\mathrm{g}\left(\nu^{N_r,\xi}_u,g^{N_r,\xi}_u\right)\right\rangle\right|\le 4\|f\|_\infty \langle \varphi, \nu^{N_r}_0\rangle^2. \end{equation} It therefore follows, by bounded convergence, that for all $t\le t_\mathrm{f}$, \begin{equation} \label{eq: part iii convergence 1} \int_0^t \int_0^s \left\langle f, L^\xi_\mathrm{g}\left(\nu^{N_r,\xi}_u,g^{N_r,\xi}_u\right)\right\rangle du ds \rightarrow \int_0^t \int_0^s \left\langle f, L^\xi_\mathrm{g}\left(\overline{\mu}^\xi_u,\overline{g}^{\xi}_u\right)\right\rangle du ds. \end{equation}  For the other terms, we note that $d_0(\nu^{N_r}_0, \nu_0)\rightarrow 0$, and that $d_0(\nu^{N_r}_t, \nu_t)\rightarrow 0$ for almost all $t\le t_\mathrm{f}$. Therefore, by (C1-2.) and bounded convergence, \begin{equation} \int_0^t\langle f, \nu^{N_r,\xi}_s\rangle ds \rightarrow \int_0^t \langle f, \nu^\xi_s\rangle ds;\hspace{1cm} \langle f, \nu^{N_r,\xi}_0\rangle \rightarrow \langle f, \nu^\xi_0\rangle = \langle f, \mu^\xi_0\rangle.  \end{equation} Combined with (\ref{eq: part iii convergence 1}), these prove the first claim. An identical argument proves the second claim.
\end{proof}

We will now show that for any subsequential limit point $\overline{\mu}$, the pair $(\overline{\mu}^\xi, \overline{g}^\xi)$ solves the restricted dynamics (\ref{eq:rE1}, \ref{eq: rE2}) whenever $\xi$ satisfies (C1-2.).

\begin{lem}\label{lem: mild}
Suppose conditions (B1-2.) hold, and that some subsequence $(\mu^{N_r})_{r\ge 1}$ of $(\mu^N)_{N\ge 1}$ converges in distribution to a random variable $\overline{\mu}$ in $\mathbb{D}([0,t_\mathrm{f}],(\mathcal{M},d_0))$. Suppose also that $\xi$ is such that (C1-2.) hold for $\overline{\mu}$. Write $\overline{\mu}^\xi$ for the truncated measures and $\overline{g}^\xi$ for the truncated gel, as above, and similarly $\mu^\xi_0, g^\xi_0$. Then, almost surely, for all $f\in C_c(S)$ and all $t\le t_\mathrm{f}$,
\begin{equation}
   \left<f, \overline{\mu}^\xi_t\right> - \left<f, \mu^\xi_0\right> - \int_0^t \left<f, L^\xi_\mathrm{g}\left(\overline{\mu}^\xi_s, \overline{g}^\xi_s\right) \right> ds = 0;
\end{equation}
\begin{equation} \overline{g}^\xi_t - {g}^\xi_0 - \int_0^t \widehat{L}^\xi_\mathrm{g}\left(\overline{\mu}^\xi_s, \overline{g}^\xi_s\right) d s= 0.
\end{equation} It follows that, almost surely, $(\overline{\mu}^{\xi}_t,\overline{g}^\xi_t)_{0\le t\le t_\mathrm{f}}$ is the unique solution $(\mu^\xi_t, g^\xi_t)_{0\le t\le t_\mathrm{f}}$ to (\ref{eq:rE1}, \ref{eq: rE2}) started at $(\mu^\xi_0,g^\xi_0)$. In particular, $(\overline{\mu}^\xi_t, \overline{g}^\xi_t)_{0\le t\le t_\mathrm{f}}$ is deterministic.
\end{lem}
\begin{proof}
Fix $f\in C_c(S)$. We first estimate the martingale terms $\mathfrak{M}^{f,N,\xi}$,  $\widehat{\mathfrak{M}}^{N,\xi}$. From \cite[Thrm. 26.12]{K02} and Lemma~\ref{lemma: qvar}, there exists a  constant $c=c(\kappa,\gamma)>0$ such that
\begin{equation}
    \EE\left[\sup_{t\in[0,\infty)} \left(\mathfrak{M}_t^{f,N,\xi}\right)^2\right]
    \leq c \EE\left[\sup_{t\in[0,\infty)} \left[\mathfrak{M}^{f,N,\xi}\right]_t\right]
    \leq \frac{9c \|f\|_\infty^2}{N}.
\end{equation} It follows that, for all $t\le t_\mathrm{f}$,\begin{equation} \int_0^{t} M^{f,N,\xi}_s d s \rightarrow 0\end{equation} in distribution.
We now estimate the self-interaction term: for some other constant $c$, we bound
\begin{equation}
   \abs{\frac{1}{2N}\int_0^t \int_S \left[f(2x)1[\varphi(2x)\leq \xi] - 2f(x)\right]\overline{K}(x,x)\mu^{N,\xi}_s(d x) d s}
   \leq \frac{c}{N}\|f\varphi^2\|_\infty.
\end{equation}
Therefore, from (\ref{eq:mg1}), it follows that for all $t \le t_\mathrm{f}$, \begin{equation} \int_0^t\left[\langle f,\mu^{N_r, \xi}_{s}\rangle -\langle f, \mu^{N_r,\xi}_0\rangle -\int_0^{s} \langle f, L^\xi_\mathrm{g}(\mu^{N_r,\xi}_u, g^{N_r,\xi}_u)\rangle du \right]ds\rightarrow 0 \end{equation} in distribution. Comparing this to item iii). of Lemma \ref{lemma: convergences in distribution}, we identify the two limits to conclude that almost surely, for all $t\le t_\mathrm{f}$, \begin{equation} \int_0^t\left[\langle f, \overline{\mu}^{\xi}_s\rangle -\langle f, {\mu}^{\xi}_0\rangle -\int_0^s \langle f, L^\xi_\mathrm{g}(\overline{\mu}^{\xi}_u, \overline{g}^{\xi}_u)\rangle \right]ds =0.\end{equation} Since $t\le t_\mathrm{f}$ is arbitrary and the integrand is right-continuous, this implies that, almost surely, for all $t\le t_\mathrm{f}$, \begin{equation} \langle f, \overline{\mu}^{\xi}_t\rangle -\langle f, {\mu}^{\xi}_0\rangle -\int_0^t \langle f, L^\xi_\mathrm{g}(\overline{\mu}^{\xi}_s, \overline{g}^{\xi}_s)\rangle  =0.\end{equation} Taking an intersection over $f$ belonging to a countable dense subset of $(C_c(S), \|\cdot\|_\infty)$ proves the first claim. The argument for the second claim is identical. It follows that $(\overline{\mu}^\xi_t, \overline{g}^\xi_t)_{0\le t\le t_\mathrm{f}}$ satisfies (\ref{eq:rE1}, \ref{eq: rE2}) almost surely, which implies that $(\overline{\mu}^\xi_t, \overline{g}^\xi_t)_{0\le t\le t_\mathrm{f}}=(\mu^\xi_t, g^\xi_t)_{0\le t\le t_\mathrm{f}}$ almost surely, by uniqueness in Lemma \ref{lemma: restricted dynamics}.
\end{proof}

We may now prove Lemma \ref{lemma: local uniform convergence of stochastic coagulent}, using the well-known combination of tightness and the identification of the limit. Tightness of the processes $\mu^N$ was proven in Lemma \ref{lemma: tight processes}, and so it is sufficient to characterise possible limit paths. Suppose a subsequence $(\mu^{N_r}_t)_{0\le t\le t_\mathrm{f}}$ converges in distribution to a limit $(\overline{\mu}_t)_{0\le t\le t_\mathrm{f}}$. From Lemma \ref{lemma: regularity condition}, there is an unbounded set $\mathcal{X}\subset [1,\infty)$ such that (C1-2.) hold for each $\xi\in\mathcal{X}$. By Lemma \ref{lem: mild}, it follows that the limit process $(\overline{\mu}_t1_{S_{\xi}})_{0\le t\le t_\mathrm{f}}$ is, almost surely, the measure part $(\mu^{\xi}_t)_{0\le t\le t_\mathrm{f}}$ of the unique solution to (\ref{eq:rE1}, \ref{eq: rE2}) which starts at $\mu_01_{S_\xi}$, for each such $\xi$. Moreover, we recall from Lemma \ref{lemma: E and U} that the full solution $(\mu_t)_{0\le t\le t_\mathrm{f}}$ to (\ref{eq: E+G}) is characterised by \begin{equation} \mu_t=\lim_{\xi\uparrow \infty} \mu^\xi_t \end{equation} in the sense of monotone limits. Therefore, almost surely, for all $t\le t_\mathrm{f}$, \begin{equation}  \overline{\mu}_t = \lim_{\xi \uparrow \infty; \hspace{0.1cm} \xi \in \mathcal{X}} \overline{\mu}_t1_{S_\xi} =  \lim_{\xi \uparrow \infty; \hspace{0.1cm} \xi \in \mathcal{X}} \mu^\xi_t = \mu_t. \end{equation} Therefore, the \emph{only} possible subsequential limit point is the deterministic path $(\mu_t)_{0\le t\le t_\mathrm{f}}$; together with tightness, this implies that $(\mu^N_t)_{0\le t\le t_\mathrm{f}} \rightarrow (\mu_t)_{0
\le t\le t_\mathrm{f}}$ in distribution in the Skorohod topology. Since the limit path is nonrandom, it follows that, for all $\epsilon>0$, \begin{equation}\label{eq: convergence in skorohod met} \mathbb{P}\left(d_\mathrm{Sk}((\mu^N_t)_{0\le t\le t_\mathrm{f}},(\mu_t)_{0\le t\le t_\mathrm{f}})>\epsilon\right) \rightarrow 0\end{equation} where we recall that $d_\mathrm{Sk}$ is a complete metric compatible with the Skorohod topology. \medskip \\ Returning to the cutoff dynamics in (\ref{eq:rE1}, \ref{eq: rE2}), one can see that each cutoff solution $(\mu^{\xi}_t)_{0\le t\le t_\mathrm{f}}=(\mu_t1_{S_\xi})_{0\le t\le t_\mathrm{f}}$ is continuous in total variation norm. Now, if $f\in C_c(S)$, one may choose $\xi$ such that $\text{supp}(f)\subset S_\xi$, so that for all $t\le t_\mathrm{f}$, \begin{equation} \langle f, \mu_t\rangle =\langle f, \mu^\xi_t\rangle.\end{equation} It follows that each process $\langle f, \mu_t\rangle$ is continuous for $f\in C_c(S)$, which implies that $\mu_t$ is continuous in the metric $d_0$. Therefore, every uniformly open neighbourhood of $(\mu_t)_{0\le t\le t_\mathrm{f}}$ contains a Skorohod-open neighbourhood of $(\mu_t)_{0\le t\le t_\mathrm{f}}$, and together with (\ref{eq: convergence in skorohod met}), this implies that, for all $\epsilon>0$, \begin{equation}  \mathbb{P}\left(\sup_{t\le t_\mathrm{f}} d_0(\mu^N_t, \mu_t)>\epsilon\right)\rightarrow 0\end{equation} as desired. 
\fi 

\section{\textbf{Convergence of the Stochastic Coagulant}}
\label{sec: csc}

We now turn to a preliminary version of Theorem \ref{thrm: convergence of stochastic coagulent}.
In this section, we will outline the proof, of the convergence of the stochastic coagulant $\mu^N_t$ to a solution $\mu_t$ of (\ref{eq: E+G}), locally uniformly in time. Most of the arguments are well-known for the Smoluchowski equation \cite{N99,N00}, and for brevity, we will only mention the nontrivial technical details. Throughout, we fix $\mu_0$ satisfying (A1-4.). As mentioned in the introduction, we will consider a mild generalisation of the stochastic coagulant, which will be helpful for future reference: we allow initial data of the form 
\begin{equation}
    \mu^N_0 := \frac1N \sum_{i=1}^{l_N} \delta_{\left(1,v_i, \frac12 \abs{v_i}^2\right)}
\end{equation}
where the $(v_i)_{i=1}^{l_N}$ are the initial velocities for the underlying Kac process, $l_N\le N$, and we ask that the following conditions hold.
\\B1. As $N\rightarrow \infty$, the initial measures $\mu^N_0=\frac1N\sum_{i\le N}\delta_{(1,v_i,\frac{1}{2}|v_i|^2)}$ converge in probability to $\mu_0$ in the vague topology in probability: \begin{equation}d_0(\mu^N_0, \mu_0)\rightarrow 0 \hspace{1cm}\text{in probability}. \end{equation} 
\\B2. We also have the convergence \begin{equation} \langle x, \mu^N_0\rangle \rightarrow \langle x, \mu_0\rangle \hspace{1cm} \text{in probability}\end{equation} where we recall that $x$ is the identity function on $S$, and \begin{equation} \sup_{N\ge 1} \EE\langle \varphi^2, \mu^N_0\rangle <\infty. \end{equation}
We have the following corollary of the second point of B2. For $\eta>0$, let $X^N_\eta$ be the number of particles with kinetic energy exceeding $\eta:$ \begin{equation}
    X^N_\eta= \#\left\{i\le l_N: \frac{1}{2}|v_i|^2>\eta\right\}.
\end{equation}Then, as $\eta\rightarrow \infty$, we have the convergence \begin{equation}\label{eq: b3} \sup_{N\ge 1} \mathbb{E}\left[\frac{1}{N}X^N_\eta\right]\rightarrow 0. \end{equation} \medskip \\ For example, it is straightforward to verify that these hold in the case where $l_N=N$, and $v_1,....,v_N$ are independent samples from the distribution $m$ described in Section \ref{sec: results}. However, the more general hypotheses will be useful for a `duality' argument in Section \ref{sec: finiteness of second moment}. Our result is as follows:
\begin{lem}\label{lemma: local uniform convergence of stochastic coagulent}
Suppose $\mu_0$ satisfies ({A1}-{4.}), and let $(\mu_t)_{t\ge 0}$ be the solution to (\ref{eq: E+G}) starting at $\mu_0$. Let $(\mu^N_t)_{t\ge 0}$ be stochastic coalescents with initial data $\mu^N_0$ satisfying (B1-2.). Then we have the local uniform convergence
\begin{equation}
\forall t_\mathrm{f}\ge 0 \hspace{0.5cm} \sup_{t\le t_\mathrm{f}} d_0(\mu^N_t, \mu_t)\rightarrow 0 \text{ in probability} 
\end{equation} where recall that $d_0$ is a complete metric inducing the vague topology.
\end{lem} \begin{rmk}
We will later upgrade the \emph{local} uniform convergence to \emph{full} uniform convergence in Lemma \ref{lemma: uniform convergence of coagulant}. We also remark that this does not immediately imply the convergence of the gel terms in Theorem \ref{thrm: convergence of stochastic coagulent}, as the test functions involved are  neither compactly supported nor even bounded. This will be dealt with in Sections \ref{sec: ECT}, \ref{sec: COG}, where the proofs build on this result.
\end{rmk}
\begin{proof} The proof proceeds as follows, based on the well known method of proving tightness and identifying possible limit paths. \medskip \\  Firstly, the jump rates can bounded, uniformly in time, in terms of the initial second moment $\langle \varphi^2, \mu^N_0\rangle$ and, thanks to (B2.), these are stochastically bounded $\langle \varphi^2, \mu^N_0\rangle \in \mathcal{O}_\mathrm{p}(1)$ as $N\rightarrow \infty$. As a result, it follows that for all  $t_\mathrm{f}\ge 0$, the processes $(\mu^N_t)_{0\le t\le t_\mathrm{f}}$ are tight in the Skorohod topology of $\mathbb{D}([0,t_\mathrm{f}],(\mathcal{M},d_0))$.  \medskip \\ Next, we wish to argue that if $\overline{\mu}$ is any subsequential limit point, then $\overline{\mu}$ coincides with the solution $\mu_t$ to (\ref{eq: E+G}). For this stage, we show that for certain, well-chosen $\xi>0$, the pair \begin{equation} \mu^{N,\xi}_t=\mu^N_t1_{S_\xi}; \hspace{1cm} g^{N,\xi}_t=\langle x, \mu^N_t-\mu^{N,\xi}_t\rangle  \end{equation} converge to a pair $\overline{\mu}^\xi_t=\overline{\mu}_t1_{S_\xi}, \overline{g}^\xi_t$ which solve the restricted evolution equations (\ref{eq:rE1},  \ref{eq: rE2}), started at \begin{equation} \overline{\mu}^\xi_0=\mu_01_{S_\xi}; \hspace{1cm} \overline{g}^\xi_0=\int_{x\not \in S_\xi} x\mu_0(dx). \end{equation} In order to prove this convergence, we will need a pair of regularity conditions (C1-C2.) which will be displayed below. These allow us to obtain vague convergence of $\mu^{N,\xi}_t$, despite the discontinuity of the cutoff $S_\xi$.  Moreover, one can show that these conditions are satisfied for almost all $\xi>0$.  \begin{enumerate}[label=\roman{*}).]\item Almost surely, for  almost all $t\le t_\mathrm{f}$, \begin{equation} \label{eq: reg} \tag{{C1.}}
   \overline{\mu}_t\left(\left\{x\colon \varphi(x) = \xi\right\}\right) + 
    \overline{\mu}_t \otimes \overline{\mu}_t \left(\left\{(x,y) \colon \varphi(x+y) = \xi\right\}\right) = 0;
\end{equation} \item This also holds for $t=0$. That is, almost surely, \begin{equation} \label{eq: reg2} \tag{{C2.}}
   \overline{\mu}_0\left(\left\{x\colon \varphi(x) = \xi\right\}\right) + 
    \overline{\mu}_0 \otimes \overline{\mu}_0 \left(\left\{(x,y) \colon \varphi(x+y) = \xi\right\}\right) = 0. 
\end{equation} \end{enumerate} Thanks to the construction of solutions to the global equation (\ref{eq: E+G}) in Lemma \ref{lemma: E and U}, we know that for all such $\xi$, $\overline{\mu}_t1_{S_\xi}$ coincides with $\mu_t1_{S_\xi}$. Finally, we take a limit of such $\xi \uparrow \infty$, to conclude the equality $\overline{\mu}_t=\mu_t, t\le t_\mathrm{f}$.  Since the limit process $(\mu_t)_{0\le t\le t_\mathrm{f}}$ is continuous in the vague topology $(\mathcal{M},d_0)$, it follows that we may upgrade to uniform convergence: \begin{equation} \sup_{0\le t\le t_\mathrm{f}}\hspace{0.1cm}d_0\left(\mu^N_t, \mu_t\right) \rightarrow 0 \hspace{0.5cm} \text{in probability}  \end{equation} as claimed.  
\end{proof} 


\section{\textbf{Introduction to Inhomogenous Random Graphs}}\label{sec: IRG}
As discussed in the introduction, the connection between gelation and random graphs is well-understood, and the multiplicative kernel corresponds to the well-known Erd\H{o}s-R\'eyni random graphs \cite{Flo41,ER60,A99}. However, for our purposes, not all particles are equal: particles with large velocities will undergo more collisions and exhibit quantitatively different behaviour if $\gamma>0$, and so we will need a more sophisticated model of random graphs to accommodate this inhomogeneity. In this section, we will review the theory of \emph{inhomogenous random graphs} developed in \cite{BJR07}, which will play the same r\^ole for our model that the Erd\H{o}s-R\'eyni model does for the multiplicative kernel. We will introduce the relevant definitions and state, without proof, the main results of \cite{BJR07} which we use in our work.
\begin{defn} \label{def: Generalised vertex space} A \emph{generalised vertex space} is a triple $\mathcal{V}=(\mathcal{S}, m, (\mathbf{v}_N)_{N\geq 1})$, consisting of \begin{itemize}
    \item A separable metric space $\mathcal{S}$, equipped with its Borel $\sigma$-algebra;
    \item A measure $m$ on $\mathcal{S}$, with $m(\mathcal{S}) \in (0, \infty)$; 
    \item A family of random variables $\mathbf{v}_N=(v^{(N)}_1,...,v^{(N)}_{l_N})$ taking values in $\mathcal{S}$, and of potentially random length $l_N$,  such that the empirical measures \begin{equation}
        m_N=\frac{1}{N}\sum_{k=1}^{l_N} \delta_{v^{(N)}_k} \end{equation} converge to $m$ in the weak topology $\mathcal{F}(C_b(\mathcal{S}))$, in probability.

\end{itemize} In the special case where $m(\mathcal{S})=1$ and $l_N=N$, we say that $(\mathcal{S}, m, (\mathbf{v}_N)_{N\geq 1})$ is a \emph{vertex space}. \end{defn}  
\begin{defn}
    A \emph{kernel} is a symmetric, measurable map $k: \mathcal{S}\times \mathcal{S} \rightarrow [0, \infty).$ We say that $k$ is \emph{positive} if, for all distinct $x,y \in S$, $k(x,y)>0$. 
    \end{defn} This is sufficient for our results, and is a slightly requirement stronger than the notion of irreducibility in \cite{BJR07}; this difference allows us to eliminate the possibility of exceptional sets of measure $0$. \begin{defn}[Inhomogenous random graphs]\label{definition of GN} Given a kernel $k$ and a generalised vertex space $\mathcal{V}$, we let
    $G^N$ be a random graph on $\{1, 2,..,N\}$ given as follows. Conditional on the values of $\mathbf{v}_N$, the edge $e=(ij)$ is included with probability \begin{equation}
        p_{ij}=1-\exp\left(-\frac{k(v^{(N)}_i,v^{(N)}_j)}{N} \right)
    \end{equation} and such that the presence of different edges is (conditionally) independent. We write $G^N\sim\mathcal{G}^\mathcal{V}(N,k)$. We also consider the \emph{vertex data}  $\mathbf{v}_N=(v^{(N)}_i)_{i=1}^N$ to be part of the data of $G^N_t$, so that an equality of random graphs $G=G'$ includes the equality of the vertex data.
\end{defn} To treat a general class of kernels $k$, additional regularity is required, to prevent pathologies. This is the content of the following defintion: \begin{defn}[Graphical Kernel]
    We say that a kernel $k$ on a vertex space $\mathcal{V}=(\mathcal{S}, m, (\mathbf{v}_N)_{N\geq 1})$ is \emph{graphical} if the following hold. 
    \begin{enumerate}[label=\roman{*}).]
        \item $k$ is almost everywhere continuous on $\mathcal{S}\times\mathcal{S};$
        \item $k \in L^1(\mathcal{S}\times \mathcal{S}, m \times m)$;
        \item If $G^N \sim \mathcal{G}^\mathcal{V}(N,k)$, then
        \begin{equation}
            \frac{1}{N}\mathbb{E}\left[e\left(G^N\right)\right]\rightarrow \frac{1}{2}\int_{\mathcal{S}\times \mathcal{S}} k(v,w)m(dv)m(dw)
        \end{equation} where $e(\cdot)$ denotes the number of edges of the graph.
    \end{enumerate}
\end{defn}  \begin{defn}
 Given a graph $G$, we write $\mathcal{C}_j(G): j=1, 2...$ for the connected components of $G$, in decreasing order of their sizes $\#\mathcal{C}_j(G)=C_j(G)$. If there are fewer than $j$ connected components, then $\mathcal{C}_j(G)=\emptyset$ and $C_j(G)=0$.
\end{defn}The phase transition is given in terms of the convolution operator
\begin{equation}\label{eq: T}
       (T f)(v)=\int_{\mathbb{R}^d} k(v,w)f(w)m(dw) 
   \end{equation} for functions $f$ such that the right-hand side is defined (i.e., finite or $+\infty$) for $m$-almost all $v$; for instance, if $f\ge 0$ then $Tf$ is well-defined, possibly taking the value $\infty$. We define \begin{equation} \|T\|=\sup\{\|Tf\|_{L^2(m)}: \|f\|_{L^2(m)}\le 1, f\ge 0\}. \end{equation} If $T$ defines a bounded linear map from $L^2(m)$ to itself, then $\|T\|$ is precisely its operator norm in this setting; otherwise, $\|T\|=\infty.$ It is straightforward to show that if $k\in L^2(S\times S, m\times m)$ then $T: L^2(m)\rightarrow L^2(m)$ is a Hilbert-Schmidt operator, and that $\|T\|_\text{HS}=\|k\|_{L^2(m)}<\infty$. In this case, $\|T\|$ is certainly finite, and is the operator norm of $T: L^2(m)\rightarrow L^2(m)$. The example of interest to us will fall into this case. \bigskip\\  The analysis of the random graphs uses a branching process, similar to that used in the standard analysis of Erd\H{o}s-R\'enyi graphs. Many quantities of the graph can be expressed in terms of the `survival probability' $\rho(k, v)$ when the data $v$ of the first vertex is given. To avoid the unnecessary complication of making this into a precise definition, we use the following characterisation, which is equivalent by \cite[Theorem 6.2]{BJR07}.
   \begin{lem}\label{lemma: survival function}
       Let $k$ be a positive kernel on a generalised vertex space $\mathcal{V}$, such that $k \in L^1(\mathcal{S}\times \mathcal{S}, m \times m)$, and such that, for all $v,$ \begin{equation} \label{eq: BJR 51}
           \int_S k(v,w)m(dw)<\infty.
       \end{equation} Consider the nonlinear fixed-point equation 
      \begin{equation} \label{eq: nonlinear fixed point equation} 
        \forall v \in S,\hspace{1cm}  {\rho}(v)=1-e^{-(T{\rho})(v)}
      \end{equation} where $T$ is the convolution operator (\ref{eq: T}). Then (\ref{eq: nonlinear fixed point equation}) has a maximal solution $\rho_k(v)=\rho(k;v)$; that is, for any other solution $\tilde{\rho}$, \begin{equation}
          \forall v \in S, \hspace{1cm} \tilde{\rho}(v)\leq \rho(k,v).
      \end{equation} It therefore follows that $0\leq \rho_k(v)\leq 1$ for all $v$. The maximal solution is necessarily unique, and so this uniquely defines $\rho_k.$ Moreover, we have the following dichotomy:
      \begin{enumerate}[label=\roman{*}).]
          \item If $\|T\|\leq 1$, then $\rho(k, v)=0$ for all $v$;
          \item If $\|T\|> 1$, then $\rho(k, v)>0$ for all $v$.
      \end{enumerate} This can be stated dynamically as follows. Consider the survival function `at time $t$', given by $\rho(tk,v)$, which we will write throughout as $\rho_t(v)$. Then 
      \begin{itemize}
          \item If $t\leq \|T\|^{-1}$, then $\rho_t(v)=0$ for all $v$;
          \item If $t>\|T\|^{-1}$, then $\rho_t(v)>0$ for all $v$.
      \end{itemize}
      
   \end{lem} We can now state the main results on the phase transition, given by \cite[Theorem 3.1 and Corollary 3.2]{BJR07}.
   \begin{thm}[Phase Transition]\label{thrm: RG1} Let $k$ be a graphical and positive kernel for a vertex space $\mathcal{V}$, with $0<\|T\|< \infty.$ Let $G^N\sim \mathcal{G}^\mathcal{V}(N, k)$ be random graphs on a common probability space. Then we have the convergence \begin{equation}
       \frac{1}{N}C_1(G^N_t)\rightarrow \int_{\mathcal{S}} \rho(tk, v) m(dv) \hspace{1cm} \text{in probability.}
   \end{equation}
   Therefore, if $(G^N_t)_{t\geq 0}$ is a dynamic family of random graphs $
       G^N_t \sim \mathcal{G}^\mathcal{V}(N, tk)$, then we have the following dichotomy:  \begin{enumerate}[label=\roman{*}).]
       \item If $t\leq \|T\|^{-1}$, then there is no giant component, in particular \begin{equation}
           \frac{C_1(G^N_t)}{N} \rightarrow 0
       \end{equation} in probability.
       \item If $t>\|T\|^{-1}$, then there is a giant component: there exists $c=c(t)>0$ such that
       \begin{equation}
           \mathbb{P}(C_1(G^N_t)>cN)\rightarrow 1.
       \end{equation}
   \end{enumerate}\end{thm}
   \begin{rmk} Following \cite{BJR07}, based on this dichotomy, we say that \begin{enumerate}[label=\roman{*}).]
       \item $G^N$ is \emph{subcritical} if $\|T\|<1;$
       \item $G^N$ is \emph{critical} if $\|T\|=1;$
       \item $G^N$ is \emph{supercritical} if $\|T\|>1.$
   \end{enumerate} \end{rmk} We also have the following result, which implies the uniqueness of the giant component \cite[Theorem 3.6]{BJR07}. This result considers clusters of a scale $\xi_N\ll N$, excluding the largest; we term these \emph{mesoscopic} clusters.
   \begin{thm}\label{thrm: RG2} Let $G^N\sim \mathcal{G}^\mathcal{V}(N, k)$, for a (generalised) vertex space $\mathcal{V}$ and an positive graphical kernel $k$. Let $\xi_N$ be a sequence with  \begin{equation}
       \xi_N\rightarrow \infty; \hspace{1cm} \frac{\xi_N}{N}\rightarrow 0.
   \end{equation} Then \begin{equation}
       \frac{1}{N}\sum_{j\geq 2: C_j(G^N)\geq \xi_N}C_j(G^N) \rightarrow 0
   \end{equation} in probability. \end{thm}   We will also make use of the following monotonicity and continuity properties, from \cite[Theorem 6.4]{BJR07}.
   \begin{thm}\label{thrm: continuity of rho} Let $k$ be a kernel on a vertex space $\mathcal{V}$, and let $\rho_t(\cdot)=\rho(tk,\cdot)$ be the survival function defined above. Then the map $t\mapsto \rho_t(\cdot)$ is monotonically increasing, in the sense that for all $0\leq s \leq t$ and for all $v$, $\rho_s(v)\le \rho_t(v).$ We also have the following continuity property. Let $t_n\rightarrow t$ be a monotone sequence, either increasing or decreasing. Then \begin{equation}
       \rho_{t_n}(v)\rightarrow \rho_t(v) \hspace{1cm} \text{for $m$- almost all }v, \text{ and}
   \end{equation} \begin{equation}
       \int_{\mathcal{S}}\rho_{t_n}(v)m(dv)\rightarrow \int_{\mathcal{S}}\rho_t(v)m(dv).
   \end{equation} \end{thm} The final result which we will need is a `duality' result, connecting the supercritical and subcritical behaviours. This is given by \cite[Theorem 12.1]{BJR07}.
   \begin{thm}\label{thrm: coupling supercritical and subcritical} Let $k$ be an positive graphical kernel on a generalised vertex space $\mathcal{V}$, such that $\|T\|>1$. Let $G^N \sim \mathcal{G}^\mathcal{V}(N, k)$, and form $\widetilde{G}^N$ by deleting all vertexes in the largest component $\mathcal{C}_1(G^N).$ Then, defined on the same underlying probability space, there is a generalised vertex space $\widehat{\mathcal{V}}=(\mathcal{S}, \widehat{m}, (\mathbf{w}_N)_{N\geq 1})$ with \begin{equation}
       \widehat{m}(dv)=(1-\rho(k;v))m(dv)
   \end{equation} and such that $\mathbf{w}_N$ is an enumeration of those $v_i$ not belonging to the component $\mathcal{C}_1(G^N)$, and a random graph $\widehat{G}^N \sim \mathcal{G}^{\widehat{\mathcal{V}}}(N,k)$ such that \begin{equation}
       \mathbb{P}(\widetilde{G}^N=\widehat{G}^N)\rightarrow 1.
   \end{equation}  Furthermore, if $k\in L^2(\mathcal{S}\times \mathcal{S}, m\times m)$, then $\widehat{G}^N$ is subcritical.\end{thm} We emphasise here that we have defined the equality $\widetilde{G}^N=\widehat{G}^N$ includes equality of the velocities $v_i$ associated to each vertex; this follows from the construction in \cite{BJR07}, since the velocities $\mathbf{w}_N$ associated to $\widehat{G}^N$ are exactly those $v_i$ not belonging to the giant component. This generalises the standard `duality result' of Bollob\'as \cite{BB84} for Erd\H{o}s-R\'enyi graphs. 
\section{\textbf{Coupling of the Stochastic Coagulant to Random Graphs}} \label{sec: coupling_to_random_graph}
In this section, we will show that the stochastic coagulant defined in \S\ref{s:jump_procs} may be coupled to a \emph{dynamic} version of the random graphs $\mathcal{G}^\mathcal{V}(N,tk)$ discussed above. This allows us to apply the results quoted above to analyse the stochastic coagulant process and the limit equation.
\begin{defn}\label{def: GNT}[Dynamic Inhomogenous Random Graphs] Fix a measure $\mu_0$ satisfying (A1-4.). Let $\mathbf{v}_N=(v_i, i=1,2,...,l_N)$ be random velocities such that the empricial measures \begin{equation} \mu^N_0=\frac{1}{N}\sum_{i\le l_N} \delta_{(1,v_i,\frac{1}{2}|v_i|^2)}\end{equation}  satisfy the conditions (B1-2.), and sample $\tau_e \sim \text{Exponential}(1)$, independently of each other, and of $\mathbf{v}_N$. We define the kernel \begin{equation} k(v,w)=2\overline{K}\left(\left\{1,v, \frac{1}{2}\norm{v}^2\right\},\left\{1,w, \frac{1}{2}\norm{w}^2\right\}\right) \end{equation} where the right-hand side is the interaction kernel defined in (\ref{eq: smoluchowski kernel}). We form the random graphs $(G^N_t)_{t \ge 0}$ on $\{1,2,...,l_N\}$ by including the edge $e=(ij)$ if \begin{equation}
    t\ge \frac{N \tau_e}{k(v_i,v_j)}.
\end{equation} We emphasise that the $v_i$ do \emph{not} change during the dynamics. \end{defn} This has the following immediate consequences. Firstly, if we define $ \mathcal{V}=(\mathbb{R}^d, m, \mathbf{v}_N)$, then $\mathcal{V}$ is a vertex space in the sense of Definition \ref{def: Generalised vertex space} and, for all times $t$, $G^N_t$ is an instance of the inhomogenous random graph $\mathcal{G}^{\mathcal{V}}(N, tk)$ defined in Definition \ref{definition of GN}. Moreover, the process $(G^N_t)_{t\ge 0}$ is increasing, and is a Markov process, by the memoryless property of the exponential variables $\tau_e$.  We write $T$ for the convolution operator and $\|T\|$ for the associated operator norm, as defined following (\ref{eq: T}). We write also $t_\mathrm{c}$ for the critical time for the phase transition in Theorem \ref{thrm: RG1}, which is given by $\|T\|^{-1}$. \medskip \\ For a cluster $\mathcal{C}$ of the graph $G^N_t$, we will write $M(\mathcal{C}), P(\mathcal{C}), E(\mathcal{C})$ to denote the unnormalised mass, momentum and energy \begin{equation}\label{eq: cluster quantities}
   M(\mathcal{C})=\#\mathcal{C}; \hspace{1cm} P(\mathcal{C})=\sum_{i\in \mathcal{C}} v_i;\hspace{1cm} E(\mathcal{C})=\sum_{i\in \mathcal{C}} \frac{1}{2}|v_i|^2.
\end{equation} We write $\delta(\mathcal{C})$ for the point mass in $S$ \begin{equation}
    \delta(\mathcal{C})=\delta_{(M(\mathcal{C}),P(\mathcal{C}),E(\mathcal{C}))}
\end{equation} and $\mu^N(G^N_t)$ for the normalised empirical measure \begin{equation} \mu^N(G^N_t)=\frac{1}{N}\sum_\text{Clusters} \delta(\mathcal{C})\end{equation} where the sum is over all clusters $\mathcal{C}$ of $G^N_t$. This is connected to the stochastic coagulants as follows:
\begin{lem}[Coupling of Random Graphs and Stochastic Coagulants]\label{lemma: coupling} Let $(G^N_t)$ be the random graph process described in Definition \ref{def: GNT}. Then the processes \begin{equation}
    \mu^N_t=\mu^N(G^N_t)
\end{equation} are stochastic coagulants for the kernel $K$.
\end{lem}

We also note that, since the rates are bounded, the stochastic coagulant has uniqueness in law. As a consequence, if we wish to prove any property of a general stochastic coagulant $\mu^N_t$, we may assume that it is given by $\mu^N_t=\mu^N(G^N_t)$, and appeal to an analysis of the random graphs.
\begin{proof}[Sketch of proof of Lemma~\ref{lemma: coupling}]
One may verify that the two processes undergo the same transitions at the same rates; this essentially follows from the calculation (\ref{e:cluster-merge-rate}). We will give here an alternative proof, which we feel offers more insight. \medskip \\ From the calculation (\ref{e:cluster-merge-rate}), the rates of the stochastic coagulants do not depend on the collision kernel $B$ beyond the total rates $B(v,\mathbb{S}^{d-1})$. In particular, if $B, \tilde{B}$ are two kernels with the same total rates, then given random initial velocities $(v_i(0))_{i=1}^N$, one may couple the Kac processes $(v_i(t): t\ge 0)_{i=1}^N, (\tilde{v}_i(t): t\ge 0)_{i=1}^N$ so that the two processes have the same coagulation structure, and the initial velocities coincide. We also note that the total mass, momentum and energy for each cluster are the same in each model, thanks to the conservation properties of the kernel. Therefore, the processes of empirical measures $\mu^N_t$ coincide for the two models. \medskip \\ Now, we choose $\tilde{B}$ to be the degenerate kernel such that the outgoing velocities are the same as the incoming velocities: in particular, each $v_i(t)$ is constant in time. For this case, we construct a graph on $\{1, ...,N\}$ by including $(i,j)$ if particles $i, j$ have collided before, or at, time $t$, according to the Kac process for $\tilde{B}$. It is immediate that the resulting graph $G^N_t$ is distributed according to the distribution $\mathcal{G}^\mathcal{V}(N,tk)$ constructed above, and by construction, \begin{equation} \mu^N_t=\mu^N(G^N_t). \end{equation} Since this is true for a particular realisation of the two processes, we have equality in law, as claimed.\end{proof} Combining this with the approximation result Lemma \ref{lemma: local uniform convergence of stochastic coagulent} for the stochastic coagulant, we may connect the random graph process to the limit equation as follows. \begin{lem}[Convergence of the Random Graphs]\label{lemma: convergence of random graphs} Let $(G^N_t)_{t\ge 0}$ be the random graph processes constructed above, such that the initial velocities $\mathbf{v}_N=(v_1,...v_{l_N})$ satisfy (B1-2.). Let $\mu_0$ be the corresponding initial measure on $S$ under ({A2}.), and assume that (A1-4.) hold. Let $(\mu_t)_{t\ge 0}$ be the solution to the Smoluchowski Equation (\ref{eq: E+G}) starting at $\mu_0$; then we have the local uniform convergence \begin{equation}\sup_{s\le t} \hspace{0.1cm}d_0(\mu^N(G^N_t), \mu_t) \rightarrow 0 \end{equation} in probability, for all $t<\infty,$  where we recall that $d_0$ is a metric for the vague topology $\mathcal{F}(\mathcal{M}_{\le 1}(S), C_c(S))$. We emphasise here that we do not require exactly $N$ particles, or that $\mu_0$ is a probability measure. \end{lem} 

We can also compute the critical time associated to $G^N_t$ explicitly:
\begin{lem}[Computation of critical time]\label{lemma: computation of tcrit} Suppose condition (A3.) holds for $m$, and let $G^N_t$ be the random graphs defined above. Then $T$ is a bounded linear map from $L^2(m)$ to itself, and the critical time for the graph phase transition is
\begin{equation}\label{eq: closed form for tc}
        t_\text{c}= \left(\kappa\sigma_0(m) +2\gamma\sigma_2(m) + \sqrt{(\kappa\sigma_0(m)+2\gamma\sigma_2(m))^2+4\gamma^2(\sigma_0(m)\sigma_4(m)-\sigma_2^2(m))}\right)^{-1}
   \end{equation} where we recall the notation $\sigma_k(m)$ for the $k^\text{th}$ moment $\langle |v|^k, m\rangle.$
\end{lem}
\begin{rmk} We remark that, in the case where $m$ is a probability measure, this is the closed form claimed for $t_\mathrm{g}$ in Theorem \ref{thrm: Smoluchowski equation}. However, we have not yet established that $t_\mathrm{c}=t_\mathrm{g}$; this is the content of Lemma \ref{lemma: connect critical times}. \end{rmk}
\begin{proof}[Proof of Lemma~\ref{lemma: computation of tcrit}]
Firstly, by (A3.), it is easy to see that $k\in L^2(\mathbb{R}^d\times \mathbb{R}^d,m\times m)$. Therefore, as remarked in Section \ref{sec: IRG}, $\|T\|$ is precisely the operator norm of $T: L^2(m)\rightarrow L^2(m)$. \medskip \\ Construct a basis $\{e_n\}_{n\geq 1}$ of $L^2(m)$, such that \begin{equation}
       e_i(v)=v_i, \hspace{0.2cm}i=1,2,..d;
   \end{equation}
   \begin{equation}
       e_{d+1}(v)=1, \hspace{1cm}e_{d+2}(v)=|v|^2
   \end{equation} and such that, for $n\geq d+3$, $e_n$ is orthogonal to $E=\text{Span}(e_1,...,e_{d+2})$.
   By expanding the quadratic term $|v-w|^2$, we see that, for all $f\in L^2(m),$ \begin{equation} 
       (Tf)(v)  =2\left[\kappa\langle f, e_{d+1}\rangle e_{d+1}(v)
       +\gamma \langle f, e_{d+1}\rangle e_{d+2}(v)
       -2\gamma \sum_{i=1}^d \langle f, e_i\rangle e_i(v)
       +\gamma \langle f, e_{d+2}\rangle e_{d+1}(v)\right]
   \end{equation} 
 where $\langle\cdot,\cdot\rangle$ denotes the $L^2(m)$ inner product. Therefore, $T$ maps into the subspace $E$, and $T(e_n)=0$ if $n\geq d+3.$ It follows from the definition (\ref{eq: T}) that $T$ is self-adjoint, and so the operator norm $\|T\|$ is given by the largest modulus of an eigenvalue of $T|_E.$ We write $\sigma_{2,i}(m)$ for the second moment $\sigma_{2,i}(m)=\int_{\mathbb{R}^d} v_i^2 m(dv)$. In this notation, and with respect to the basis $\{e_1,...,e_d, e_{d+1}, e_{d+2}\}$,  $T|_E$ has the matrix representation  \begin{equation}
       \left[T|_E\right]=\begin{bmatrix}
    -4\gamma\sigma_{2,1}(m) & 0  & \dots  &0 & 0 &0 \\
    0&-4\gamma\sigma_{2,2}(m)   & \dots  &0 & 0 &0 \\
    
    \vdots & \vdots  & \ddots & \vdots &\vdots & \vdots \\
    0 & 0&  \dots & -4\gamma \sigma_{2,d}(m) & 0 &0\\
    0&0   & \dots &0 & 2\kappa\sigma_0(m)+2\gamma\sigma_2(m) & 2\kappa\sigma_2(m)+2\gamma \sigma_4(m) \\ 
    0& 0 & \dots & 0& 2\gamma \sigma_0(m) & 2\gamma \sigma_2(m)
\end{bmatrix}.
   \end{equation} Therefore, the eigenvalues are $-4\gamma \sigma_{2,i}(m), i=1,..,d$, and the two roots  $\lambda_\pm$ to a quadratic equation which simplifies to \begin{equation}
       \lambda^2-2\lambda(\kappa\sigma_0(m)+2\gamma\sigma_2(m))+4\gamma^2(\sigma_2^2(m)-\sigma_0(m)\sigma_4(m))=0,
   \end{equation} 
   where $\sigma_0(m), \sigma_2(m), \sigma_4(m)$ are as defined in Assumption (A3.).
   The two roots are \begin{equation}
       \lambda_\pm=\kappa\sigma_0(m) +2\gamma\sigma_2(m) \pm \sqrt{(\kappa\sigma_0(m)+2\gamma\sigma_2(m))^2+4\gamma^2(\sigma_0(m)\sigma_4(m)-\sigma_2^2(m))}.
   \end{equation} It is straightforward to check that the largest eigenvalue in modulus is $\lambda_+$, so we find \begin{equation}
       \|T\|=\lambda_+=\kappa\sigma_0(m) +2\gamma\sigma_2(m) + \sqrt{(\kappa\sigma_0(m)+2\gamma\sigma_2(m))^2+4\gamma^2(\sigma_0(m)\sigma_4(m)-\sigma_2^2(m))}.
   \end{equation} This gives the critical time as \begin{equation}
       t_\text{c}= \left(\kappa\sigma_0(m) +2\gamma\sigma_2(m) + \sqrt{(\kappa\sigma_0(m)+2\gamma\sigma_2(m))^2+4\gamma^2(\sigma_0(m)\sigma_4(m)-\sigma_2^2(m))}\right)^{-1}
   \end{equation} as claimed.\end{proof}
It is straightforward to show that (\ref{eq: BJR 51}) holds, and we may there define $\rho_t$ as the survival function from Lemma \ref{lemma: survival function}, for kernel $tk$. We note, for future use, the following properties where $k$ is the kernel given above.
\begin{lem}\label{lemma: form of rho-t}
    The survival function $\rho_t(v)=\rho(tk,v)$ takes the form \begin{equation}
        \rho_t(v)=1-e^{-a_t-b_t|v|^2}
    \end{equation} for some $a_t, b_t \ge 0$. Moreover, the functions $t\mapsto a_t$, $t\mapsto b_t$ are continuous.
\end{lem} This proves the first two assertions of item 4 of Theorem \ref{thrm: Smoluchowski equation}.
\begin{proof} Using the symmetry $k(-v,-w)=k(v,w)$ and hypothesis ({A1}.), it is simple to verify that $\tilde{\rho}(v):=\rho_t(-v)$ also satisfies the fixed point equation (\ref{eq: nonlinear fixed point equation}). By maximality of $\rho_t$, we must have $\rho_t(-v)\le \rho_t(v)$ for all $v\in \mathbb{R}^d$, which implies that $\rho_t$ is an even function of $v\in\mathbb{R}^d$. \medskip \\ Using the identification of the range of $T$ as in Lemma \ref{lemma: computation of tcrit}, we see that there exist $c^i_t: 1\le i\le d+2$ such that \begin{equation}
    (T\rho_t)(v)=\sum_{i=1}^d c^i_t v_i+ c^{d+1}_t +c^{d+2}_t|v|^2
\end{equation}and therefore, from the equation (\ref{eq: nonlinear fixed point equation}) defining $\rho$, \begin{equation}
    \rho_t(v)=1-\exp\left(\sum_{i=1}^d c^i_t v_i+ c^{d+1}_t +c^{d+2}_t|v|^2\right).
\end{equation} Since $\rho_t$ is even, the linear term $\sum_{i\le d} c^i_t v_i$ must identically vanish, which gives the claimed representation of $\rho_t$ by relabelling $a_t=c^{d+1}_t, b_t=c^{d+2}_t$. Since $\rho_t(v)\le 1$ everywhere, it follows that \begin{equation}
   \forall v \in \mathbb{R}^d\hspace{1cm} a_t+b_t|v|^2 \ge 0
\end{equation}which is only possible if $a_t, b_t\ge 0.$ The continuity follows immediately from Theorem \ref{thrm: continuity of rho}. \end{proof} 

\section{\textbf{Equality of the Critical Times}} \label{sec: ECT}

In this section, we will prove that the critical time $t_\mathrm{c}$ or the graph process, introduced in Section \ref{sec: coupling_to_random_graph}, coincides with the gelation time for the limiting equation, defined in Section \ref{sec:SE} as the time at which mass and energy begin to escape to infinity. 
\begin{lem}\label{lemma: connect critical times} Let $\mu_0$ be a measure on $S$ satisfying ({A1-4}.). Let $(\mu_t)_{t\ge 0}$ be the solution to (\ref{eq: E+G}) starting at $\mu_0$, with associated mass $M_t$ and energy $E_t$ of the gel; recall that $t_\mathrm{g}$ is defined by \begin{equation}
    t_\mathrm{g}:=\inf\{t\ge 0: \left<\varphi, \mu_t\right> < \left<\varphi, \mu_0\right>\} 
    = \inf\{t\ge 0: M_t+E_t>0\}.
\end{equation} Let $(G^N_t)$ be the random graph processes constructed above, and suppose that (B1-2.) hold for $G^N_t, \mu_0$. Then the critical time $t_\mathrm{c}$ for the graph transition process coincides with the gelation time $t_\mathrm{g}$: $ t_\mathrm{c}=t_\mathrm{g}. $ \end{lem} The following is a straightforward corollary. \begin{cor}\label{corr: actual expression for tg} Let $\mu_0$ be a sub-probability measure on $S$ satisfying (A1-4.), for a base measure $m$. Let $(\mu_t)_{t\ge 0}$ be the solution to (\ref{eq: E+G}) starting at $\mu_0$, and $t_\mathrm{g}$ the associated gelation time. Then $t_\mathrm{g}$ is given explicitly by the closed-form expression (\ref{eq: closed form for tc}). In the case where $m$ is a probability measure, this reduces to the expression (\ref{eq: closed form for tg}).   \end{cor} \begin{proof}[Proof of Corollary \ref{corr: actual expression for tg}] Let $l_N=\lceil N m(\mathbb{R}^d)\rceil$, and form $\mathbf{v}_N$ by sampling $l_N$ velocities independently from $m(\cdot)/m(\mathbb{R}^d).$ It is immediate that the resulting vertex space $\mathcal{V}=(\mathbb{R}^d, (\mathbf{v}_N)_{N\ge 1}, m)$ satisfies (B1-2.) for the measure $\mu_0$, and the critical time $t_\mathrm{c}$ of the associated graphs $G^N_t$ is given by the claimed expression (\ref{eq: closed form for tc}). From the previous lemma, it now follows that the gelation time $t_\mathrm{g}=t_\mathrm{c}$, which proves the claimed result. \end{proof} 
The proof of Lemma \ref{lemma: connect critical times} is based on the following weak version of the convergence of the gel in Theorem \ref{thrm: convergence of stochastic coagulent}, and may be taken as preliminary reading for the more detailed arguments in Section \ref{sec: COG}.
\begin{lem} \label{lemma: WCOG} Let $(\mu_t)_{t\ge 0}, M_t, E_t$ and $G^N_t$ be as above. Fix $t>0$, and write $M^N_t, E^N_t$ for the scaled mass and energy of the largest component of $G^N_t$, as in Theorem \ref{thrm: convergence of stochastic coagulent}: \begin{equation} M^N_t:=\frac{1}{N}C_1(G^N_t); \hspace{1cm} E^N_t=\frac{1}{N}E(\mathcal{C}_1(G^N_t)). \end{equation} Then $M^N_t \rightarrow M_t$ and $E^N_t\rightarrow E_t$ in probability.
\end{lem}
We first show that Lemma \ref{lemma: WCOG} implies Lemma \ref{lemma: connect critical times}; the remainder of this section is dedicated to the proof of Lemma \ref{lemma: WCOG}.
\begin{proof}[Proof of Lemma \ref{lemma: connect critical times}] Throughout, let $(v_i)_{i=1}^N$ be the velocities associated to the nodes of the random graph process, which we recall are independent of time. \medskip \\Firstly, suppose for a contradiction that $t_\mathrm{g}< t_\mathrm{c}$. Then $M_{t_\mathrm{c}}+E_{t_\mathrm{c}}>0$, but we bound \begin{equation}\label{eq: use of CS 0} M^N_{t_\mathrm{c}}+E^N_{t_\mathrm{c}} \le \left(\frac{1}{N}C_1(G^N_{t_\mathrm{c}})\right)^\frac{1}{2}\left(\frac{1}{N}\sum_{i=1}^N(1+\frac{1}{2}|v_i|^2)^2\right)^\frac{1}{2}.\end{equation} The first term converges to $0$ in probability, by definition of the phase transition, and the second term is bounded in $L^2$ by hypothesis (B2.). This implies that $M^N_{t_\mathrm{c}}+E^N_{t_\mathrm{c}}\rightarrow 0$ in probability, which contradicts Lemma \ref{lemma: WCOG}; we must therefore have that $t_\mathrm{g}\ge t_\mathrm{c}.$ \medskip \\ Conversely, if $t< t_\mathrm{g}$, then $M_t=0$ by definition. Now, the convergence \begin{equation} \frac{1}{N}C_1(G^N_t) \le M^N_t\rightarrow 0\end{equation} in probability implies that the largest cluster is of the order $o_\mathrm{p}(N)$, which is only possible if $t\le t_\mathrm{c}$. Since $t<t_\mathrm{g}$ was arbitrary, we must have $t_\mathrm{g}\le t_\mathrm{c}$, and together with the previous argument, we have shown that $t_\mathrm{g}=t_\mathrm{c}$ as claimed.  \end{proof} 
\subsection{\textbf{Preparatory Lemmas}} The proof of Lemma \ref{lemma: WCOG} is based on the following argument. We know, from Theorem \ref{thrm: RG2}, that any `mesoscopic' clusters contain negligable mass; thanks to the integrability assumption (A3.), the same is true for the energy. Therefore, almost all mass and energy either belongs to the `microscopic' scale, whose convergence is quantified by Lemma \ref{lemma: local uniform convergence of stochastic coagulent}, or the giant component, whose convergence is the subject of interest here. Therefore, with a suitable approximation argument, the claimed convergence will follow from the quoted results.  \medskip\\ We begin with some preparatory lemmas; throughout, we will assume the notation of Lemma \ref{lemma: WCOG}. Firstly, we  bound the size of the largest cluster \emph{below}, even in the cases where there is no giant component. 
\begin{lem}\label{lemma: lower bound on largest cluster} Fix $t>0$ and $r\in \mathbb{N}$, and let $G^N_t$ be the random graph process described above. Then \begin{equation}
    \mathbb{P}(C_1(G^N_t)\leq r)\rightarrow 0.
\end{equation} \end{lem}
\begin{proof} We remark that, due to (A1, A4.), $m$ is not a point mass. Therefore, we can find open sets $V, W$ of positive $m$- measure such that $\inf_{v \in V, w \in W} |v-w|>0$; define \begin{equation} I_1=\{i \in \{1,2,...,N\}: v_i \in V\}; \hspace{1cm} N_1=\#I_1
\end{equation} and similarly, $I_2, N_2$, with $V$ replaced by $W$. By weak convergence (B1.) and openness of $V$ and $W$, $N_1\geq cN$ and $N_2\geq cN$ with high probability, for some constant $c>0$. Now, for all $i\in I_1, j\in I_2$ the edge $e=(ij)$ is present in $G^N_t$ with probability \begin{equation}
    \mathbb{P}(e \text{ present in }G^N_t) \geq \left(1-\exp\left(-\frac{\epsilon}{N}\right)\right) \geq \frac{\delta}{N}
\end{equation} for all $N$ large enough, for some positive $\epsilon, \delta>0$, possibly depending on $t,\kappa,\gamma, V,W$. Therefore, we can construct a random bipartite graph $H^N$, with vertex sets size $cN$ and independent edges occurring with probability $\frac{\delta}{N}$, such that $H^N \subset G^N_t$ with high probability. It is straightforward to see that the maximum degree $\Delta(H^N)\rightarrow \infty$ in probability, which implies the claim.  \end{proof}

For the proof of of Lemma \ref{lemma: connect critical times}, and later Theorem \ref{thrm: convergence of stochastic coagulent}, we will wish to study the convergence of integrals $\langle \varphi f, \mu^N_t\rangle$, for bounded continuous functions $f$ with non-compact support. However, the convergence result Lemma \ref{lemma: local uniform convergence of stochastic coagulent} only gives us information when the support of $f$ is compact. Our second preparatory lemma allows us to approximate the integrals $\langle \varphi f, \mu^N_t\rangle$ for functions \emph{whose support is bounded in the $\pi_n$-direction}.

\iffalse \begin{lem}\label{lemma: etar} There exists a sequence $\eta_r \rightarrow \infty$ such that the sequence \begin{equation}
    \beta(r)=\sup_{N\geq 1} \hspace{0.1cm} \mathbb{E}\left[\sup_{t\geq 0}\hspace{0.1cm}\langle \varphi 1[\pi_e(x)>\eta_r, \pi_n(x)< r], \mu^N_t\rangle \right]
\end{equation} converges to $0$ as $r \rightarrow \infty.$ \end{lem} \begin{proof} Let $\mu^N_t$ be a stochastic coagulant coupled to a random graphs process $G^N_t$. Using Cauchy-Schwarz, \begin{equation} \begin{split}
  & \langle \pi_e 1[\pi_e(x)>\eta_r, \pi_n(x)< r], \mu^N_t\rangle  = \frac{1}{N} \sum_{\substack{j\geq 1: C_j(G^N_t)\le r \\ E(\mathcal{C}_j(G^N_t))>\eta_r}} \hspace{0.2cm} E(\mathcal{C}_j(G^N_t))  %\\ & \leq \left(\frac{1}{N} \sum_{\substack{j\geq 1: C_j(G^N_t)\le r \\[1ex] E(\mathcal{C}_j(G^N_t))>\eta_r}} C_j(G^N_t)\right)^\frac{1}{2}\left(\frac{1}{N}\sum_{i=1}^N \frac{1}{4}|v_i|^4\right)^\frac{1}{2}  
   \\ & \hspace{3cm} \leq \langle \pi_n 1[\pi_e(x)>\eta_r, \pi_n(x)< r], \mu^N_t\rangle^\frac{1}{2}\left(\frac{1}{N}\sum_{i=1}^N \frac{1}{4}|v_i|^4\right)^\frac{1}{2}.
\end{split} \end{equation}
As remarked in Definition \ref{def: GNT}, the velocities $v_i$ associated with the graph nodes are constant in time, so the second factor is constant and bounded in $L^2$ by ({A3.}). Therefore, it is sufficient to prove the claim with $\varphi$ replaced by $\pi_n.$ \medskip \\ Choose $\eta_r\rightarrow \infty$ quickly enough that \begin{equation} \label{eq: choice of etar}
    r\hspace{0.1cm}m\left(\frac{1}{2}|v|^2>\frac{\eta_r}{r}\right)\rightarrow 0.
\end{equation}We observe that \begin{equation}
    \sup_{t\geq 0} \hspace{0.1cm} \langle \pi_n 1[\pi_e(x)>\eta_r, \pi_n(x)< r], \mu^N_t\rangle \leq \frac{1}{N}\sum_{i=1}^N 1_{A_i}  
\end{equation} where \begin{equation}
    A_i=\left\{\exists \hspace{0.1cm} t: C(i, G^N_t)< r, \hspace{0.2cm} E\left(\mathcal{C}(i, G^N_t)\right)> \eta_r \right\}.
\end{equation}
Let
\begin{equation}
    \mathcal{X}_{N,r}=\left\{i \in \{1,2,..,N\}: \hspace{0.1cm} \frac{1}{2}|v_i|^2>\frac{\eta_r}{r}\right\}; \hspace{1cm}X_{N,r}=\#\mathcal{X}_{N,r}.
\end{equation} For $i\in \mathcal{X}_{N,r}$, let $T_{i}$ be the first time that the cluster $\mathcal{C}(i, G^N_t)$ is of size $r$, and define \begin{equation}
    \mathcal{C}^*(i,r,N)=\mathcal{C}(i, G^N_{T_i});\hspace{1cm} \mathcal{C}^*(r,N)=\bigcup_{i\in \mathcal{X}_{N,r}} \mathcal{C}^*(i,r,N).
\end{equation} Then we bound $\#\mathcal{C}^*(r,N)\le r X_{N,R}$.  \medskip \\ With this notation, we can prove the claimed convergence. Fix $i$, and suppose that, for some time $t\geq 0$, $C(i, G^N_t)< r$ and $E(\mathcal{C}(i, G^N_t))> \eta_r.$ Then there exists some $j \in \mathcal{C}(i, G^N_t)$ with $\frac{1}{2}|v_j|^2 >\frac{\eta_r}{r}$, and so $j\in \mathcal{X}_{N,r}$. Since $C(j, C^N_t)=C(i, G^N_t)< r$, we must also have $t\le T_j$ and so $i\in\mathcal{C}(j,G^N_t) \subset \mathcal{C}^*(r,N).$ Therefore, \begin{equation}
    \sup_{t\geq 0} \hspace{0.1cm} \langle \pi_n 1[\pi_e(x)>\eta_r, \pi_n(x)\le r], \mu^N_t\rangle \le \frac{1}{N}\sum_{i=1}^N 1_{A_i} \le \frac{1}{N}\#\mathcal{C}(r,N) \le \frac{r}{N}X_{N,r}.
\end{equation}
Taking expectations, and using that the initial velocities are sampled independently from $m$,
\begin{equation} \begin{split}
    \mathbb{E}\left[\sup_{t\geq 0} \hspace{0.1cm} \langle \pi_n 1[\pi_e(x)>\eta_r, \pi_n(x)\le r], \mu^N_t\rangle\right] \le \frac{r}{N} \mathbb{E} X_{N,r}  = r\hspace{0.1cm}m\left(\frac{1}{2}|v|^2>\frac{\eta_r}{r}\right) \rightarrow 0. 
\end{split} \end{equation}
\end{proof}

\fi
\begin{lem}[A step towards uniform integrability]\label{lemma: STUI}
Suppose that $\mu^N$ are stochastic coagulants satisfying (B1-2.). Then, for every positive $r \in \NN$,
\begin{equation}
    \beta(r,\eta):= \sup_{N\geq 1}
      \mathbb{E}\left[\sup_{t\geq 0}\bigg\langle \varphi 1[\pi_e(x)>\eta, \pi_n(x)\leq r], \mu^N_t\bigg\rangle \right]
    \rightarrow 0\hspace{1cm} \text{as }\eta\rightarrow \infty.
\end{equation}
\end{lem}
\begin{proof}
Let $\mu^N_t$ be a stochastic coagulant coupled to a random graphs process $G^N_t$. Using Cauchy-Schwarz we note firstly that
\begin{equation} \begin{split}
 &\sup_{t\geq 0}\hspace{0.1cm}\bigg\langle \varphi 1[\pi_e(x)>\eta, \pi_n(x)\leq r], \mu^N_t\bigg\rangle
  \\[1ex] & \hspace{0.5cm}
\leq
  \left(\frac1N\sum_{j=1}^N \sum_{i \in \mathcal{C}_j(G_t^N)}\frac14 |v_i|^4 \right)^{\frac12}
  \left(\sup_{t\geq 0}\frac1N \sum_{j=1}^N    1[E(\mathcal{C}_j(G_t^N)) > \eta, C_j(G_t^N)\leq r]\right)^{\frac12} \\[1ex] & \hspace{0.5cm}
=
  \left(\frac1N\sum_{i=1}^N \frac14 |v_i|^4 \right)^{\frac12}
  \left(\sup_{t\geq 0}\hspace{0.1cm}\bigg\langle \pi_n 1[\pi_e(x)>\eta, \pi_n(x)\leq r], \mu^N_t\bigg\rangle\right)^{\frac12}.
\end{split}\end{equation}
As remarked in Definition \ref{def: GNT}, the velocities $v_i$ associated with the graph nodes are constant in time, so the first factor is independent of $t\ge 0$, and is bounded in $L^2$ by the second assertion of (B2.).
Therefore, it is sufficient to prove the claim with $\varphi$ replaced by $\pi_n$.
\bigskip \\  Recall from the discussion below (B2.) that $X^N_{\eta/r}$ is given by \begin{equation} X^N_{\eta/r}=\#\left\{i\le N: \frac{1}{2}|v_i|^2>\frac{\eta}{r}\right\}.\end{equation} Fix $t\ge 0$. Now, if a cluster at time $t$ contains at most $r$ particles, but has total kinetic energy greater than $\eta$, then it must contain a particle of velocity $v$ with $\frac{1}{2}|v|^2>\frac{\eta}{r}$. Since each such cluster contains at most $r$ particles, the contribution of these clusters is at most \begin{equation}\bigg \langle \pi_n 1[\pi_e(x)>\eta, \pi_n(x)\le r], \hspace{0.005cm}\mu^N_t\bigg \rangle \le \frac{r}{N} \hspace{0.005cm}X^N_{\eta/r}.   \end{equation}  
Now, the right-hand side is independent of $t \ge 0$, and so is an upper bound when we maximise over $t.$ Taking expectations, we obtain
\begin{equation}
\EE\left[\sup_{t\ge 0} \bigg\langle \pi_n 1[\pi_e(x)>\eta, \pi_n(x)\leq r], \mu^N_t\bigg\rangle \right]
\leq
\frac{r}{N} \mathbb{E}\left[X^N_{\eta/r}\right].
\end{equation}
From (\ref{eq: b3}), this vanishes as $\eta\rightarrow \infty$, uniformly in $N$.
\end{proof}


\subsection{\textbf{Proof of Lemma \ref{lemma: WCOG}}} Using the two preparatory lemmas developed above, we now prove Lemma \ref{lemma: WCOG}. 
\begin{proof}[Proof of Lemma \ref{lemma: WCOG}]   Throughout, we let $(\mu^N_t)_{t\geq 0}$ be a stochastic coagulant coupled to a random graph process $(G^N_t)_{t\geq 0}$, as described in Section \ref{sec: coupling_to_random_graph}. We write $(v_i)_{i=1}^N$ for the velocities associated to the graph vertices. The case $t=0$ is trivial, and can be omitted. We deal first with the mass term $M^N_t$ and show later how this may be modified for the energy $E^N_t.$ \medskip \\ Fix $t> 0$, and let $\xi_N$ be a sequence, to be constructed later, such that \begin{equation}\label{eq: choice of xiN for WCOG}
       \xi_N\rightarrow \infty; \hspace{1cm} \frac{\xi_N}{N}\rightarrow 0; \hspace{1cm}\mathbb{P}(C_1(G^N_t)\geq \xi_N)\rightarrow 1.
   \end{equation}  We now construct `bump functions' as follows.  Let $\eta_r\rightarrow \infty$ be a sequence growing sufficiently fast that, in the notation of Lemma \ref{lemma: STUI}, $\beta(r, \eta_r)\rightarrow 0$, and let
 \begin{equation}
       S_{(r)} := \{x \in S: \pi_n(x)< r, |\pi_p(x)|\leq \sqrt{2r\eta_r}, \pi_e(x)\leq \eta_r\}.
 \end{equation}
 Let $\widetilde{g}_r$ be the indicator $\widetilde{g}_r=1[\pi_n(x)< r]$, and construct a continuous, compactly supported function $\widetilde{f}_r$ such that
 \begin{equation}
      0\leq \widetilde{f}_r\leq 1;\hspace{1cm} \widetilde{f}_r=1 \hspace{0.1cm} \text{ on } S_{(r)};\hspace{1cm} \widetilde{f}_r(x)=0 \hspace{0.1cm} \text{ if } \pi_n(x)\ge r.
 \end{equation}
 The final condition is compatible with continuity because $\pi_n:S\rightarrow \mathbb{N}$ is continuous and integer valued. We define $f_N=\widetilde{f}_{\xi_N}$ and $g_N=\widetilde{g}_{\xi_N}$.  We now decompose the difference $M^N_t-M_t:$ \begin{equation}\label{eq: decomposition of erorr in WCOG}\begin{split} M^N_t-M_t &= \underbrace{(\langle \pi_n, \mu_t\rangle -\langle \pi_n f_N, \mu_t\rangle)}_{:=\mathcal{T}^1_N} + \underbrace{\langle \pi_n f_N, \mu_t-\mu^N_t\rangle}_{:=\mathcal{T}^2_N} \\[1ex]&\hspace{2cm}+ \underbrace{\langle \pi_n (f_N-g_N), \mu^N_t\rangle}_{:=\mathcal{T}^3_N} +\underbrace{(
   \langle \pi_n g_N, \mu^N_t\rangle - (\langle \pi_n, \mu^N_0\rangle-M^N_t)}_{:=\mathcal{T}^4_N}
   \\[1ex]&\hspace{3cm}+ \underbrace{\langle \pi_n, \mu^N_0-\mu_0\rangle}_{:=\mathcal{T}^5_N} .\end{split} \end{equation} where we recall that $M_t=\langle \pi_n, \mu_0-\mu_t\rangle$. We now estimate the errors $\mathcal{T}^i_N$, $i=1,3,4,5;$ the remaining term $\mathcal{T}^2_N$ will be dealt with separately, and requires careful construction of the sequence $\xi_N$. \paragraph{1. Estimate on $\mathcal{T}^1_N$.} Let $h_N=1_{S_{(\xi_N)}}$, so that $h_N \le f_N \le 1$. As $N\rightarrow \infty$, $\pi_n h_N \uparrow \pi_n$, and so by monotone convergence, $
       \langle \pi_n h_N, \mu_t\rangle \uparrow \langle \pi_n, \mu_t\rangle$ This implies immediately that the (nonrandom) error $\mathcal{T}^1_N \rightarrow 0$.
\paragraph{2. Estimate on $\mathcal{T}^3_N$.} From the definitions of $f_N, g_N$, we observe that \begin{equation}
       |\mathcal{T}^3_N(t)|=\langle \pi_n(g_N-f_N), \mu^N_t\rangle \le  \langle \pi_n 1[\pi_n(x)<\xi_N, \pi_e(x)>\eta_{\xi_N}], \mu^N_t\rangle.
   \end{equation} Therefore, in the notation of Lemma \ref{lemma: STUI}, \begin{equation}
       \mathbb{E}\left[|\mathcal{T}^3_N(t)|\right] \leq \beta(\xi_N, \eta_{\xi_N}).
   \end{equation} By construction of $\eta_r$, and since $\xi_N \rightarrow \infty$, it follows that $\mathbb{E}[ |\mathcal{T}^3_N(t)|] \rightarrow 0,$ which implies convergence to $0$ in probability.
\paragraph{3. Estimate on $\mathcal{T}^4_N$.} By the choice (\ref{eq: choice of xiN for WCOG}) of $\xi_N$, we have that $ C_1(G^N_t) \ge \xi_N$ with high probability. On this event, we have the equality \begin{equation}
           \begin{split}
               \langle \pi_n g_N, \mu^N_t\rangle &=\langle \pi_n, \mu^N_t\rangle - \langle \pi_n 1_{\pi_n\geq \xi_N}, \mu^N_t\rangle \\[2ex] & = \langle \pi_n, \mu^N_0\rangle-\frac{1}{N}\sum_{j\geq 1: C_j(G^N_t)\ge \xi_N} \hspace{0.1cm}\sum_{i \in C_j(G^N_t)} 1 \\[2ex] & = \langle \pi_n, \mu^N_0\rangle-M^N_t-\frac{1}{N}\sum_{j\ge 2:C_j(G^N_t)\ge \xi_N}\hspace{0.1cm}\sum_{i\in C_j(G^N_t)}1. 
           \end{split} 
       \end{equation} Therefore, with high probability, \begin{equation}\label{eq: bound on T4} \mathcal{T}^4_N(t) \le \frac{1}{N} \sum_{j\ge 2:C_j(G^N_t)\ge \xi_N} \hspace{0.1cm}\sum_{i\in C_j(G^N_t)} (1+\frac{1}{2}|v_i|^2) \end{equation} and we bound, on this event, \begin{equation}\mathcal{T}^4_N(t) \le \left(\frac{1}{N}\sum_{j\ge 2: C_j(G^N_t)\ge \xi_N} C_j(G^N_t)\right)^\frac{1}{2}\left(\frac{1}{N}\sum_{i=1}^N (1+\frac{1}{2}|v_i|^2)^2\right)^\frac{1}{2}. \end{equation} The first term is the mass of the mesoscopic clusters, which converges to $0$ in probability, by Theorem \ref{thrm: RG2}, and the second term is bounded in $L^2$ by the second part of (B2.). Together, these imply that $\mathcal{T}^4_N(t)\rightarrow 0$ in probability.
 \paragraph{4. Estimate on $\mathcal{T}^5_N$.} Using the first part of (B2.), we have the convergence in distribution \begin{equation} \langle \pi_n, \mu^N_0\rangle \rightarrow \langle \pi_n, \mu_0\rangle \end{equation} which implies that $\mathcal{T}^5_N\rightarrow 0$ in probability as desired. 
\paragraph{5. Construction of $\xi_N$, and convergence of $\mathcal{T}^2_N$.} It remains to show how a sequence $\xi_N$ can be constructed such that $\mathcal{T}^2_N \rightarrow 0$ in probability. Let $A^1_{r,N}, A^2_{r,N}$ be the events \begin{equation} \label{eq: definition of A1rn for WCOG}
       A^1_{r,N}=\left\{ |\langle \varphi \widetilde{f}_r, \mu^N_t-\mu_t\rangle|<\frac{1}{r}\right\}; \hspace{1cm}
       A^2_{r,N}=\left\{C_1(G^N_{t}) \geq r\right\}.
   \end{equation}
 Then, as $N\rightarrow \infty$ with $r$ fixed, both $\mathbb{P}(A^1_{r,N}), \mathbb{P}(A^2_{r,N}) \rightarrow 1$, by Lemma \ref{lemma: local uniform convergence of stochastic coagulent} and Lemma \ref{lemma: lower bound on largest cluster}. We now define $N_r$ inductively for $r\geq 1$ by setting $N_1=1$, and letting $N_{r+1}$ be the minimal $N>N_r$ such that, for all $N'\ge N$,
 \begin{equation}
       \label{eq: recursive definition of Nr for WCOG} N\geq (r+1)^2; \hspace{1cm} \mathbb{P}(A^1_{r+1,N'})>\frac{r}{r+1}, \hspace{1cm} \mathbb{P}(A^2_{r+1,N'})>\frac{r}{r+1}.
 \end{equation} 
 Now, we set $\xi_N=r$ for $N\in [N_r, N_{r+1})\cap\mathbb{N}.$ It follows that $\xi_N \rightarrow \infty$ and $\xi_N\leq \sqrt{N}\ll N$, and
 \begin{equation}
       \mathbb{P}\left(C_1(G^N_t))\geq \xi_N\right)\ge 1-\frac{1}{\xi_N} \rightarrow 1. 
 \end{equation} Therefore, $\xi_N$ satisfies the requirements (\ref{eq: choice of xiN for WCOG}) above. Moreover, \begin{equation}
       \mathbb{P}\left(|\mathcal{T}^2_N| <\frac{1}{\xi_N}\right) \ge \mathbb{P}\left(A^1_{\xi_N,N}\right) > 1-\frac{1}{\xi_N}\rightarrow 1
 \end{equation}
 and so, with this choice of $\xi_N$, $\mathcal{T}^2_N \rightarrow 0$ in probability. Since we have now dealt with every term appearing in the decomposition (\ref{eq: decomposition of erorr in WCOG}), it follows that $M^N_t\rightarrow M_t$ in probability, as claimed. \medskip \\ The arguments for the energy $E^N_t$ are identical to those above, using the same bound (\ref{eq: bound on T4}) on $\mathcal{T}^4_N$. \end{proof} We also note, for future use, an important corollary of this argument.
\begin{cor}\label{corr: gel at tgel} At the instant of gelation, the gel is negligible: $ M_{t_\mathrm{g}}=E_{t_\mathrm{g}}=0.$  \end{cor} \begin{proof} This follows from the critical case of Theorem~\ref{thrm: RG1} exactly as in (\ref{eq: use of CS 0}). \end{proof}
\section{\textbf{Behaviour of the Second Moment}}
\label{sec: finiteness of second moment} In this section, we consider part 2 of Theorem \ref{thrm: Smoluchowski equation}, concerning the behaviour of the second moment $\mathcal{E}(t)=\langle \varphi^2, \mu_t\rangle$. Following \cite{N00}, one might expect that the gelation time $t_\mathrm{g}$ corresponds to a divergence of $\mathcal{E}(t)$ as $t\uparrow t_\mathrm{g}$; by an approximation argument, we will show that this is indeed the case. We also introduce a \emph{duality argument}, corresponding to Theorem \ref{thrm: coupling supercritical and subcritical}, which allows us to prove that $\mathcal{E}$ is finite on $(t_\mathrm{g}, \infty)$. The final assertion follows from the fact that $M_{t_\mathrm{g}}=E_{t_\mathrm{g}}=0$, which is the content of Corollary \ref{corr: gel at tgel}.
\subsection{\textbf{Subcritical Regime}} We first deal with the subcritical regime $[0, t_\mathrm{g})$, to show that the second moment $\mathcal{E}(t)$ is finite and increasing on this interval, and that $t_\mathrm{g}$ is exactly the first time at which $\mathcal{E}$ diverges.
\begin{lem}\label{lemma: second moment before tgel} Suppose $\mu_0$ satisfies (A1-4.), and let $(\mu_t)_{t\ge 0}$ be the corresponding solution to (\ref{eq: E+G}). The second moment $\mathcal{E}(t)=\langle \varphi^2, \mu_t\rangle$ is finite and increasing on $[0, t_\mathrm{g})$, and increases to infinity as $t\uparrow t_\mathrm{g}$, where $t_\mathrm{g}$ is the associated gelation time. \end{lem} 

The ideas of this argument follow \cite{N00}, where there is a similar result for \emph{approximately multiplicative} kernels, for which the total rate $\overline{K}(x,y)$ is bounded above \emph{and below} by nonzero multiples of $\widetilde{\varphi}(x)\widetilde{\varphi}(y)$, where $\widetilde{\varphi}$ is a mass function playing the same r\^ole as our $\varphi$. Unfortunately, this cannot be applied directly, for two reasons. \begin{enumerate}[label=\roman{*}).]
    \item Firstly, the total rate in (\ref{eq: overline K}) contains the term $\pi_p(x)\cdot\pi_p(y)$ of indefinite sign.
    \item Secondly, the remaining combination of $\pi_n, \pi_e$ is not in an approximately multiplicative form: particles of either very low or very high energy prevent the desired lower bound from holding for any positive constant.
\end{enumerate}
We deal with these as follows. To deal with item i)., we introduce a symmetrised kernel $K^\mathrm{m}$, and use the symmetry (A1.) to argue that the solutions coincide exactly with solutions to the original equation (\ref{eq: E}). To deal with the degeneracy in point ii), we consider a truncated state space $S^\epsilon$, which excludes particles of either very high or very low energy. In this context, the kernel $K^\mathrm{m}$ is approximately multiplicative, so the results of \cite{N00} apply; we then carefully justify taking the limit $\epsilon \downarrow 0$. \medskip \\ 
We first consider the symmetrised equation. Let $K^\mathrm{m}$ be the kernel on $S\times S\times S$ given by
 \begin{equation}\label{eq: modified K} 
 \begin{split}
 K^\mathrm{m}(x,y,dz)
 =\frac{1}{4}K(Rx, y, dz)+\frac{1}{2}K(x,y,dz)+\frac{1}{4}K(x,Ry, dz). \end{split} 
\end{equation}
Let $L^\mathrm{m}$ be the drift operator for the modified kernel $K^\mathrm{m}$, and consider the modified equation \begin{equation} \tag{mE-G}\label{eq: mE}
    \mu_t=\mu_0+\int_0^t L^\mathrm{m}(\mu_s)ds.
\end{equation}The total rate of the modified kernel is \begin{equation}
    \label{eq: modified Kbar} 
    \overline{K^\mathrm{m}}(x,y)=\kappa \pi_n(x) \pi_n(y) + 2\gamma(\pi_n(x)\pi_e(y)+\pi_e(x)\pi_n(y)).
\end{equation}
Consider a modified state space, which truncates the velocity distribution by excluding clusters with extreme kinetic energies: for $\epsilon>0$, let \begin{equation}
    S^\epsilon= \{x\in S: \epsilon \pi_n(x) \leq \pi_e(x) \leq \epsilon^{-1} \pi_n(x)\}.
\end{equation} Note that this state space is preserved under both kernels $K, K^\mathrm{m}$. Moreover, on the reduced state space $S^\epsilon$, the modified kernel $K^\mathrm{m}$ is \emph{approximately multiplicative} \cite{N00} in the sense that, for some $\delta_\epsilon>0$ and $\Delta_\epsilon<\infty$, we have \begin{equation}
    \delta_\epsilon\hspace{0.1cm}\varphi(x)\varphi(y) \leq \overline{K^\mathrm{m}}(x,y) \leq  \Delta_\epsilon\hspace{0.1cm}\varphi(x)\varphi(y)
\end{equation}for all $x,y \in S^\epsilon$. For any $\epsilon$, let $\mu_0^\epsilon$ denote the restriction $\mu_0^\epsilon(dx)=1_{x\in S^\epsilon}\hspace{0.1cm}\mu_0(dx).$ We now appeal to \cite[Theorem 2.2]{N00}, on existence and uniqueness for approximately multiplicative kernels, to obtain the following, which provides the connection between gelation and explosion of a second moment.
\begin{lem}\label{lemma: solution to modified equation}
    Suppose (A1-4.) hold, and let $\mu^\epsilon_0$ be as above. For all $\epsilon>0$, there is a unique maximal conservative solution  $(\nu^\epsilon_t)_{t< t_\mathrm{e}^\epsilon}$ in $S^\epsilon$ to the modified equation (\ref{eq: mE}), starting from $\mu_0^\epsilon$. Moreover, the map $t\mapsto \langle \varphi^2, \nu^\epsilon_t\rangle$ is finite and increasing on $[0,t_\mathrm{e}^\epsilon)$, and increases to $\infty$ as $t\uparrow t_\mathrm{e}^\epsilon$. 
\end{lem}

Similarly, we can also apply Corollary \ref{cor: maximal conservative solutions} to see that there exist maximal conservative solutions $(\mu^\epsilon_t)_{t<t_\mathrm{g}^\epsilon}$ to (\ref{eq: E}) starting at $\mu^\epsilon_0$, which are given by initial segments of a global solution $(\mu^\epsilon_t)_{t\geq 0}$ to (\ref{eq: E+G}). Repeatedly exploiting uniqueness, we show that these coincide with the solution to (\ref{eq: mE}):

\begin{lem}[Relationship of equations]\label{lemma: Relationship}
Let $\mu^\epsilon_0$ be as above, for initial data $\mu_0$ satisfying ({A1}-4.).  Then the maximal conservative solutions $(\mu^\epsilon_t)_{t<t_\mathrm{g}^\epsilon}$ and $(\nu^\epsilon_t)_{t<t_\mathrm{e}^\epsilon}$, to (\ref{eq: E}) and (\ref{eq: mE}) respectively, coincide. In particular, $t_\mathrm{e}^\epsilon = t^\epsilon_\mathrm{g}$, and the map \begin{equation}
    t\mapsto \langle \varphi^2, \mu^\epsilon_t\rangle
\end{equation} is finite and increasing on $[0, t_\mathrm{g}^\epsilon)$, and increases to $\infty$ as $t\uparrow t_\mathrm{g}^\epsilon.$ \end{lem}

\begin{proof} Firstly, we note that $( \mu^\epsilon_t \circ R^{-1})_{t<t_\mathrm{g}^\epsilon}$ also solves (\ref{eq: E}), starts at $\mu^\epsilon_0$ by (A1.), and is conservative. Therefore, by uniqueness in Corollary \ref{cor: maximal conservative solutions}, we must have $\mu^\epsilon_t \circ R^{-1}=\mu^\epsilon_t$ for all $t<t_\mathrm{g}^\epsilon.$ Therefore, for any bounded, measurable function $f$, and $t<t^\epsilon_\mathrm{g},$ it follows from elementary manipulations that
\begin{equation} \begin{split} \label{eq: symmetry under R}
        &\int_{S^3} (f(z)-f(x)-f(y))K(x,y,dz)\mu^\epsilon_t(dx)\mu^\epsilon_t(dy)
 \\ &\hspace{1cm}=  \int_{S^3} (f(z)-f(x)-f(y))K(Rx,y,dz)\mu^\epsilon_t(dx)\mu^\epsilon_t(dy) \\
 &\hspace{1cm}=  \int_{S^3} (f(z)-f(x)-f(y))K(x,Ry,dz)\mu^\epsilon_t(dx)\mu^\epsilon_t(dy).
\end{split} \end{equation} Combining these, we see that $(\mu^\epsilon_t)_{t<t^\epsilon_\mathrm{g}}$ solves the modified equation (\ref{eq: mE}), and so, by uniqueness of the maximal conservative solution $(\nu^\epsilon_t)_{t<t_\mathrm{e}^\epsilon}$ in Lemma \ref{lemma: solution to modified equation}, we have \begin{equation}
        t^\epsilon_\mathrm{g} \leq t_\mathrm{e}^\epsilon; \hspace{1cm} \mu^\epsilon_t=\nu^\epsilon_t \hspace{0.5cm}\forall t<t^\epsilon_\mathrm{g}.
    \end{equation} The other implication is identical, using the uniqueness of the maximal conservative solution $(\nu^\epsilon_t)_{t<t_\mathrm{e}^\epsilon}$ in Lemma \ref{lemma: solution to modified equation} to deduce that $\nu^\epsilon_t= \nu^\epsilon_t\circ R^{-1}$ for $t<t_\mathrm{e}^\epsilon$. Hence, the equations \eqref{eq: symmetry under R} hold with $\nu^\epsilon_t$ in place of $\mu^\epsilon_t$, for any bounded, measurable $f$ and $t<t_\mathrm{e}^\epsilon.$ Therefore, $(\nu^\epsilon_t)_{t<t_\mathrm{e}^\epsilon}$ is a conservative solution to the unmodified equation (\ref{eq: E}), and so by Corollary \ref{cor: maximal conservative solutions}, \begin{equation}
        t_\mathrm{e}^\epsilon \leq t^\epsilon_\mathrm{g}; \hspace{1cm} \nu^\epsilon_t=\mu^\epsilon_t \hspace{0.5cm}\forall t<t_\mathrm{e}^\epsilon.
    \end{equation} \end{proof}  
    
Using previous arguments, we make the following remark on the gelation times $t^\epsilon_\mathrm{g}$.

\begin{lem}\label{lemma: continuity of tcrit} Suppose $\mu_0$ satisfies (A1-4.). Let $\mu^\epsilon_0$ be as above, and let $t^\epsilon_\mathrm{g}$ be the corresponding gelation time. Then $t^\epsilon_\mathrm{g}\rightarrow t_\mathrm{g}$ as $\epsilon \downarrow 0$.  \end{lem}\begin{proof} This follows from the calculation in Lemma \ref{lemma: computation of tcrit}. Let $m^\epsilon$ be the cutoff measure \begin{equation} m^\epsilon(dv)=1[\sqrt{2\epsilon}\le|v|\le\sqrt{2\epsilon^{-1}}]\hspace{0.1cm}m(dv) \end{equation} so that $m^\epsilon$ corresponds to the initial particle velocities for $\mu^\epsilon_0$. By Corollary \ref{corr: actual expression for tg}, the gelation times $t^\epsilon_\mathrm{g}, t_\mathrm{g}$ are given by a continuous function of the moments $\sigma_0, \sigma_2, \sigma_4$ of $m^\epsilon, m$.  Using dominated convergence and hypotheses (A3-4.), we have \begin{equation} \sigma_k(m^\epsilon)\rightarrow \sigma_k(m);\hspace{1cm}0\le k\le 4 \end{equation} which implies the claimed convergence $t^\epsilon_\mathrm{g}\rightarrow t_\mathrm{g}.$ \end{proof}

%\iffalse Repeating the calculation we have already done, we can calculate the gelation times explicitly: \begin{lem}[Identification of Gelation Times]\label{lemma: calculation of gelation} For $\epsilon>0$, the gelation time $t_\mathrm{g}^\epsilon$ is given by \begin{equation} t_\mathrm{g}^\epsilon = \frac{1}{4\gamma}\sqrt{\frac{\langle \pi_n^2, \mu_0^\epsilon\rangle}{\langle \pi_e^2, \mu_0^\epsilon\rangle}}\left(\langle \pi_n^2, \mu_0^\epsilon\rangle+\sqrt{\frac{\langle \pi_n^2, \mu_0^\epsilon\rangle}{\langle \pi_e^2, \mu_0^\epsilon\rangle}}\langle \pi_n\pi_e, \mu_0^\epsilon\rangle \right)^{-1}. \end{equation} As $\epsilon \downarrow 0$, we have \begin{equation}
%    t_\mathrm{g}^\epsilon \rightarrow t_\mathrm{g} =\frac{1}{4\gamma}\sqrt{\frac{\langle \pi_n^2, \mu_0\rangle}{\langle \pi_e^2, \mu_0\rangle}}\left(\langle \pi_n^2, \mu_0\rangle+\sqrt{\frac{\langle \pi_n^2, \mu_0\rangle}{\langle \pi_e^2, \mu_0\rangle}}\langle \pi_n\pi_e, \mu_0\rangle \right)^{-1}.
%\end{equation}  \end{lem} \begin{proof} \textcolor{red}{{Here's what we had before:}} From Lemma \ref{lemma: integral equation}, and standard regularity arguments, we have 
%\begin{align}
%    \frac{\dd}{\dd t}\langle \pi_n^2, \mu_t\rangle &=
%    2\kappa \langle \pi_n^2, \mu_t\rangle^2
%    + 8 \gamma \langle \pi_n \pi_e, \mu_t\rangle\langle \pi_n^2, \mu_t\rangle\\
%    \frac{\dd}{\dd t}\langle \pi_n \pi_e, \mu_t\rangle &=
%    2\kappa \langle \pi_n^2, \mu_t\rangle \langle \pi_n \pi_e, \mu_t\rangle
%    + 4 \gamma \langle \pi_n \pi_e, \mu_t\rangle^2
%    + 4 \gamma \langle \pi_n^2, \mu_t\rangle\langle \pi_e^2, \mu_t\rangle\\
%    \frac{\dd}{\dd t}\langle \pi_e^2, \mu_t\rangle &=
%    2\kappa \langle \pi_n \pi_e, \mu_t\rangle^2
%    + 8 \gamma \langle \pi_n \pi_e, \mu_t\rangle\langle \pi_e^2, \mu_t\rangle.
%\end{align}
%In the case $\kappa = 0$ note that
%\begin{equation}\label{eq:proportional}
%    \langle \pi_e^2, \mu_t\rangle = \langle \pi_e^2, \mu_0\rangle
%    \frac{\langle \pi_n^2, \mu_t\rangle}{\langle \pi_n^2, \mu_0\rangle}
%\end{equation}
%and then one easily checks that
%\begin{equation}
%    f(t) := \langle \pi_n^2, \mu_t\rangle
%           + \sqrt{\frac{\langle \pi_n^2, \mu_0\rangle}
%                        {\langle \pi_e^2, \mu_0\rangle}}
%             \langle \pi_n \pi_e, \mu_t\rangle
%\end{equation}
%satisfies
%\begin{equation}
%    \frac{\dd }{\dd t}f(t) = 4\gamma 
%    \sqrt{\frac{\langle \pi_e^2, \mu_0\rangle}
%                        {\langle \pi_n^2, \mu_0\rangle}}f(t)^2,
%\end{equation}
%which has the solution
%\begin{equation}\label{eq:quadmoment-ode}
%    f(t) = \left(\frac{1}{f(0)} - 4\gamma 
%    \sqrt{\frac{\langle \pi_e^2, \mu_0\rangle}
%                        {\langle \pi_n^2, \mu_0\rangle}}t \right)^{-1}.
%\end{equation}
%The solution to \eqref{eq:quadmoment-ode} blows up when
%\begin{equation}\label{eq:quad-tgel}
%    t = \frac{1}{4\gamma f(0)}
%             \sqrt{\frac{\langle \pi_n^2, \mu_0\rangle}
%                        {\langle \pi_e^2, \mu_0\rangle}}
%\end{equation}

%Assuming the initial condition is independent Maxwell-Boltzmann velocities in $d$ dimensions and all clusters of size 1 one has
%\begin{align*}
%    \langle \pi_n^2, \mu_0\rangle &= 1\\
%    \langle \pi_n \pi_e, \mu_0\rangle &= d\frac{\sigma^2}{2}\\
%    \langle \pi_e^2, \mu_0\rangle &= \frac{\sigma^4}{4}\left(2d + d^2\right)
%\end{align*}
%and thus $f(0) = 1 + d / \sqrt{2d + d^2}$.
%In this case the critical time given by \eqref{eq:quad-tgel} is
%\begin{equation}
%    \frac{1}{2\gamma \sigma^2 \left(d + \sqrt{2d +d^2}\right)}.
%\end{equation}
%This should be compared to the expression for the mean free time \cite[equation 4.38]{PSW17}, which is $1/4\gamma \sigma^2 d$. \end{proof} \fi 



We now turn to the proof of Lemma \ref{lemma: second moment before tgel}. We say that a local solution $(\nu_t)_{t<T}$ to (\ref{eq: E}) is \emph{strong} if, for all times $t<T$, 
\begin{equation}
    \int_0^t  \hspace{0.1cm} \langle\varphi^2, \nu_s\rangle  \hspace{0.1cm} ds<\infty.
\end{equation} We use the following result from \cite{N00} on the existence and uniqueness of strong solutions. 
\begin{lem}\label{lemma: strong solutions} Any strong solution to (\ref{eq: E}) is conservative. For any finite measure $\mu_0$ with $\langle \varphi^2, \mu_0\rangle <\infty$, there is a unique maximal strong solution $(\mu'_t)_{t<t_\mathrm{e}(\mu_0)}$ to (\ref{eq: E}), starting at $\mu_0$, and with $t_\mathrm{e}(\mu_0)>0$, such that $ \langle \varphi^2, \mu'_t\rangle$ is increasing on $[0,t_\mathrm{e}(\mu_0))$. If $t_\mathrm{e}(\mu_0)<\infty$, then $\langle \varphi^2, \mu'_t\rangle$ increases to $\infty$  as $t\uparrow t_\mathrm{e}(\mu_0)$.  \end{lem} Here, the subscript `e' denotes explosion: $t_\mathrm{e}(\mu_0)$ is exactly the blow-up time of the second moment. When the measure $\mu_0$ is clear, we will omit the argument of $t_\mathrm{e}.$ 
\begin{proof}
    This is almost a special case of \cite[Theorem 2.1]{N00}. From the cited result, for any finite measure $\mu_0$ with $\langle \varphi^2, \mu_0\rangle <\infty$, there exists a maximal strong solution $(\mu'_t)_{t<t_\mathrm{e}(\mu_0)}$. Moreover, there exists a constant $C=C(\kappa, \gamma)>0$ such that, for all such $\mu_0$,  $t_\mathrm{e}(\mu_0) \ge C \langle \varphi^2, \mu_0\rangle^{-1}$. By applying this bound to $\mu'_t$, if $t_\mathrm{e}(\mu_0)<\infty,$ then $\langle \varphi^2, \mu'_t\rangle \ge (C(t_\mathrm{e}(\mu_0)-t))^{-1}$  which implies the claimed divergence.
\end{proof}

By Corollary \ref{cor: maximal conservative solutions}, since $(\mu'_t)_{t<t_\mathrm{e}}$ is conservative, it follows that $t_\mathrm{e} \le t_\mathrm{g}$, and $\mu'_t=\mu_t$ for all $t<t_\mathrm{e}$. It remains to show that $t_\mathrm{e}\ge t_\mathrm{g}$. \medskip \\ Following the ideas of \cite[Proposition 2.7]{N00}, we obtain the integral relations, for all $t<t_\mathrm{e}$, \begin{equation} \label{eq: ODE1}
    \langle \pi_n^2, \mu_t\rangle =
    \langle \pi_n^2, \mu_0\rangle + \int_0^t \left[\kappa\langle \pi_n^2, \mu_s\rangle^2+4\gamma\langle \pi_n\pi_e, \mu_t\rangle\langle\pi_n^2, \mu_s\rangle \right] ds;
\end{equation} 

\begin{equation}\begin{split}\label{eq: ODE2}
    &\langle \pi_n \pi_e, \mu_t\rangle =
    \langle \pi_n\pi_e, \mu_0\rangle \\&\hspace{1cm}+ \int_0^t \left[\kappa\langle \pi_n^2, \mu_s\rangle\langle \pi_n\pi_e, \mu_s\rangle+2\gamma\langle \pi_n\pi_e, \mu_t\rangle^2+2\gamma\langle\pi_n^2, \mu_s\rangle\langle \pi_e^2, \mu_s \rangle \right] ds;
\end{split}\end{equation}

\begin{equation} \label{eq: ODE3}
    \langle \pi_e^2, \mu_t\rangle =
    \langle \pi_e^2, \mu_0\rangle + \int_0^t \left[\kappa\langle \pi_n\pi_e, \mu_s\rangle^2+4\gamma\langle \pi_n\pi_e, \mu_t\rangle\langle\pi_e^2, \mu_s \rangle \right] ds.
\end{equation} These immediately imply that $\mathcal{E}(t)$ is bounded on compact subsets of $[0, t_\mathrm{e})$, and in particular cannot diverge before $t_\mathrm{e}$. Combining this with Lemma \ref{lemma: strong solutions}, the maximal time $t_\mathrm{e}$ of existence of a strong solution is precisely the first time at which the second moment $\mathcal{E}(t)$ diverges, or $\infty$ if there is no divergence. \medskip \\ We also remark on the relationship of this result to the solutions $(\mu^\epsilon)_{t<t_\mathrm{g}^\epsilon}=(\nu^\epsilon_t)_{t<t_\mathrm{e}^\epsilon}$ discussed in Lemmas \ref{lemma: solution to modified equation}, \ref{lemma: Relationship}. It is clear, from Lemma \ref{lemma: solution to modified equation}, that $(\nu^\epsilon_t)_{t<t_\mathrm{e}^\epsilon}$ is a strong solution. Moreover, in view of the comments above, and since $\langle \varphi^2, \nu^\epsilon_t\rangle \uparrow \infty$ as $t\uparrow t_\mathrm{e}^\epsilon$, it follows that that $(\nu^{\epsilon}_t)_{t<t_\mathrm{e}^\epsilon}$ is the maximal strong solution with initial data $\mu^\epsilon_0.$ This justifies the use of the notation $t^\epsilon_\mathrm{e}$ in Lemma \ref{lemma: solution to modified equation}. \bigskip \\  As discussed at the beginning of this section, our argument is now as follows. From general considerations in \cite{N00}, we argued above that the gelation time and explosion time coincide for the restricted dynamics: $t_\mathrm{g}^\epsilon=t_\mathrm{e}^\epsilon$, and we now wish to justify the limit $\epsilon\downarrow 0$. The limiting behaviour of $t^\epsilon_\mathrm{g}$ is understood from Lemma \ref{lemma: continuity of tcrit}, and so we wish to understand the behaviour of $t_\mathrm{e}^\epsilon.$ \medskip \\  Using standard regularity arguments, we may view (\ref{eq: ODE1} - \ref{eq: ODE3}) as a differential equation for the three moments $q_t=(\langle \pi_n^2, \mu_t\rangle, \langle \pi_n \pi_e, \mu_t\rangle, \langle \pi_e^2, \mu_t\rangle)$ and, from the discussion above, the blow-up time to the ODE system is exactly $t_\mathrm{e}$. An identical argument holds for $\mu^\epsilon_t$, which blows up at $t^\epsilon_\mathrm{e}$. By analysing this system of ODEs, we will show that the explosion time is continuous in the initial data, which implies that $t^\epsilon_\mathrm{e}\rightarrow t_\mathrm{e}.$
\begin{lem}\label{lemma: ODE considerations} Consider the ordinary differential equation $\dot{q}_t=b(q_t)$ in $\mathbb{R}^3$, where $b$ is the locally Lipschitz field given by \begin{equation} \label{eq: system of ODEs} b(q_1,q_2,q_3)=\begin{pmatrix}\kappa q_1^2+4\gamma q_1q_2 \\ \kappa q_1q_2+2\gamma q_2^2+2\gamma q_1q_3 \\ \kappa q_2^2 + 4\gamma q_2q_3 \end{pmatrix}. \end{equation} Then, for all $q_0\in \mathbb{R}^3$, there exists a unique maximal solution $\psi(q_0, t)$ starting at $q_0$, defined until time $\zeta(q_0)\in (0, \infty]$. Consider the sets \begin{equation} E=(0, \infty)^3; \hspace{1cm} E_\delta=[\delta,\infty)^3.\end{equation} Then, if $q_0 \in E_\delta$ for some $\delta>0$, then the solution $(\psi(q_0,t))_{t<\zeta(q_0)} \subset E_\delta$. We have the following properties: \begin{enumerate}[label=\roman{*}).] \item Let $J_\epsilon$ be the set \begin{equation} J_\epsilon  =\{q \in E: \hspace{0.2cm} \zeta(q)\ge\epsilon\}.\end{equation} If $\gamma>0$, then for all $\epsilon, \delta>0$, the set $E_\delta \cap J_\epsilon $ is bounded. Moreover, $\zeta<\infty$ everywhere. \item Suppose $q^n_0 \in E$ and $q^n_0 \rightarrow q_0 \in E$. Then $\zeta(q^n_0)\rightarrow \zeta(q_0).$ \item Suppose $I\subset \mathbb{R}_+$ is an open interval, and the map $q_0: I\rightarrow E$ is continuous, and such that $t<\zeta(q_0(t))$ for all $t\ge 0.$ Then the map $I\rightarrow E, t\mapsto \psi(q_0(t), t)$ is continuous. 
 \end{enumerate} \end{lem} 

\begin{proof} For all three items, the case where $\gamma=0, \kappa>0$ may be checked by an elementary explicit calculation. For the remainder of the proof, we exclude this case, and consider only the case $\gamma>0.$ \begin{enumerate}[label=\roman{*}).]
    \item Let $\zeta_0$ denote the blowup time for the dynamics (\ref{eq: system of ODEs}) with $\kappa=0$. It is straightforward to see that $\zeta(q)\le\zeta_0(q)$ for all $q\in E$, and so it is sufficient to show that $E_\delta\cap \{q: \zeta_0(q)\ge \epsilon\}$ is bounded. We argue using the following explicit computation. \medskip \\ Let $q(0)=(q_1(0),q_2(0),q_3(0))\in E$, and let $q(t)=(q_1(t),q_2(t),q_3(t))$ be the solution to (\ref{eq: system of ODEs}) with $\kappa=0$, starting at $q(0)$. It is then straightforward to see that \begin{equation}
        \frac{1}{q_1(t)}\frac{d}{dt}q_1(t)=\frac{1}{q_3(t)}\frac{d}{dt}q_3(t)
    \end{equation} which implies that $q_3(t)=q_1(t)q_3(0)/q_1(0)$ for all $t\ge 0$. Now, the linear combination $\widetilde{q}(t)$ given by \begin{equation}
        \widetilde{q}(t)=q_1(t)+\sqrt{\frac{q_1(0)}{q_3(0)}}q_2(t)
    \end{equation} has the same blowup time as $q(t)$, and satisfies the ordinary differential equation \begin{equation} \frac{d}{dt}\widetilde{q}(t)=4\gamma\sqrt{\frac{q_3(0)}{q_1(0)}}\hspace{0.1cm}\widetilde{q}(t)^2. \end{equation} This has the unique solution \begin{equation} \widetilde{q}(t)=\left(\frac{1}{\widetilde{q}(0)}-4\gamma\sqrt{\frac{q_3(0)}{q_1(0)}}t\right)^{-1}; \hspace{1cm} t< \frac{1}{4\gamma\widetilde{q}(0)}\sqrt{\frac{q_1(0)}{q_3(0)}}. \end{equation} In terms of the initial data $q(0)$, this gives the blowup time as \begin{equation} \zeta(q(0))=\frac{1}{4\gamma}\left(\sqrt{q_1(0)q_3(0)}+q_2(0)\right)^{-1} \end{equation} which converges to $0$ as $q(0)\rightarrow \infty$ in $E_\delta.$ This shows that $E_\delta \cap \{q: \zeta_0(q)\ge \epsilon\}$ is bounded, as claimed. The same computation also shows that $\zeta(q)<\infty$ for all $q\in E$.
    \item The lower semicontinuity of explosion times is standard, and follows from the continuous dependence on the initial data. Therefore, it is sufficient to prove that $\limsup_{n\rightarrow \infty} \zeta(q^n)\le \zeta(q).$  \medskip \\ Suppose, for a contradiction, that for some $\epsilon>0$, we have $\limsup_{n\rightarrow \infty} \zeta(q^n)>\zeta(q)+\epsilon$; write $\tau=\zeta(q)$. By passing to a subsequence, we may assume that $\zeta(q^n)>\tau+\epsilon$ for all $n$, and some fixed $\epsilon>0$. Moreover, since $q^n\rightarrow q \in E$, we may assume that $q^n, q \in E_\delta$ for all $n$, for some $\delta>0$, which implies that $\psi(q^n,t)\in E_\delta$ for all $t<\zeta(q^n)$ and all $n\in \mathbb{N}$.\medskip\\  Now, if $t\le \tau$, we have $\zeta(\psi(t,q^n))=\zeta(q^n)-t \ge \epsilon$, which implies the containment \begin{equation} \{\psi(t,q^n): t\le \tau, n\ge 1\} \subset E_\delta\cap J_\epsilon \end{equation} which we know, from item i)., to be bounded: for some $C<\infty$, \begin{equation}
        \{\psi(t,q^n): t\le \tau, n\ge 1\} \subset [0,C]^3.
    \end{equation} By the lemma of leaving compact sets, there exists $s<\tau$ such that, for all $t\in (s,\tau)$, $\psi_t(q)\not \in [0,C]^3.$ However, if we pick $t\in (s,\tau)$, we have $\psi_t(q^n) \rightarrow \psi_t(q)$, by the continuity of the dependence in the initial conditions, which is a contradiction. Therefore, $\limsup_{n\rightarrow \infty} \zeta(q^n)\le \zeta(q)$, which proves the claimed convergence.      
    \item  Firstly, we note that by ii)., the map $t\mapsto \zeta(q_0(t))$ is continuous on $I$. Therefore, fixing $t\in I$, we may choose choose  $\epsilon, \delta > 0$ such that, if $\abs{t-s} \le \delta$, then $s\in I$ and $s < \min \left(\zeta\left(q_0(s)\right), \zeta\left(q_0(t)\right)\right)-\epsilon$. Now, we observe that, for $s\in [t-\delta, t+\delta],$\begin{equation}
    |\psi(t,q_0(t))-\psi(s,q_0(s))|\le|\psi(t,q_0(t))-\psi(t,q_0(s), )|+|\psi(t,q_0(s))-\psi(s,q_0(s))|.
\end{equation} As $s\rightarrow t$, the first term converges to $0$ by continuity of the solution $s\mapsto \psi(x_0(t),s)$; it is therefore sufficient to control the second term. We observe that, for all $s\in[t-\delta, t+\delta],$ we have $\zeta(\psi(s,q_0(s)))=\zeta(q_0(s))-s>\epsilon$. Moreover, by compactness, there exists some $\eta>0$ such that $q_0(s) \in E_\eta$ for all $s\in [t-\delta, t+\delta]$, and so $\psi(q_0(s),u)\in E_\eta$ for all $0\le u\le \zeta(q_0(s)).$ However, we showed in point i). above that the region $E_\eta \cap J_\epsilon=\{q \in E_\eta: \zeta(q)\geq\epsilon\}$ is compact  and so there exists a constant $M=M(\epsilon)$: for all $s\in[t-\delta, t+\delta]$, and for all $u \le t+\delta$, \begin{equation}u< \zeta(q_0(s));\hspace{1cm} |b(\psi(u,q_0(s))| \le M. \end{equation}
This implies the bound, for all $s\in[t-\delta,t+\delta]$, \begin{equation} |\psi(t,q_0(s))-\psi(s,q_0(s))| \le M|t-s|\end{equation} which implies the claimed continuity.
\end{enumerate}  \end{proof}



We can now use this to prove our main result Lemma \ref{lemma: second moment before tgel} on the second moment $\mathcal{E}(t)$ in the subcritical phase.





\begin{proof}[Proof of Lemma \ref{lemma: second moment before tgel}]
Let $\mu_0$ be any measure on $S$ satisfying ({A1}-{4}.), and let $(\mu_t)_{t\ge 0}$ be the associated solution to (\ref{eq: E+G}). From the discussion following Lemma \ref{lemma: strong solutions}, it is sufficient to show that $t_\mathrm{g}=t_\mathrm{e}<\infty$. We also recall that $t_\mathrm{e}$ is characterised as the explosion time $\zeta$ of the ODE system (\ref{eq: ODE1}-\ref{eq: ODE3}). \medskip \\ Since the base measure $m$ of $\mu_0$ is not a multiple of the point mass $\delta_0$, all the quadratic moments $q$ of $\mu_0$ are strictly positive, and so $q \in E$. Therefore, by the second point of Lemma \ref{lemma: ODE considerations}, the explosion time $t_\mathrm{e}=\zeta(q)<\infty.$
As $\epsilon \downarrow 0$, the quadratic moments $q^\epsilon$ of $\mu^\epsilon_0$ converge to the quadratic moments $q\in E$ of $\mu_0$ by dominated convergence, and using (A4.).  Therefore, by Lemma~\ref{lemma: ODE considerations},  $t^\epsilon_\mathrm{e}=\zeta(q^\epsilon)\rightarrow \zeta(q)= t_\mathrm{e}$.
By Lemma \ref{lemma: Relationship}, we know that $t_\mathrm{e}^\epsilon=t_\mathrm{g}^\epsilon$, and by Lemma \ref{lemma: continuity of tcrit}, $t_\mathrm{g}^\epsilon \rightarrow t_\mathrm{g}$. Together, these imply that $t_\mathrm{e} = t_\mathrm{g}$, as claimed.

\end{proof} 

\subsection{\textbf{The Critical Point}} Using the concepts introduced above, we next consider the behaviour at and near the critical time $t_\mathrm{g}$. \begin{lem} Assume that $\mu_0$ is a probability measure. In the notation of Lemma \ref{lemma: second moment before tgel}, we have \begin{equation} \mathcal{E}(t_\mathrm{g})=\infty=\lim_{t\rightarrow t_\mathrm{g}} \mathcal{E}(t). \end{equation} \end{lem} 
\begin{proof}We first show that $\mathcal{E}(t_\mathrm{g})=\infty$. Suppose, for a contradiction, that $\mathcal{E}(t_\mathrm{g})<\infty.$ Then, applying \cite[Proposition 2.7]{N00} as in Lemma \ref{lemma: strong solutions}, we see that, for some positive $\delta>0$, there exists a strong solution $(\nu_t)_{t<\delta}$ to (\ref{eq: E}), starting at $\mu_{t_\mathrm{g}}.$ This solution is conservative, so is an initial segment of the solution $(\nu_t)_{t\ge 0}$ to (\ref{eq: E+G}) starting at $\mu_{ t_\mathrm{g}}$. By uniqueness in Lemma \ref{lemma: E and U}, \begin{equation}
    \nu_t=\mu_{t_\mathrm{g}+t} \hspace{1cm} \text{ for all }t\ge 0.
\end{equation}
By Corollary \ref{corr: gel at tgel}, $\langle \varphi, \mu_{t_\mathrm{g}}\rangle = \langle \varphi, \mu_0\rangle$, and by definition of $t_\mathrm{g}$, \begin{equation} \langle \varphi, \mu_{t_\mathrm{g}+t}\rangle < \langle \varphi, \mu_{0}\rangle = \langle \varphi, \mu_{t_\mathrm{g}}\rangle \text{ for all }t>0. \end{equation}This contradicts the fact that $(\nu_t)_{t<\delta}$ is strong, which therefore shows that $\mathcal{E}(t)=\infty$.

The second point follows, because $t\mapsto \mu_t$ is continuous, and $\mu \mapsto \langle \varphi^2, \mu\rangle$ is lower semicontinuous, when $\mathcal{M}$
is equipped with the vague topology. \end{proof}

\subsection{\textbf{The Supercritical Regime}} We finally turn to the supercritical case; our result is as follows. 
\begin{lem}\label{lemma: second moment finite after tgel} In the notation of Lemma \ref{lemma: second moment before tgel}, the map $t\mapsto \mathcal{E}(t)$ is finite and continuous, and therefore locally bounded, on  $(t_\mathrm{g},\infty)$. \end{lem} 
The proof is based on a \emph{duality argument} following Theorem \ref{thrm: coupling supercritical and subcritical}, which connects the measures in the supercritical regime to an auxiliary process in the subcritical case.  Let $(G^N_t)_{t\geq 0}$ be the random graph processes described in Section  \ref{sec: coupling_to_random_graph} with $N$ particles sampled independently from $m$, and fix $t>t_\mathrm{g}$. Let $\widetilde{G}^N_{t}$ be the graph $G^N_{t}$ with the giant component deleted. \medskip \\ 
Let $\rho_{t}(v)=\rho(t, v)$ be the survival function defined in Lemma \ref{lemma: survival function}, and let $\widehat{m}^t(dv)=(1-\rho_{t}(v))m(dv)$ and $\widehat{\mu}^t_0$ the corresponding measure on $S$ under pushforward by $\iota: v\mapsto (1, v, \frac{1}{2}|v|^2)$. By Lemma \ref{lemma: E and U}, there exists a unique solution $(\widehat{\mu}^t_s)_{s\geq 0}$ to the equation (\ref{eq: E+G}) starting at $\widehat{\mu}^t_0$; write $\widehat{t_\mathrm{g}}(t)$ for its gelation time. By Theorem \ref{thrm: coupling supercritical and subcritical}, we can construct a generalised vertex space $\widehat{\mathcal{V}}=(\mathbb{R}^d, \widehat{m}, (\mathbf{w}_N)_{N\ge 1})$ and a graph $\widehat{G}^N_{t}\sim \mathcal{G}^{\widehat{\mathcal{V}}}(N,tK)$ such that $\mathbb{P}(\widehat{G}^N_{t}=\widetilde{G}^N_{t})\rightarrow 1$, and where $\mathbf{w}_N$ is an enumeration of the vertexes $v_i$ not belonging to the giant component. 
\medskip \\
In order to appeal to Lemmas \ref{lemma: convergence of random graphs}, \ref{lemma: connect critical times}, we will now verify that the desired regularity conditions (A1-4, B1-2.) hold for the vertex space $\widehat{\mathcal{V}}$.
\begin{lem}\label{lemma: conditions B1-3 for duality} Fix $t>0$, and let $\mu_0, \widehat{m}^t, \widehat{\mu}^t_0$ and $\widehat{\mathcal{V}}$ be as described above. Then the regularity conditions (B1-2.) hold for $\mathbf{w}_N$ and $\widehat{\mu}^t_0.$ \end{lem} \begin{proof} To ease notation, we write $\widehat{\mu}_0, \widehat{m}$ for $\widehat{\mu}^t_0, \widehat{m}^t$, $\mu^N_0$ for the initial empirical measure of the unmodified process corresponding to $\mathbf{v}_N$, and $\widehat{\mu}^N_0$ for the reduced empirical measure corresponding to $\mathbf{w}_N$: \begin{equation} \widehat{\mu}^N_0 =\frac{1}{N}\sum_{i=1}^{l_N} \delta_{(1,w_i, \frac{1}{2}|w_i|^2)}.\end{equation} It is straightforward to see that $\widehat{\mu}^t_0$ inherits the properties (A1-4.) from $\mu_0$, and so it is sufficient to establish (B1-2.). We will also appeal to the fact that the unmodified empirical measures $\mu^N_0$ satisfy (B1-2.), which is straightforward to verify. \medskip \\ For (B1.), we note that part of the content of Theorem \ref{thrm: coupling supercritical and subcritical} is that $\widehat{\mathcal{V}}$ is a generalised vertex space, as defined in Definition~\ref{def: Generalised vertex space}, which includes the weak convergence
\begin{equation}
    \widehat{m}_N=\frac{1}{N}\sum_{i\le l_N} \delta_{w_i} \rightarrow \widehat{m} \hspace{1cm}\text{weakly, in probability}.
\end{equation} Since the map $\iota: \mathbb{R}^d\rightarrow S$ is continuous and the vague topology is weaker than the weak topology, (B1.) follows.

We will now show that (B2.) follows from the previous point, together with the moment estimates for the original initial measure $\mu^N_0$. 

Fix $R<\infty$, and let $\chi_R \in C_c(S)$ be such that $1_{S_R} \leq \chi \leq1_{S_{R+1}}$. We observe that
\begin{equation} \begin{split}
\abs{\langle x, \widehat{\mu}_0^N\rangle - \langle x, \widehat{\mu}_0\rangle}  &\leq
\abs{\langle x\chi_R, \widehat{\mu}_0^N-\widehat{\mu}_0\rangle } 
 +\langle \abs{x} 1_{S_R^\mathrm{c}},\widehat{\mu}_0^N\rangle
 +\langle \abs{x} 1_{S_R^\mathrm{c}},\widehat{\mu}_0\rangle \\[1ex]
 &\leq
 \abs{\langle x\chi_R, \widehat{\mu}_0^N-\widehat{\mu}_0\rangle} 
 +\frac{\sqrt{2}}{R}\langle \varphi^2,{\mu}_0^N\rangle
 +\frac{\sqrt{2}}{R}\langle \varphi^2,{\mu}_0\rangle.
\end{split} \end{equation}
We now fix $\epsilon, \delta>0$. Since $\langle \varphi^2,{\mu}_0^N\rangle$ converges almost surely by (A3.), and in particular is $O_\mathrm{p}(1)$, we may choose $R<\infty$ such that the second and third terms are at most $\epsilon/3$ with probability exceeding $1-\delta/2$, for all $N$. For this choice of $R$, the first term vanishes as $N\rightarrow\infty$ by vague convergence in probability, and so is at most $\frac{\epsilon}{3}$ with probability exceeding $1-\delta/2$ for all $N$ large enough. Therefore, for all such $N$, we have \begin{equation} \PP\left(|\langle x, \widehat{\mu}^N_0-\widehat{\mu}_0\rangle|>\epsilon\right)\le \delta \end{equation} which proves the desired convergence in probability.  

For the second assertion of (B2.), we note that $\langle \varphi^2, \widehat{\mu}^N_0\rangle \le \langle \varphi^2, \mu^N_0\rangle$ by the construction of $\mathbf{w}_N$, and $\langle \varphi^2, \mu^N_0\rangle$ is bounded in $L^1$ by (A3.), recalling that the velocities in $\mathbf{v}_N$ are sampled independently from $m$.
\end{proof}


We now use this preparatory result to prove Lemma \ref{lemma: second moment finite after tgel}. \begin{proof}[Proof of Lemma \ref{lemma: second moment finite after tgel}] Let $G^N_t, \widetilde{G}^N_t, \widehat{G}^N_t$ be as above. Recalling that we consider equality of graphs to include equality of the vertex data, it follows from Theorem \ref{thrm: coupling supercritical and subcritical} that \begin{equation} \mathbb{P}(\mu^N(\widehat{G}^N_t)=\mu^N(\widetilde{G}^N_t))\rightarrow 1. \end{equation}  From Lemmas \ref{lemma: convergence of random graphs}, \ref{lemma: conditions B1-3 for duality}, we obtain the following convergences in probability:
\begin{equation}
    \mu^N(G^N_{t})\rightarrow {\mu}_{t};\hspace{1cm}
    \mu^N(\widehat{G}^N_{t})\rightarrow \widehat{\mu}^t_{t}
\end{equation} in the vague topology, in probability.  Moreover, the difference \begin{equation}
    \mu^N(G^N_{t})-\mu^N(\widetilde{G}^N_{t})=\frac{1}{N}\delta(\mathcal{C}_1(G^N_{t}))
\end{equation} converges to $0$ in the vague topology in probability, since the support is eventually disjoint from any compact set, with high probability. It follows that \begin{equation}
    \mu^N(\widetilde{G}^N_{t})\rightarrow \mu_{t}
\end{equation} in the vague topology, in probability, and by uniqueness of limits, we have $\widehat{\mu}^t_{t}=\mu_{t}$. Using assumption ({A3.}), we can see that $t k\in L^2(\mathbb{R}^d\times \mathbb{R}^d,m\times m)$, and so it follows from Theorem \ref{thrm: coupling supercritical and subcritical} that the graphs $\widehat{G}^N_{t}$ are subcritical. By Lemma \ref{lemma: connect critical times}, it follows that that  $t<\widehat{t_\mathrm{g}}(t)$, and so by Lemma \ref{lemma: second moment before tgel}, we have \begin{equation}
   \langle \varphi^2, \mu_t\rangle = \langle \varphi^2, \widehat{\mu}^t_{t}\rangle <\infty.
\end{equation} Using Theorem \ref{thrm: continuity of rho} and dominated convergence, the map \begin{equation}\begin{split}
   & t\mapsto q^t_0=\left(\left\langle \pi_n^2, \widehat{\mu}^t_0\right\rangle,\left\langle \pi_n\pi_e, \widehat{\mu}^t_0\right\rangle,\left\langle \pi_e^2, \widehat{\mu}^t_0\right\rangle\right)\\&\hspace{1cm}=\left(\left\langle 1-\rho_t,m\right\rangle,\left\langle \frac{1}{2}|v|^2(1-\rho_t),m\right\rangle,\left\langle \frac{1}{4}|v|^4(1-\rho_t),m\right\rangle\right) \end{split}
\end{equation} is continuous, and it is clear that it takes values in $E=(0,\infty)^3$. Therefore, by the general ODE considerations in Lemma \ref{lemma: ODE considerations} point iii)., it follows that the map \begin{equation}
    t\mapsto q^t(t)= \left(\langle \pi_n^2, \widehat{\mu}^t_t\rangle,\langle \pi_n\pi_e, \widehat{\mu}^t_t\rangle,\langle \pi_e^2, \widehat{\mu}^t_t\rangle\right)
\end{equation} is finite and continuous on $(t_\mathrm{g}, \infty).$  Since $\widehat{\mu}^t_t=\mu_t$, this implies that $t\mapsto \mathcal{E}(t)$ is finite and continuous on $(t_\mathrm{g}, \infty)$, which implies that it is bounded on compact subsets. \end{proof}

\begin{rmk} The same argument also shows that $t\mapsto \widehat{t}_\mathrm{g}(t)$ is continuous. This fact will be used later in the proof of Lemma \ref{lemma: anomalous clusters 2}. \end{rmk}


\section{\textbf{Representation and Dynamics of the Gel}} \label{sec: gel dynamics}
\subsection{\textbf{Representation Formula}}

The duality construction used in the proof of Lemma \ref{lemma: second moment finite after tgel} gives us a natural way to identify $M_t, E_t$ in terms of the survival function $\rho_t$ from Sections \ref{sec: IRG}, \ref{sec: coupling_to_random_graph}. This is the content of the following lemma. \begin{lem}\label{lemma: representation of M, E} Let $\mu_0$ be an initial data satisfying (A1-4.) for some measure $m$, and let $M_t, E_t$ be the gel data for the corresponding solution to (\ref{eq: E+G}). Let $\rho_t(\cdot)$ be the corresponding survival function defined in Sections \ref{sec: IRG}, \ref{sec: coupling_to_random_graph}. Then we have the equalities \begin{equation}\label{eq: formula for M, E}
    M_t = \int_{\mathbb{R}^d} \rho_t(v)m(dv); \hspace{1cm} E_t=\int_{\mathbb{R}^d} \frac{1}{2}|v|^2\rho_t(v)m(dv).
\end{equation} In particular, both $M_t$ and $E_t$ are continuous, and if $t>t_\mathrm{g}$ then $M_t>0$ and $E_t>0$. \end{lem} It is immediate from the symmetry (A1.), or from Lemma \ref{lemma: E and U of Restricted}, that $P_t=0$. Together with the identification of $\rho_t$ in Lemma \ref{lemma: form of rho-t}, this proves part 3 of Theorem \ref{thrm: Smoluchowski equation}. \begin{proof} We deal with the supercritical and subcritical/critical cases, $t>t_\mathrm{g}, t\le t_\mathrm{g}$ separately. 
\paragraph{1. Supercritical Case $t>t_\mathrm{g}$.}  Let $(\widehat{\mu}^t_s)_{s\geq 0}$ and $\widehat{t_\mathrm{g}}(t)$ be as in the proof of Theorem \ref{lemma: second moment finite after tgel}. Then, since $(\widehat{\mu}^t_s)_{s\geq 0}$ is conservative on $[0, \widehat{t_\mathrm{g}})$, and $t<\widehat{t_\mathrm{g}}(t)$, we have \begin{equation}
    \langle \pi_n, \widehat{\mu}^t_t\rangle =\langle \pi_n, \widehat{\mu}^t_0\rangle = \int_{\mathbb{R}^d} (1-\rho(t,v))m(dv).
\end{equation} But since $\mu_t=\widehat{\mu}^t_t$, we have \begin{equation}
    M_t:=\langle \pi_n, \mu_0\rangle -\langle \pi_n, \mu_t\rangle =\langle \pi_n, \mu_0\rangle - \int_{\mathbb{R}^d} (1-\rho(t,v))m(dv)
\end{equation} which implies the result for $M_t$. The argument for $E_t$ is identical. 

\paragraph{2. Subcritical and Critical Cases $t\le t_\mathrm{g}$.} For $t<t_\mathrm{g}$, the result is immediate: we have $M_t=E_t=0$ by definition of $t_\mathrm{g}$, and $\rho_t=0$ by Theorem \ref{lemma: survival function}. The critical case is identical, recalling from Corollary \ref{corr: gel at tgel} that $M_{t_\mathrm{g}}=E_{t_\mathrm{g}}=0.$ \medskip \\ Continuity follows from Theorem \ref{thrm: continuity of rho} by using dominated convergence. For the final claim, if $t>t_\mathrm{g}$ then at least one of $M_t, E_t$ is strictly positive. If $M_t>0$, then by (\ref{eq: formula for M, E}), $\rho_t$ is not $0$ $m$-almost everywhere, and by (A4.), \begin{equation} m\left(v: \frac{1}{2}|v|^2\rho_t(v)>0\right)>0 \end{equation} and it follows that $E_t>0$. The case where $E_t>0$ is identical.   \end{proof}


\subsection{\textbf{Gel Dynamics Beyond the Critical Time}} We now obtain point 4 of Theorem \ref{thrm: Smoluchowski equation}  as a consequence of the previous results. We have already proven the continuity of $M_t, E_t$ on the whole time interval $[0,\infty)$ and the finiteness of the second moments $q_t=(\langle  \pi_n^2, \mu_t\rangle, \langle  \pi_n\pi_e, \mu_t\rangle, \langle  \pi_e^2, \mu_t\rangle)$ in the supercritical regime. Therefore, it is sufficient to prove the following result.
\begin{lem}\label{lemma: dynamics after tgel} In the notation of Lemma \ref{lemma: second moment finite after tgel}, let $(M_t, E_t)$ be the mass and energy of the gel associated to $(\mu_t)_{t\ge 0}$. Then, for  $t\ge t_\mathrm{g}$, we have
\begin{equation}
    M_t=\int_{t_\mathrm{g}}^t 
    \left(
      \kappa \left<\pi_n^2,\mu_s\right>M_s +
      2\gamma \left[
        \left<\pi_n \pi_e,\mu_s \right>M_s +
        \left<\pi_n^2,\mu_s \right>E_s \right]
    \right)ds;
\end{equation}
\begin{equation}
    E_t=\int_{t_\mathrm{g}}^t 
    \left(
      \kappa \left<\pi_n \pi_e,\mu_s\right>M_s +
      2\gamma \left[
        \left<\pi_e^2,\mu_s \right>M_s +
        \left<\pi_n \pi_e,\mu_s \right>E_s \right]
    \right)ds.
\end{equation}\end{lem} \begin{rmk}\label{rmk: continuity of tgelt} We interpret Lemma \ref{lemma: dynamics after tgel} as as saying that the growth of the gel away from $t_\mathrm{g}$ is due entirely to the absorption of finite clusters, rather than an additional blow-up due to coagulation of small particles. This may be expected following the relationship between gelation and blowup of the second moment $\mathcal{E}(t)$ in Lemma \ref{lemma: second moment before tgel}, and the finiteness of $\mathcal{E}$ in the supercritical regime. \end{rmk}
\begin{proof} We return to the truncated dynamics (\ref{eq:rE1}, \ref{eq: rE2}) used in the proof of Lemma \ref{lemma: E and U}. We recall that, starting at \begin{equation} \mu^\xi_0 = 1_{S_\xi}\mu_0; \hspace{1cm} g^\xi_0=\int_{x\not\in S_\xi} x\mu_0(dx)\end{equation} the solution $(\mu^\xi_t, g^\xi_t)$ to (\ref{eq:rE1}, \ref{eq: rE2}) exists and is unique, and as $\xi\uparrow \infty$, we have \begin{equation} \label{eq: convergence to E+G}
    \mu^\xi_t\uparrow \mu_t; \hspace{1cm} (M^\xi_t, E^\xi_t)\downarrow (M_t, E_t)
\end{equation} where $(\mu_t)_{t\ge 0}$ is the solution to (\ref{eq: E+G}) starting at $\mu_0$, and $(M_t, E_t)$ are the associated gel data. \medskip \\ Fix $s, t$ such that $t_\mathrm{g}<s<t$. It is immediate from (\ref{eq: rE2}) that \begin{equation}
    M^\xi_t-M^\xi_s=\int_{s}^t 
    \left(
      \kappa \left<\pi_n^2,\mu^\xi_u\right>M^\xi_u +
      2\gamma \left[
        \left<\pi_n \pi_e,\mu^\xi_u \right>M^\xi_u +
        \left<\pi_n^2,\mu^\xi_u \right>E^\xi_u \right]
    \right)du;
\end{equation}
\begin{equation}
    E^\xi_t-E^\xi_s=\int_{s}^t 
    \left(
      \kappa \left<\pi_n \pi_e,\mu^\xi_u\right>M^\xi_u +
      2\gamma \left[
        \left<\pi_e^2,\mu^\xi_u \right>M^\xi_u +
        \left<\pi_n \pi_e,\mu^\xi_u \right>E^\xi_u \right]
    \right)du.
\end{equation} By the monotonicity $\mu^\xi_u \le \mu_u$, and local boundedness in Lemma \ref{lemma: second moment finite after tgel},  $\langle \varphi^2, \mu^\xi_u\rangle $ is bounded, uniformly in $\xi<\infty$ and $u\in [s,t]$. It is also  straightforward to see that the truncated gel data are bounded by $M^\xi_u \le 1;\hspace{0.2cm} E^\xi_u \le \langle \pi_e, \mu_u\rangle.$ Together, these imply that the integrands are bounded. Using (\ref{eq: convergence to E+G}) and bounded convergence, we take the limit $\xi \rightarrow \infty$ to obtain  \begin{equation}
    M_t-M_s=\int_{s}^t 
    \left(
      \kappa \left<\pi_n^2,\mu_u\right>M^\xi_u +
      2\gamma \left[
        \left<\pi_n \pi_e,\mu_u \right>M_u +
        \left<\pi_n^2,\mu_u \right>E_u \right]
    \right)du;
\end{equation}
\begin{equation}
    E_t-E_s=\int_{s}^t 
    \left(
      \kappa \left<\pi_n \pi_e,\mu_u\right>M_u +
      2\gamma \left[
        \left<\pi_e^2,\mu_u \right>M_u +
        \left<\pi_n \pi_e,\mu_u \right>E_u \right]
    \right)du.
\end{equation} Taking $s\downarrow t_\mathrm{g}$, and using the continuity $(M_s, E_s)\downarrow (0,0)$ established in Lemma \ref{lemma: representation of M, E}, we obtain the claimed result. \end{proof}
\section{\textbf{Uniform Convergence of the Stochastic Coagulant}} \label{sec: uniform convergence} We now show how previous results, describing the dynamics of $E_t, M_t$, imply convergence to the limiting values \begin{equation} M_t\rightarrow 1;\hspace{1cm} E_t\rightarrow \frac{1}{2}\sigma_2(m) \end{equation} at an exponential rate.  From this, we will be able to upgrade the previous result, Lemma \ref{lemma: local uniform convergence of stochastic coagulent}, on the convergence of the stochastic coagulant to \emph{uniform} convergence. \begin{lem}\label{lemma: M and E at infinity} Let $\mu_0$ be an initial measure satisfying (A1-4.) for a probability measure $m$, and let $M_t, E_t$ be the mass and energy of the associated gel. As $t\uparrow \infty$, we have \begin{equation}
    M_t\uparrow 1; \hspace{1cm}E_t\uparrow  \frac{1}{2}\sigma_2(m)
\end{equation} where $m$ is given by (A2.). \end{lem} \begin{proof} We recall that at least one of the rate parameters $\kappa, \gamma$ is strictly positive. Therefore, it is sufficient to prove the result in the cases where $\gamma>0$ and $\kappa \ge 0$ is arbitrary, or where $\kappa>0$ and $\gamma \ge 0$ is arbitrary.
\paragraph{1. ${\gamma>0}$.} We deal first with the case where $\gamma>0$ and $\kappa \ge 0$ is arbitrary. Choose $t_0>t_\text{gel}$, and let $\lambda=E_{t_0}$; it follows from Lemma \ref{lemma: representation of M, E} that $\lambda>0$, and since $E_t$ is increasing, $E_t\ge \lambda>0$ for all $t\ge t_0$. By Lemma \ref{lemma: dynamics after tgel}, for all $t\ge t_0$, we have \begin{equation} \begin{split} M_t\ge M_{t_0}+\int_{t_0}^t 2\gamma \langle \pi_n^2, \mu_s\rangle E_s ds & \ge M_{t_0}+2\lambda \gamma \int_{t_0}^t \langle \pi_n, \mu_s\rangle ds \\ & = M_{t_0}+2\lambda \gamma \int_{t_0}^t (1-M_s)ds\end{split}  \end{equation} where the second inequality uses the fact that $\pi_n \ge 1$ everywhere, and the final equality is the definition of $M_t$. It follows that, for all $t\ge t_0$, \begin{equation} M_t \ge 1-e^{-2\lambda \gamma (t-t_0)}(1-M_{t_0}). \end{equation} From the definition, it is immediate that $M_t\le 1$, which implies that $M_t\rightarrow 1$ as claimed. A similar argument for the energy shows that \begin{equation} E_t \ge E_{t_0} + 2\lambda \gamma  \int_{t_0}^t(\langle \pi_e, \mu_0\rangle -\langle \pi_e, \mu_s\rangle)ds\end{equation} from which it follows that, for $t\ge t_0$, \begin{equation} E_t\ge \langle \pi_e, \mu_0\rangle-e^{-2\lambda\gamma(t-t_0)}(\langle \pi_e, \mu_0\rangle - E_{t_0})  \end{equation}  which, as above, implies the claim.
\paragraph{2. ${\kappa>0}$.} The case $\kappa>0$ is essentially identical. Let $t_0>t_\text{gel}$,  and let $\lambda'$ be given instead by $\lambda'=M_{t_0}$. Then the same argument as above gives the inequalities, for all $t\ge t_0$, \begin{equation} M_t \ge M_{t_0}+\kappa \lambda'\int_{t_0}^t (1-M_s)ds; \end{equation} \begin{equation}  E_t \ge E_{t_0}+\kappa \lambda'\int_{t_0}^t (\langle \pi_e, \mu_0\rangle-E_s)ds \end{equation} which yield the claimed convergences as above. \end{proof} 
\begin{lem} \label{lemma: uniform convergence of coagulant} Let $\mu^N_t$ be the stochastic coagulants constructed in the introduction, with $N$ particles sampled independently from a probability measure $m$, and let $\mu_t$ be the corresponding solution to the Smoluchowski equation (\ref{eq: E+G}), with conditions (A1-4.). Then we have the \emph{uniform} convergence \begin{equation} \sup_{t\ge 0} \hspace{0.1cm} d_0(\mu^N_t, \mu_t) \rightarrow 0\end{equation} in probability.  \end{lem} 
\begin{proof} 

From the definition of the vague topology, it is sufficient to prove that, for any $f\in C_c(S)$ with $0\leq f \leq 1$, we have the uniform convergence
$\sup_{t\geq 0}\hspace{0.1cm} \langle f, \mu^N_t-\mu_t \rangle \rightarrow 0$ in probability.

Fix $\epsilon>0$. By Lemma~\ref{lemma: M and E at infinity}, we can find $t_+\in (t_\text{g}, \infty)$ such that $M_{t_+}>1-\frac{\epsilon}{2}.$
Let $A^1_N$ be the event $A^1_N=\{M^N_{t_+}>1-\frac{\epsilon}{2} \}$; by Lemma~\ref{lemma: WCOG}, it follows that $\mathbb{P}(A^1_N)\rightarrow 1$.
On this event, we have
\begin{equation} \begin{split}
\sup_{t\geq 0}\hspace{0.1cm} \langle f, \mu^N_t-\mu_t \rangle
&\leq
\sup_{0 \leq t \leq t_+}\hspace{0.1cm} \langle f, \mu^N_t-\mu_t \rangle + \sup_{t >t_+}\hspace{0.1cm} \langle f, \mu^N_t-\mu_t \rangle \\ &
\leq
\sup_{0 \leq t \leq t_+}\hspace{0.1cm} \langle f, \mu^N_t-\mu_t \rangle + \left(1 - M_{t_+}^N\right) + \left(1 - M_{t_+}\right)
\\ & \leq
\sup_{0 \leq t \leq t_+}\hspace{0.1cm} \langle f, \mu^N_t-\mu_t \rangle + \epsilon
\end{split} \end{equation}
and the first term converges to $0$ in probability by Lemma~\ref{lemma: local uniform convergence of stochastic coagulent}.

\iffalse
\begin{equation}
    \sup_{t\geq 0}\hspace{0.1cm} \langle f, \mu^N_t-\mu_t \rangle \rightarrow 0 \hspace{1cm} \text{in probability.}
\end{equation} Following Lemma \ref{lemma: coupling}, we let $(G^N_t)_{t\ge 0}$ be a random graph process coupled to $(\mu^N_t)_{t\ge 0}$. Without loss of generality, we assume that $\|f\|_\infty=1.$
Let $0<\epsilon<1$, by Lemma~\ref{lemma: M and E at infinity} one can find $t_+\in (t_\text{gel}, \infty)$ such that $M_{t_+}>1-\frac{\epsilon}{2}.$

Consider the events $
        A^1_N=\{M^N_{t_+}>1-\frac{\epsilon}{2} \}$; by Lemma \ref{lemma: WCOG}, $\mathbb{P}(A^1_N)\rightarrow 1$. Since $f$ is compactly supported, we can choose $R<\infty$ such that $\text{Supp}(f)\subset \{x\in S: \pi_n(x)<R\}$. We write $\widetilde{\mu}^N_t$ for the stochastic coagulant with the largest cluster removed, corresponding to the random graphs $\widetilde{G}^N_t$ in Theorem \ref{thrm: coupling supercritical and subcritical}. Then, for $N$ large enough, $(1-\frac{\epsilon}{2})N>R$, and so, on $A^1_N$, \begin{equation}
    \langle f,\mu^N_t\rangle = \langle f,\widetilde{\mu}^N_t\rangle+\frac{1}{N}f(Ng^N_t) =  \langle f,\widetilde{\mu}^N_t\rangle.
\end{equation}
On $A^1_N$, for all $t\ge t_+$, we have $|\langle f, \widetilde{\mu}^N_t\rangle | \le \langle \pi_n, \widetilde{\mu}^N_t\rangle = 1-M^N_t<\frac{\epsilon}{2}.$
For $t\geq t_+$, it is immediate that $|\langle f, \mu_t\rangle| \leq \langle \pi_n, \mu_t\rangle = 1-M_t <\frac{\epsilon}{2}.$
Therefore, on $A^1_N$, for all $N > \frac{2}{2-\epsilon}R$, we have \begin{equation} \begin{split}
    \sup_{t\ge t_+} |\langle f, \mu^N_t-\mu_t\rangle \le \epsilon.   \end{split}
\end{equation} Finally, let $A^2_N$ be the event \begin{equation}
    A^2_N=\left\{\sup_{t\leq t_+} |\langle f, \mu^N_t-\mu_t\rangle |<\epsilon\right\}.
\end{equation} Then $\mathbb{P}(A^2_N)\rightarrow 1$ by Lemma \ref{lemma: local uniform convergence of stochastic coagulent}. For $N > \frac{2}{2-\epsilon}R$, on the event $A^1_N\cap A^2_N$, we have $\sup_{t\geq 0} |\langle f, \mu^N_t-\mu_t\rangle |<\epsilon,$ which proves the claimed convergence.
\fi
\end{proof}
\section{\textbf{Behaviour Near the Critical Point}}\label{sec: BNCP} We now prove item 5 of Theorem \ref{thrm: Smoluchowski equation}, concerning the phase transition: we will show that the gel data $(M_t, E_t)$ have strictly positive right-derivatives at the gelation time $t_\mathrm{gel}$. We start from the nonlinear fixed point equation (\ref{eq: NLFP 1}), which we rewrite as \begin{equation}\label{eq: NLPF again}c_t=tF(c_t); \hspace{1cm} F\left(\begin{matrix} a \\ b \end{matrix}\right)=2\int_{\mathbb{R}^d}(1-e^{-a-b|v|^2})\left(\begin{matrix}\kappa+\gamma|v|^2 \\ \gamma  \end{matrix}\right)m(dv). \end{equation} The following proof is a modification of the arguments in \cite[Theorem 3.17]{BJR07}, which itself generalises an analagous, well-known result for the phase transition of Erd\H{o}s-R\'eyni graphs.
\begin{lem}\label{lemma: BNCP} Suppose that $\mu_0$ satisfies (A1-4.) for a probability measure $m$, and let $c_t=(a_t, b_t)$ be as in Lemma \ref{lemma: form of rho-t}.  Then $c_t$ is right-differentiable at $t_\mathrm{g}$, $a'_{t_\mathrm{g}^+}>0$, and \begin{equation}\label{eq: ratio of r derivatives} \lambda = \frac{b'_{t_\mathrm{g}^+}}{a'_{t_\mathrm{g}^+}}=\frac{\sqrt{\kappa^2+4\gamma(\kappa\sigma_2(m)+\gamma \sigma_4(m))}-\kappa}{2(\kappa\sigma_2(m)+\gamma\sigma_4(m))}. \end{equation}  \end{lem} \begin{proof} We first assume that $|v|$ is not constant $m$-almost everywhere, and equip $\mathbb{R}^2$ with the inner product \begin{equation} \left((a,b),(a',b')\right)_m=\int_{\mathbb{R}^d} (a+b|v|^2)(a+b'|v|^2)m(dv)  \end{equation} and write $|\cdot|_m$ for the associated norm.  Differentiating under the integral sign twice, and using (A3.), we write \begin{equation} F\left(\begin{matrix} a \\ b \end{matrix}\right) = \Lambda\left(\begin{matrix} a \\ b \end{matrix}\right)-\Sigma \left(\begin{matrix} a \\ b \end{matrix}\right) + R\left(\begin{matrix} a \\ b \end{matrix}\right) \end{equation} where $\Lambda(\cdot), \Sigma(\cdot)$ are the linear and quadratic terms, and $R$ is a remainder term:   \begin{gather} \label{eq: defn of L}
    \Lambda\left(\begin{matrix} a \\ b \end{matrix}\right)=2\left(\begin{matrix} \kappa+\gamma\sigma_2(m) & \kappa \sigma_2(m)+\gamma\sigma_4(m)\\ \gamma & \gamma \sigma_2(m) \end{matrix}\right)\left(\begin{matrix} a \\ b \end{matrix}\right);
\\[2ex]
    \label{eq: defn of B}\Sigma\left(\begin{matrix} a \\ b \end{matrix}\right)=\int_{\mathbb{R}^d}(a+b|v|^2)^2\left(\begin{matrix} \kappa+\gamma|v|^2 \\ \gamma \end{matrix}\right)m(dv); \\[2ex]
 \left|R(c)\right|_m = o\left(|c|^2_m\right) \text{as  }|c|\rightarrow 0.\end{gather} The signs here are chosen to guarantee that, if $c>0$, then $\Lambda c, B(c)>0$, and it is straightforward to verify that $\Lambda$ is self-adjoint with respect to $(\cdot,\cdot)_m$. We also note that we have already found the spectrum of $\Lambda$ in the computation in Lemma \ref{lemma: computation of tcrit}: $\Lambda$ has exactly two eigenvalues, of which the larger is $t_\mathrm{g}^{-1}$, and the corresponding $|\cdot|_m$-unit eigenvector is given by \begin{equation} \psi=(t_\mathrm{g}^{-2}/4+\gamma^2(\sigma_4(m)-\sigma_2(m)^2) )^{-1/2} \left(\begin{matrix} t_\mathrm{g}^{-1}/2-\gamma\sigma_2(m) \\ \gamma \end{matrix}\right). \end{equation} From Lemma \ref{lemma: form of rho-t}, Theorem \ref{thrm: RG1} and Theorem \ref{thrm: continuity of rho}, we know that $c_{t_\mathrm{g}}=0$, that $c_{t_\mathrm{g}+\epsilon}\in [0,\infty)^2\setminus \{(0,0)\}$ for all $\epsilon>0$, and that $t\mapsto c_t$ is continuous at $t_\mathrm{g}.$ \bigskip \\  It is straightforward to see that $\psi$ is an eigenvector of $\Lambda$ of eigenvalue $t_\mathrm{g}^{-1}$, and $|\psi|_m=1.$ Writing $\psi^\bot$ for the orthogonal compliment of $\text{Span}(\psi)$ with respect to $(\cdot, \cdot)_m$, it follows from the self-adjointness of $\Lambda$ that $\Lambda$ maps $\psi^\bot$ into itself. Moreover, for $t>t_\mathrm{g}$ small enough, $(t\Lambda-1)|_{\psi^\bot}$ is invertible, and that the operator norm $\|(t\Lambda-1)|_{\psi^\bot}^{-1}\|_{m\rightarrow m}$ is bounded as $t\downarrow t_\mathrm{g}$. \bigskip \\ Let $Q:\mathbb{R}^2\rightarrow\mathbb{R}^2$ be the orthogonal projection onto $\psi^\bot$ with respect to $(\cdot,\cdot)_m$, and write $c^*_t=Qc_t$ so that we have the orthogonal decomposition \begin{equation}\label{eq: orth dec} c_t=\alpha_t \psi + c^*_t \end{equation} for some $\alpha_t \in \mathbb{R}$. Noting that $\Lambda Q=Q\Lambda$, it follows from (\ref{eq: NLPF again}, \ref{eq: orth dec}) that \begin{equation} c^\star_t=Q(tF(c_t))=t\Lambda c^\star_t + tQ\left(-\Sigma(c_t)+R(c_t)\right). \end{equation} The function $-\Sigma(c)+R(c)$ is of quadratic growth as $|c|_m\rightarrow 0$, and using the invertibility of $(t\Lambda-I)|_{\psi^\bot}$ described above, it follows that there exists $\beta>0$ such that $
    |c^*_t|_m \le \beta |c_t|_m^2$
  whenever $|c_t|_m\le 1$. In turn, it follows that $|c_t|_m\sim \alpha_t$ as $t\downarrow t_\mathrm{g}.$ Now, using (\ref{eq: NLPF again}) and the self-adjointness of $\Lambda$, we obtain \begin{equation}
      \begin{split}
          \alpha_t&=(t_\mathrm{g}\Lambda\psi,c_t)_m=t_\mathrm{g}(\psi, \Lambda c_t)_m \\[1ex] &=\frac{t_\mathrm{g}}{t}(\psi, c_t)_m-t_\mathrm{g}\left(\psi,-\Sigma(c_t)+R(c_t)\right)_m \\[2ex]&=\frac{t_\mathrm{g}}{t}\alpha_t-t_\mathrm{g}\left(\psi,-\Sigma(c_t)+R(c_t)\right)_m.
      \end{split}
  \end{equation} We now expand to second order in $\alpha_t$; for clarity, we will number the error terms $\mathcal{T}^i_t.$ Since $|c_t|\sim \alpha_t$, it follows that that $|c^*_t|_m=\mathcal{O}(\alpha_t^2)$ and that $R(c_t)=o(\alpha_t^2)$. Expanding $\Sigma(c_t)$ using (\ref{eq: orth dec}),\begin{equation} -\Sigma(c_t)+R(c_t)=-\alpha_t^2\Sigma(\psi)+\mathcal{T}^1_t; \hspace{1cm} |\mathcal{T}^1_t|_m=o(\alpha_t^2). \end{equation} It therefore follows that \begin{equation}
      \begin{split}
          \alpha_t=t_\mathrm{g}\left(\frac{\alpha_t}{t}+\alpha_t^2(\psi, \Sigma(\psi))_m\right)+ \mathcal{T}^2_t; \hspace{1cm} \mathcal{T}^2_t=o(\alpha_t^2).
      \end{split} 
  \end{equation} For $t>t_\mathrm{g}$ small enough, $\alpha_t>0$, and we may rearrange to find \begin{equation} t-t_\mathrm{g}=t\hspace{0.1cm}t_\mathrm{g}\hspace{0.1cm}\alpha_t(\psi,\Sigma(\psi))+\mathcal{T}^3_t;\hspace{1cm} \mathcal{T}^3_t=o(\alpha_t) \end{equation}  and in particular $\alpha_t=\Theta(t-t_\mathrm{g})$ as $t\downarrow t_\mathrm{g}$, since $(\psi, \Sigma(\psi))_m>0.$ Finally, we obtain \begin{equation} \frac{\alpha_t}{t-t_\mathrm{g}}\rightarrow\frac{1}{t_\mathrm{g}^2(\psi,\Sigma(\psi))_m}\hspace{1cm} \text{as }t\downarrow t_\mathrm{g}.  \end{equation} The calculations above show that $|c_t-\alpha_t\psi|=\mathcal{O}((t-t_\mathrm{g})^2)$, and the claimed right-differentiability now follows. Observing that $\alpha'_{t_\mathrm{g}^+}>0$ and that the first component of $\psi$ is strictly positive, it follows that $a'_{t_\mathrm{g}^+}>0$ as claimed. Finally, the expression (\ref{eq: ratio of r derivatives}) for the ratio of the right-derivatives follows from the definition of $\psi$ with some elementary algebra, and using the computation of $t_\mathrm{g}$ in Lemma \ref{lemma: computation of tcrit}. \bigskip \\  We now deal with the degenerate case in which $|v|$ is constant almost everywhere; say,  $|v|=\upsilon$ for $m$-almost all $v$. In this case, let $\widetilde{a}_t=a_t+b_t\upsilon^2$; taking $v=(\upsilon, 0,....0)$ in (\ref{eq: NLPF again}), we obtain \begin{equation} \begin{split} \label{eq: degenerate NLFP}
 \widetilde{a}_t&=2\int_{\mathbb{R}^d} (1-e^{-a_t-b_t|w|^2})(\kappa+\gamma|w-(\upsilon,0...,0)|^2)m(dw) \\ & \hspace{1cm} = 2(\kappa+2\gamma\upsilon^2)(1-e^{-\widetilde{a}_t}).\end{split}\end{equation} This may be rearranged into the nonlinear fixed point equation for Erd\H{o}s-R\'eyni random graphs, and analysed using the standard argument, or viewed as a special case of (\ref{eq: NLPF again}) with  \begin{equation} \widetilde{\kappa}=\kappa+2\gamma\upsilon^2; \hspace{1cm}\widetilde{\gamma}=0. \end{equation} Either argument shows that $\widetilde{a}_t$ is right-differentiable at $t_\mathrm{g}$, with right-derivative \begin{equation} \widetilde{a}'_{t_\mathrm{g}^+}=\frac{2}{t_\mathrm{g}}=\frac{4}{\widetilde{\kappa}}.
 \end{equation} To go from $\widetilde{a}_t$ to the original parameters $(a_t, b_t)$, we observe that from (\ref{eq: NLPF again}), we have \begin{equation}\label{eq: NLFP from atilde to ab} \left(\begin{matrix} a_t\\b_t \end{matrix}\right)=2(1-e^{-\widetilde{a}_t})\left(\begin{matrix} \kappa+\gamma\upsilon^2 \\ \gamma \end{matrix}\right) \end{equation} and so the right-differentiability of $\widetilde{a}_t$ implies the right-differentiability of $c_t=(a_t, b_t)$ at $t_\mathrm{g}.$ Since $\upsilon>0$ by (A4.) and at least one of $\kappa, \gamma$ is strictly positive, it follows that $a'_{t_\mathrm{g}^+}>0$. Finally, (\ref{eq: NLFP from atilde to ab}) implies that \begin{equation} \frac{b'_{t_\mathrm{g}^+}}{a'_{t_\mathrm{g}^+}}=\frac{\gamma}{\kappa+\gamma\upsilon^2} \end{equation} which may be seen to coincide with claimed expression (\ref{eq: ratio of r derivatives}). \end{proof} We now show how this implies item 5 of Theorem \ref{thrm: Smoluchowski equation}. From  Lemmas \ref{lemma: form of rho-t}, \ref{lemma: representation of M, E}, we have that \begin{equation} M_t=\int_{\mathbb{R}^d}(1-e^{-a_t-b_t|v|^2})m(dv); \hspace{1cm} E_t=\int_{\mathbb{R}^d}|v|^2(1-e^{-a_t-b_t|v|^2})m(dv). \end{equation} Differentiating under the integral sign using hypothesis (A3.), we obtain \begin{equation}
    M_t=a_t+b_t\sigma_2(m)+o(a_t+b_t); \hspace{1cm} E_t=\frac{1}{2}a_t\sigma_2(m)+\frac{1}{2}b_t\sigma_4(m)+o(a_t+b_t).
\end{equation} From the previous result, we see that for $t>t_\mathrm{g}$, \begin{gather} M_t=(t-t_\mathrm{g})(a'_{t_\mathrm{g}^+}+b'_{t_\mathrm{g}^+}\sigma_2(m))+o(t-t_\mathrm{g}); \\[1ex] E_t=\frac{1}{2}(t-t_\mathrm{g})(a'_{t_\mathrm{g}^+}\sigma_2(m)+b'_{t_\mathrm{g}^+}\sigma_4(m))+o(t-t_\mathrm{g})\end{gather} which proves the desired right-differentiability, and the positivity of the right-derivatives. Since $M'_{t_\mathrm{g}^+}>0$, we may take a quotient and let $t\downarrow t_\mathrm{g}$ to obtain the claimed size-biasing effect. 
\section{\textbf{Convergence of the Gel}} \label{sec: COG}
Finally, we prove the remaining part of Theorem \ref{thrm: convergence of stochastic coagulent}, concerning the \emph{uniform} convergence of the stochastic gel, drawing on other results we have proven. We recall that $g^N_t=(M^N_t, P^N_t, E^N_t)$ are the normalised mass, momentum and energy of the cluster with the largest mass, corresponding to the largest component $\mathcal{C}_1(G^N_t).$ To conclude the proof of Theorem \ref{thrm: convergence of stochastic coagulent}, we must extend Lemma \ref{lemma: WCOG}, to show that $g^N_t\rightarrow g_t$ uniformly in time, in probability. \medskip \\ This subsection is structured as follows. We recall that, in the proof of Lemma \ref{lemma: WCOG}, we used the result on mesoscopic clusters from \cite{BJR07}: if $\xi_N\rightarrow \infty$ and $\frac{\xi_N}{N}\rightarrow 0$, then for all $t\ge 0$,  \begin{equation} \frac{1}{N}\sum_{j\ge 2: C_j(G^N_t)\ge \xi_N} C_j(G^N_t)\rightarrow 0 \end{equation} in probability. In the first subsection, we improve this to uniform convergence in probability, and collect some corollaries. In the second subsection, we then use these to prove the claimed convergence.
\subsection{\textbf{Bounds on the Mesoscopic Clusters}} The main result of this section is the following.
\begin{lem} \label{lemma: anomalous clusters} Let $G^N_t$ be the random graph process constructed in Section \ref{sec: coupling_to_random_graph}, with $N$ particles sampled independely according to a probability measure $m$ satisfying (A1-4.). For any sequence $\xi_N\rightarrow \infty$ with $\xi_N\ll N$, we have the convergence \begin{equation}
       \sup_{t \geq t_\mathrm{g}}\left[\frac{1}{N}\sum_{j\geq 2: C_j(G^N_t)\geq \xi_N} C_j(G^N_t)\right] \rightarrow 0 \hspace{1cm}\text{in probability.}\end{equation} 
        \end{lem} 
       
       We prove this lemma as follows. First, we prove uniform convergence  on compact subsets $I\subset (t_\mathrm{g}, \infty)$ in Lemma \ref{lemma: anomalous clusters 2}. We will then show how this may be extended to the whole interval $[t_\mathrm{g}, \infty)$. The subcritical case is also true (Lemma \ref{lemma: large clusters below criticality}), and we will then deduce analogous results for the momentum and energy.
       
\begin{lem}\label{lemma: anomalous clusters 2}
       Let $G^N_t$ and $\xi_N$ be as above. Fix a compact subset $I\subset (t_\mathrm{g}, \infty)$. Then we have the convergence \begin{equation} \sup_{t \in I}\left[\frac{1}{N}\sum_{j\geq 2: C_j(G^N_t)\geq \xi_N} C_j(G^N_t)\right] \rightarrow 0 \hspace{1cm}\text{in probability.}\end{equation}
\end{lem}
\begin{proof}[Proof of Lemma \ref{lemma: anomalous clusters 2}]  It is sufficient to show that for every $t>t_\mathrm{g}$ the claim holds for some $I$ of the form $I=(t_-, t_+) \subset (t_\mathrm{g}, \infty)$ containing $t$.
As in Theorem \ref{lemma: second moment finite after tgel}, let $\widehat{m}^t$ be the measure on $\mathbb{R}^d$ given by $\widehat{m}^t(dv)=(1-\rho_t(v))m(dv)$. We also let $\widehat{\mu}^t_0$ be the pushforward of $\widehat{m}^t$ by $v\mapsto (1,v,\frac{1}{2}|v|^2)$, and $\widehat{t_\mathrm{g}}(t)$ the gelation time of the solution $(\widehat{\mu}^t_s)_{s\ge 0}$ to (\ref{eq: E+G}) starting at $\widehat{\mu}^t_0$. We showed in the proof of Theorem \ref{lemma: second moment finite after tgel} that, for all $t>t_\mathrm{g}$,  $\widehat{t_\mathrm{g}}(t)>t$, and the map $t\mapsto \widehat{t_\mathrm{g}}(t)$ is continuous. Therefore, for any $t>t_\mathrm{g}$, we can choose $t_\pm$ such that
\begin{equation}
    t_\mathrm{g}<t_-<t<t_+<\widehat{t_\mathrm{g}}(t_-).
\end{equation} We form $\widetilde{G}^N_{t_-}$ from $G^N_{t-}$ by deleting all vertexes of the giant component of $C_1(G^N_{t_-})$. We now form a new graph, $\widetilde{G}^N_{t_-,t_+}$ by including all edges between vertexes of $\widetilde{G}^N_{t_-}$ which are present in the graph $G^N_{t_+}$. \medskip \\ From Theorem \ref{thrm: coupling supercritical and subcritical}, we can construct a generalised vertex space $\widehat{\mathcal{V}}$ and graphs $\widehat{G}^N_{t_-}\sim \mathcal{G}^{\widehat{\mathcal{V}}}(N,t_-K)$, with the same vertex data as $\widetilde{G}^N_{t_-}$, such that \begin{equation}
    \mathbb{P}\left(\widehat{G}^N_{t_-}=\widetilde{G}^N_{t_-}\right)\rightarrow 1.
\end{equation} We now form $\widehat{G}^N_{t_-,t_+}$ by adding those edges present in $G^N_{t_+}$. By the Markov property of the graph process $(G^N_s)_{t\geq 0}$, these edges are independent of the construction of $\widehat{G}^N_{t_-}$, and so $\widehat{G}^N_{t_-,t_+}\sim \mathcal{G}^{\widehat{\mathcal{V}}}(N,t_+K)$. By the choices of $t_\pm$, $\widehat{G}^N_{t_-,t_+}$ is still subcritical, and by construction, \begin{equation}
    \mathbb{P}\left(\widehat{G}^N_{t_-, t_+}=\widetilde{G}^N_{t_-, t_+}\right)\rightarrow 1.
\end{equation} For $s\in [t_-, t_+]$, let $\mathcal{C}_1'(G^N_s)$ be the connected component of $G^N_s$ which contains $\mathcal{C}_1(G^N_{t_-})$. By considering the cases $\mathcal{C}_1'(G^N_s)=\mathcal{C}_1(G^N_s)$ and $\mathcal{C}_1'(G^N_s)\neq \mathcal{C}_1(G^N_s)$ separately, for all $s\in[t_-,t_+]$ we bound \begin{equation}
    \sum_{j\geq 2: C_j(G^N_s)\geq \xi_N} C_j(G^N_s) \leq \sum_{\substack{j\geq 1: C_j(G^N_s)\geq \xi_N \\[1ex] \mathcal{C}_j(G^N_s)\neq \mathcal{C}_1'(G^N_s)}} C_j(G^N_s)
\end{equation} Observe that we can rewrite the sum as \begin{equation}
    \sum_{\substack{j\geq 1: C_j(G^N_s)\geq \xi_N \\[1ex] \mathcal{C}_j(G^N_s)\neq \mathcal{C}_1'(G^N_s)}}  C_j(G^N_s) = \sum_{i=1}^N 1[C(i; G^N_s)\geq \xi_N; i \not \in \mathcal{C}'_1(G^N_s)].
\end{equation}  Fet $s\in [t_-, t_+]$, and $i\in\{1, 2,..,N\}$. Suppose that $C(i; G^N_s)\geq \xi_N$ and that $i \not \in \mathcal{C}'_1(G^N_s)$. It follows that $\mathcal{C}(i, G^N_s)$ is disjoint from $\mathcal{C}_1(G^N_{t_-})$, and so all vertexes of $\mathcal{C}(i, G^N_s)$ are present in $\widetilde{G}^N_{t_-,t_+}$. Moreover, all edges in $\mathcal{C}(i, G^N_s)$ are also present in $\widetilde{G}^N_{t_-,t_+}$, and so $\mathcal{C}(i, G^N_s) \subset \mathcal{C}(i, \widetilde{G}^N_{t_-,t_+})$. Therefore, for all $i$, and all $s\in [t_-, t_+]$ \begin{equation}
    1[C(i; G^N_s)\geq \xi_N; i \not \in \mathcal{C}'_1(G^N_t)] \leq 1[i \in V(\widetilde{G}^N_{t_-t_+}); C(i; \widetilde{G}^N_{t_-,t_+})\geq \xi_N].
\end{equation} Summing, we have the bound \begin{equation}\begin{split}
&\frac{1}{N}\sum_{j\geq 2: C_j(G^N_s)\geq \xi_N} C_j(G^N_s) \\&\hspace{3cm} \leq \frac{1}{N}\sum_{j\geq 1: C_j(\widetilde{G}^N_{t_-,t_+})\geq \xi_N} C_j(\widetilde{G}^N_{t_-,t_+}) 
\end{split} \end{equation} Therefore, \begin{equation}\begin{split}
    &\sup_{s\in [t_-, t_+]} \left[\frac{1}{N}\sum_{j\geq 2: C_j(G^N_s)\geq \xi_N} C_j(G^N_s)\right] \\[1ex] &\hspace{2cm} \leq \frac{1}{N}C_1(\widetilde{G}^N_{t_-, t_+})+\frac{1}{N}\sum_{j\geq 2: C_j(\widetilde{G}^N_{t_-,t_+})\geq \xi_N} C_j(\widetilde{G}^N_{t_-,t_+}).
\end{split} \end{equation} Since $\widehat{G}^N_{t_-,t_+}=\widetilde{G}^N_{t_-,t_+}$ with high probability, we obtain the bound \begin{equation}\begin{split}
    &\sup_{s\in [t_-, t_+]} \left[\frac{1}{N}\sum_{j\geq 2: C_j(G^N_s)\geq \xi_N} C_j(G^N_s)\right] \\[1ex]&\hspace{2cm} \leq \frac{1}{N}C_1(\widehat{G}^N_{t_-, t_+})+\frac{1}{N}\sum_{j\geq 2: C_j(\widehat{G}^N_{t_-,t_+})\geq \xi_N} C_j(\widehat{G}^N_{t_-,t_+})
\end{split} \end{equation} with high probability. The first term of the right-hand side converges to $0$ in probability  because $\widehat{G}^N_{t_-,t_+}$ is subcritical, and the second term converges to $0$ in probability by Theorem \ref{thrm: RG2}. \end{proof} 
\begin{proof}[Proof of Lemma \ref{lemma: anomalous clusters}] Fix $\epsilon >0$; without loss of generality, assume that $\epsilon<1.$ By continuity from Lemma \ref{lemma: representation of M, E} and Lemma \ref{lemma: M and E at infinity}, we can choose $t_\pm \in (t_\mathrm{g}, \infty)$ such that \begin{equation}
    M_{t_-}<\frac{\epsilon}{3};\hspace{1cm}M_{t_+}>1-\frac{\epsilon}{3}.
\end{equation} Consider now the events \begin{equation}
    A^1_N=\left\{M^N_{t_-}<\frac{2\epsilon}{3}; \hspace{0.2cm} M^N_{t_+}>1-\epsilon \right\};
\end{equation} \begin{equation}
    A^2_N=\left\{\frac{1}{N}\sum_{j\geq 2: C_j(G^N_{t_-})\geq \xi_N} C_j(G^N_{t_-})<\frac{\epsilon}{3} \right\}.
\end{equation} We know that $\mathbb{P}(A^1_N)\rightarrow 1 $ from Lemma \ref{lemma: WCOG}, and that $\mathbb{P}(A^2_N)\rightarrow 1$ from Theorem \ref{thrm: RG2}. On the event $A^1_N \cap A^2_N$, we bound as follows. 
\begin{enumerate}[label=\roman{*}).]
    \item For the initial interval $[t_\mathrm{g}, t_-]$, an argument similar to that of Lemma \ref{lemma: anomalous clusters 2} shows that, on this event,
 
    \begin{equation} \begin{split}
        \sup_{t\in [t_\mathrm{g}, t_-]} \left[\frac{1}{N} \sum_{j\geq 2: C_j(G^N_t)\geq \xi_N} C_j(G^N_t)\right] & \leq \frac{1}{N}\sum_{j\geq 1: C_j(G^N_{t_-}) \geq \xi_N} C_j(G^N_{t_-}) \\ & = M^N_{t_-}+\frac{1}{N}\sum_{j\geq 2: C_j(G^N_{t_-}) \geq \xi_N} C_j(G^N_{t_-})\\ & <\epsilon.
   \end{split} \end{equation}
    \item For late times $t\in [t_+, \infty)$, the largest cluster $\mathcal{C}_1(G^N_t)$ is at least the size of the cluster containing $\mathcal{C}_1(G^N_{t_+})$. Therefore, \begin{equation}
        \inf_{t\geq t_+} \frac{1}{N}C_1(G^N_t)\geq M^N_{t_+}>1-\epsilon
    \end{equation} and so 
    \begin{equation}\begin{split}
        \sup_{t\geq t_+} \left[\frac{1}{N} \sum_{j\geq 2: C_j(G^N_t)\geq \xi_N} C_j(G^N_t)\right] & \leq \sup_{t\geq t_+} \left[ \frac{1}{N} \sum_{j\geq 2} C_j(G^N_t)\right]  <\epsilon. \end{split}
    \end{equation}
\end{enumerate}
Now, consider the events
\begin{equation}
    A^3_N=\left\{\sup_{t\in [t_-, t_+]}\left[\frac{1}{N}\sum_{j\geq 2: C_j(G^N_{t})\geq \xi_N} C_j(G^N_{t})\right]<\epsilon \right\};\end{equation}
    \begin{equation}
    A_N=A^1_N\cap A^2_N\cap A^3_N.\end{equation} By Lemma \ref{lemma: anomalous clusters 2}, $\mathbb{P}(A^3_N)\rightarrow 1$, and so $\mathbb{P}(A_N) \rightarrow 1$. On the event $A_N$, we have \begin{equation}
        \sup_{t\geq t_\mathrm{g}}\left[\frac{1}{N}\hspace{0.1cm} \sum_{j\geq 2: C_j(G^N_t)\geq \xi_N} C_j(G^N_t)\right] <\epsilon
    \end{equation} which proves the claimed convergence in probability. 
\end{proof} 
We also remark that this is also true, and much simpler, in the subcritical and critical regimes $t\le t_\mathrm{g}:$ 
\begin{lem}\label{lemma: large clusters below criticality} Let $G^N_t$ and $\xi_N$ be as above. Then \begin{equation}
    \sup_{t\le t_\mathrm{g}} \left[\frac{1}{N}\sum_{j\geq 2: C_j(G^N_t)\ge \xi_N} C_j(G^N_t)\right]\rightarrow 0\hspace{1cm}\text{in probability}.
\end{equation} \end{lem}\begin{proof} Following a similar argument as in the supercritical case, we bound for $t\leq t_\mathrm{g}$, \begin{equation}
    \frac{1}{N}\sum_{j\geq 2: C_j(G^N_t)\geq \xi_N} C_j(G^N_t) \leq \frac{1}{N}\sum_{j\geq 1: C_j(G^N_{t_\mathrm{g}})\geq \xi_N} C_j(G^N_{t_\mathrm{g}}).
\end{equation} Therefore \begin{equation}
    \sup_{t\le t_\mathrm{g}} \left[\frac{1}{N}\sum_{j\geq 2: C_j(G^N_t)\geq \xi_N} C_j(G^N_t)\right] \leq  \frac{1}{N}C_1(G^N_{t_\mathrm{g}})+ \frac{1}{N}\sum_{j\geq 2: C_j(G^N_{t_\mathrm{g}})\geq \xi_N} C_j(G^N_{t_\mathrm{g}}).
\end{equation} By Theorems \ref{thrm: RG1}, \ref{thrm: RG2}, both terms converge to $0$ in probability. \end{proof}  
We now use these results to deduce an analogous result for the momentum and energy of these `mesoscopic clusters'. \begin{cor}\label{corr: anomalous clusters 3}  Let $G^N_t$ be as above. In the notation of (\ref{eq: cluster quantities}), the momentum and energy of the anomalous clusters satisfy \begin{equation}
           \sup_{t\ge 0} \left|\hspace{0.1cm}\frac{1}{N}\sum_{j\ge 2: C_j(G^N_t)\ge \xi_N} \hspace{0.1cm} P\left(\mathcal{C}_j(G^N_t)\right) \hspace{0.1cm}\right| \rightarrow 0 \hspace{1cm} \text{in probability};
       \end{equation} \begin{equation}
           \sup_{t\ge 0} \left[\hspace{0.1cm}\frac{1}{N}\sum_{j\ge 2: C_j(G^N_t)\ge \xi_N} \hspace{0.1cm} E\left(\mathcal{C}_j(G^N_t)\right) \hspace{0.1cm}\right] \rightarrow 0 \hspace{1cm} \text{in probability}.
       \end{equation}   \end{cor} \begin{proof} By the law of large numbers and (A3.), we can find a constant $\eta<\infty$ such that the events \begin{equation}
        A_N =\left\{\frac{1}{N}\sum_{i=1}^N |v_i|^2 \le \eta;\hspace{0.5cm} \frac{1}{N}\sum_{i=1}^N \frac{1}{4}|v_i|^4\le \eta\right\}
    \end{equation} have probability $\mathbb{P}(A_N)\rightarrow 1$. On this event, we use Cauchy-Schwarz to estimate, for any $t\ge 0$, \begin{equation}\begin{split} \label{eq: use of CS 1}\left|\hspace{0.1cm}\frac{1}{N}\sum_{j\geq 2: C_j(G^N_t)\geq \xi_N} \hspace{0.1cm}\sum_{i \in \mathcal{C}_j(G^N_t)} v_i \hspace{0.1cm}\right| & \leq \left(\frac{1}{N}\sum_{j\geq 2: C_j(G^N_t)\geq \xi_N} C_j(G^N_t)\right)^\frac{1}{2}\left(\frac{1}{N}\sum_{i=1}^N |v_i|^2\right)^\frac{1}{2} \\[2ex] & \le  \sqrt{\eta}\left(\frac{1}{N}\sum_{j\geq 2: C_j(G^N_t)\geq \xi_N} C_j(G^N_t)\right)^\frac{1}{2}; \end{split} \end{equation} and similarly \begin{equation} \label{eq: use of CS 2} \begin{split} \sum_{j\geq 2: C_j(G^N_t)\geq \xi_N} \hspace{0.1cm}\sum_{i \in \mathcal{C}_j(G^N_t)} \frac{1}{2}|v_i|^2 \leq \sqrt{\eta}\left(\frac{1}{N}\sum_{j\geq 2: C_j(G^N_t)\geq \xi_N} C_j(G^N_t)\right)^\frac{1}{2}. \end{split}\end{equation} In each case, the right-hand side converges uniformly to $0$ in probability by Lemmas \ref{lemma: anomalous clusters}, \ref{lemma: large clusters below criticality}, which proves the claimed convergence. \end{proof} 

  \subsection{\textbf{Proof of Theorem \ref{thrm: convergence of stochastic coagulent}}}
   We can now prove the claimed convergence of the gel. The result is broken up into several Lemmas, which together prove the remainder of Theorem \ref{thrm: convergence of stochastic coagulent}.
  \begin{lem}\label{lemma: COG} Let $\mu_0$ be a measure satisfying ({A1}-{4}.) for a probability measure $m$, and let $g_t=(M_t,P_t,E_t)$ be the mass, momentum, and energy of the gel in the solution $(\mu_t)_{t\ge 0}$ to (\ref{eq: E+G}). Let $(G^N_t)_{t\ge 0}$ be the graph process constructed in Section \ref{sec: coupling_to_random_graph}, with $N$ particles sampled independely for the same base measure $m$, and write $g^N_t=(M^N_t, P^N_t, E^N_t)$ for the data of the stochastic gel: \begin{equation} g^N_t=\frac{1}{N}(M(\mathcal{C}_1(G^N_t),P(\mathcal{C}_1(G^N_t),E(\mathcal{C}_1(G^N_t))=\frac{1}{N}\sum_{i \in \mathcal{C}_1(G^N_t)} \left(1,v_i,\frac{1}{2}|v_i|^2\right). \end{equation} Then we have uniform convergence in probability: \begin{equation} \sup_{t\ge 0}\hspace{0.1cm}\left|g^N_t-g_t\right|\rightarrow 0 \hspace{1cm}\text{ in probability.}\end{equation}  \end{lem} Combined with Lemma \ref{lemma: uniform convergence of coagulant}, this proves the first assertion of Theorem \ref{thrm: convergence of stochastic coagulent}. \begin{proof} The following proof is very similar to that of Lemma \ref{lemma: WCOG}, with suitable modifications to work in a uniform setting; for ease of readability, we will recall all necessary constructions.   Throughout, we let $(\mu^N_t)_{t\geq 0}$ be a stochastic coagulant coupled to a random graphs process $(G^N_t)_{t\geq 0}$, as described in Section \ref{sec: coupling_to_random_graph}. We write $(v_i)_{i=1}^N$ for the velocities associated to the graph vertexes. \medskip \\ Throughout, we will deal with the mass $M^N_t$; we will discuss the necessary modifications for momentum $P^N_t$ and energy $E^N_t$ at the end of the proof. \medskip \\ We will deal first with the critical and supercritical regimes $t\ge t_\mathrm{g}$. Following the proof of Lemma \ref{lemma: WCOG}, let $\xi_N$ be a sequence, to be constructed later, such that \begin{equation}\label{eq: choice of xiN}
       \xi_N\rightarrow \infty; \hspace{1cm} \frac{\xi_N}{N}\rightarrow 0; \hspace{1cm}\mathbb{P}(C_1(G^N_{t_\mathrm{g}})\geq \xi_N)\rightarrow 1.
   \end{equation}  We recall the following construction of `bump functions' from Lemma \ref{lemma: WCOG}. First, using Lemma \ref{lemma: STUI}, we construct a sequence $\eta_r \rightarrow \infty$ such that \begin{equation}\label{eq: choice of etar 1}
    \beta(r, \eta_r):=\sup_{N\geq 1} \hspace{0.1cm} \mathbb{E}\left[\sup_{t\geq 0}\hspace{0.1cm}\langle \varphi 1[\pi_e(x)>\eta_r, \pi_n(x)< r], \mu^N_t\rangle \right]\rightarrow 0;
\end{equation} We let $S_{(r)}$ be the set \begin{equation}
       \{x: \pi_n(x)< r, |\pi_p(x)|\leq \sqrt{2r\eta_r}, \pi_e(x)\leq \eta_r\}.
   \end{equation} Let $\widetilde{g}_r$ be the indicator $\widetilde{g}_r=1[\pi_n(x)< r]$, and construct a continuous, compactly supported function $\widetilde{f}_r$ such that \begin{equation}
      0\leq \widetilde{f}_r\leq 1;\hspace{1cm} \widetilde{f}_r=1 \hspace{0.1cm} \text{ on } S_{(r)};\hspace{1cm} \widetilde{f}_r(x)=0 \hspace{0.1cm} \text{ if } \pi_n(x)\ge r.
   \end{equation} By symmetrising if necessary, we also assume that $f_r(Rx)=f_r(x)$, for all $x\in S$. We define $f_N=\widetilde{f}_{\xi_N}$ and $g_N=\widetilde{g}_{\xi_N}$. \medskip \\ As in Lemma \ref{lemma: WCOG}, we can decompose the difference $M^N_t-M_t$ as \begin{equation}\label{eq: decomposition of erorr}\begin{split} M^N_t-M_t &= \underbrace{(1-M_t-\langle \pi_n f_N, \mu_t\rangle)}_{:=\mathcal{T}^1_N(t)} + \underbrace{\langle \pi_n f_N, \mu_t-\mu^N_t\rangle}_{:=\mathcal{T}^2_N(t)} \\& \hspace{2cm}+ \underbrace{\langle \pi_n (f_n-g_n), \mu^N_t\rangle}_{:=\mathcal{T}^3_N(t)} +\underbrace{(
   \langle \pi_n g_N, \mu^N_t\rangle - (1-M^N_t)).}_{:=\mathcal{T}^4_N(t)}\end{split} \end{equation} We now modify the argument of Lemma \ref{lemma: WCOG} to estimate the errors $\mathcal{T}^i_N$, $i=1,3,4$, \emph{uniformly in $t\ge t_\mathrm{g}$}.  As in Lemma \ref{lemma: WCOG}, we ensure uniform convergence of $\mathcal{T}^2_N$ later with a particular construction of the sequence $\xi_N$. \paragraph{1. Estimate on $\mathcal{T}^1_N$.} Let $h_N=1_{S_{(\xi_N)}}$, so that $h_N \le f_N \le 1$. As $N\rightarrow \infty$, $\pi_n h_N \uparrow \pi_n$, and so by monotone convergence, \begin{equation}
       \langle \pi_n h_N, \mu_t\rangle \uparrow \langle \pi_n, \mu_t\rangle =1-M_t.
   \end{equation} Moreover, each function $t\mapsto \langle \pi_n h_N, \mu_t\rangle$ is continuous on $[0,\infty)$ by the definition (\ref{eq: E+G}) of the Smoluchowski dynamics, and the limit function $t\mapsto 1-M_t$ is continuous on $[0,\infty)$ by Lemma \ref{lemma: representation of M, E}.  We extend the maps $t\mapsto \langle \pi_n h_N, \mu_t\rangle, t\mapsto \langle \pi_n, \mu_t\rangle$ to the compactification $[0,\infty]$, by defining both to be equal to $0$ at infinity. By Lemma \ref{lemma: M and E at infinity}, it follows that $\langle \pi_n, \mu_t\rangle, \langle \pi_n h_N, \mu_t\rangle \rightarrow 0$ as $t\rightarrow \infty$, so both of these extensions are continuous on the whole compactification $[0,\infty].$  Therefore, by Dini's Theorem, the convergence is uniform in $t\geq t_\mathrm{g}$, and since \begin{equation}
       \langle \pi_n h_N, \mu_t\rangle \le \langle \pi_n f_N, \mu_t \rangle \le \langle \pi_n, \mu_t\rangle = 1-M_t, 
   \end{equation} it follows that $\mathcal{T}^1_N(t) \rightarrow 0$, uniformly in $t\geq t_\mathrm{g}$.
   \paragraph{2. Estimate on $\mathcal{T}^3_N$.} From the definitions of $f_N, g_N$, we observe that \begin{equation}
       |\mathcal{T}^3_N(t)|=\langle \pi_n(g_N-f_N), \mu^N_t\rangle \le  \langle \pi_n 1[\pi_n(x)<\xi_N, \pi_e(x)>\eta_{\xi_N}], \mu^N_t\rangle.
   \end{equation} Therefore, in the notation of (\ref{eq: choice of etar 1}), \begin{equation}
       \mathbb{E}\left[\sup_{t\geq t_\mathrm{g}} |\mathcal{T}^3_N(t)|\right] \leq \beta(\xi_N, \eta_{\xi_N}).
   \end{equation} By construction of $\eta_r$, and since $\xi_N \rightarrow \infty$, the right hand side converges to $0$. Therefore, $\mathcal{T}^3_N$ converges to $0$, uniformly in probability.
       \paragraph{3. Estimate on $\mathcal{T}^4_N$.} By the choice (\ref{eq: choice of xiN}) of $\xi_N$, we have that $\mathbb{P}(\forall  t\geq t_\mathrm{g}, C_1(G^N_t)\geq \xi_N)\rightarrow 1.$ On this event, we have the equality \begin{equation}
           \begin{split}
               \langle \pi_n g_N, \mu^N_t\rangle &=\langle \pi_n, \mu^N_t\rangle - \langle \pi_n 1[\pi_n\geq \xi_N], \mu^N_t\rangle  \\[2ex] & = 1-M^N_t-\frac{1}{N}\sum_{j\ge 2:C_j(G^N_t)\ge \xi_N} C_j(G^N_t). 
           \end{split} 
       \end{equation} where $(G^N_t)_{t\geq 0}$ is the random graph process coupled to the stochastic coagulant. Therefore, with high probability, for all $t\ge t_\mathrm{g}$, \begin{equation} \mathcal{T}^4_N(t) = \frac{1}{N}\sum_{j\ge 2:C_j(G^N_t)\ge \xi_N} C_j(G^N_t) \end{equation} which converges to $0$, uniformly in probability on $t\geq t_\mathrm{g}$, by Lemma \ref{lemma: anomalous clusters}.
       \paragraph{4. Construction of $\xi_N$, and convergence of $\mathcal{T}^2_N$.} To conclude the proof of the supercritical case, it remains to show how a sequence $\xi_N$ can be constructed such that $\mathcal{T}^2_N \rightarrow 0$ uniformly, in probability. Let $A^1_{r,N}, A^2_{r,N}$ be the events \begin{equation} \label{eq: definition of A1rn}
       A^1_{r,N}=\left\{\sup_{t\geq 0} |\langle \pi_n \widetilde{f}_r, \mu^N_t-\mu_t\rangle|<\frac{1}{r}\right\}; \hspace{1cm}
       A^2_{r,N}=\left\{C_1(G^N_{t_\mathrm{g}}) \geq r\right\}.
   \end{equation} Then, as $N\rightarrow \infty$, both $\mathbb{P}(A^1_{r,N}), \mathbb{P}(A^2_{r,N}) \rightarrow 1$, by Lemmas \ref{lemma: uniform convergence of coagulant}, \ref{lemma: lower bound on largest cluster}. We now define $N_r$ inductively for $r\geq 1$ inductively, as in Lemma \ref{lemma: WCOG}, by setting $N_1=1$ and letting $N_{r+1}$  be the minimal $N>N_r$ such that, for all $N'\ge N$, \begin{equation}
       \label{eq: recursive definition of Nr} N\geq \max((r+1)^2, N_r+1);\hspace{1cm}  \mathbb{P}(A^1_{r+1,N})>\frac{r}{r+1};\hspace{1cm}  \mathbb{P}(A^2_{r+1,N})>\frac{r}{r+1}.
   \end{equation} Now, we set $\xi_N=r$ for $N\in [N_r, N_{r+1})\cap\mathbb{N}.$ It follows that $\xi_N \rightarrow \infty$ and $\xi_N\leq \sqrt{N}\ll N$, and \begin{equation}
       \mathbb{P}\left(C_1(G^N_{t_\mathrm{g}}))\geq \xi_N\right)\ge 1-\frac{1}{\xi_N} \rightarrow 1. 
   \end{equation} Therefore, $\xi_N$ satisfies the requirements (\ref{eq: choice of xiN}) above. Moreover, \begin{equation}
       \mathbb{P}\left(\sup_{t\geq t_\mathrm{g}} |\mathcal{T}^2_N| <\frac{1}{\xi_N}\right) \ge \mathbb{P}\left(A^1_{\xi_N,N}\right) > 1-\frac{1}{\xi_N}\rightarrow 1
   \end{equation} and so, with this choice of $\xi_N$, $\mathcal{T}^2_N \rightarrow 0$ uniformly in probability on $t\ge t_\mathrm{g}.$ \bigskip \\ This concludes the argument for the mass $M^N_t$ of the stochastic gel, in the critical and supercritical regime $t\ge t_\mathrm{g}$. The cases for the momentum $P^N_t$ and energy $E^N_t$ are essentially identical, with the following minor differences. \begin{enumerate}[label=\roman{*}).]
       \item For the momentum, by symmetry under $R$ we have \begin{equation}
           \langle \pi_p f_N, \mu_t\rangle =0; \hspace{0.5cm} \langle \pi_p g_N, \mu_t\rangle =0.
       \end{equation} Therefore, for the momentum term, $\mathcal{T}^3_N$ is identically $0$. For the energy, we argue by Dini as above.
       \item For the momentum, we relate the error term $\mathcal{T}^3_N$ to the equivalent terms for the mass and energy, using the bound $|\pi_p(x)|\le\sqrt{2\pi_n(x)\pi_e(x)} \le \varphi(x)$ by Cauchy-Schwarz. In the notation of (\ref{eq: choice of etar 1}), this produces the bound \begin{equation}
           \mathbb{E}\left[\sup_{t\geq t_\mathrm{g}} |\mathcal{T}^3_N(t)|\right] \leq \beta(\xi_N,\eta_{\xi_N})\rightarrow 0.
       \end{equation}
       \item In the decomposition (\ref{eq: decomposition of erorr}) for the momentum (resp. energy) terms, there is an additional error \begin{equation}
           \mathcal{T}^5_N = \langle \pi_p, \mu^N_0-\mu_0\rangle \hspace{1cm}\left(\text{resp. } \langle \pi_e, \mu^N_0-\mu_0\rangle\right).
       \end{equation} Each of these converge to $0$ in probability, by the law of large numbers.
   \end{enumerate} \bigskip  To conclude the proof for $M^N_t$, we show uniform convergence in the subcritical phase $t<t_\mathrm{g}$. In this region, we have $M_t=E_t=0$ and $P_t=0$. Hence \begin{equation}
       \sup_{t<t_\mathrm{g}} \left[|M^N_t-M_t|\right] = \sup_{t<t_\mathrm{g}}\left[ \frac{1}{N}C_1(G^N_t)\right]=\frac{1}{N}C_1(G^N_{t_\mathrm{g}}) \rightarrow 0
   \end{equation} in probability, by Theorem \ref{thrm: RG1}. The cases for momentum and energy are similar, using Cauchy-Schwarz as in (\ref{eq: use of CS 1}, \ref{eq: use of CS 2}). \end{proof} 
   
  In order to complete the proof of Theorem \ref{thrm: convergence of stochastic coagulent}, we wish to prove a similar result, when $g^N_t$ is replaced by cutting off at a suitable deterministic scale $\xi_N$: \begin{equation}\label{eq: alternative stochastic gel} \begin{split} \widetilde{g}^N_t&=(\widetilde{M}^N_t,\widetilde{P}^N_t,\widetilde{E}^N_t)\\& =\frac{1}{N}\sum_{j\ge 1: C_j(G^N_t)\ge \xi_N}(M(\mathcal{C}_j(G^N_t)),P(\mathcal{C}_j(G^N_t)),E(\mathcal{C}_j(G^N_t))) \\[1ex] & = \left(\langle \pi_n1[\pi_n\ge \xi_N], \mu^N_t\rangle,\langle \pi_p1[\pi_n\ge \xi_N], \mu^N_t\rangle,\langle \pi_e1[\pi_n\ge \xi_N], \mu^N_t\rangle\right). \end{split} \end{equation} 
   Together with the previous lemma, it suffices to prove the following. \begin{lem} Let $\xi_N$ be a sequence such that \begin{equation} \xi_N\rightarrow \infty; \hspace{1cm} \frac{\xi_N}{N}\rightarrow 0. \end{equation} Let $g^N_t$ be as in Lemma \ref{lemma: COG}, and $\widetilde{g}^N_t$ be as in (\ref{eq: alternative stochastic gel}). Then \begin{equation}
       \sup_{t\ge 0}\hspace{0.1cm} \left|g^N_t-\widetilde{g}^N_t\right|\rightarrow 0\hspace{1cm}\text{in probability.}
   \end{equation} \end{lem} 
   \begin{proof} Let $\mathcal{K}_t$ be the symmetric diference of the clusters considered: \begin{equation}
      \mathcal{K}_t=\{1\}\triangle\left\{j: \hspace{0.1cm}C_j(G^N_t)\ge \xi_N \right\}=\begin{cases} \{1\} & \text{ if } C_1(G^N_t)< \xi_N; \\ \{j\ge 2: C_j(G^N_t) \ge \xi_N\} & \text{ if } C_1(G^N_t) \ge \xi_N \end{cases}  
  \end{equation} By considering the two cases separately, we observe that \begin{equation}
          \left|M^N_t-\widetilde{M}^N_t\right|=\frac{1}{N}\sum_{j\in\mathcal{K}_t} C_j(G^N_t) 
      \end{equation} with similar equalities for the momentum and energy. As usual, we deal with subcritical and supercritical cases separately; in this case, we will see that it is more natural to split at a time $t_+$ slightly larger than $t_\mathrm{g}.$\paragraph{1. Subcritical Case.} Let $\epsilon>0$. By continuity in Lemma \ref{lemma: representation of M, E}, we can choose $t_+>t_\mathrm{g}$ such that $M_{t_+}<\epsilon.$ Then, for $t\le t_+$, we bound \begin{equation}
          \left|M^N_t-\widetilde{M}^N_t\right| \le M^N_t +\frac{1}{N}\sum_{j\ge 2: C_j(G^N_t)\ge \xi_N} C_j(G^N_t).
      \end{equation} By Lemma \ref{lemma: COG}, the first term converges to $M_t <\epsilon$ uniformly in probability, and by Lemmas \ref{lemma: anomalous clusters}, \ref{lemma: large clusters below criticality}, the second term converges in probability to $0$, uniformly in $t\le t_+$. Therefore, the event \begin{equation}
          A^1_N=\left\{\sup_{t\le t_+} \left|M^N_t-\widetilde{M}^N_t\right|<\epsilon\right\}
      \end{equation} has $\mathbb{P}(A_N)\rightarrow 1.$ \paragraph{2. Supercritical Case.} Since $t_+>t_\mathrm{g}$, there is a giant component, and so $C_1(G^N_{t_+})>\xi_N$ with high probability. On this event, for all $t\ge t_+$, we have \begin{equation}
          \left|M^N_t-\widetilde{M}^N_t\right|=\frac{1}{N}\sum_{j\ge 2: C_j(G^N_t)\ge \xi_N} C_j(G^N_t)
      \end{equation} which converges to $0$, uniformly in probability, by Lemma \ref{lemma: anomalous clusters}. Therefore, \begin{equation}
          A^2_N=\left\{\sup_{t\ge t_+} \left|M^N_t-\widetilde{M}^N_t\right|<\epsilon \right\}
      \end{equation} has probability $\mathbb{P}(A^2_N)\rightarrow 1$. \medskip \\ Combining the two cases above, we have shown that \begin{equation}
          \mathbb{P}\left(\sup_{t\ge 0}\hspace{0.1cm}\left|M^N_t-\widetilde{M}^N_t\right|>\epsilon\right)\rightarrow 1
      \end{equation} as desired. For the momentum and energy, we bound \begin{equation}
          \left| P^N_t-\widetilde{P}^N_t \right| \le \frac{1}{N} \sum_{j\in\mathcal{K}_t} \left|P\left(\mathcal{C}_j(G^N_t)\right)\right|;\hspace{1cm}\left| E^N_t-\widetilde{E}^N_t \right| \le \frac{1}{N} \sum_{j\in\mathcal{K}_t} \left|E\left(\mathcal{C}_j(G^N_t)\right)\right|.
      \end{equation} By using the case for the mass, and using Cauchy-Schwartz as in (\ref{eq: use of CS 1}, \ref{eq: use of CS 2}), we see that both right-hand-sides converge to $0$, uniformly in probability, as $N\rightarrow \infty.$ \end{proof}
   
   
  


%\begin{acknowledgements}
%If you'd like to thank anyone, place your comments here
%and remove the percent signs.
%\end{acknowledgements}

% BibTeX users please use one of
%\bibliographystyle{spbasic}      % basic style, author-year citations
%\bibliographystyle{spmpsci}      % mathematics and physical sciences
%\bibliographystyle{spphys}       % APS-like style for physics
%\bibliography{}   % name your BibTeX data base

% Non-BibTeX users please use
\begin{thebibliography}{}
%
% and use \bibitem to create references. Consult the Instructions
% for authors for reference list style.
%
\bibitem{RefJ}
% Format for Journal Reference
Author, Article title, Journal, Volume, page numbers (year)
% Format for books
\bibitem{RefB}
Author, Book title, page numbers. Publisher, place (year)

\bibitem{Ald16} Aldous, D., The Incipient Giant Component in Bond Percolation on General Finite Weighted graphs, Electronic Communications in Probability, pp21 (2016).

\bibitem{A99} Aldous, D.J., Deterministic and Stochastic Models for Coalescence (Aggregation and Coagulation): a Review of the Mean-Field Theory for Probabilists, Bernoulli, 5(1), pp.3-48 (1999).

\bibitem{BB84} Bollob\'as, B., The Evolution of Random Graphs, Transactions of the American Mathematical Society, 286(1), pp.257-274 (1984).

\bibitem{B01} Bollob\'as, B., Random graphs (No. 73), Cambridge University Press, Cambridge (2001).


\bibitem{BJR07} Bollob\'as, B., Janson, S. and Riordan, O., The Phase transition in Inhomogeneous Random Graphs, Random Structures \& Algorithms, 31(1), pp.3-122 (2007).

\bibitem{B96} Boltzmann, L., Vorlesungen \"uber Gastheorie (Vol. 1). JA Barth, Leipzig (1896).

\bibitem{BP91} Buffet, E. and Pul{\'e}, J. V., Polymers and Random Graphs, Journal of Statistical Physics, 64, pp.87--110 (1991).

\bibitem{D&N} Darling, R.W.R. and Norris, J.R., Differential Equation Approximations for Markov Chains. Probability surveys, 5, pp.37-79 (2008).

\bibitem{ER60} Erd\H{o}s, P. and R\'{e}nyi, A., On the Evolution of Random Graphs, Magyar Tud. Akad. Mat. Kutat\'{o} Int. K\"{o}zl, 5, pp.17-61 (1960).

\bibitem{EK86} Ethier, S.N. and Kurtz, T.G., Markov Processes: Characterization and Convergence Vol. 282. John Wiley \& Sons, New York (2009).


\bibitem{Flo41} Flory, P. J., Molecular Size Distribution in Three Dimensional Polymers {I}. Gelation, Journal of the American Chemical Society, 63, pp.3083-3090 (2008).


\bibitem{GKSZ08} Gabrielov, A., Keilis-Borok, V., Sinai, Y. and Zaliapin, I., Statistical Properties of the Cluster Dynamics of the Systems of Statistical Mechanics, Boltzmann's Legacy, pp.203-215 (2008).

\bibitem{Grunbaum} Gr\"unbaum, F.A., Propagation of Chaos for the Boltzmann Equation, Archive for Rational Mechanics and Analysis, 42(5), pp.323-345 (1971).

%\bibitem{HPR16} Heida, M., Patterson, R.I.A. and Renger, D.R.M., 2016. The space of bounded variation with infinite-dimensional codomain. Work in progress.

\bibitem{J86} Jakubowski, A., On the Skorokhod Topology, Ann. Inst. H. Poincar\'e Probab. Statist, 22(3), pp.263-285 (1986).

\bibitem{JLR} Janson, S., \L{}uczak, T. and Ruc\'{i}nski, A., Random Graphs, Vol. 45. John Wiley \& Sons, New York (2011).


\bibitem{J98} Jeon, I., Existence of Gelling Solutions for Coagulation-Fragmentation Equations, Communications in Mathematical Physics, 194, pp.541-567 (1998).

\bibitem{K56} Kac, M., Foundations of Kinetic Theory, Proceedings of the Third Berkeley Symposium on Mathematical Statistics and Probability, Vol. 3, pp. 171-197 (1956). 

\bibitem{K02} Kallenberg, O., Foundations of Modern Probability. Springer Science \& Business Media, New York (2006).

\bibitem{Lanford} Lanford III, O.E., On a Derivation of the Boltzmann Equation, Nonequilibrium Phenomena I: The Boltzmann Equation, p.1, 1986.

\bibitem{L75} Lanford, O.E., Time Evolution of Large Classical Systems, Dynamical systems, Theory and Applications, pp. 1-111. Springer, Berlin, Heidelberg (1975).

\bibitem{L35} Leontovich, M.A., Basic equations of the Kinetic Gas Theory from the Point of View of the Theory of Random Processes. Zh. Teoret. Ehksper. Fiz, 5, pp.211-231 (1935).

\bibitem{Lu} Lu, X. and Mouhot, C., On measure solutions of the Boltzmann Equation, Part I: Moment Production and Stability Estimates. Journal of Differential Equations, 252(4), pp.3305-3363 (2012).


\bibitem{L78} Lushnikov, A.A., Coagulation in Finite Systems, Journal of Colloid and interface science, 65(2), pp.276-285 (1978).

\bibitem{McKean} McKean Jr, H.P., An Exponential Formula for Solving Boltzmann's Equation for a Maxwellian Gas, Journal of Combinatorial Theory, 2(3), pp.358-382 (1967).

\bibitem{MischlerMouhot} Mischler, S. and Mouhot, C., Kac's Program in Kinetic Theory, Inventiones Mathematicae, 193(1), pp.1-147 (2013).


\bibitem{Nm09} Normand, R., A Model for Coagulation with Mating, Journal of Statistical Physics, 137, pp.343--371 (2009).

\bibitem{Nm11} Normand, R. and Zambotti, L., Uniqueness of Post-Gelation Solutions of a Class of Coagulation Equations, Annales de l'Institut Henri Poincar\'e (C) Non Linear Analysis, 28(2), pp.189-215 (2011).


\bibitem{N99} Norris, J.R., Smoluchowski's Coagulation Equation: Uniqueness, Nonuniqueness and a Hydrodynamic Limit for the Stochastic Coalescent, Annals of Applied Probability, pp.78-109 (1999).

\bibitem{N00} Norris, J.R., Cluster Coagulation, Communications in Mathematical Physics, 209(2), pp.407-435 (2000).

\bibitem{N16} Norris, J., A Consistency Estimate for Kac's Model of Elastic Collisions in a Dilute Gas, The Annals of Applied Probability, 26(2), pp.1029-1081 (2016).


\bibitem{PSW16} Patterson, R.I.A., Simonella, S. and Wagner, W., Kinetic Theory of Cluster Dynamics, Physica D: Nonlinear Phenomena, 335, pp.26-32 (2016).

\bibitem{PSW17} Patterson, R.I.A., Simonella, S. and Wagner, W., A Kinetic Equation for the Distribution of Interaction Clusters in Rarefied Gases, Journal of Statistical Physics, 169(1), pp.126-167 (2017).

\bibitem{PS17} Pulvirenti, M. and Simonella, S., The Boltzmann-Grad Limit of a Hard Sphere System: Analysis of the Correlation Error. Inventiones mathematicae, 207(3), pp.1135-1237 (2017).

\bibitem{vS16} von Smoluchowski, M., Drei Vortr\"age \"uber Diffusion, Brownsche Bewegung und Koagulation von Kolloidteilchen, Z. Phys., 17, pp.557-585 (1916).

\bibitem{Sznitman} Sznitman, A.S., \'Equations de type de Boltzmann, Spatialement Homogenes, Zeitschrift für Wahrscheinlichkeitstheorie und Verwandte Gebiete, 66(4), pp.559-592 (1984).

\bibitem{Villani} Villani, C., Cercignani's Conjecture is Sometimes True and Always Almost True, Communications in Mathematical Physics, 234(3), pp.455-490 (2003).


\bibitem{ZS80} Ziff, R.M. and Stell, G., Kinetics of Polymer Gelation, The Journal of Chemical Physics, 73(7), pp.3492-3499 (1980).
% etc
\end{thebibliography}

\end{document}
% end of file template.tex

