\subsection{Definitions}
\subsubsection{Markov jump processes}\label{s:jump_procs}
The Kac process with parameter $N$ defines Markovian dynamics for $N$ molecules, each with mass $N^{-1}$ and a velocity $v_i(t)\in\RR^d$, undergoing pairwise collisions that conserve momentum and energy. We write $B$ for the kernel on $\RR^d \times \mathbb{S}^{d-1}$ underlying the dynamics of the Kac process, and $\left(v_i^N(t)\right)_{i=1}^N$ for the state of the Kac process at time $t\ge 0$.
At time $t$ the instantaneous collision rate between an unordered pair of molecules $i$ and $j$ is given by $\frac2NB(v_i^N(t) - v_j^N(t), \mathbb{S}^{d-1})$, and the post-collision velocities are
\begin{equation*}
    \widetilde{v}_i^N(t)=\frac{v_i^N(t) + v_j^N(t)}{2} + \sigma \frac{|v_i^N(t) - v_j^N(t)|}{2}
    \qquad
    \widetilde{v}_j^N(t)=\frac{v_i^N(t) + v_j^N(t)}{2} - \sigma \frac{|v_i^N(t) - v_j^N(t)|}{2},
\end{equation*}
with $\sigma$ distributed according to $B(v_i^N(t) - v_j^N(t), \cdot)$.
For the purposes of this work we consider kernels whose total mass is of the form
\begin{equation} \label{eq: total collision rate}
    B(v, \mathbb{S}^{d-1}) =
  \left[\kappa + \gamma |v|^2 \right]
    \quad \kappa,\gamma \geq 0.
\end{equation}
The case $\gamma = 0$ is that of cutoff Maxwell molecules, which arises in $d$ dimensions when molecules are modelled as repelling each other with a force inversely proportional to the $(2d-1)^\text{th}$ power of their separation.
Replacing \eqref{eq: total collision rate} with $B(v, \mathbb{S}^{d-1}) \propto \abs{v}$ would give the \emph{hard spheres} model, which can be seen as the large exponent limit of an inverse power law force model, and arises when molecules are modelled as non-overlapping `billiards'.
The contribution from $\gamma |v|^2$ is a `very hard spheres' term, which we will refer to as quadratic.

The quadratic model has appeared previously in the literature \cite{Lu,PSW17} and is mathematically interesting, although it does not arise from a physical interaction model \cite{Villani}.
Since Maxwell molecules and the quadratic model (with $\kappa = 0$) are the cases  $B(v, \mathbb{S}^{d-1}) \propto \abs{v}^n$ for $n=0,2$ respectively it is reasonable to regard the hard spheres ($n=1$) as an intermediate case and to suppose that much qualitative behaviour shared by the Maxwell and quadratic cases will also be seen for hard spheres.
This supposition is supported by numerical observations in \cite{PSW17}.
For our analysis, the Maxwell and quadratic cases have the crucial property that the collision rates can be expressed entirely in terms of mass, momentum and energy, which are all conserved during collisions, leading to closed form expressions for the gelation time and other quantities of interest, and we restrict ourselves to this setting.
%Our analysis is restricted to this case, although we will also discuss the implications for hard sphere case.
To avoid triviality, we assume throughout that $\kappa + \gamma > 0$.


Formally, we write $\sim_t$ for the equivalence relation on $\{1,2,\dotsc, N\}$ generated by the relation containing pairs $(i, j)$ such that molecules $i$ and $j$ have collided at, or before, time $t$. The (Kac-$N$) interaction clusters at time $t$ are the equivalence classes of $\sim_t$.
If $I$ and $J$ are two distinct, unordered interaction clusters, the instantaneous merger rate is
\begin{multline}\label{e:cluster-merge-rate}
    \frac2N\sum_{i\in I}\sum_{j\in J}
     \left(\kappa + \gamma \norm{v_i^N(t) - v_j^N(t)}^2 \right)\\
    =
    \frac2N\kappa (\#I)(\#J)
    +\frac2N\gamma (\#{I})\left( \sum_{j\in J}\norm{v_j^N(t)}^2\right)
    +\frac2N\gamma (\#{J})\left( \sum_{i\in I}\norm{v_i^N(t)}^2\right)
    -\frac4N\gamma \left(\sum_{i\in I}v_i^N(t)\right)\cdot
    \left(\sum_{j\in J}v_j^N(t)\right).
\end{multline}

From \eqref{e:cluster-merge-rate} one sees that Kac interaction cluster dynamics do not require a full knowledge of the Kac process, but only the number of molecules, combined momentum and combined kinetic energy of the molecules indexed by each interaction cluster; for $t\ge 0$, let $I^N_j(t): 1\le j \le k^N(t)$ be an enumeration of the equivalence classes of $\sim_t$, and $x^N_j(t)$ be the associated data of the clusters \begin{equation}
    x^N_j(t)=\sum_{i \in I^N_j(t)}\left(1,v^N_i(t),\frac{1}{2}\left|v^N_i(t)\right|^2\right).
\end{equation} 
Thanks to the conservation properties, these quantities are added when two particles belonging to distinct clusters collide, and no change occurs when if the two particles belong to the same cluster. Therefore, the interaction cluster size dynamics can be replicated by a system of coagulating particles $(x^N_j(t): j\le k^N(t))$ in the state space
\begin{equation}
    S=\{x=(n, p, e) \in \mathbb{N}\times \mathbb{R}^d\times [0,\infty): |p|\leq \sqrt{2ne} \}. 
\end{equation} We study the associated empirical measure, given by \begin{equation}\label{eq: sc1}\mu^N_t = \frac{1}{N}\sum_{j=1}^{k^N(t)} \delta_{x^N_j(t)}. \end{equation} We equip the the state space $S$ with the maps $\pi_n, \pi_p, \pi_e$ for the projection onto the respective factors, and $R:S\rightarrow S$ for the parity map $(n,p,e)\mapsto (n,-p,e)$. Rewriting the calculation (\ref{e:cluster-merge-rate}) in this notation, the rate at which unordered pairs of particles $\left\{ x,y \right\}$ in $S$ merge to form a new particle in $A \subset S$ is $2K(x,y, A)/N$ with
\begin{equation}\label{eq: smoluchowski kernel}
    K(x,y,dz)=\overline{K}(x,y)\delta_{x+y}(dz);
    \end{equation} \begin{equation} \label{eq: overline K}  \overline{K}(x,y)=\kappa \pi_n(x)\pi_n(y)+2\gamma\left(\pi_n(x)\pi_e(y)-\pi_p(x)\cdot \pi_p(y)+\pi_e(x)\pi_n(y)\right).
\end{equation} 
This is a Marcus--Lushnikov coagulation process \cite{L78} on $S$, which we will refer to as the \emph{stochastic coagulant}.
Note that a $1/N$ scaling of the pair interaction rate is used, which ensures that each molecule has a total collision rate that is of order 1; this is to be expected when modelling a gas where the mean free time is a positive real number.
Dividing jump rates by $N$ is equivalent to accelerating time by the same factor and this alternative formulation means that the jump rates in the definition of the ``stochastic coalescent'' in  \cite{A99} as well as of the ``stochastic $K$-coagulant'' in \cite{N00} omit the $1/N$ from the rates and rescale time when taking the $N\rightarrow \infty$ limit. \medskip \\ Finally, we remark  that the definitions (\ref{eq: sc1}, \ref{eq: smoluchowski kernel}, \ref{eq: overline K}) make sense in the more general case when the number of particles in the underlying Kac process is replaced by a random number $l_N$ of the same order as $N$; in this case, the scaling factors $N, N^{-1}$ in (\ref{eq: sc1}, \ref{eq: smoluchowski kernel}) remain fixed. This mild generalisation will be helpful for our arguments.

\subsubsection{Limiting kinetic equations}
We now consider various forms of the limiting Smoluchowski equation.  Define a drift operator $L$, by specifying for all bounded measurable $f\colon S \rightarrow \RR$, 
 \begin{equation} \label{eq: drift wo gel}
    \langle f,L(\mu)\rangle=\frac{1}{2}\int_{S^2}\{f(x+y)-f(x)-f(y)\}\overline{K}(x,y)\mu(dx)\mu(dy).
\end{equation}
The associated evolution equation for $(\mu_t)_{t<T}$ is that, for all $t<T$,
\begin{equation}
    \tag{E-G}\label{eq: E} \mu_t =  \mu_0 +\int_0^t L(\mu_s) ds.
\end{equation} Following \cite{N00}, we say that a family $(\mu_t)_{t<T}$ of positive measures is a solution to (\ref{eq: E}) if the following hold: \begin{enumerate}[label=\roman{*}).] \item For all Borel sets $A\subset S$, the map $t\mapsto \mu_t(A)$ is measurable; \item For all bounded, measurable functions $f:S\rightarrow \RR_+$ of compact support, $\langle f, \mu_0\rangle<\infty$; \item For all compact subsets $S'\subset S$ and all $t<T$, \begin{equation} \int_0^t ds \int_{S'\times S}\overline{K}(x,y)\mu_s(dx)\mu_s(dy)<\infty; \end{equation}  \item For all bounded, compactly supported functions $f:S\rightarrow \RR$, and $t<T$, \begin{equation}
    \langle f, \mu_t \rangle =  \langle f,\mu_0\rangle +\int_0^t \langle f, L(\mu_s)\rangle ds
\end{equation} \end{enumerate} 
 This captures the effects of coagulations between finite clusters. However, as discussed above, we wish to include the possibility of a macroscopic component, which we term \emph{gel}. To include this effect, we define a modified drift operator $L_\mathrm{g}(\mu_t)$ by specifying, for bounded, measurable $f:S\rightarrow \RR$, \begin{equation} \langle f,L_\mathrm{g}(\mu_t)\rangle =\langle f, L(\mu_t)\rangle -\int_{S}f(x)\overline{K}(x,y)\mu_t(dx)(\mu_0-\mu_t)(dy). \end{equation} A \emph{coagulant} is then a solution $(\mu_t)_{t<T}$ to \begin{equation} \tag{E+G} \label{eq: E+G}
    \mu_t= \mu_0 + \int_0^t L_\mathrm{g}(\mu_s)ds.
\end{equation} Here, the additional term comes into play only after $\mu_t$ ceases to conserve the quantities $\langle \pi_n, \mu_t\rangle, \langle \pi_p, \mu_t\rangle ,\langle \pi_e, \mu_t\rangle$, and the extra term represents the interaction with the gel.
This may be interpreted concretely in a similar sense to  (\ref{eq: E}) above.
This generalises the Smoluchowski coagulation equations \cite{vS16} analagous to that of Flory \cite{ZS80}.
The solution to this deterministic evolution problem is a `$K$-coagulant' in the language of \cite{N00}. \medskip \\ We write \begin{equation}\label{eq: gel data} \begin{split}
   g_t= (M_t, P_t, E_t)&=\langle x, \mu_0-\mu_t\rangle \\ &= \left(\langle \pi_n, \mu_0-\mu_t\rangle,\langle \pi_p, \mu_0-\mu_t\rangle,\langle \pi_e, \mu_0-\mu_t\rangle\right) \end{split}
\end{equation} for the mass, energy and momentum of the gel. Following remarks in \cite{N00}, one may show that if $\mu_t$ is a solution to (\ref{eq: E+G}), then the maps $t\mapsto \langle \pi_n, \mu_t\rangle, \langle \pi_e, \mu_t\rangle$ are non-increasing, which guarantees that $M_t, E_t\ge 0$. We write  $S_\mathrm{g}$ for the continuum analogue of the state space $S$, given by \begin{equation}
    S_\mathrm{g}=[0,\infty)\times \RR^d\times [0,\infty)
\end{equation} and use the same notation $\pi_n, \pi_p, \pi_e$ for the projections onto the factors, as for $S$. When $x\in S$ and $g\in S_\mathrm{g}$, we use $\overline{K}(x,g)$ for the rate of absorption, given by (\ref{eq: overline K}) with the new meanings of $\pi_n(g), \pi_p(g),\pi_e(g).$ We will also write $\varphi$ for the linear combination $\varphi=\pi_n+\pi_e$, defined on both $S$ and $ S_\mathrm{g}$.

\begin{definition}[Conservative Solutions] As noted above, the functions $t\mapsto \langle \pi_n, \mu_t\rangle$ and $t\mapsto \langle \pi_e, \mu_t\rangle$ are non-increasing, whenever $(\mu_t)_{t<T}$ is a local solution to either \eqref{eq: E} or \eqref{eq: E+G}. We say that a solution $(\mu_t)_{t<T}$ is \emph{conservative} if both are constant on $[0,T)$, or equivalently, if $\langle \varphi, \mu_t\rangle$ is constant on $[0,T)$. \end{definition}
 Thus, any solution to (\ref{eq: E+G}) is conservative up to some time $0\le t_\mathrm{g}\leq \infty$, and non-conservative thereafter.
 
\begin{definition}[Metrisation of Convergence] Let $\mathcal{M}=\mathcal{M}_{\le 1}(S)$ be the space of measures on $S$ with total mass at most 1. We equip $\mathcal{M}$ with the \emph{vague} topology $\mathcal{F}(\mathcal{M}, C_c(S))$ induced by continuous, compactly supported functions on $S$, and fix a complete metric $d_0$ compatible with this topology. Let $\mathcal{M}^\star$ be the space $\mathcal{M}\times S_\mathrm{g}$, and define the complete, separable metric \begin{equation} d^\star\left((\mu, g), (\mu', g')\right):=d_0(\mu, \mu')+|g-g'| \end{equation} where $|\cdot|$ is the Euclidean distance on $S_\mathrm{g}\subset \RR^{d+2}.$ \end{definition}

\subsection{Statement of Results}\label{sec: results}

We make the following hypotheses on the initial data $\mu_0$.
\\A1. The initial data $\mu_0 \in \mathcal{M}$ is equal to its pushforward under $R$, that is, $\mu_0 = \mu_0 \circ R^{-1}$.
\\ A2. The initial data is given in terms of a sub-probability measure $m$ of particle velocities, by pushforward under the map
\begin{equation}
    \iota: \RR^d \rightarrow S; \hspace{1cm} v\mapsto (1, v, |v|^2).
\end{equation}
\\ {A3.} For $0\leq k\leq 6$, we have
\begin{equation}
    \sigma_k(m)=\langle |v|^k, m\rangle <\infty.
\end{equation} 
\\ {A4.} The underlying measure $m$ has $m(\{0\})=0.$
\\We remark that much of our analysis can be done without the assumption (A2.); for instance, this would allow us to deal with the case where not all clusters initially have size 1. However, this assumption is natural in the context of Kac interaction clusters, as it guarantees that the number of particles $N$ initially in the stochastic coagulant corresponds to the number of particles in the underlying Kac process. 
\\ We summarise our results on the analysis of the Smoluchowski equation (\ref{eq: E+G}) as follows.
\begin{theorem}\label{thrm: Smoluchowski equation}
Let $\mu_0$ be an initial measure on $S$ satisfying ({A1-4}.), and assume that $m$ is a probability measure. Then the equation (\ref{eq: E+G}) has a unique solution $(\mu_t)_{t\geq 0}$ starting at $\mu_0$; we write $g_t=(M_t, P_t, E_t)$ for the gel data defined in (\ref{eq: gel data}). This solution has the following properties.
\paragraph{1. Phase Transition.} Let $t_\mathrm{g}$ be the first time at which the solution $\mu_t$ fails to be conservative, that is:
\begin{equation} t_\mathrm{g}:=\inf\{t\ge 0: \langle \varphi, \mu_t\rangle < \langle \varphi, \mu_0\rangle \}.
\end{equation}
Then
$t_\mathrm{g}\in (0,\infty)$, and can be given explicitly in terms of the moments of $m$ as \begin{equation}\label{eq: closed form for tg}
       t_\mathrm{g}= \frac{1}{\kappa +2\gamma\sigma_2(m) + \sqrt{(\kappa+2\gamma\sigma_2(m))^2+4\gamma^2(\sigma_4(m)-\sigma_2^2(m))}}.
   \end{equation}

\paragraph{2. Behaviour of the Second Moment.} Consider the second moment
\begin{equation} \mathcal{E}(t)=\langle \varphi^2, \mu_t\rangle.
\end{equation}
Then \begin{enumerate}[label=\roman{*}).]
    \item $\mathcal{E}(t)$ is finite and continuous, and so locally bounded, on $[0, \infty)\setminus\{t_\mathrm{g}\}.$ 

    \item On $[0, t_\mathrm{g})$, $\mathcal{E}$ is monotonically increasing.
    
    \item At the gelation time, $\mathcal{E}(t_\mathrm{g})=\infty$, and $\mathcal{E}(t)\rightarrow \infty$ as $t\rightarrow t_\mathrm{g}.$ 
\end{enumerate}


\paragraph{3. Representation of Gel Data.} Let $m$ be the underlying distribution of initial velocities, as in ({A2}.) For each $t\ge 0$, there exists a unique maximal pair $c_t=(a_t, b_t) \ge 0$ such that, for all $v\in \mathbb{R}^d$, \begin{equation}\label{eq: NLFP 1} a_t+b_t|v|^2=2t \int_{\mathbb{R}^d} (1-e^{-a_t-b_t|w|^2})(\kappa+\gamma|v-w|^2)m(dw). \end{equation} $c_t$ undergoes a phase transition at time $t_\mathrm{g}$: if $t\le t_\mathrm{g}$, then $c_t=0$, and if $t>t_\mathrm{g}$ then $a_t>0$. If, in addition, $\gamma>0$, then if $t>t_\mathrm{g}$ then both components $a_t, b_t>0$. Moreover, the map $t\mapsto c_t$ is continuous. \medskip \\  The gel data are given in terms of $c_t$ by \begin{equation}\label{eq: formula for M, E_0}
    g_t =(M_t, P_t, E_t)= \int_{\mathbb{R}^d} \left(1,0, \frac{1}{2}|v|^2\right)(1-e^{-a_t-b_t|v|^2})m(dv).
\end{equation} Therefore, if $t>t_\mathrm{g}$ then both $M_t, E_t>0$. Moreover, the map $t\mapsto g_t$ is continuous, and $g_{t_\mathrm{g}}=0$.
\paragraph{4. Gel Dynamics.} The map $t\mapsto g_t$ is differentiable on $t\in(t_\mathrm{g}, \infty)$, and
\begin{equation}
    \frac{d}{dt}M_t=\kappa\left<\pi_n^2,\mu_t\right>M_t +2\gamma 
        \left<\pi_n \pi_e,\mu_t \right>M_t +
        \left<\pi_n^2,\mu_t \right>E_t ;
\end{equation}
\begin{equation}
    \frac{d}{dt}E_t=
      \kappa \left<\pi_n \pi_e,\mu_t\right>M_t +
      2\gamma 
        \left<\pi_e^2,\mu_t \right>M_t +
        \left<\pi_n \pi_e,\mu_t \right>E_t.
\end{equation}
\paragraph{5. Order of the Phase Transition, and the Size-Biasing Effect} The map $t\mapsto c_t=(a_t,b_t)$ is right-differentiable at $t_\mathrm{g}$, and $a'_{t_\mathrm{g}^+}>0$. The ratio of the components is \begin{equation} \frac{b'_{t_\mathrm{g}^+}}{a'_{t_\mathrm{g}^+}}=\lambda =\frac{\sqrt{\kappa^2+4\gamma(\kappa\sigma_2(m)+\gamma \sigma_4(m))}-\kappa}{2(\kappa\sigma_2(m)+\gamma\sigma_4(m))}.\end{equation} As a consequence, the phase transition is first order; that is, the right-derivative of the gel data $M_t, E_t$ exists and is positive at $t_\mathrm{g}$: \begin{equation}M'_{t_\mathrm{g}^+}>0; \hspace{1cm} E'_{t_\mathrm{g}^+}>0 \end{equation} and we have the following \emph{size-biasing} effect, which further quantifies the way in which gelation is driven by the fast particles:  \begin{equation}
    \lim_{t\downarrow t_\mathrm{g}}\frac{E_t}{M_t}=\frac{1}{2}\frac{\sigma_2(m)+\lambda\sigma_4(m)}{1+\sigma_2(m)\lambda}.
\end{equation} In particular, unless $\gamma=0$ or $|v|$ is constant $m$-almost everywhere, we have a positive bias \begin{equation}
    \lim_{t\downarrow t_\mathrm{g}}\frac{E_t}{M_t}>\frac{1}{2}\int_{\mathbb{R}^d}|v|^2m(dv).
\end{equation}

A notable case of physical interest is when we take $m$ to be a Gaussian density $N_d(0, \sigma^2 I)$, which certainly satisfies (A1-4.). In this case, the gelation time evaluates to
\begin{equation}
    \label{eq: formula of tgel} t_\mathrm{g} = \frac{1}{\kappa+2\gamma d \sigma^2+
    \sqrt{(\kappa+2\gamma d \sigma^2)^2+8\gamma^2 d \sigma^2}}.
\end{equation} \end{theorem}
We also prove the following convergence theorem, relating the stochastic coagulant to the solution of the limit equation. Firstly, following ideas of \cite{N00}, we show that the empirical measure $\mu^N_t$ converges to the limiting solution $(\mu_t)_{t\ge 0}$ in the vague topology, uniformly in time. \medskip \\ One might also ask about the connection between the stochastic coagulants and the limiting gel $(g_t)_{t\ge 0}$; since each stochastic coagulant preserves $\langle \varphi, \mu^N_t\rangle$, there is no natural analogue of the representation (\ref{eq: gel data}). However, we will show that the limiting gel $(g_t)_{t\ge 0}$ is closely related to the data of the largest (by particle number) cluster. In particular, the convergence of this \emph{stochastic gel} may be viewed as a phase transition at time $t_\mathrm{g}$.
\begin{theorem} \label{thrm: convergence of stochastic coagulent} Let $\mu_0$ be a probability measure on $S$ satisfying ({A1-4}.), for some probability measure $m$, and let $(\mu_t)_{t\ge 0}, (g_t)_{t\ge 0}$ be the associated solution to (\ref{eq: E+G}) and corresponding gel. For $N\ge 1$, let $\mu^N_t$ be the stochastic coagulant, where the initial velocities of particles are sampled independently from $m$. Define \begin{equation} g^N_t=(M^N_t, P^N_t, E^N_t)\end{equation}  as the data of the largest cluster in the stochastic system, normalised by $N^{-1}$. Then we have the convergence \begin{equation} \label{eq: convergence of stochastic system}
    \sup_{t\ge 0}\hspace{0.1cm}d^\star\left((\mu^N_t, g^N_t),(\mu_t, g_t)\right) \rightarrow 0
\end{equation} in probability. In particular, we have the following phase transition: \begin{enumerate}[label=\roman{*}).]
    \item If $t\le t_\mathrm{g}$, then the largest cluster has size of the order $o_\mathrm{p}(N)$;
    \item If $t>t_\mathrm{g}$, the largest cluster has size of the order $\Theta_\mathrm{p}(N)$.
\end{enumerate} 

Moreover, if $\xi_N$ is any sequence with $\xi_N\rightarrow \infty$ and $\frac{\xi_N}{N}\rightarrow 0$, then we may define $\widetilde{g}^N_t=(\widetilde{M}^N_t, \widetilde{P}^N_t, \widetilde{E}^N_t)$ by summing the data of all clusters with mass at least $\xi_N$, and normalising by $N$. Then the same result holds when we replace $g^N_t$ by $\widetilde{g}^N_t$ in (\ref{eq: convergence of stochastic system}).
\end{theorem} 

Here, and throughout, we use the notation $o_\mathrm{p}(\cdot), \mathcal{O}_\mathrm{p}(\cdot), \Theta_\mathrm{p}(\cdot)$ for the probabilistic equivalents of $o(\cdot), \mathcal{O}(\cdot), \Theta(\cdot)$. Precise definitions can be found in \cite{JLR}.

\subsection{Connection of results to the Literature}
We briefly discuss Theorems \ref{thrm: Smoluchowski equation} and \ref{thrm: convergence of stochastic coagulent} in relation to some other results in the literature. \medskip \\ The starting point for this project was the conjecture raised in \cite{PSW16}, that gelation occurs before or at the mean free time $t_\mathrm{mf}$. In \cite{PSW16}, this is supported by numerical evidence for a variety of collision kernels, including the hard spheres and quadratic cases. For the case of the quadratic kernel (\ref{eq: total collision rate}) considered above, the mean free time is \begin{equation} t_\mathrm{mf}=\frac{1}{2\kappa+4\gamma\sigma_2(m)}. \end{equation} The first item Theorem \ref{thrm: Smoluchowski equation} therefore gives a positive, analytical solution to this conjecture. Moreover, except in the special cases where $\gamma=0$ or $|v|$ is constant $m$- almost everywhere, we have the strict inequality $t_\mathrm{g}<t_\mathrm{mf}$. This potentially counterintuitive result may be understood as saying that large velocity tails of $m$ reduce the gelation time more than they reduce the mean-free time; heuristically, gelation is driven by the fastest particles.

Together with Theorem \ref{thrm: convergence of stochastic coagulent}, item 2 of Theorem \ref{thrm: Smoluchowski equation} follows exactly the idea of Lushnikov: the formation of a giant particle at $t_\mathrm{g}$ corresponds exactly to blowup of the second moment $\mathcal{E}(t)$ and breakdown of conservation. We go further, and show that the second moment is finite after $t_\mathrm{g}$, since the giant particle does not correspond to anything in the limit measure. Indeed, the only time when the second moment diverges is at the critical point $t_\mathrm{g}$ when a giant particle is about to form; following the percolation literature, this may be thought of as an `incipient giant component' \cite{Ald16}. \medskip \\ Theorem \ref{thrm: convergence of stochastic coagulent}, concerning the convergence of the stochastic coagulant and the giant particle, follows ideas of \cite[Theorem 4.1]{N00}. However, in our case, we can use the uniqueness statement from Theorem \ref{thrm: Smoluchowski equation} to conclude a local uniform convergence result in Lemma \ref{lemma: local uniform convergence of stochastic coagulent}. The remainder of this theorem follows from a careful analysis of the gel through an associated random graphs process in Sections \ref{sec: uniform convergence}, \ref{sec: COG}. \medskip \\ Following well-known results on Erd\H{o}s-Renyi graphs (see, for example, \cite{B01}), we could further ask about the size of the largest component at and below criticality, and of the small components above criticality.  The results of \cite{BJR07} address both of these results for certain classes of graph processes, but unfortunately the results do not cover our kernel in general.   
\subsection{A Note on Generalisations} As commented above, our analysis rests on the \emph{bilinear} form of the total rate $\overline{K}$, which allows us to connect the Smoluchowski equation to random graphs in Sections \ref{sec: IRG}, \ref{sec: coupling_to_random_graph}. This motivates the following definition.
\begin{definition} For any measurable space $S$, we say that a kernel $K$ on $S\times S\times S$ is \emph{bilinear} if there exists a finite set of measurable mappings $(\pi_i)_{i\le n+m}$ and a symmetric real matrix $(a_{ij})_{i, j\le n+m}$ satisfying the following. 
\begin{enumerate}[label=\roman{*}).]
\item For all $i\le n+m$ and all $x,y\in S$, \begin{equation} \pi_i=\pi_i(x)+\pi_i(y) \hspace{1cm}K(x,y,\cdot)\text{- almost everywhere}. \end{equation} \item For $i\le n$, the map $\pi_i:S\rightarrow \mathbb{R}$ takes only nonnegative values.
\item There exists a measurable map $R: S\rightarrow S$ such that $R\circ R$ is the identity on $S$, and  \begin{equation} \pi_i \circ R=\begin{cases} \pi_i & 1\le i\le n; \\ -\pi_i, & n+1\le i \le n+m \end{cases}.  \end{equation}
\item There exists a constant $C$ such that, for all $x\in S$, \begin{equation} \sum_{i> n} \pi_i(x)^2 \le C \sum_{i\le n} \pi_i(x)^2. \end{equation} \item For all $i, j\le n$, $a_{ij}\ge 0$, and there is at least one pair $i, j\le n$ such that $a_{ij}>0.$
\item For all $x,y\in S$, the total rate $\overline{K}(x,y)=K(x,y,S)$ may be expressed as \begin{equation} \overline{K}(x,y)=\sum_{i,j\le n+m}a_{ij}\pi_i(x)\pi_j(y). \end{equation}  
\end{enumerate}  \end{definition}
In this setting, we define $\varphi:=\sum_{i\le n} \pi_i$, and seek processes of measures $(\mu_t)_{t\ge 0}$ on $S$ solving the equation analagous to (\ref{eq: E+G}). Similarly, one can consider a stochastic coagulant $(\mu^N_t)_{t\ge 0}$ defined analagously to (\ref{eq: smoluchowski kernel}, \ref{eq: overline K}), with $\lceil N \mu_0(S)\rceil$ particles initially sampled independely from $\mu_0(S)^{-1}\mu_0(\cdot)$. In this context, we would require the following conditions on the initial data $\mu_0$: 
\\ (A1'.) The measure $\mu_0$ is invariant under the transformation $R$ in point iii).: $\mu_0\circ R^{-1}=\mu_0.$
\\ (A3'.) For all $i\le n$, we have $\langle \pi_i^3, \mu_0\rangle <\infty.$
\\ (A4'). For all $i \le n$, \begin{equation} \mu_0(x\in S: \pi_i(x)=0)=0.\end{equation} In this setting, no assumption analagous to (A2.) is necessary. As remarked above this assumption is included in Theorems \ref{thrm: Smoluchowski equation}, \ref{thrm: convergence of stochastic coagulent} in order to guarantee that the number of particles initially in the stochastic coagulant coincides with the number of particles in the underlying Kac process. \medskip \\ Under these hypotheses, all of the arguments used in this paper may be adapted to prove results analagous to Theorems \ref{thrm: Smoluchowski equation}, \ref{thrm: convergence of stochastic coagulent}. In this context, a closed form expression for the gelation time $t_\mathrm{g}$, right-derivative $\lim_{t\downarrow t_\mathrm{g}} \frac{g_t}{t-t_\mathrm{g}}$, or quantities analagous to the bias $\lim_{t\downarrow t_\mathrm{g}} \frac{E_t}{M_t}$ may not be available, but these will instead be characterised in terms of finite-dimensional, explicit eigenvalue problems, and can therefore be found numerically if an explicit value is desired.

\subsection{Plan of the Paper} Our programme will be as follows. \begin{enumerate} \item In Section \ref{sec:SE}, we will prove that the limiting equation (\ref{eq: E+G}) has unique, globally defined solutions, based on a truncation argument from \cite{N99,N00}.
\item In Section \ref{sec: csc}, we prove an initial result Lemma \ref{lemma: local uniform convergence of stochastic coagulent} on the convergence of the stochastic coagulant, using the ideas of \cite[Theorem 4.1]{N00}. This will later be used to prove later points of Theorem \ref{thrm: Smoluchowski equation} based on probabilistic arguments.
\item In Sections \ref{sec: IRG}, \ref{sec: coupling_to_random_graph}, we introduce the theory of inhomogenous random graphs set out in \cite{BJR07}, and show how a particlar example of these graphs may be coupled to the stochastic coagulant. The critical time $t_\mathrm{crit}$ for these graphs may be found exactly, leading to the explicit expression in Theorem \ref{thrm: Smoluchowski equation}. \item A weakness of the preceding sections is that, a priori, the critical time $t_\mathrm{crit}$ for the graph processes may differ from the gelation time $t_\mathrm{g}$; in Section \ref{sec: ECT}, we show that this cannot happen. This is based on a preliminary version of Theorem \ref{thrm: convergence of stochastic coagulent}, which shows convergence of $(\mu^N_t, g^N_t)$ at a single fixed time $t\ge 0$. 
\item Section \ref{sec: finiteness of second moment} is dedicated to a proof of item 2 of Theorem \ref{thrm: Smoluchowski equation}. The statements about the subcritical regime follow general ideas in \cite{N99,N00}, while statements about the critical and supercritical cases use additional ideas from the theory of random graphs.
\item Section \ref{sec: gel dynamics} uses the ideas of previous sections to prove items 3 and 4 of Theorem \ref{thrm: Smoluchowski equation}, concerning the gel data $g_t$ beyond the critical point. \item Section \ref{sec: uniform convergence} uses the analysis of the gel to extend Lemma \ref{lemma: local uniform convergence of stochastic coagulent} to show that convergence is uniform in time.
\item Section \ref{sec: BNCP} proves item 5 of Theorem \ref{thrm: Smoluchowski equation}, concerning the behaviour near the critical point. This completes the proof of this theorem. \item To finish the proof of Theorem \ref{thrm: convergence of stochastic coagulent}, we revisit the ideas of Section \ref{sec: ECT} to prove convergence of the stochastic gel $g^N_t, \widetilde{g}^N_t$, uniformly in time. This is the focus of Section \ref{sec: COG}, and builds further on ideas of previous sections. \end{enumerate}