\section{The Phase Transition}\label{sec: BNCP} We now prove item 5 of Theorem \ref{thrm: Smoluchowski equation}, concerning the phase transition: we will show that the gel data $(M_t, E_t)$ have strictly positive right-derivatives at the gelation time $t_\mathrm{gel}$. We start from the nonlinear fixed point equation (\ref{eq: NLFP 1}), which we rewrite as \begin{equation}\label{eq: NLPF again}c_t=tF(c_t); \hspace{1cm} F\left(\begin{matrix} a \\ b \end{matrix}\right)=2\int_{\RR^d}(1-e^{-a-b|v|^2})\left(\begin{matrix}\kappa+\gamma|v|^2 \\ \gamma  \end{matrix}\right)m(dv). \end{equation} The following proof is a modification of the arguments in \cite[Theorem 3.17]{BJR07}, which itself generalises an analagous, well-known result for the phase transition of Erd\H{o}s-R\'eyni graphs.
\begin{lemma}\label{lemma: BNCP} Suppose that $\mu_0$ satisfies (A1-4.) for a probability measure $m$, and let $c_t=(a_t, b_t)$ be as in Lemma \ref{lemma: form of rho-t}.  Then $c_t$ is right-differentiable at $t_\mathrm{g}$, $a'_{t_\mathrm{g}^+}>0$, and \begin{equation}\label{eq: ratio of r derivatives} \lambda = \frac{b'_{t_\mathrm{g}^+}}{a'_{t_\mathrm{g}^+}}=\frac{\sqrt{\kappa^2+4\gamma(\kappa\sigma_2(m)+\gamma \sigma_4(m))}-\kappa}{2(\kappa\sigma_2(m)+\gamma\sigma_4(m))}. \end{equation}  \end{lemma} \begin{proof} We first assume that $|v|$ is not constant $m$-almost everywhere, and equip $\RR^2$ with the inner product \begin{equation} \left((a,b),(a',b')\right)_m=\int_{\RR^d} (a+b|v|^2)(a+b'|v|^2)m(dv)  \end{equation} and write $|\cdot|_m$ for the associated norm.  Differentiating under the integral sign twice, and using (A3.), we write \begin{equation} F\left(\begin{matrix} a \\ b \end{matrix}\right) = \Lambda\left(\begin{matrix} a \\ b \end{matrix}\right)-\Sigma \left(\begin{matrix} a \\ b \end{matrix}\right) + R\left(\begin{matrix} a \\ b \end{matrix}\right) \end{equation} where $\Lambda(\cdot), \Sigma(\cdot)$ are the linear and quadratic terms, and $R$ is a remainder term:   \begin{gather} \label{eq: defn of L}
    \Lambda\left(\begin{matrix} a \\ b \end{matrix}\right)=2\left(\begin{matrix} \kappa+\gamma\sigma_2(m) & \kappa \sigma_2(m)+\gamma\sigma_4(m)\\ \gamma & \gamma \sigma_2(m) \end{matrix}\right)\left(\begin{matrix} a \\ b \end{matrix}\right);
\\[2ex]
    \label{eq: defn of B}\Sigma\left(\begin{matrix} a \\ b \end{matrix}\right)=\int_{\RR^d}(a+b|v|^2)^2\left(\begin{matrix} \kappa+\gamma|v|^2 \\ \gamma \end{matrix}\right)m(dv); \\[2ex]
 \left|R(c)\right|_m = o\left(|c|^2_m\right) \text{as  }|c|\rightarrow 0.\end{gather} The signs here are chosen to guarantee that, if $c>0$, then $\Lambda c, B(c)>0$, and it is straightforward to verify that $\Lambda$ is self-adjoint with respect to $(\cdot,\cdot)_m$. We also note that we have already found the spectrum of $\Lambda$ in the computation in Lemma \ref{lemma: computation of tcrit}: $\Lambda$ has exactly two eigenvalues, of which the larger is $t_\mathrm{g}^{-1}$, and the corresponding $|\cdot|_m$-unit eigenvector is given by \begin{equation} \psi=(t_\mathrm{g}^{-2}/4+\gamma^2(\sigma_4(m)-\sigma_2(m)^2) )^{-1/2} \left(\begin{matrix} t_\mathrm{g}^{-1}/2-\gamma\sigma_2(m) \\ \gamma \end{matrix}\right). \end{equation} From Lemma \ref{lemma: form of rho-t}, Theorem \ref{thrm: RG1} and Theorem \ref{thrm: continuity of rho}, we know that $c_{t_\mathrm{g}}=0$, that $c_{t_\mathrm{g}+\epsilon}\in [0,\infty)^2\setminus \{(0,0)\}$ for all $\epsilon>0$, and that $t\mapsto c_t$ is continuous at $t_\mathrm{g}.$ \bigskip \\  It is straightforward to see that $\psi$ is an eigenvector of $\Lambda$ of eigenvalue $t_\mathrm{g}^{-1}$, and $|\psi|_m=1.$ Writing $\psi^\bot$ for the orthogonal compliment of $\text{Span}(\psi)$ with respect to $(\cdot, \cdot)_m$, it follows from the self-adjointness of $\Lambda$ that $\Lambda$ maps $\psi^\bot$ into itself. Moreover, for $t>t_\mathrm{g}$ small enough, $(t\Lambda-1)|_{\psi^\bot}$ is invertible, and that the operator norm $\|(t\Lambda-1)|_{\psi^\bot}^{-1}\|_{m\rightarrow m}$ is bounded as $t\downarrow t_\mathrm{g}$. \bigskip \\ Let $Q:\RR^2\rightarrow\RR^2$ be the orthogonal projection onto $\psi^\bot$ with respect to $(\cdot,\cdot)_m$, and write $c^*_t=Qc_t$ so that we have the orthogonal decomposition \begin{equation}\label{eq: orth dec} c_t=\alpha_t \psi + c^*_t \end{equation} for some $\alpha_t \in \mathbb{R}$. Noting that $\Lambda Q=Q\Lambda$, it follows from (\ref{eq: NLPF again}, \ref{eq: orth dec}) that \begin{equation} c^\star_t=Q(tF(c_t))=t\Lambda c^\star_t + tQ\left(-\Sigma(c_t)+R(c_t)\right). \end{equation} The function $-\Sigma(c)+R(c)$ is of quadratic growth as $|c|_m\rightarrow 0$, and using the invertibility of $(t\Lambda-I)|_{\psi^\bot}$ described above, it follows that there exists $\beta>0$ such that $
    |c^*_t|_m \le \beta |c_t|_m^2$
  whenever $|c_t|_m\le 1$. In turn, it follows that $|c_t|_m\sim \alpha_t$ as $t\downarrow t_\mathrm{g}.$ Now, using (\ref{eq: NLPF again}) and the self-adjointness of $\Lambda$, we obtain \begin{equation}
      \begin{split}
          \alpha_t&=(t_\mathrm{g}\Lambda\psi,c_t)_m=t_\mathrm{g}(\psi, \Lambda c_t)_m \\[1ex] &=\frac{t_\mathrm{g}}{t}(\psi, c_t)_m-t_\mathrm{g}\left(\psi,-\Sigma(c_t)+R(c_t)\right)_m \\[2ex]&=\frac{t_\mathrm{g}}{t}\alpha_t-t_\mathrm{g}\left(\psi,-\Sigma(c_t)+R(c_t)\right)_m.
      \end{split}
  \end{equation} We now expand to second order in $\alpha_t$; for clarity, we will number the error terms $\mathcal{T}^i_t.$ Since $|c_t|\sim \alpha_t$, it follows that that $|c^*_t|_m=\mathcal{O}(\alpha_t^2)$ and that $R(c_t)=o(\alpha_t^2)$. Expanding $\Sigma(c_t)$ using (\ref{eq: orth dec}),\begin{equation} -\Sigma(c_t)+R(c_t)=-\alpha_t^2\Sigma(\psi)+\mathcal{T}^1_t; \hspace{1cm} |\mathcal{T}^1_t|_m=o(\alpha_t^2). \end{equation} It therefore follows that \begin{equation}
      \begin{split}
          \alpha_t=t_\mathrm{g}\left(\frac{\alpha_t}{t}+\alpha_t^2(\psi, \Sigma(\psi))_m\right)+ \mathcal{T}^2_t; \hspace{1cm} \mathcal{T}^2_t=o(\alpha_t^2).
      \end{split} 
  \end{equation} For $t>t_\mathrm{g}$ small enough, $\alpha_t>0$, and we may rearrange to find \begin{equation} t-t_\mathrm{g}=t\hspace{0.1cm}t_\mathrm{g}\hspace{0.1cm}\alpha_t(\psi,\Sigma(\psi))+\mathcal{T}^3_t;\hspace{1cm} \mathcal{T}^3_t=o(\alpha_t) \end{equation}  and in particular $\alpha_t=\Theta(t-t_\mathrm{g})$ as $t\downarrow t_\mathrm{g}$, since $(\psi, \Sigma(\psi))_m>0.$ Finally, we obtain \begin{equation} \frac{\alpha_t}{t-t_\mathrm{g}}\rightarrow\frac{1}{t_\mathrm{g}^2(\psi,\Sigma(\psi))_m}\hspace{1cm} \text{as }t\downarrow t_\mathrm{g}.  \end{equation} The calculations above show that $|c_t-\alpha_t\psi|=\mathcal{O}((t-t_\mathrm{g})^2)$, and the claimed right-differentiability now follows. Observing that $\alpha'_{t_\mathrm{g}^+}>0$ and that the first component of $\psi$ is strictly positive, it follows that $a'_{t_\mathrm{g}^+}>0$ as claimed. Finally, the expression (\ref{eq: ratio of r derivatives}) for the ratio of the right-derivatives follows from the definition of $\psi$ with some elementary algebra, and using the computation of $t_\mathrm{g}$ in Lemma \ref{lemma: computation of tcrit}. \bigskip \\  We now deal with the degenerate case in which $|v|$ is constant almost everywhere; say,  $|v|=\upsilon$ for $m$-almost all $v$. In this case, let $\widetilde{a}_t=a_t+b_t\upsilon^2$; taking $v=(\upsilon, 0,....0)$ in (\ref{eq: NLPF again}), we obtain \begin{equation} \begin{split} \label{eq: degenerate NLFP}
 \widetilde{a}_t&=2\int_{\RR^d} (1-e^{-a_t-b_t|w|^2})(\kappa+\gamma|w-(\upsilon,0...,0)|^2)m(dw) \\ & \hspace{1cm} = 2(\kappa+2\gamma\upsilon^2)(1-e^{-\widetilde{a}_t}).\end{split}\end{equation} This may be rearranged into the nonlinear fixed point equation for Erd\H{o}s-R\'eyni random graphs, and analysed using the standard argument, or viewed as a special case of (\ref{eq: NLPF again}) with  \begin{equation} \widetilde{\kappa}=\kappa+2\gamma\upsilon^2; \hspace{1cm}\widetilde{\gamma}=0. \end{equation} Either argument shows that $\widetilde{a}_t$ is right-differentiable at $t_\mathrm{g}$, with right-derivative \begin{equation} \widetilde{a}'_{t_\mathrm{g}^+}=\frac{2}{t_\mathrm{g}}=\frac{4}{\widetilde{\kappa}}.
 \end{equation} To go from $\widetilde{a}_t$ to the original parameters $(a_t, b_t)$, we observe that from (\ref{eq: NLPF again}), we have \begin{equation}\label{eq: NLFP from atilde to ab} \left(\begin{matrix} a_t\\b_t \end{matrix}\right)=2(1-e^{-\widetilde{a}_t})\left(\begin{matrix} \kappa+\gamma\upsilon^2 \\ \gamma \end{matrix}\right) \end{equation} and so the right-differentiability of $\widetilde{a}_t$ implies the right-differentiability of $c_t=(a_t, b_t)$ at $t_\mathrm{g}.$ Since $\upsilon>0$ by (A4.) and at least one of $\kappa, \gamma$ is strictly positive, it follows that $a'_{t_\mathrm{g}^+}>0$. Finally, (\ref{eq: NLFP from atilde to ab}) implies that \begin{equation} \frac{b'_{t_\mathrm{g}^+}}{a'_{t_\mathrm{g}^+}}=\frac{\gamma}{\kappa+\gamma\upsilon^2} \end{equation} which may be seen to coincide with claimed expression (\ref{eq: ratio of r derivatives}). \end{proof} We now show how this implies item 5 of Theorem \ref{thrm: Smoluchowski equation}. From  Lemmas \ref{lemma: form of rho-t}, \ref{lemma: representation of M, E}, we have that \begin{equation} M_t=\int_{\RR^d}(1-e^{-a_t-b_t|v|^2})m(dv); \hspace{1cm} E_t=\int_{\RR^d}|v|^2(1-e^{-a_t-b_t|v|^2})m(dv). \end{equation} Differentiating under the integral sign using hypothesis (A3.), we obtain \begin{equation}
    M_t=a_t+b_t\sigma_2(m)+o(a_t+b_t); \hspace{1cm} E_t=\frac{1}{2}a_t\sigma_2(m)+\frac{1}{2}b_t\sigma_4(m)+o(a_t+b_t).
\end{equation} From the previous result, we see that for $t>t_\mathrm{g}$, \begin{gather} M_t=(t-t_\mathrm{g})(a'_{t_\mathrm{g}^+}+b'_{t_\mathrm{g}^+}\sigma_2(m))+o(t-t_\mathrm{g}); \\[1ex] E_t=\frac{1}{2}(t-t_\mathrm{g})(a'_{t_\mathrm{g}^+}\sigma_2(m)+b'_{t_\mathrm{g}^+}\sigma_4(m))+o(t-t_\mathrm{g})\end{gather} which proves the desired right-differentiability, and the positivity of the right-derivatives. Since $M'_{t_\mathrm{g}^+}>0$, we may take a quotient and let $t\downarrow t_\mathrm{g}$ to obtain the claimed size-biasing effect. 