\section{Analysis of Smoluchowski Equations}\label{sec:SE}  This chapter is dedicated to a first analysis of the Smolochowski equations (\ref{eq: E}, \ref{eq: E+G}), following Norris \cite{N99, N00}. Our proofs build heavily on these works; to avoid unnecessary duplication, we will sometimes cite the corresponding proof and indicate how it should be modified for our case, or simply indicate the corresponding argument where no modification is necessary.
\medskip \\ This section is divided into four subsections as follows.\begin{enumerate}[label=\roman{*}).]
    \item Firstly, we show global existence and uniqueness for the equation (\ref{eq: E+G}) on the full state space $S$.
    \item  Next, we show that, if the dynamics are restricted to a truncated state space $S^\epsilon$, then the second moment blows up exactly at the associated gelation time $t^\epsilon_\mathrm{gel}$.
    \item Thirdly, we show that in the limit as the truncation is removed, we recover the full solution, which implies the convergence of the gelation times $t^\epsilon_\mathrm{gel}\rightarrow t_\mathrm{gel}$, and that for $t>t_\mathrm{gel}$, both the mass $M_t$ and energy $E_t$ are strictly positive.
    \item Finally, we argue the existence of a blowup time $t_\mathrm{expl}$ for the second moment of the non-truncated dynamics and argue, by analysing a system of ODEs, that the explosion times also converge: $t^\epsilon_\mathrm{gel}\rightarrow t_\mathrm{expl}$. Together with the third subsection, this implies that the explosion time $t_\mathrm{expl}$ of the second moment coincides with the gelation time $t_\mathrm{gel}.$
\end{enumerate}  
This proves existence and uniqueness in Theorem \ref{thrm: Smoluchowski equation}, as well as part 1 i)--iii)., and part 2 i--ii). restricted to $[0,t_\mathrm{gel})$. 
The remainder of Theorem \ref{thrm: Smoluchowski equation}, including the explicit formula for $t_\mathrm{gel}$ relies on the convergence of the stochastic coalescent, and a coupling to an \emph{inhomogenous random graph process}, which will be developed in later sections. In particular, the rest of part 2 uses a duality property which relates the subcritical and supercritical phases, and will be proven in is proven in Section \ref{sec: finiteness of second moment}.
\subsection{Existence and Uniqueness}

\begin{lemma}\label{lemma: E and U} For any measure $\mu_0$ satisfying (\textbf{A1}-\textbf{4}.), the equation with gel (\ref{eq: E+G}) has a unique global solution $(\mu_t)_{t\geq 0}$ starting at $\mu_0$. Moreover, the momentum $P_t=0$ for all time $t\ge 0$. \end{lemma}
\begin{corollary}\label{cor: maximal conservative solutions} Suppose $(\mu'_t)_{t<T}$ is a conservative local solution to the equation without gel, (\ref{eq: E}), starting at $\mu_0$. Then $\mu_t=\mu'_t$ for all $t<T$, and $T<t_\mathrm{gel}$. Hence, (\ref{eq: E}) has a unique maximal conservative solution, given by $(\mu_t)_{t<t_\mathrm{gel}}$.
\end{corollary}

Our proof of Lemma~\ref{lemma: E and U} is an adaptation of the arguments in \cite[Section 2]{N99} and \cite[Section 2]{N00}. 
Fix $R>0$, and let $S_R=\{x\in S: \left(\pi_n(x) + 2 \pi_e(x)\right) \le R\}$. We consider the following problem, `truncated at level $R$': Find measures $\mu^R_t$ supported on $S_R$ and gel data $g^R_t=(M^R_t, P^R_t, E^R_t)\in S_\mathrm{gel}$ such that, for all bounded measurable $f$ on $S_R$, \begin{equation} \tag{E$|^1_\mathrm{R}$} \label{eq:rE1} \frac{d}{dt} \langle f, \mu^R_t\rangle =\frac{1}{2}\int_{S_R\times S_R} (f(x+y)1_{x+y\in S_R}-f(x)-f(y))\overline{K}(x,y)\mu^R_t(dx)\mu^R_t(dy);\end{equation} and
\begin{equation} \label{eq: rE2} \tag{E$|^2_\mathrm{R}$}
 \frac{d}{dt}g^R_t=\frac{1}{2}\int_{S_R\times S_R} (x+y)1_{x+y\in S_R}\overline{K}(x,y)\mu^R_t(dx)\mu^R_t(dy)+\int_{S_R} x \overline{K}(x, g^R_t) \mu^R_t(dx).
\end{equation} 
We will use the following existence and uniqueness result for the restricted dynamics (\ref{eq:rE1}, \ref{eq: rE2}).
\begin{lemma}[Existence and Uniqueness of Restricted Dynamics]\label{lemma: restricted dynamics} Suppose $\mu^R_0$ is a finite measure on $S_R$ which satisfies (\textbf{A1}.), and $g^R_0 = (M^R_0, 0, E^R_0)\in S_\mathrm{gel}$. Then there exists a unique map $(\mu^R_t, g^R_t)$ on $[0, \infty)$, which solves the restricted dynamics (\ref{eq:rE1}, \ref{eq: rE2}). Moreover, for all $t\le T$, $\mu^R_t$ is a positive, finite measure on $S_R$, $P^R_t=0$ and $g^R_t \in S_\mathrm{gel}$. 
\end{lemma}
\begin{proof}[Sketch Proof of Lemma \ref{lemma: restricted dynamics}]This may be proved by a trivial modification of the arguments in \cite[Proposition 2.2]{N99}. One shows that, given a bound $\langle \pi_n+\pi_e, \mu^R_0\rangle +M^R_0+E^R_0\le C$, there is a positive time $T=T(R,C)>0$ such that the Picard iterates $(\mu^{(R,n)}_t)_{t\le T}$ converge uniformly in total variation on $[0,T]$, and that the limit $\mu^R_t$ solves (\ref{eq:rE1}, \ref{eq: rE2}), possibly allowing $\mu^R_t$ to be a signed measure. Moreover, bilinear continuity arguments in total variation norm $\|\cdot\|$ imply that the solution is unique on this interval. We now note that the quantity $\langle \pi_n+\pi_e, \mu^R_t\rangle +M^R_t+E^R_t$ is constant in time, and therefore this construction can be repeated on $[T, 2T]$, $[2T, 3T]$, etc, which proves global existence and uniqueness. Finally, an integrating factor is introduced to argue that $\mu_t$ is a positive measure.
\rpcomment{I think an easier positivity argument is that for $f$ positive and compactly supported $\frac{\dd}{\dd t} \left<f,\mu_t^R\right> \geq -\left<f,\mu_t^R\right> \times 2 \sup\left\{ \varphi(x) \colon x \in \mathrm{supp}(f)\right\}\left<\varphi, \mu_t^R\right>$, but I may be missing something.}
In our case, it is also straightforward to see that the gel data $M^R_t, E^R_t \ge 0$, and that $P^R_t=0$ from the symmetry (\textbf{A1}.)
\end{proof}

\begin{proof}[Proof of Lemma \ref{lemma: E and U}] 
We first show existence: For all $R<\infty$, we let $(\mu^R_t, g^R_t)$ be the solution to the dynamics (\ref{eq:rE1}, \ref{eq: rE2}) restricted to $S_R$, with initial data \begin{equation} \mu^R_0(dx)=1_{x\in S_R}\hspace{0.1cm}\mu_0(dx); \end{equation} \begin{equation} g^R_0 = \int_{x\not\in S_R} x\mu_0(dx). \end{equation}  Observe that, if $R<R'$, then $\widetilde{\mu}^R_t, \widetilde{g}^R_t$ given by \begin{equation} \widetilde{\mu}^R_t(dx)=1_{x\in S_R}\hspace{0.1cm}\mu^{R'}_t(dx);\end{equation} \begin{equation} \widetilde{g}^R_t=g^{R'}_t+\int_{x\in S_{R'}\setminus S_R} x \mu^{R'}_t(dx)\end{equation} solve the dymanics (\ref{eq:rE1},\ref{eq: rE2}) with the same initial data $\mu^R_0, g^R_0$. From uniqueness in Lemma \ref{lemma: restricted dynamics}, it follows that $\widetilde{\mu}^R_t=\mu^R_t; \widetilde{g}^R_t=g^R_t$. This implies that the measures $\mu^R_t$ are increasing in $R$, while the gel data $M^R_t, E^R_t$ are decreasing, and $P^R_t$ is identically $0$, by symmetry (\textbf{A1}). This implies that the limits \begin{equation} \mu_t=\lim_{R\uparrow \infty} \mu^R_t; \hspace{1cm} M_t=\lim_{R\rightarrow \infty} M^R_t; \hspace{1cm} E_t=\lim_{R\rightarrow \infty} E^R_t \end{equation} exist in the sense of monotone limits. One can then check that $\mu_t$ and $g_t=(M_t,0,E_t)$ satisfy the full equation (\ref{eq: E+G}), with initial values $\mu_0, g_0=0.$

To see uniqueness let $\mu_t$ be the solution constructed above, and write $g_t=(M_t,P_t,E_t)$ for the mass, momentum and energy of the gel. Let $\mu'_t$ be any solution to (\ref{eq: E+G}) starting at $\mu_0$, and let $g'_t=(M'_t, P'_t,E'_t)$ be the associated data of the gel. We will show that $(\mu'_t,g'_t)=(\mu_t, g_t)$ for all $t\ge 0.$ \medskip \\ For all $R<\infty$, it is simple to verify that \begin{equation} \widetilde{\mu}^R_t(dx)=1_{x\in S_R}\hspace{0.1cm} \mu'_t(dx);\hspace{1cm} \widetilde{g}^R_t=g'_t+\int_{S^\mathrm{c}_R} x \mu'_t(dx) \end{equation} is a solution to the dynamics (\ref{eq:rE1}, \ref{eq: rE2}) on $S_R$. By uniqueness in Lemma \ref{lemma: restricted dynamics}, it follows that $\widetilde{\mu}^R_t=\mu^R_t$, and taking monotone limits, we see that $\mu'_t=\lim_{R\rightarrow \infty} \widetilde{\mu}^R_t=\lim_{R\rightarrow \infty} {\mu}^R_t=\mu_t$. The argument for $g'$ is identical.
\end{proof}

\subsection{Gelation and Explosion of Truncated Dynamics}
 For a further analysis of the equation (\ref{eq: E+G}), we consider a modified equation, resulting from a more tractable kernel, which is symmetrised with respect to momentum. Let
 \begin{equation}\label{eq: modified K} 
 \begin{split}
 K^\mathrm{m}(x,y,dz)& \\  = & \left[\kappa \pi_n(x) \pi_n(y) +\gamma\left(\pi_n(x)\pi_e(y)-\pi_p(x)\cdot \pi_p(y)+ \pi_e(x)\pi_n(y)\right)\right]\delta_{x+y}(dz) \\ & + \gamma(\pi_n(x)\pi_e(y)+\pi_p(x)\cdot \pi_p(y)+ \pi_e(x)\pi_n(y))\left(\frac{\delta_{x+Ry}+\delta_{Rx+y}}{2}\right)(dz) \\ & 
 =\frac{1}{4}K(Rx, y, dz)+\frac{1}{2}K(x,y,dz)+\frac{1}{4}K(x,Ry, dz). \end{split} 
\end{equation}
Let $L^\mathrm{m}$ be the drift operator for the modified kernel $K^\mathrm{m}$, and consider the modified equation \begin{equation} \tag{mE-G}\label{eq: mE}
    \mu_t=\mu_0+\int_0^t L^\mathrm{m}(\mu_s)ds.
\end{equation}The total mass of the modified kernel is \begin{equation}
    \label{eq: modified Kbar} 
    \overline{K^\mathrm{m}}(x,y)=\kappa \pi_n(x) \pi_n(y) + 2\gamma(\pi_n(x)\pi_e(y)+\pi_e(x)\pi_n(y)).
\end{equation}
Consider a modified state space, which truncates the velocity distribution by excluding clusters with extreme kinetic energies.  For $\epsilon>0$, \begin{equation}
    S^\epsilon= \{x\in S: \epsilon \pi_n(x) \leq \pi_e(x) \leq \epsilon^{-1} \pi_n(x)\}.
\end{equation} Note that this state space is preserved under both kernels $K, K^\mathrm{m}$. Moreover, on the reduced state space $S^\epsilon$, the modified kernel $K^\mathrm{m}$ is \emph{approximately multiplicative} \cite{N00} in the sense that, for some $\delta_\epsilon>0$ and $\Delta_\epsilon<\infty$, we have \begin{equation}
    \delta_\epsilon\hspace{0.1cm} (\pi_n(x)+\pi_e(x))\hspace{0.1cm}(\pi_n(y)+\pi_e(y)) \leq \overline{K^\mathrm{m}}(x,y) \leq  \Delta_\epsilon\hspace{0.1cm} (\pi_n(x)+\pi_e(x))\hspace{0.1cm}(\pi_n(y)+\pi_e(y))
\end{equation}for all $x,y \in S^\epsilon$. For any $\epsilon$, let $\mu_0^\epsilon$ denote the restriction $\mu_0^\epsilon(dx)=1_{x\in S^\epsilon}\hspace{0.1cm}\mu_0(dx).$ We now appeal to \cite[Theorem 2.2]{N00}, on existence and uniqueness for approximately multiplicative kernels, to obtain the following, which provides the connection between gelation and explosion of a second moment.
\begin{lemma}\label{lemma: solution to modified equation}
    For all $\epsilon>0$, there is a unique maximal conservative solution  $(\nu^\epsilon_t)_{t< t_\mathrm{expl}^\epsilon}$ in $S^\epsilon$ to the modified equation (\ref{eq: mE}). Moreover, the map $t\mapsto \langle \pi_n^2+\pi_e^2, \nu^\epsilon_t\rangle$ is finite and increasing on $[0,t_\mathrm{expl}^\epsilon)$, and increases to $\infty$ as $t\uparrow t_\mathrm{expl}^\epsilon$. 
\end{lemma}

Similarly, we can also apply Corollary \ref{cor: maximal conservative solutions} to see that there exist maximal conservative solutions $(\mu^\epsilon_t)_{t<t_\mathrm{gel}^\epsilon}$ to (\ref{eq: E}) starting at $\mu^\epsilon_0$, which are given by initial segments of a global solution $(\mu^\epsilon_t)_{t\geq 0}$ to (\ref{eq: E+G}). Repeatedly exploiting uniqueness, we show that these coincide with the solution to (\ref{eq: mE}):

\begin{lemma}[Relationship of equations]\label{lemma: Relationship}
Let $\mu^\epsilon_0$ be as above, for initial data $\mu_0$ satisfying (\textbf{A1}).  Then the maximal conservative solutions $(\mu^\epsilon_t)_{t<t_\mathrm{gel}^\epsilon}$ and $(\nu^\epsilon_t)_{t<t_\mathrm{expl}^\epsilon}$, to (\ref{eq: E}) and (\ref{eq: mE}) respectively, coincide. In particular, $t_\mathrm{expl}^\epsilon = t^\epsilon_{gel}$, and the map \begin{equation}
    t\mapsto \langle \pi_n^2+\pi_e^2, \mu^\epsilon_t\rangle
\end{equation} is finite and increasing on $[0, t_\mathrm{gel}^\epsilon)$, and increases to $\infty$ as $t\uparrow t_\mathrm{gel}^\epsilon.$ \end{lemma}

\begin{proof} Firstly, we note that $( \mu^\epsilon_t \circ R^{-1})_{t<t_\mathrm{gel}^\epsilon}$ also solves (\ref{eq: E}), starts at $\mu^\epsilon_0$ and is conservative. Therefore, by uniqueness in Corollary \ref{cor: maximal conservative solutions}, we must have $\mu^\epsilon_t \circ R^{-1}=\mu^\epsilon_t$ for all $t<t_\mathrm{gel}^\epsilon.$ Therefore, for any bounded, measurable function $f$, and $t<t^\epsilon_\mathrm{gel},$ it follows from elementary manipulations that \begin{equation}\label{eq: symmetry under R1}
        \int_{S\times S\times S} (f(z)-f(x)-f(y))K(x,y,dz)\mu^\epsilon_t(dx)\mu^\epsilon_t(dy) =  \int_{S\times S\times S} (f(z)-f(x)-f(y))K(Rx,y,dz)\mu^\epsilon_t(dx)\mu^\epsilon_t(dy)
    \end{equation} and
\begin{equation}\label{eq: symmetry under R2}
        \int_{S\times S\times S} (f(z)-f(x)-f(y))K(x,y,dz)\mu^\epsilon_t(dx)\mu^\epsilon_t(dy) =  \int_{S\times S\times S} (f(z)-f(x)-f(y))K(x,Ry,dz)\mu^\epsilon_t(dx)\mu^\epsilon_t(dy).
    \end{equation} Combining these, we see that $(\mu^\epsilon_t)_{t<t^\epsilon_\mathrm{gel}}$ solves the modified equation (\ref{eq: mE}), and so, by uniqueness of the maximal conservative solution $(\nu^\epsilon_t)_{t<t_\mathrm{expl}^\epsilon}$ in Lemma \ref{lemma: solution to modified equation}, we have \begin{equation}
        t^\epsilon_\mathrm{gel} \leq t_\mathrm{expl}^\epsilon; \hspace{1cm} \mu^\epsilon_t=\nu^\epsilon_t \hspace{0.5cm}\forall t<t^\epsilon_\mathrm{gel}.
    \end{equation} The other implication is identical, using the uniqueness of the maximal conservative solution $(\nu^\epsilon_t)_{t<t_\mathrm{expl}^\epsilon}$ in Lemma \ref{lemma: solution to modified equation} to deduce that $\nu^\epsilon_t= \nu^\epsilon_t\circ R^{-1}$ for $t<t_\mathrm{expl}^\epsilon$. Hence, the equations (\ref{eq: symmetry under R1}, \ref{eq: symmetry under R2}) hold with $\nu^\epsilon_t$ in place of $\mu^\epsilon_t$, for any bounded, measurable $f$ and $t<t_\mathrm{expl}^\epsilon.$ Therefore, $(\nu^\epsilon_t)_{t<t_\mathrm{expl}^\epsilon}$ is a conservative solution to the unmodified equation (\ref{eq: E}), and so by Corollary \ref{cor: maximal conservative solutions}, \begin{equation}
        t_\mathrm{expl}^\epsilon \leq t^\epsilon_\mathrm{gel}; \hspace{1cm} \nu^\epsilon_t=\mu^\epsilon_t \hspace{0.5cm}\forall t<t_\mathrm{expl}^\epsilon.
    \end{equation} \end{proof}  

%\iffalse Repeating the calculation we have already done, we can calculate the gelation times explicitly: \begin{lemma}[Identification of Gelation Times]\label{lemma: calculation of gelation} For $\epsilon>0$, the gelation time $t_\mathrm{gel}^\epsilon$ is given by \begin{equation} t_\mathrm{gel}^\epsilon = \frac{1}{4\gamma}\sqrt{\frac{\langle \pi_n^2, \mu_0^\epsilon\rangle}{\langle \pi_e^2, \mu_0^\epsilon\rangle}}\left(\langle \pi_n^2, \mu_0^\epsilon\rangle+\sqrt{\frac{\langle \pi_n^2, \mu_0^\epsilon\rangle}{\langle \pi_e^2, \mu_0^\epsilon\rangle}}\langle \pi_n\pi_e, \mu_0^\epsilon\rangle \right)^{-1}. \end{equation} As $\epsilon \downarrow 0$, we have \begin{equation}
%    t_\mathrm{gel}^\epsilon \rightarrow t_\mathrm{gel} =\frac{1}{4\gamma}\sqrt{\frac{\langle \pi_n^2, \mu_0\rangle}{\langle \pi_e^2, \mu_0\rangle}}\left(\langle \pi_n^2, \mu_0\rangle+\sqrt{\frac{\langle \pi_n^2, \mu_0\rangle}{\langle \pi_e^2, \mu_0\rangle}}\langle \pi_n\pi_e, \mu_0\rangle \right)^{-1}.
%\end{equation}  \end{lemma} \begin{proof} \textcolor{red}{\textbf{Here's what we had before:}} From Lemma \ref{lemma: integral equation}, and standard regularity arguments, we have 
%\begin{align}
%    \frac{\dd}{\dd t}\langle \pi_n^2, \mu_t\rangle &=
%    2\kappa \langle \pi_n^2, \mu_t\rangle^2
%    + 8 \gamma \langle \pi_n \pi_e, \mu_t\rangle\langle \pi_n^2, \mu_t\rangle\\
%    \frac{\dd}{\dd t}\langle \pi_n \pi_e, \mu_t\rangle &=
%    2\kappa \langle \pi_n^2, \mu_t\rangle \langle \pi_n \pi_e, \mu_t\rangle
%    + 4 \gamma \langle \pi_n \pi_e, \mu_t\rangle^2
%    + 4 \gamma \langle \pi_n^2, \mu_t\rangle\langle \pi_e^2, \mu_t\rangle\\
%    \frac{\dd}{\dd t}\langle \pi_e^2, \mu_t\rangle &=
%    2\kappa \langle \pi_n \pi_e, \mu_t\rangle^2
%    + 8 \gamma \langle \pi_n \pi_e, \mu_t\rangle\langle \pi_e^2, \mu_t\rangle.
%\end{align}
%In the case $\kappa = 0$ note that
%\begin{equation}\label{eq:proportional}
%    \langle \pi_e^2, \mu_t\rangle = \langle \pi_e^2, \mu_0\rangle
%    \frac{\langle \pi_n^2, \mu_t\rangle}{\langle \pi_n^2, \mu_0\rangle}
%\end{equation}
%and then one easily checks that
%\begin{equation}
%    f(t) := \langle \pi_n^2, \mu_t\rangle
%           + \sqrt{\frac{\langle \pi_n^2, \mu_0\rangle}
%                        {\langle \pi_e^2, \mu_0\rangle}}
%             \langle \pi_n \pi_e, \mu_t\rangle
%\end{equation}
%satisfies
%\begin{equation}
%    \frac{\dd }{\dd t}f(t) = 4\gamma 
%    \sqrt{\frac{\langle \pi_e^2, \mu_0\rangle}
%                        {\langle \pi_n^2, \mu_0\rangle}}f(t)^2,
%\end{equation}
%which has the solution
%\begin{equation}\label{eq:quadmoment-ode}
%    f(t) = \left(\frac{1}{f(0)} - 4\gamma 
%    \sqrt{\frac{\langle \pi_e^2, \mu_0\rangle}
%                        {\langle \pi_n^2, \mu_0\rangle}}t \right)^{-1}.
%\end{equation}
%The solution to \eqref{eq:quadmoment-ode} blows up when
%\begin{equation}\label{eq:quad-tgel}
%    t = \frac{1}{4\gamma f(0)}
%             \sqrt{\frac{\langle \pi_n^2, \mu_0\rangle}
%                        {\langle \pi_e^2, \mu_0\rangle}}
%\end{equation}

%Assuming the initial condition is independent Maxwell-Boltzmann velocities in $d$ dimensions and all clusters of size 1 one has
%\begin{align*}
%    \langle \pi_n^2, \mu_0\rangle &= 1\\
%    \langle \pi_n \pi_e, \mu_0\rangle &= d\frac{\sigma^2}{2}\\
%    \langle \pi_e^2, \mu_0\rangle &= \frac{\sigma^4}{4}\left(2d + d^2\right)
%\end{align*}
%and thus $f(0) = 1 + d / \sqrt{2d + d^2}$.
%In this case the critical time given by \eqref{eq:quad-tgel} is
%\begin{equation}
%    \frac{1}{2\gamma \sigma^2 \left(d + \sqrt{2d +d^2}\right)}.
%\end{equation}
%This should be compared to the expression for the mean free time \cite[equation 4.38]{PSW17}, which is $1/4\gamma \sigma^2 d$. \end{proof} \fi 


\subsection{Convergence of the Truncated Dynamics}
We have just shown that for $\epsilon > 0$ the time at which conservation breaks down, $t_\mathrm{gel}^\epsilon$, is equal to the time $t_\mathrm{expl}^\epsilon$ at which the second moment explodes.
We now show that $t_\mathrm{gel}^\epsilon \rightarrow t_\mathrm{gel}$ and $t_\mathrm{expl}^\epsilon \rightarrow t_\mathrm{expl}$, the time at which $\mathcal{E}(t)=\langle \pi_n^2+\pi_e^2, \mu_t\rangle$ explodes in order to see that the explosion coincides with $t_\mathrm{gel}$.
\begin{lemma}\label{lemma: connecting mu-epsilon and mu}
    For $\epsilon>0$, let $(\mu^\epsilon_t)_{t\geq 0}$ be the global solution to (\ref{eq: E+G}), starting at $\mu_0^\epsilon.$ Let $M^\epsilon_t$ and $E^\epsilon_t$ be the mass and energy of the corresponding gel. Then we have  the monotonicity \begin{equation}\label{eq: monontonicity for muepsilont}
        \mu^\epsilon_t \uparrow \mu_t \text{ as }\epsilon\downarrow 0
    \end{equation} in the sense of monotone limits, and\begin{equation} \label{eq: monotonicity for gel data}
        M^\epsilon_t \uparrow M_t; \hspace{1cm} E^\epsilon_t \uparrow E_t \hspace{1cm} \text{as }\epsilon\downarrow 0.
    \end{equation}
It therefore follows that $t^\epsilon_\mathrm{gel}\downarrow t_\mathrm{gel}$. Noting the relation \begin{equation}\label{eq: Mepsilon and Eepsilon}
        \epsilon M^\epsilon_t \leq E^\epsilon_t \leq \epsilon^{-1}M^\epsilon_t
    \end{equation}it follows that, if $t>t_\mathrm{gel}$, then $M_t >0$ and $E_t>0$.
\end{lemma}
The proof of the monotonicity principle closely follows \cite[Propositions 2.4,  2.7]{N00}. We indicate below how these arguments should be modified to prove our result.
\begin{proof} Let $0<\epsilon'<\epsilon$, so that $\mu_0^\epsilon\le\mu_0^{\epsilon'}.$ For $R<\infty$, let $\mu^{\epsilon, R}_t, g^{\epsilon, R}_t=(M^{\epsilon,R}_t,P^{\epsilon,R}_t,E^{\epsilon,R}_t)$ be the approximations used in the proof of Lemma \ref{lemma: E and U} by restricting to $S_R$, given as the solution to (\ref{eq:rE1}, \ref{eq: rE2}) with initial data
\begin{equation} \mu^{\epsilon,R}_0(dx)=1_{x\in S_R \cap S^\epsilon}\hspace{0.1cm}\mu_0(dx);\hspace{1cm} g^{\epsilon,R}_0=\int_{S^\epsilon\setminus S_R}x\mu_0(dx).\end{equation}
and define $\mu^{\epsilon',R}_t, g^{\epsilon',R}_t$ similarly, with $\epsilon'$ in place of $\epsilon$. We observe that $\mu^{\epsilon,R}_0\le \mu^{\epsilon',R}_0$ and similarly $M^{\epsilon,R}_0\le M^{\epsilon',R}_0$, $E^{\epsilon,R}_0\le E^{\epsilon',R}_0$. \medskip \\ 
We introduce an integrating factor as follows. For $x\in S_R$, we define $\theta_t(x)$ by 
\begin{equation} \theta_t(x)=\exp\left( \int_0^t\left[\int_{S_R}\overline{K}(x,y) (\mu^{\epsilon',R}_s+\mu^{\epsilon,R}_s(d y)+\overline{K}\left(x,g^{\epsilon',R}_s\right)\right]ds\right)
\end{equation}
and define the signed measure on $S_R$ \begin{equation}
    \xi_t(\dd x) = \theta_t(x) \left(\mu^{\epsilon',R}_t - \mu^{\epsilon,R}_t\right)(\dd x).
\end{equation} Using a Leibniz rule analagous to \cite[Proposition 2.5]{N00}, one can check that, for any bounded, measurable function $f$ on $S_R$, \begin{equation} \begin{split} &\frac{d}{dt}\langle f, \xi_t\rangle = \left\langle f \theta_t, \frac{d}{dt}(\mu^{\epsilon',R}_t-\mu^{\epsilon, R}_t)\right\rangle +\left\langle f\frac{\partial \theta_t}{\partial t},\mu^{\epsilon',R}_t-\mu^{\epsilon, R}_t  \right\rangle \\[1ex]& = \frac{1}{2}\int_{S_R\times S_R}\left\{(f\theta_t)(x+y)1_{x+y\in S_R}-(f\theta_t)(x)-(f\theta_t)(y)\right\}\overline{K}(x,y)\left(\mu^{\epsilon',R}_t(dx)\mu^{\epsilon',R}_t(dy)-\mu^{\epsilon,R}_t(dx)\mu^{\epsilon,R}_t(dy)\right) \\ &\hspace{2cm} -\int_{S_R}(f\theta_t)(x)\left(\overline{K}\left(x, g^{\epsilon',R}_t\right)\mu^{\epsilon',R}_t(dx)-\overline{K}\left(x,g^{\epsilon,R}_t\right)\mu^{\epsilon,R}_t(dx)\right) \\  & \hspace{2cm}+ \int_{S_R\times S_R} (f\theta_t)(x)\overline{K}(x,y)\left(\mu^{\epsilon',R}_t(dx)-\mu^{\epsilon,R}_t(dx)\right)\left(\mu^{\epsilon',R}_t(dx)-\mu^{\epsilon,R}_t(dx)\right) \\&\hspace{2cm}  + \int_{S_R} (f\theta_t)(x)\overline{K}\left(x, g^{\epsilon',R}_t\right)\mu^{\epsilon',R}_t(dx). \end{split} \end{equation}We observe that this simplifies to a differential equation \begin{equation} \label{eq: ODE for xit}\frac{d}{dt}\xi_t=H_t(\xi_t)+\alpha_t;\hspace{1cm} \xi_0=\mu^{\epsilon',R}_0-\mu^{\epsilon,R}_0\end{equation} where we define signed measures $H_t(\xi), \alpha_t$ for signed measures $\xi$ on $S_R$, by specifying, for all bounded measurable functions $f$, \begin{equation} H_t(\xi) = \frac{1}{2}\int_{S_R\times S_R}(f\theta_t)(x+y)1_{x+y\in S_R}\overline{K}(x,y)(\mu^{\epsilon',R}_t(dx)+\mu^{\epsilon,R}_t(dx))(\theta_t^{-1}\xi)(dy) \end{equation} and  \begin{equation} \langle f, \alpha_t\rangle = \int_{S_R} (f\theta_t)(x)\overline{K}\left(x,g^{\epsilon,R}_t\right)\mu^{\epsilon,R}_t(dx). \end{equation} We now write $\mathcal{M}(S_R)$ for the space of signed measures on $S_R$, equipped with the total variation norm, and write $\mathcal{M}^+(S_R)$ for the positive measures. It is straightforward to see that that $H_t: \mathcal{M}(S_R)\rightarrow \mathcal{M}(S_R)$ is a bounded linear map, and that both the norm $\|\alpha_t\|$ and the operator norm $\|H_t\|$ are bounded on compact time intervals. It therefore follows, by standard Gr\"onwall-style techniques, that, for any initial data $\xi_0$, the solution $\xi_t$ to (\ref{eq: ODE for xit}) is unique, and that the Picard iterates \begin{equation} \label{eq: Picard iterates} \xi^{0}_t=\xi_0; \hspace{1cm} \xi^{n+1}_t=\xi_0+\int_0^t \left(H_s\left(\xi^n_s\right)+\alpha_s\right) ds \end{equation} converge to $\xi_t$ in total variation, uniformly on compact time intervals. \medskip \\ We now note that $\alpha_t \in \mathcal{M}^+(S_R)$ is a positive measure for all $t\ge 0$, and that if $\xi\in \mathcal{M}^+(S_R)$, then $H_t(\xi) \in \mathcal{M}^+$ for all $t\ge 0$. Therefore, by induction, each iterate $\xi^n_t$ is a positive measure: $\xi^n_t\in \mathcal{M}^+(S_R)$ for all $t\ge 0$. Since $\xi^n_t\rightarrow \xi_t=\theta_t(\mu^{\epsilon',R}_t-\mu^{\epsilon,R}_t)$ in total variation for all $t\ge 0$, this implies that $\mu^{\epsilon,R}_t\le \mu^{\epsilon',R}_t$ for all $t\ge 0.$ Taking a limit $R\rightarrow \infty,$ $\mu^{\epsilon,R}_t\uparrow \mu^{\epsilon}_t$ in the sense of monotone limits, and similarly for $\mu^{\epsilon'}_t$. Together, these imply that $\mu^{\epsilon}_t \le \mu^{\epsilon'}_t$, and therefore the map $\epsilon\mapsto \mu^{\epsilon}_t$ is increasing as $\epsilon \downarrow 0.$ If we define \begin{equation} \widetilde{\mu}_t = \lim_{\epsilon \downarrow 0} \mu^{\epsilon}_t \end{equation} then it is straightforward to verify that $\widetilde{\mu}_t$ satisfies the Smoluchowski dynamics (\ref{eq: E+G}), with $\widetilde{\mu}_0=\mu_0$; by uniqueness, this implies that $\widetilde{\mu}_t=\mu_t$. \medskip \\  This proves the monotonicity principle (\ref{eq: monontonicity for muepsilont}) for $\mu^\epsilon_t$ and $\mu_t$. For the equivalent (\ref{eq: monotonicity for gel data}) for the gel data, the argument is similar. We note first that that, for all $R>0$ and $0<\epsilon'<\epsilon$, the initial data satisfy \begin{equation} M^{\epsilon,R}_0 \le M^{\epsilon',R}_0; \hspace{1cm} E^{\epsilon,R}_0 \le E^{\epsilon',R}_0.\end{equation} Therefore, from the monotonicity $\mu^{\epsilon,R}_t \le \mu^{\epsilon',R}_t$ and the evolution equation (\ref{eq: rE2}), it immediately follows that \begin{equation} M^{\epsilon,R}_t \le M^{\epsilon',R}_t; \hspace{1cm} E^{\epsilon,R}_t \le E^{\epsilon',R}_t\end{equation} and taking $R\rightarrow \infty$, \begin{equation} M^{\epsilon}_t \le M^{\epsilon'}_t; \hspace{1cm} E^{\epsilon}_t \le E^{\epsilon'}_t.\end{equation} Therefore, the map $\epsilon\mapsto M^{\epsilon}_t$ is increasing as $\epsilon\downarrow 0$. Now, using monotone convergence twice as $\epsilon\downarrow 0$, \begin{equation} \begin{split} M^\epsilon_t &= \langle \pi_n, \mu^\epsilon_0\rangle -\langle \pi_n, \mu^\epsilon_t\rangle \\& \rightarrow\langle \pi_n, \mu_0\rangle -\langle \pi_n, \mu_t\rangle = M_t \end{split}\end{equation} which implies that $M^\epsilon_t\uparrow M_t$ as $\epsilon\downarrow 0$. The case for the energy is identical. \medskip \\  For the relation (\ref{eq: Mepsilon and Eepsilon}), we use a similar limiting argument. By definition, $\mu^{\epsilon, R}_t$ is supported on $S^\epsilon$, and the initial data satisfy \begin{equation} \epsilon M^{\epsilon,R}_0\le E^{\epsilon,R}_0\le \epsilon^{-1}M^{\epsilon,R}_0. \end{equation} Therefore, immediately from (\ref{eq: rE2}), \begin{equation} \epsilon M^{\epsilon,R}_t\le E^{\epsilon,R}_t\le \epsilon^{-1}M^{\epsilon,R}_t. \end{equation} The claimed bound (\ref{eq: Mepsilon and Eepsilon}) follows on taking the limit $R\rightarrow \infty$. \medskip \\ For the final point, let $t>t_\mathrm{gel}$. Then, for some $\epsilon>0$ small enough, $t>t^\epsilon_\mathrm{gel}$. From the bound (\ref{eq: Mepsilon and Eepsilon}), if one of $M^\epsilon_t$ and $E^\epsilon_t$ is strictly positive, then they both are, and so both $M^\epsilon_t, E^\epsilon_t>0$. Therefore \begin{equation} M_t\ge M^\epsilon_t >0;\hspace{1cm} E_t\ge E^\epsilon_t >0 \end{equation} as claimed \end{proof}
\subsection{Gelation and Explosion of non-Truncated Dynamics} We now turn to the behaviour of the second moment $\mathcal{E}(t)=\langle \pi_n^2+\pi_e^2, \mu_t\rangle$ in the subcritical phase $t<t_\mathrm{gel}$. Using the previous results, we show that the gelation coincides exactly with the blowup of the second moment. 
\begin{lemma}\label{lemma: second moment before tgel} The second moment $\mathcal{E}(t)=\langle \pi_n^2+\pi_e^2, \mu_t\rangle$ is finite and increasing on $[0, t_\mathrm{gel})$, and increases to infinity as $t\uparrow t_\mathrm{gel}$. \end{lemma}  We say that a local solution $(\nu_t)_{t<T}$ to (\ref{eq: E}) is \emph{strong} if, for all times $t<T$, 
\begin{equation}
    \int_0^t  \hspace{0.1cm} \langle\pi_n^2+\pi_e^2, \nu_s\rangle  \hspace{0.1cm} ds<\infty.
\end{equation} We use the following result on the existence and uniqueness of strong solutions. 
\begin{lemma}\label{lemma: strong solutions} For any finite measure $\mu_0$ with $\langle \pi_n^2+\pi_e^2, \mu_0\rangle <\infty$, there is a unique maximal strong solution $(\mu'_t)_{t<t_\mathrm{expl}(\mu_0)}$ to (\ref{eq: E}), starting at $\mu_0$, and with $t_\mathrm{expl}(\mu_0)>0$, such that $ \langle \pi_n^2+\pi_e^2, \mu'_t\rangle$ is increasing on $[0,t_\mathrm{expl}(\mu))$. If $t_\mathrm{expl}(\mu)<\infty$, then $\langle \pi_n^2+\pi_e^2, \mu'_t\rangle$ increases to $\infty$  as $t\uparrow t_\mathrm{expl}(\mu)$.  \end{lemma} When the measure $\mu_0$ is clear, we will omit the argument of $t_\mathrm{expl}.$ 
\begin{proof}
    This is almost a special case of \cite[Thrm. 2.1]{N00}. From \cite[Theorem 2.1]{N00} it follows that, for any finite measure $\mu_0$ with $\langle \pi_n^2+\pi_e^2, \mu_0\rangle <\infty$, there exists a maximal strong solution $(\mu'_t)_{t<t_\mathrm{expl}(\mu_0)}$. Moreover, there exists a constant $C=C(\kappa, \gamma)>0$ such that, for all such $\mu_0$,  $t_\mathrm{expl}(\mu_0) \ge C \langle \pi_n^2+\pi_e^2, \mu_0\rangle$. By applying this bound to $\mu'_t$, if $t_\mathrm{expl}(\mu_0)<\infty,$ then $\langle \pi_n^2 +\pi_e^2, \mu'_t\rangle \ge (C(t_\mathrm{expl}(\mu_0)-t))^{-1}$  which implies the claimed divergence.
\end{proof}

It follows immediately that $(\mu'_t)_{t<t_\mathrm{expl}}$ is a  conservative solution to (\ref{eq: E}), and so $t_\mathrm{expl} \le t_\mathrm{gel}$, and $\mu'_t=\mu_t$ for all $t<t_\mathrm{expl}$, by Corollary \ref{cor: maximal conservative solutions}. It remains to show that $t_\mathrm{expl}\ge t_\mathrm{gel}$. \medskip \\ Following the ideas of \cite[Proposition 2.7]{N00}, we obtain the integral relations, for all $t<t_\mathrm{expl}$, \begin{equation} \label{eq: ODE1}
    \langle \pi_n^2, \mu_t\rangle =
    \langle \pi_n^2, \mu_0\rangle + \int_0^t \left[\kappa\langle \pi_n^2, \mu_s\rangle^2+4\gamma\langle \pi_n\pi_e, \mu_t\rangle\langle\pi_n^2, \mu_s\rangle \right] ds;
\end{equation} 

\begin{equation}\label{eq: ODE2}
    \langle \pi_n \pi_e, \mu_t\rangle =
    \langle \pi_n\pi_e, \mu_0\rangle + \int_0^t \left[\kappa\langle \pi_n^2, \mu_s\rangle\langle \pi_n\pi_e, \mu_s\rangle+2\gamma\langle \pi_n\pi_e, \mu_t\rangle^2+2\gamma\langle\pi_n^2, \mu_s\rangle\langle \pi_e^2, \mu_s \rangle \right] ds;
\end{equation}

\begin{equation} \label{eq: ODE3}
    \langle \pi_e^2, \mu_t\rangle =
    \langle \pi_e^2, \mu_0\rangle + \int_0^t \left[\kappa\langle \pi_n\pi_e, \mu_s\rangle^2+4\gamma\langle \pi_n\pi_e, \mu_t\rangle\langle\pi_e^2, \mu_s \rangle \right] ds.
\end{equation} First of all, these imply that $\mathcal{E}(t)=\langle \pi_n^2+\pi_e^2, \mu_t\rangle$ is bounded on compact subsets of $[0, t_\mathrm{expl})$, and in particular cannot diverge before $t_\mathrm{expl}$. Combining this with Lemma \ref{lemma: strong solutions}, the maximal time $t_\mathrm{expl}$ of existence of a strong solution is precisely the first time at which the second moment $\mathcal{E}(t)$ diverges, or $\infty$ if there is no divergence. \medskip \\ We also remark on the relationship of this result to the solutions $(\mu^\epsilon)_{t<t_\mathrm{expl}^\epsilon}$ discussed in Lemmas \ref{lemma: solution to modified equation}, \ref{lemma: Relationship}. It is clear, from Lemma \ref{lemma: solution to modified equation}, that $(\nu^\epsilon_t)_{t<t_\mathrm{expl}^\epsilon}$ is a strong solution. Moreover, in view of the comments above, and since $\langle \pi_n^2+\pi_e^2, \nu^\epsilon_t\rangle \uparrow \infty$ as $t\uparrow t_\mathrm{expl}^\epsilon$, it follows that that $(\mu^{\epsilon}_t)_{t<t_\mathrm{expl}^\epsilon}$ is the maximal strong solution with initial data $\mu^\epsilon_0.$ This justifies the use of the notation $t^\epsilon_\mathrm{expl}$ in Lemma \ref{lemma: solution to modified equation}.  \medskip \\  Using standard regularity arguments, we may view (\ref{eq: ODE1}, \ref{eq: ODE2}, \ref{eq: ODE3}) as a differential equation for the three moments $\langle \pi_n^2, \mu_t\rangle, \langle \pi_n \pi_e, \mu_t\rangle, \langle \pi_e^2, \mu_t\rangle$ and, from the discussion above, the blow-up time to the ODE system is exactly $t_\mathrm{expl}$. An identical argument holds for $\mu^\epsilon_t$, which blows up at $t^\epsilon_\mathrm{expl}$, which we know coincides with the corresponding gelation time $t^\epsilon_\mathrm{gel}$. To connect the explosion times for truncated and non-truncated data, we now analyse the system of ODEs (\ref{eq: ODE1} - \ref{eq: ODE3}).
\begin{lemma}\label{lemma: ODE considerations} Consider the ordinary differential equation $\dot{q}_t=b(q_t)$ in $\mathbb{R}^3$, where $b$ is the locally Lipschitz field given by \begin{equation} \label{eq: system of ODEs} b(q_1,q_2,q_3)=\begin{pmatrix}\kappa q_1^2+4\gamma q_1q_2 \\ \kappa q_1q_2+2\gamma q_2^2+2\gamma q_1q_3 \\ \kappa q_2^2 + 4\gamma q_2q_3 \end{pmatrix}. \end{equation} Then, for all $q_0\in \mathbb{R}^3$, there exists a unique maximal solution $\phi(q_0, t)$ starting at $q_0$, defined until time $\zeta(q_0)\in (0, \infty]$. Consider the sets \begin{equation} E=(0, \infty)^3; \hspace{1cm} E_\delta=[\delta,\infty)^3.\end{equation} Then, if $q_0 \in E_\delta$ for some $\delta>0$, then the solution $(\phi(q_0,t))_{t<\zeta(q_0)} \subset E_\delta$. We have the following properties: \begin{enumerate}[label=\roman{*}).] \item Let $J_\epsilon$ be the set \begin{equation} J_\epsilon  =\{q \in E: \hspace{0.2cm} \zeta(q)\ge\epsilon\}.\end{equation} If $\gamma>0$, then for all $\epsilon, \delta>0$, the set $E_\delta \cap J_\epsilon $ is bounded. Moreover, $\zeta<\infty$ everywhere. \item Suppose $q^n_0 \in E$ and $q^n_0 \rightarrow q_0 \in E$. Then $\zeta(q^n_0)\rightarrow \zeta(q_0).$ \item Suppose $I\subset \RR_+$ is an open interval, and the map $q_0: I\rightarrow E$ is continuous, and such that $t<\zeta(q_0(t))$ for all $t\ge 0.$ Then the map $I\rightarrow E, t\mapsto \phi(q_0(t), t)$ is continuous. 
 \end{enumerate} \end{lemma} 

\begin{proof} \begin{enumerate}[label=\roman{*}).]
    \item For the claimed boundedness, we let $\zeta_0$ denote the blowup time for the dynamics (\ref{eq: system of ODEs}) with $\kappa=0$. It is straightforward to see that $\zeta(q)\le\zeta_0(q)$ for all $q\in E$, and so it is sufficient to show that $E_\delta\cap \{q: \zeta_0(q)\ge \epsilon\}$ is bounded. We argue using the following explicit computation. \medskip \\ Let $q(0)=(q_1(0),q_2(0),q_3(0))\in E$, and let $q(t)=(q_1(t),q_2(t),q_3(t))$ be the solution to (\ref{eq: system of ODEs}) starting at $q(0)$. It is then straightforward to see that \begin{equation}
        \frac{1}{q_1(t)}\frac{d}{dt}q_1(t)=\frac{1}{q_3(t)}\frac{d}{dt}q_3(t)
    \end{equation} which implies that $q_3(t)=q_1(t)q_3(0)/q_1(0)$ for all $t\ge 0$. Now, the linear combination $\widetilde{q}(t)$ given by \begin{equation}
        \widetilde{q}(t)=q_1(t)+\sqrt{\frac{q_1(0)}{q_3(0)}}q_2(t)
    \end{equation} has the same blowup time as $q(t)$, and satisfies the ODE \begin{equation} \frac{d}{dt}\widetilde{q}(t)=4\gamma\sqrt{\frac{q_3(0)}{q_1(0)}}\hspace{0.1cm}\widetilde{q}(t)^2 \end{equation} which has the unique solution \begin{equation} \widetilde{q}(t)=\left(\frac{1}{\widetilde{q}(0)}-4\gamma\sqrt{\frac{q_3(0)}{q_1(0)}}t\right)^{-1}; \hspace{1cm} t< \frac{1}{4\gamma\widetilde{q}(0)}\sqrt{\frac{q_1(0)}{q_3(0)}}. \end{equation} In terms of the initial data $q(0)$, this gives the blowup time as \begin{equation} \zeta(q(0))=\frac{1}{4\gamma}\left(\sqrt{q_1(0)q_3(0)}+q_2(0)\right)^{-1} \end{equation} which converges to $0$ as $q(0)\rightarrow \infty$ in $E_\delta.$ This shows that $E_\delta \cap \{q: \zeta_0(q)\ge \epsilon\}$ is bounded, as claimed. \medskip \\ For the case where $\gamma>0$, the same computation also shows that $\zeta(q)<\infty$ for all $q\in E$ . If $\gamma=0, \kappa>0$, this is an elementary computation.
    \item The lower semicontinuity of explosion times is standard, and follows from the continuous dependence on the initial data. Therefore, it is sufficient to prove that $\limsup_{n\rightarrow \infty} \zeta(q^n)\le \zeta(q).$ In the case $\gamma=0, \kappa>0$, this is an elementary explicit calculation; for the rest of this item, we exclude this case and consider only $\gamma>0.$ \medskip \\ Suppose, for a contradiction, that for some $\delta>0$, we have $\limsup_{n\rightarrow \infty} \zeta(q^n)>\zeta(q)+\delta$; write $\tau=\zeta(q)$. By passing to a subsequence, we may assume that $\zeta(q^n)>\tau+\delta$ for all $n$. Moreover, since $q^n\rightarrow q \in E$, we may assume that $q^n, q \in E_\delta$ for all $n$, for some positive $\delta>0$, which implies that $\phi(q^n,t)\in E_\delta$ for all $t<\zeta(q^n)$ and all $n\in \mathbb{N}$.\medskip\\  Now, if $t\le \tau$, we have $\zeta(\phi_t(q^n))=\zeta(q^n)-t \ge \delta$, which implies the containment \begin{equation} \{\phi_t(q^n): t\le \tau, n\ge 1\} \subset E_\delta\cap J_\epsilon \end{equation} which we know, from item i)., to be bounded: for some $C<\infty$, \begin{equation}
        \{\phi_t(q^n): t\le \tau, n\ge 1\} \subset [0,C]^3.
    \end{equation} By the lemma of leaving compact sets, there exists $s<\tau$ such that, for all $t\in (s,\tau)$, $\phi_t(q)\not \in [0,C]^3.$ However, if we pick $t\in (s,\tau)$, we have $\phi_t(q^n) \rightarrow \phi_t(q)$, by the continuity of the dependence in the initial conditions, which is a contradiction. Therefore, $\limsup_{n\rightarrow \infty} \zeta(q^n)\le \zeta(q)$, which proves the claimed convergence.      
    \item As in item ii)., the case $\gamma=0, \kappa>0$ can be checked by an explicit, elementary computation. For the remainder of this point, we consider only $\gamma>0$. \medskip \\ Firstly, we note that by ii)., the map $t\mapsto \zeta(q_0(t))$ is continuous on $I$. Therefore, fixing $t\in I$, we may choose choose  $\epsilon, \delta > 0$ such that, if $\abs{t-s} \le \delta$, then $s\in I$ and $s < \min \left(\zeta\left(q_0(s)\right), \zeta\left(q_0(t)\right)\right)-\epsilon$. Now, we observe that, for $s\in [t-\delta, t+\delta],$\begin{equation}
    |\phi(q_0(t),t)-\phi(q_0(s), s)|\le|\phi(q_0(t),t)-\phi(q_0(s), t)|+|\phi(q_0(s),t)-\phi(q_0(s), s)|.
\end{equation} As $s\rightarrow t$, the first term converges to $0$ by continuity of the ODE solution $s\mapsto \phi(x_0(t),s)$; it is therefore sufficient to control the second term. We observe that, for all $s\in[t-\delta, t+\delta],$ we have $\zeta(\phi(x_0(s),s))=\zeta(x_0(s))-s>\epsilon$. Moreover, by compactness, there exists some $\eta>0$ such that $q_0(s) \in E_\eta$ for all $s\in [t-\delta, t+\delta]$, and so $\phi(q_0(s),u)\in E_\eta$ for all $u\ge 0.$ However, we showed in point i). above that the region $E_\eta \cap J_\epsilon=\{q \in E_\eta: \zeta(q)\geq\epsilon\}$ is compact  and so there exists a constant $M=M(\epsilon)$: for all $s\in[t-\delta, t+\delta]$, and for all $u \le t+\delta$, \begin{equation} |b(\phi(q_0(s), u)| \le M. \end{equation}
This implies the bound, for all $s\in[t-\delta,t+\delta]$, \begin{equation} |\phi(x_0(s), t)-\phi(x_0(s), s)| \le M|t-s|\end{equation} which implies the claimed continuity.
\end{enumerate}  \end{proof}



We can now use this to prove our main result Lemma \ref{lemma: second moment before tgel} on the second moment $\mathcal{E}(t)$ in the subcritical phase.





\begin{proof}[Proof of Lemma \ref{lemma: second moment before tgel}]
Let $\mu_0$ be any measure on $S$ satisfying (\textbf{A1}-\textbf{4}.), and let $(\mu_t)_{t\ge 0}$ be the associated solution to (\ref{eq: E+G}). From the discussion following Lemma \ref{lemma: strong solutions}, it is sufficient to show that $t_\mathrm{gel}=t_\mathrm{expl}<\infty$. We also recall that $t_\mathrm{expl}$ is characterised as the explosion time $\zeta$ of the ODE system (\ref{eq: ODE1}-\ref{eq: ODE3}). \medskip \\ Firstly, since the base measure $m$ of $\mu_0$ is not a multiple of the point mass $\delta_0$, all the quadratic moments $q$ of $\mu_0$ are strictly positive, and so $q \in E$. Therefore, by the second point of Lemma \ref{lemma: ODE considerations}, the explosion time $t_\mathrm{expl}=\zeta(q)<\infty.$\medskip \\ 
As $\epsilon \downarrow 0$, the quadratic moments $q^\epsilon$ of $\mu^\epsilon_0$ converge to the quadratic moments $q\in E$ of $\mu_0$ by dominated convergence.  Therefore, by Lemma~\ref{lemma: ODE considerations},  $t^\epsilon_\mathrm{expl}=\zeta(q^\epsilon)\rightarrow \zeta(q)= t_\mathrm{expl}$.
By Lemmas \ref{lemma: Relationship}, ~\ref{lemma: connecting mu-epsilon and mu}, we know that $t_\mathrm{expl}^\epsilon=t_\mathrm{gel}^\epsilon$ and that $t_\mathrm{gel}^\epsilon \rightarrow t_\mathrm{gel}$. Together, these imply that $t_\mathrm{expl} = t_\mathrm{gel}$.

\end{proof} 

%Reference for lowersemicontinuity: https://mathoverflow.net/questions/114747/dependence-of-the-blow-up-time-of-existence-of-an-ode-with-respect-to-initial-co