\section{Behaviour of the Second Moment}
\label{sec: finiteness of second moment} In this section, we consider part 2 of Theorem \ref{thrm: Smoluchowski equation}, concerning the behaviour of the second moment $\mathcal{E}(t)=\langle \varphi^2, \mu_t\rangle$. Following \cite{N00}, one might expect that the gelation time $t_\mathrm{g}$ corresponds to a divergence of $\mathcal{E}(t)$ as $t\uparrow t_\mathrm{g}$; by an approximation argument, we will show that this is indeed the case. We also introduce a \emph{duality argument}, corresponding to Theorem \ref{thrm: coupling supercritical and subcritical}, which allows us to prove that $\mathcal{E}$ is finite on $(t_\mathrm{g}, \infty)$. The final assertion follows from the fact that $M_{t_\mathrm{g}}=E_{t_\mathrm{g}}=0$, which is the content of Corollary \ref{corr: gel at tgel}.
\subsection{Subcritical Regime} We first deal with the subcritical regime $[0, t_\mathrm{g})$, to show that the second moment $\mathcal{E}(t)$ is finite and increasing on this interval, and that $t_\mathrm{g}$ is exactly the first time at which $\mathcal{E}$ diverges.
\begin{lemma}\label{lemma: second moment before tgel} Suppose $\mu_0$ satisfies (A1-4.), and let $(\mu_t)_{t\ge 0}$ be the corresponding solution to (\ref{eq: E+G}). The second moment $\mathcal{E}(t)=\langle \varphi^2, \mu_t\rangle$ is finite and increasing on $[0, t_\mathrm{g})$, and increases to infinity as $t\uparrow t_\mathrm{g}$, where $t_\mathrm{g}$ is the associated gelation time. \end{lemma} 

The ideas of this argument follow \cite{N00}, where there is a similar result for \emph{approximately multiplicative} kernels, for which the total rate $\overline{K}(x,y)$ is bounded above \emph{and below} by nonzero multiples of $\widetilde{\varphi}(x)\widetilde{\varphi}(y)$, where $\widetilde{\varphi}$ is a mass function playing the same r\^ole as our $\varphi$. Unfortunately, this cannot be applied directly, for two reasons. \begin{enumerate}[label=\roman{*}).]
    \item Firstly, the total rate in (\ref{eq: overline K}) contains the term $\pi_p(x)\cdot\pi_p(y)$ of indefinite sign.
    \item Secondly, the remaining combination of $\pi_n, \pi_e$ is not in an approximately multiplicative form: particles of either very low or very high energy prevent the desired lower bound from holding for any positive constant.
\end{enumerate}
We deal with these as follows. To deal with item i)., we introduce a symmetrised kernel $K^\mathrm{m}$, and use the symmetry (A1.) to argue that the solutions coincide exactly with solutions to the original equation (\ref{eq: E}). To deal with the degeneracy in point ii), we consider a truncated state space $S^\epsilon$, which excludes particles of either very high or very low energy. In this context, the kernel $K^\mathrm{m}$ is approximately multiplicative, so the results of \cite{N00} apply; we then carefully justify taking the limit $\epsilon \downarrow 0$. \medskip \\ 
We first consider the symmetrised equation. Let $K^\mathrm{m}$ be the kernel on $S\times S\times S$ given by
 \begin{equation}\label{eq: modified K} 
 \begin{split}
 K^\mathrm{m}(x,y,dz)
 =\frac{1}{4}K(Rx, y, dz)+\frac{1}{2}K(x,y,dz)+\frac{1}{4}K(x,Ry, dz). \end{split} 
\end{equation}
Let $L^\mathrm{m}$ be the drift operator for the modified kernel $K^\mathrm{m}$, and consider the modified equation \begin{equation} \tag{mE-G}\label{eq: mE}
    \mu_t=\mu_0+\int_0^t L^\mathrm{m}(\mu_s)ds.
\end{equation}The total rate of the modified kernel is \begin{equation}
    \label{eq: modified Kbar} 
    \overline{K^\mathrm{m}}(x,y)=\kappa \pi_n(x) \pi_n(y) + 2\gamma(\pi_n(x)\pi_e(y)+\pi_e(x)\pi_n(y)).
\end{equation}
Consider a modified state space, which truncates the velocity distribution by excluding clusters with extreme kinetic energies: for $\epsilon>0$, let \begin{equation}
    S^\epsilon= \{x\in S: \epsilon \pi_n(x) \leq \pi_e(x) \leq \epsilon^{-1} \pi_n(x)\}.
\end{equation} Note that this state space is preserved under both kernels $K, K^\mathrm{m}$. Moreover, on the reduced state space $S^\epsilon$, the modified kernel $K^\mathrm{m}$ is \emph{approximately multiplicative} \cite{N00} in the sense that, for some $\delta_\epsilon>0$ and $\Delta_\epsilon<\infty$, we have \begin{equation}
    \delta_\epsilon\hspace{0.1cm}\varphi(x)\varphi(y) \leq \overline{K^\mathrm{m}}(x,y) \leq  \Delta_\epsilon\hspace{0.1cm}\varphi(x)\varphi(y)
\end{equation}for all $x,y \in S^\epsilon$. For any $\epsilon$, let $\mu_0^\epsilon$ denote the restriction $\mu_0^\epsilon(dx)=1_{x\in S^\epsilon}\hspace{0.1cm}\mu_0(dx).$ We now appeal to \cite[Theorem 2.2]{N00}, on existence and uniqueness for approximately multiplicative kernels, to obtain the following, which provides the connection between gelation and explosion of a second moment.
\begin{lemma}\label{lemma: solution to modified equation}
    Suppose (A1-4.) hold, and let $\mu^\epsilon_0$ be as above. For all $\epsilon>0$, there is a unique maximal conservative solution  $(\nu^\epsilon_t)_{t< t_\mathrm{e}^\epsilon}$ in $S^\epsilon$ to the modified equation (\ref{eq: mE}), starting from $\mu_0^\epsilon$. Moreover, the map $t\mapsto \langle \varphi^2, \nu^\epsilon_t\rangle$ is finite and increasing on $[0,t_\mathrm{e}^\epsilon)$, and increases to $\infty$ as $t\uparrow t_\mathrm{e}^\epsilon$. 
\end{lemma}

Similarly, we can also apply Corollary \ref{cor: maximal conservative solutions} to see that there exist maximal conservative solutions $(\mu^\epsilon_t)_{t<t_\mathrm{g}^\epsilon}$ to (\ref{eq: E}) starting at $\mu^\epsilon_0$, which are given by initial segments of a global solution $(\mu^\epsilon_t)_{t\geq 0}$ to (\ref{eq: E+G}). Repeatedly exploiting uniqueness, we show that these coincide with the solution to (\ref{eq: mE}):

\begin{lemma}[Relationship of equations]\label{lemma: Relationship}
Let $\mu^\epsilon_0$ be as above, for initial data $\mu_0$ satisfying ({A1}-4.).  Then the maximal conservative solutions $(\mu^\epsilon_t)_{t<t_\mathrm{g}^\epsilon}$ and $(\nu^\epsilon_t)_{t<t_\mathrm{e}^\epsilon}$, to (\ref{eq: E}) and (\ref{eq: mE}) respectively, coincide. In particular, $t_\mathrm{e}^\epsilon = t^\epsilon_\mathrm{g}$, and the map \begin{equation}
    t\mapsto \langle \varphi^2, \mu^\epsilon_t\rangle
\end{equation} is finite and increasing on $[0, t_\mathrm{g}^\epsilon)$, and increases to $\infty$ as $t\uparrow t_\mathrm{g}^\epsilon.$ \end{lemma}

\begin{proof} Firstly, we note that $( \mu^\epsilon_t \circ R^{-1})_{t<t_\mathrm{g}^\epsilon}$ also solves (\ref{eq: E}), starts at $\mu^\epsilon_0$ by (A1.), and is conservative. Therefore, by uniqueness in Corollary \ref{cor: maximal conservative solutions}, we must have $\mu^\epsilon_t \circ R^{-1}=\mu^\epsilon_t$ for all $t<t_\mathrm{g}^\epsilon.$ Therefore, for any bounded, measurable function $f$, and $t<t^\epsilon_\mathrm{g},$ it follows from elementary manipulations that
\begin{multline}\label{eq: symmetry under R}
        \int_{S^3} (f(z)-f(x)-f(y))K(x,y,dz)\mu^\epsilon_t(dx)\mu^\epsilon_t(dy)
 =  \int_{S^3} (f(z)-f(x)-f(y))K(Rx,y,dz)\mu^\epsilon_t(dx)\mu^\epsilon_t(dy) \\
 =  \int_{S^3} (f(z)-f(x)-f(y))K(x,Ry,dz)\mu^\epsilon_t(dx)\mu^\epsilon_t(dy).
\end{multline} Combining these, we see that $(\mu^\epsilon_t)_{t<t^\epsilon_\mathrm{g}}$ solves the modified equation (\ref{eq: mE}), and so, by uniqueness of the maximal conservative solution $(\nu^\epsilon_t)_{t<t_\mathrm{e}^\epsilon}$ in Lemma \ref{lemma: solution to modified equation}, we have \begin{equation}
        t^\epsilon_\mathrm{g} \leq t_\mathrm{e}^\epsilon; \hspace{1cm} \mu^\epsilon_t=\nu^\epsilon_t \hspace{0.5cm}\forall t<t^\epsilon_\mathrm{g}.
    \end{equation} The other implication is identical, using the uniqueness of the maximal conservative solution $(\nu^\epsilon_t)_{t<t_\mathrm{e}^\epsilon}$ in Lemma \ref{lemma: solution to modified equation} to deduce that $\nu^\epsilon_t= \nu^\epsilon_t\circ R^{-1}$ for $t<t_\mathrm{e}^\epsilon$. Hence, the equations \eqref{eq: symmetry under R} hold with $\nu^\epsilon_t$ in place of $\mu^\epsilon_t$, for any bounded, measurable $f$ and $t<t_\mathrm{e}^\epsilon.$ Therefore, $(\nu^\epsilon_t)_{t<t_\mathrm{e}^\epsilon}$ is a conservative solution to the unmodified equation (\ref{eq: E}), and so by Corollary \ref{cor: maximal conservative solutions}, \begin{equation}
        t_\mathrm{e}^\epsilon \leq t^\epsilon_\mathrm{g}; \hspace{1cm} \nu^\epsilon_t=\mu^\epsilon_t \hspace{0.5cm}\forall t<t_\mathrm{e}^\epsilon.
    \end{equation} \end{proof}  
    
Using previous arguments, we make the following remark on the gelation times $t^\epsilon_\mathrm{g}$.

\begin{lemma}\label{lemma: continuity of tcrit} Suppose $\mu_0$ satisfies (A1-4.). Let $\mu^\epsilon_0$ be as above, and let $t^\epsilon_\mathrm{g}$ be the corresponding gelation time. Then $t^\epsilon_\mathrm{g}\rightarrow t_\mathrm{g}$ as $\epsilon \downarrow 0$.  \end{lemma}\begin{proof} This follows from the calculation in Lemma \ref{lemma: computation of tcrit}. Let $m^\epsilon$ be the cutoff measure \begin{equation} m^\epsilon(dv)=1[\sqrt{2\epsilon}\le|v|\le\sqrt{2\epsilon^{-1}}]\hspace{0.1cm}m(dv) \end{equation} so that $m^\epsilon$ corresponds to the initial particle velocities for $\mu^\epsilon_0$. By Corollary \ref{corr: actual expression for tg}, the gelation times $t^\epsilon_\mathrm{g}, t_\mathrm{g}$ are given by a continuous function of the moments $\sigma_0, \sigma_2, \sigma_4$ of $m^\epsilon, m$.  Using dominated convergence and hypotheses (A3-4.), we have \begin{equation} \sigma_k(m^\epsilon)\rightarrow \sigma_k(m);\hspace{1cm}0\le k\le 4 \end{equation} which implies the claimed convergence $t^\epsilon_\mathrm{g}\rightarrow t_\mathrm{g}.$ \end{proof}

%\iffalse Repeating the calculation we have already done, we can calculate the gelation times explicitly: \begin{lemma}[Identification of Gelation Times]\label{lemma: calculation of gelation} For $\epsilon>0$, the gelation time $t_\mathrm{g}^\epsilon$ is given by \begin{equation} t_\mathrm{g}^\epsilon = \frac{1}{4\gamma}\sqrt{\frac{\langle \pi_n^2, \mu_0^\epsilon\rangle}{\langle \pi_e^2, \mu_0^\epsilon\rangle}}\left(\langle \pi_n^2, \mu_0^\epsilon\rangle+\sqrt{\frac{\langle \pi_n^2, \mu_0^\epsilon\rangle}{\langle \pi_e^2, \mu_0^\epsilon\rangle}}\langle \pi_n\pi_e, \mu_0^\epsilon\rangle \right)^{-1}. \end{equation} As $\epsilon \downarrow 0$, we have \begin{equation}
%    t_\mathrm{g}^\epsilon \rightarrow t_\mathrm{g} =\frac{1}{4\gamma}\sqrt{\frac{\langle \pi_n^2, \mu_0\rangle}{\langle \pi_e^2, \mu_0\rangle}}\left(\langle \pi_n^2, \mu_0\rangle+\sqrt{\frac{\langle \pi_n^2, \mu_0\rangle}{\langle \pi_e^2, \mu_0\rangle}}\langle \pi_n\pi_e, \mu_0\rangle \right)^{-1}.
%\end{equation}  \end{lemma} \begin{proof} \textcolor{red}{{Here's what we had before:}} From Lemma \ref{lemma: integral equation}, and standard regularity arguments, we have 
%\begin{align}
%    \frac{\dd}{\dd t}\langle \pi_n^2, \mu_t\rangle &=
%    2\kappa \langle \pi_n^2, \mu_t\rangle^2
%    + 8 \gamma \langle \pi_n \pi_e, \mu_t\rangle\langle \pi_n^2, \mu_t\rangle\\
%    \frac{\dd}{\dd t}\langle \pi_n \pi_e, \mu_t\rangle &=
%    2\kappa \langle \pi_n^2, \mu_t\rangle \langle \pi_n \pi_e, \mu_t\rangle
%    + 4 \gamma \langle \pi_n \pi_e, \mu_t\rangle^2
%    + 4 \gamma \langle \pi_n^2, \mu_t\rangle\langle \pi_e^2, \mu_t\rangle\\
%    \frac{\dd}{\dd t}\langle \pi_e^2, \mu_t\rangle &=
%    2\kappa \langle \pi_n \pi_e, \mu_t\rangle^2
%    + 8 \gamma \langle \pi_n \pi_e, \mu_t\rangle\langle \pi_e^2, \mu_t\rangle.
%\end{align}
%In the case $\kappa = 0$ note that
%\begin{equation}\label{eq:proportional}
%    \langle \pi_e^2, \mu_t\rangle = \langle \pi_e^2, \mu_0\rangle
%    \frac{\langle \pi_n^2, \mu_t\rangle}{\langle \pi_n^2, \mu_0\rangle}
%\end{equation}
%and then one easily checks that
%\begin{equation}
%    f(t) := \langle \pi_n^2, \mu_t\rangle
%           + \sqrt{\frac{\langle \pi_n^2, \mu_0\rangle}
%                        {\langle \pi_e^2, \mu_0\rangle}}
%             \langle \pi_n \pi_e, \mu_t\rangle
%\end{equation}
%satisfies
%\begin{equation}
%    \frac{\dd }{\dd t}f(t) = 4\gamma 
%    \sqrt{\frac{\langle \pi_e^2, \mu_0\rangle}
%                        {\langle \pi_n^2, \mu_0\rangle}}f(t)^2,
%\end{equation}
%which has the solution
%\begin{equation}\label{eq:quadmoment-ode}
%    f(t) = \left(\frac{1}{f(0)} - 4\gamma 
%    \sqrt{\frac{\langle \pi_e^2, \mu_0\rangle}
%                        {\langle \pi_n^2, \mu_0\rangle}}t \right)^{-1}.
%\end{equation}
%The solution to \eqref{eq:quadmoment-ode} blows up when
%\begin{equation}\label{eq:quad-tgel}
%    t = \frac{1}{4\gamma f(0)}
%             \sqrt{\frac{\langle \pi_n^2, \mu_0\rangle}
%                        {\langle \pi_e^2, \mu_0\rangle}}
%\end{equation}

%Assuming the initial condition is independent Maxwell-Boltzmann velocities in $d$ dimensions and all clusters of size 1 one has
%\begin{align*}
%    \langle \pi_n^2, \mu_0\rangle &= 1\\
%    \langle \pi_n \pi_e, \mu_0\rangle &= d\frac{\sigma^2}{2}\\
%    \langle \pi_e^2, \mu_0\rangle &= \frac{\sigma^4}{4}\left(2d + d^2\right)
%\end{align*}
%and thus $f(0) = 1 + d / \sqrt{2d + d^2}$.
%In this case the critical time given by \eqref{eq:quad-tgel} is
%\begin{equation}
%    \frac{1}{2\gamma \sigma^2 \left(d + \sqrt{2d +d^2}\right)}.
%\end{equation}
%This should be compared to the expression for the mean free time \cite[equation 4.38]{PSW17}, which is $1/4\gamma \sigma^2 d$. \end{proof} \fi 



We now turn to the proof of Lemma \ref{lemma: second moment before tgel}. We say that a local solution $(\nu_t)_{t<T}$ to (\ref{eq: E}) is \emph{strong} if, for all times $t<T$, 
\begin{equation}
    \int_0^t  \hspace{0.1cm} \langle\varphi^2, \nu_s\rangle  \hspace{0.1cm} ds<\infty.
\end{equation} We use the following result from \cite{N00} on the existence and uniqueness of strong solutions. 
\begin{lemma}\label{lemma: strong solutions} Any strong solution to (\ref{eq: E}) is conservative. For any finite measure $\mu_0$ with $\langle \varphi^2, \mu_0\rangle <\infty$, there is a unique maximal strong solution $(\mu'_t)_{t<t_\mathrm{e}(\mu_0)}$ to (\ref{eq: E}), starting at $\mu_0$, and with $t_\mathrm{e}(\mu_0)>0$, such that $ \langle \varphi^2, \mu'_t\rangle$ is increasing on $[0,t_\mathrm{e}(\mu_0))$. If $t_\mathrm{e}(\mu_0)<\infty$, then $\langle \varphi^2, \mu'_t\rangle$ increases to $\infty$  as $t\uparrow t_\mathrm{e}(\mu_0)$.  \end{lemma} Here, the subscript `e' denotes explosion: $t_\mathrm{e}(\mu_0)$ is exactly the blow-up time of the second moment. When the measure $\mu_0$ is clear, we will omit the argument of $t_\mathrm{e}.$ 
\begin{proof}
    This is almost a special case of \cite[Theorem 2.1]{N00}. From the cited result, for any finite measure $\mu_0$ with $\langle \varphi^2, \mu_0\rangle <\infty$, there exists a maximal strong solution $(\mu'_t)_{t<t_\mathrm{e}(\mu_0)}$. Moreover, there exists a constant $C=C(\kappa, \gamma)>0$ such that, for all such $\mu_0$,  $t_\mathrm{e}(\mu_0) \ge C \langle \varphi^2, \mu_0\rangle^{-1}$. By applying this bound to $\mu'_t$, if $t_\mathrm{e}(\mu_0)<\infty,$ then $\langle \varphi^2, \mu'_t\rangle \ge (C(t_\mathrm{e}(\mu_0)-t))^{-1}$  which implies the claimed divergence.
\end{proof}

By Corollary \ref{cor: maximal conservative solutions}, since $(\mu'_t)_{t<t_\mathrm{e}}$ is conservative, it follows that $t_\mathrm{e} \le t_\mathrm{g}$, and $\mu'_t=\mu_t$ for all $t<t_\mathrm{e}$. It remains to show that $t_\mathrm{e}\ge t_\mathrm{g}$. \medskip \\ Following the ideas of \cite[Proposition 2.7]{N00}, we obtain the integral relations, for all $t<t_\mathrm{e}$, \begin{equation} \label{eq: ODE1}
    \langle \pi_n^2, \mu_t\rangle =
    \langle \pi_n^2, \mu_0\rangle + \int_0^t \left[\kappa\langle \pi_n^2, \mu_s\rangle^2+4\gamma\langle \pi_n\pi_e, \mu_t\rangle\langle\pi_n^2, \mu_s\rangle \right] ds;
\end{equation} 

\begin{equation}\label{eq: ODE2}
    \langle \pi_n \pi_e, \mu_t\rangle =
    \langle \pi_n\pi_e, \mu_0\rangle + \int_0^t \left[\kappa\langle \pi_n^2, \mu_s\rangle\langle \pi_n\pi_e, \mu_s\rangle+2\gamma\langle \pi_n\pi_e, \mu_t\rangle^2+2\gamma\langle\pi_n^2, \mu_s\rangle\langle \pi_e^2, \mu_s \rangle \right] ds;
\end{equation}

\begin{equation} \label{eq: ODE3}
    \langle \pi_e^2, \mu_t\rangle =
    \langle \pi_e^2, \mu_0\rangle + \int_0^t \left[\kappa\langle \pi_n\pi_e, \mu_s\rangle^2+4\gamma\langle \pi_n\pi_e, \mu_t\rangle\langle\pi_e^2, \mu_s \rangle \right] ds.
\end{equation} These immediately imply that $\mathcal{E}(t)$ is bounded on compact subsets of $[0, t_\mathrm{e})$, and in particular cannot diverge before $t_\mathrm{e}$. Combining this with Lemma \ref{lemma: strong solutions}, the maximal time $t_\mathrm{e}$ of existence of a strong solution is precisely the first time at which the second moment $\mathcal{E}(t)$ diverges, or $\infty$ if there is no divergence. \medskip \\ We also remark on the relationship of this result to the solutions $(\mu^\epsilon)_{t<t_\mathrm{g}^\epsilon}=(\nu^\epsilon_t)_{t<t_\mathrm{e}^\epsilon}$ discussed in Lemmas \ref{lemma: solution to modified equation}, \ref{lemma: Relationship}. It is clear, from Lemma \ref{lemma: solution to modified equation}, that $(\nu^\epsilon_t)_{t<t_\mathrm{e}^\epsilon}$ is a strong solution. Moreover, in view of the comments above, and since $\langle \varphi^2, \nu^\epsilon_t\rangle \uparrow \infty$ as $t\uparrow t_\mathrm{e}^\epsilon$, it follows that that $(\nu^{\epsilon}_t)_{t<t_\mathrm{e}^\epsilon}$ is the maximal strong solution with initial data $\mu^\epsilon_0.$ This justifies the use of the notation $t^\epsilon_\mathrm{e}$ in Lemma \ref{lemma: solution to modified equation}. \bigskip \\  As discussed at the beginning of this section, our argument is now as follows. From general considerations in \cite{N00}, we argued above that the gelation time and explosion time coincide for the restricted dynamics: $t_\mathrm{g}^\epsilon=t_\mathrm{e}^\epsilon$, and we now wish to justify the limit $\epsilon\downarrow 0$. The limiting behaviour of $t^\epsilon_\mathrm{g}$ is understood from Lemma \ref{lemma: continuity of tcrit}, and so we wish to understand the behaviour of $t_\mathrm{e}^\epsilon.$ \medskip \\  Using standard regularity arguments, we may view (\ref{eq: ODE1} - \ref{eq: ODE3}) as a differential equation for the three moments $q_t=(\langle \pi_n^2, \mu_t\rangle, \langle \pi_n \pi_e, \mu_t\rangle, \langle \pi_e^2, \mu_t\rangle)$ and, from the discussion above, the blow-up time to the ODE system is exactly $t_\mathrm{e}$. An identical argument holds for $\mu^\epsilon_t$, which blows up at $t^\epsilon_\mathrm{e}$. By analysing this system of ODEs, we will show that the explosion time is continuous in the initial data, which implies that $t^\epsilon_\mathrm{e}\rightarrow t_\mathrm{e}.$
\begin{lemma}\label{lemma: ODE considerations} Consider the ordinary differential equation $\dot{q}_t=b(q_t)$ in $\mathbb{R}^3$, where $b$ is the locally Lipschitz field given by \begin{equation} \label{eq: system of ODEs} b(q_1,q_2,q_3)=\begin{pmatrix}\kappa q_1^2+4\gamma q_1q_2 \\ \kappa q_1q_2+2\gamma q_2^2+2\gamma q_1q_3 \\ \kappa q_2^2 + 4\gamma q_2q_3 \end{pmatrix}. \end{equation} Then, for all $q_0\in \mathbb{R}^3$, there exists a unique maximal solution $\psi(q_0, t)$ starting at $q_0$, defined until time $\zeta(q_0)\in (0, \infty]$. Consider the sets \begin{equation} E=(0, \infty)^3; \hspace{1cm} E_\delta=[\delta,\infty)^3.\end{equation} Then, if $q_0 \in E_\delta$ for some $\delta>0$, then the solution $(\psi(q_0,t))_{t<\zeta(q_0)} \subset E_\delta$. We have the following properties: \begin{enumerate}[label=\roman{*}).] \item Let $J_\epsilon$ be the set \begin{equation} J_\epsilon  =\{q \in E: \hspace{0.2cm} \zeta(q)\ge\epsilon\}.\end{equation} If $\gamma>0$, then for all $\epsilon, \delta>0$, the set $E_\delta \cap J_\epsilon $ is bounded. Moreover, $\zeta<\infty$ everywhere. \item Suppose $q^n_0 \in E$ and $q^n_0 \rightarrow q_0 \in E$. Then $\zeta(q^n_0)\rightarrow \zeta(q_0).$ \item Suppose $I\subset \RR_+$ is an open interval, and the map $q_0: I\rightarrow E$ is continuous, and such that $t<\zeta(q_0(t))$ for all $t\ge 0.$ Then the map $I\rightarrow E, t\mapsto \psi(q_0(t), t)$ is continuous. 
 \end{enumerate} \end{lemma} 

\begin{proof} For all three items, the case where $\gamma=0, \kappa>0$ may be checked by an elementary explicit calculation. For the remainder of the proof, we exclude this case, and consider only the case $\gamma>0.$ \begin{enumerate}[label=\roman{*}).]
    \item Let $\zeta_0$ denote the blowup time for the dynamics (\ref{eq: system of ODEs}) with $\kappa=0$. It is straightforward to see that $\zeta(q)\le\zeta_0(q)$ for all $q\in E$, and so it is sufficient to show that $E_\delta\cap \{q: \zeta_0(q)\ge \epsilon\}$ is bounded. We argue using the following explicit computation. \medskip \\ Let $q(0)=(q_1(0),q_2(0),q_3(0))\in E$, and let $q(t)=(q_1(t),q_2(t),q_3(t))$ be the solution to (\ref{eq: system of ODEs}) with $\kappa=0$, starting at $q(0)$. It is then straightforward to see that \begin{equation}
        \frac{1}{q_1(t)}\frac{d}{dt}q_1(t)=\frac{1}{q_3(t)}\frac{d}{dt}q_3(t)
    \end{equation} which implies that $q_3(t)=q_1(t)q_3(0)/q_1(0)$ for all $t\ge 0$. Now, the linear combination $\widetilde{q}(t)$ given by \begin{equation}
        \widetilde{q}(t)=q_1(t)+\sqrt{\frac{q_1(0)}{q_3(0)}}q_2(t)
    \end{equation} has the same blowup time as $q(t)$, and satisfies the ordinary differential equation \begin{equation} \frac{d}{dt}\widetilde{q}(t)=4\gamma\sqrt{\frac{q_3(0)}{q_1(0)}}\hspace{0.1cm}\widetilde{q}(t)^2. \end{equation} This has the unique solution \begin{equation} \widetilde{q}(t)=\left(\frac{1}{\widetilde{q}(0)}-4\gamma\sqrt{\frac{q_3(0)}{q_1(0)}}t\right)^{-1}; \hspace{1cm} t< \frac{1}{4\gamma\widetilde{q}(0)}\sqrt{\frac{q_1(0)}{q_3(0)}}. \end{equation} In terms of the initial data $q(0)$, this gives the blowup time as \begin{equation} \zeta(q(0))=\frac{1}{4\gamma}\left(\sqrt{q_1(0)q_3(0)}+q_2(0)\right)^{-1} \end{equation} which converges to $0$ as $q(0)\rightarrow \infty$ in $E_\delta.$ This shows that $E_\delta \cap \{q: \zeta_0(q)\ge \epsilon\}$ is bounded, as claimed. The same computation also shows that $\zeta(q)<\infty$ for all $q\in E$.
    \item The lower semicontinuity of explosion times is standard, and follows from the continuous dependence on the initial data. Therefore, it is sufficient to prove that $\limsup_{n\rightarrow \infty} \zeta(q^n)\le \zeta(q).$  \medskip \\ Suppose, for a contradiction, that for some $\epsilon>0$, we have $\limsup_{n\rightarrow \infty} \zeta(q^n)>\zeta(q)+\epsilon$; write $\tau=\zeta(q)$. By passing to a subsequence, we may assume that $\zeta(q^n)>\tau+\epsilon$ for all $n$, and some fixed $\epsilon>0$. Moreover, since $q^n\rightarrow q \in E$, we may assume that $q^n, q \in E_\delta$ for all $n$, for some $\delta>0$, which implies that $\psi(q^n,t)\in E_\delta$ for all $t<\zeta(q^n)$ and all $n\in \mathbb{N}$.\medskip\\  Now, if $t\le \tau$, we have $\zeta(\psi(t,q^n))=\zeta(q^n)-t \ge \epsilon$, which implies the containment \begin{equation} \{\psi(t,q^n): t\le \tau, n\ge 1\} \subset E_\delta\cap J_\epsilon \end{equation} which we know, from item i)., to be bounded: for some $C<\infty$, \begin{equation}
        \{\psi(t,q^n): t\le \tau, n\ge 1\} \subset [0,C]^3.
    \end{equation} By the lemma of leaving compact sets, there exists $s<\tau$ such that, for all $t\in (s,\tau)$, $\psi_t(q)\not \in [0,C]^3.$ However, if we pick $t\in (s,\tau)$, we have $\psi_t(q^n) \rightarrow \psi_t(q)$, by the continuity of the dependence in the initial conditions, which is a contradiction. Therefore, $\limsup_{n\rightarrow \infty} \zeta(q^n)\le \zeta(q)$, which proves the claimed convergence.      
    \item  Firstly, we note that by ii)., the map $t\mapsto \zeta(q_0(t))$ is continuous on $I$. Therefore, fixing $t\in I$, we may choose choose  $\epsilon, \delta > 0$ such that, if $\abs{t-s} \le \delta$, then $s\in I$ and $s < \min \left(\zeta\left(q_0(s)\right), \zeta\left(q_0(t)\right)\right)-\epsilon$. Now, we observe that, for $s\in [t-\delta, t+\delta],$\begin{equation}
    |\psi(t,q_0(t))-\psi(s,q_0(s))|\le|\psi(t,q_0(t))-\psi(t,q_0(s), )|+|\psi(t,q_0(s))-\psi(s,q_0(s))|.
\end{equation} As $s\rightarrow t$, the first term converges to $0$ by continuity of the solution $s\mapsto \psi(x_0(t),s)$; it is therefore sufficient to control the second term. We observe that, for all $s\in[t-\delta, t+\delta],$ we have $\zeta(\psi(s,q_0(s)))=\zeta(q_0(s))-s>\epsilon$. Moreover, by compactness, there exists some $\eta>0$ such that $q_0(s) \in E_\eta$ for all $s\in [t-\delta, t+\delta]$, and so $\psi(q_0(s),u)\in E_\eta$ for all $0\le u\le \zeta(q_0(s)).$ However, we showed in point i). above that the region $E_\eta \cap J_\epsilon=\{q \in E_\eta: \zeta(q)\geq\epsilon\}$ is compact  and so there exists a constant $M=M(\epsilon)$: for all $s\in[t-\delta, t+\delta]$, and for all $u \le t+\delta$, \begin{equation}u< \zeta(q_0(s));\hspace{1cm} |b(\psi(u,q_0(s))| \le M. \end{equation}
This implies the bound, for all $s\in[t-\delta,t+\delta]$, \begin{equation} |\psi(t,q_0(s))-\psi(s,q_0(s))| \le M|t-s|\end{equation} which implies the claimed continuity.
\end{enumerate}  \end{proof}



We can now use this to prove our main result Lemma \ref{lemma: second moment before tgel} on the second moment $\mathcal{E}(t)$ in the subcritical phase.





\begin{proof}[Proof of Lemma \ref{lemma: second moment before tgel}]
Let $\mu_0$ be any measure on $S$ satisfying ({A1}-{4}.), and let $(\mu_t)_{t\ge 0}$ be the associated solution to (\ref{eq: E+G}). From the discussion following Lemma \ref{lemma: strong solutions}, it is sufficient to show that $t_\mathrm{g}=t_\mathrm{e}<\infty$. We also recall that $t_\mathrm{e}$ is characterised as the explosion time $\zeta$ of the ODE system (\ref{eq: ODE1}-\ref{eq: ODE3}). \medskip \\ Since the base measure $m$ of $\mu_0$ is not a multiple of the point mass $\delta_0$, all the quadratic moments $q$ of $\mu_0$ are strictly positive, and so $q \in E$. Therefore, by the second point of Lemma \ref{lemma: ODE considerations}, the explosion time $t_\mathrm{e}=\zeta(q)<\infty.$
As $\epsilon \downarrow 0$, the quadratic moments $q^\epsilon$ of $\mu^\epsilon_0$ converge to the quadratic moments $q\in E$ of $\mu_0$ by dominated convergence, and using (A4.).  Therefore, by Lemma~\ref{lemma: ODE considerations},  $t^\epsilon_\mathrm{e}=\zeta(q^\epsilon)\rightarrow \zeta(q)= t_\mathrm{e}$.
By Lemma \ref{lemma: Relationship}, we know that $t_\mathrm{e}^\epsilon=t_\mathrm{g}^\epsilon$, and by Lemma \ref{lemma: continuity of tcrit}, $t_\mathrm{g}^\epsilon \rightarrow t_\mathrm{g}$. Together, these imply that $t_\mathrm{e} = t_\mathrm{g}$, as claimed.

\end{proof} 

\subsection{The Critical Point} Using the concepts introduced above, we next consider the behaviour at and near the critical time $t_\mathrm{g}$. \begin{lemma} Assume that $\mu_0$ is a probability measure. In the notation of Lemma \ref{lemma: second moment before tgel}, we have \begin{equation} \mathcal{E}(t_\mathrm{g})=\infty=\lim_{t\rightarrow t_\mathrm{g}} \mathcal{E}(t). \end{equation} \end{lemma} 
\begin{proof}We first show that $\mathcal{E}(t_\mathrm{g})=\infty$. Suppose, for a contradiction, that $\mathcal{E}(t_\mathrm{g})<\infty.$ Then, applying \cite[Proposition 2.7]{N00} as in Lemma \ref{lemma: strong solutions}, we see that, for some positive $\delta>0$, there exists a strong solution $(\nu_t)_{t<\delta}$ to (\ref{eq: E}), starting at $\mu_{t_\mathrm{g}}.$ This solution is conservative, so is an initial segment of the solution $(\nu_t)_{t\ge 0}$ to (\ref{eq: E+G}) starting at $\mu_{ t_\mathrm{g}}$. By uniqueness in Lemma \ref{lemma: E and U}, \begin{equation}
    \nu_t=\mu_{t_\mathrm{g}+t} \hspace{1cm} \text{ for all }t\ge 0.
\end{equation}
By Corollary \ref{corr: gel at tgel}, $\langle \varphi, \mu_{t_\mathrm{g}}\rangle = \langle \varphi, \mu_0\rangle$, and by definition of $t_\mathrm{g}$, \begin{equation} \langle \varphi, \mu_{t_\mathrm{g}+t}\rangle < \langle \varphi, \mu_{0}\rangle = \langle \varphi, \mu_{t_\mathrm{g}}\rangle \text{ for all }t>0. \end{equation}This contradicts the fact that $(\nu_t)_{t<\delta}$ is strong, which therefore shows that $\mathcal{E}(t)=\infty$.

The second point follows, because $t\mapsto \mu_t$ is continuous, and $\mu \mapsto \langle \varphi^2, \mu\rangle$ is lower semicontinuous, when $\mathcal{M}$
is equipped with the vague topology. \end{proof}

\subsection{The Supercritical Regime} We finally turn to the supercritical case; our result is as follows. 
\begin{lemma}\label{lemma: second moment finite after tgel} In the notation of Lemma \ref{lemma: second moment before tgel}, the map $t\mapsto \mathcal{E}(t)$ is finite and continuous, and therefore locally bounded, on  $(t_\mathrm{g},\infty)$. \end{lemma} 
The proof is based on a \emph{duality argument} following Theorem \ref{thrm: coupling supercritical and subcritical}, which connects the measures in the supercritical regime to an auxiliary process in the subcritical case.  Let $(G^N_t)_{t\geq 0}$ be the random graph processes described in Section  \ref{sec: coupling_to_random_graph} with $N$ particles sampled independently from $m$, and fix $t>t_\mathrm{g}$. Let $\widetilde{G}^N_{t}$ be the graph $G^N_{t}$ with the giant component deleted. \medskip \\ 
Let $\rho_{t}(v)=\rho(t, v)$ be the survival function defined in Lemma \ref{lemma: survival function}, and let $\widehat{m}^t(dv)=(1-\rho_{t}(v))m(dv)$ and $\widehat{\mu}^t_0$ the corresponding measure on $S$ under pushforward by $\iota: v\mapsto (1, v, \frac{1}{2}|v|^2)$. By Lemma \ref{lemma: E and U}, there exists a unique solution $(\widehat{\mu}^t_s)_{s\geq 0}$ to the equation (\ref{eq: E+G}) starting at $\widehat{\mu}^t_0$; write $\widehat{t_\mathrm{g}}(t)$ for its gelation time. By Theorem \ref{thrm: coupling supercritical and subcritical}, we can construct a generalised vertex space $\widehat{\mathcal{V}}=(\mathbb{R}^d, \widehat{m}, (\mathbf{w}_N)_{N\ge 1})$ and a graph $\widehat{G}^N_{t}\sim \mathcal{G}^{\widehat{\mathcal{V}}}(N,tK)$ such that $\mathbb{P}(\widehat{G}^N_{t}=\widetilde{G}^N_{t})\rightarrow 1$, and where $\mathbf{w}_N$ is an enumeration of the vertexes $v_i$ not belonging to the giant component. 
\medskip \\
In order to appeal to Lemmas \ref{lemma: convergence of random graphs}, \ref{lemma: connect critical times}, we will now verify that the desired regularity conditions (A1-4, B1-2.) hold for the vertex space $\widehat{\mathcal{V}}$.
\begin{lemma}\label{lemma: conditions B1-3 for duality} Fix $t>0$, and let $\mu_0, \widehat{m}^t, \widehat{\mu}^t_0$ and $\widehat{\mathcal{V}}$ be as described above. Then the regularity conditions (B1-2.) hold for $\mathbf{w}_N$ and $\widehat{\mu}^t_0.$ \end{lemma} \begin{proof} To ease notation, we write $\widehat{\mu}_0, \widehat{m}$ for $\widehat{\mu}^t_0, \widehat{m}^t$, $\mu^N_0$ for the initial empirical measure of the unmodified process corresponding to $\mathbf{v}_N$, and $\widehat{\mu}^N_0$ for the reduced empirical measure corresponding to $\mathbf{w}_N$: \begin{equation} \widehat{\mu}^N_0 =\frac{1}{N}\sum_{i=1}^{l_N} \delta_{(1,w_i, \frac{1}{2}|w_i|^2)}.\end{equation} It is straightforward to see that $\widehat{\mu}^t_0$ inherits the properties (A1-4.) from $\mu_0$, and so it is sufficient to establish (B1-2.). We will also appeal to the fact that the unmodified empirical measures $\mu^N_0$ satisfy (B1-2.), which is straightforward to verify. \medskip \\ For (B1.), we note that part of the content of Theorem \ref{thrm: coupling supercritical and subcritical} is that $\widehat{\mathcal{V}}$ is a generalised vertex space, as defined in Definition~\ref{def: Generalised vertex space}, which includes the weak convergence
\begin{equation}
    \widehat{m}_N=\frac{1}{N}\sum_{i\le l_N} \delta_{w_i} \rightarrow \widehat{m} \hspace{1cm}\text{weakly, in probability}.
\end{equation} Since the map $\iota: \mathbb{R}^d\rightarrow S$ is continuous and the vague topology is weaker than the weak topology, (B1.) follows.

We will now show that (B2.) follows from the previous point, together with the moment estimates for the original initial measure $\mu^N_0$. 

Fix $R<\infty$, and let $\chi_R \in C_c(S)$ be such that $1_{S_R} \leq \chi \leq1_{S_{R+1}}$. We observe that
\begin{equation} \begin{split}
\abs{\langle x, \widehat{\mu}_0^N\rangle - \langle x, \widehat{\mu}_0\rangle}  &\leq
\abs{\langle x\chi_R, \widehat{\mu}_0^N-\widehat{\mu}_0\rangle } 
 +\langle \abs{x} 1_{S_R^\mathrm{c}},\widehat{\mu}_0^N\rangle
 +\langle \abs{x} 1_{S_R^\mathrm{c}},\widehat{\mu}_0\rangle \\[1ex]
 &\leq
 \abs{\langle x\chi_R, \widehat{\mu}_0^N-\widehat{\mu}_0\rangle} 
 +\frac{\sqrt{2}}{R}\langle \varphi^2,{\mu}_0^N\rangle
 +\frac{\sqrt{2}}{R}\langle \varphi^2,{\mu}_0\rangle.
\end{split} \end{equation}
We now fix $\epsilon, \delta>0$. Since $\langle \varphi^2,{\mu}_0^N\rangle$ converges almost surely by (A3.), and in particular is $O_\mathrm{p}(1)$, we may choose $R<\infty$ such that the second and third terms are at most $\epsilon/3$ with probability exceeding $1-\delta/2$, for all $N$. For this choice of $R$, the first term vanishes as $N\rightarrow\infty$ by vague convergence in probability, and so is at most $\frac{\epsilon}{3}$ with probability exceeding $1-\delta/2$ for all $N$ large enough. Therefore, for all such $N$, we have \begin{equation} \PP\left(|\langle x, \widehat{\mu}^N_0-\widehat{\mu}_0\rangle|>\epsilon\right)\le \delta \end{equation} which proves the desired convergence in probability.  

For the second assertion of (B2.), we note that $\langle \varphi^2, \widehat{\mu}^N_0\rangle \le \langle \varphi^2, \mu^N_0\rangle$ by the construction of $\mathbf{w}_N$, and $\langle \varphi^2, \mu^N_0\rangle$ is bounded in $L^1$ by (A3.), recalling that the velocities in $\mathbf{v}_N$ are sampled independently from $m$.
\end{proof}


We now use this preparatory result to prove Lemma \ref{lemma: second moment finite after tgel}. \begin{proof} Recalling that we consider equality of graphs to include equality of the vertex data, it follows that \begin{equation} \mathbb{P}(\mu^N(\widehat{G}^N_t)=\mu^N(\widetilde{G}^N_t))\rightarrow 1. \end{equation}  From Lemmas \ref{lemma: convergence of random graphs}, \ref{lemma: conditions B1-3 for duality}, we obtain the following convergences in probability:
\begin{equation}
    \mu^N(G^N_{t})\rightarrow {\mu}_{t};\hspace{1cm}
    \mu^N(\widehat{G}^N_{t})\rightarrow \widehat{\mu}^t_{t}
\end{equation} in the vague topology, in probability.  Moreover, the difference \begin{equation}
    \mu^N(G^N_{t})-\mu^N(\widetilde{G}^N_{t})=\frac{1}{N}\delta(\mathcal{C}_1(G^N_{t}))
\end{equation} converges to $0$ in the vague topology in probability, since the support is eventually disjoint from any compact set, with high probability. It follows that \begin{equation}
    \mu^N(\widetilde{G}^N_{t})\rightarrow \mu_{t}
\end{equation} in the vague topology, in probability, and by uniqueness of limits, we have $\widehat{\mu}^t_{t}=\mu_{t}$. Using assumption ({A3.}), we can see that $t k\in L^2(\mathbb{R}^d\times \mathbb{R}^d,m\times m)$, and so it follows from Theorem \ref{thrm: coupling supercritical and subcritical} that the graphs $\widehat{G}^N_{t}$ are subcritical. By Lemma \ref{lemma: connect critical times}, it follows that that  $t<\widehat{t_\mathrm{g}}(t)$, and so by Lemma \ref{lemma: second moment before tgel}, we have \begin{equation}
   \langle \varphi^2, \mu_t\rangle = \langle \varphi^2, \widehat{\mu}^t_{t}\rangle <\infty.
\end{equation} Using Theorem \ref{thrm: continuity of rho} and dominated convergence, the map \begin{equation}
    t\mapsto q^t_0=\left(\left\langle \pi_n^2, \widehat{\mu}^t_0\right\rangle,\left\langle \pi_n\pi_e, \widehat{\mu}^t_0\right\rangle,\left\langle \pi_e^2, \widehat{\mu}^t_0\right\rangle\right)=\left(\left\langle 1-\rho_t,m\right\rangle,\left\langle \frac{1}{2}|v|^2(1-\rho_t),m\right\rangle,\left\langle \frac{1}{4}|v|^4(1-\rho_t),m\right\rangle\right)
\end{equation} is continuous, and it is clear that it takes values in $E=(0,\infty)^3$. Therefore, by the general ODE considerations in Lemma \ref{lemma: ODE considerations} point iii)., it follows that the map \begin{equation}
    t\mapsto q^t(t)= \left(\langle \pi_n^2, \widehat{\mu}^t_t\rangle,\langle \pi_n\pi_e, \widehat{\mu}^t_t\rangle,\langle \pi_e^2, \widehat{\mu}^t_t\rangle\right)
\end{equation} is finite and continuous on $(t_\mathrm{g}, \infty).$  Since $\widehat{\mu}^t_t=\mu_t$, this implies that $t\mapsto \mathcal{E}(t)$ is finite and continuous on $(t_\mathrm{g}, \infty)$, which implies that it is bounded on compact subsets. \end{proof}

\begin{remark} The same argument also shows that $t\mapsto \widehat{t}_\mathrm{g}(t)$ is continuous. This fact will be used later in the proof of Lemma \ref{lemma: anomalous clusters 2}. \end{remark}

