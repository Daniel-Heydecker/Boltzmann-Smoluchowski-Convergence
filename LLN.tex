\section{LLN}
Let $\Phi \in \lip{\rca{\RR^{d+2}}}$ and $\Psi \in \lip{\RR^{d+2}}$.  Define $\Phi \oplus \Psi \in \lip{\sX}$ by
$\Phi \oplus \Psi(\nu,x) = \Phi(\nu) + \Psi(x)$.  Then $\PP^N$-almost surely
\begin{equation}
    M^{\Phi \oplus \Psi}_t := \Phi \oplus \Psi\left(\tmu_t, \tl_t\right) - \Phi \oplus \Psi\left(\tmu_0, \tl_0\right)
    - \int_0^t \widetilde{L}^N (\Phi \oplus \Psi)\left(\tmu_s, \tl_s\right) \dd s
\end{equation}
is a Martingale and its quadratic variation process $[M^{\Phi \oplus \Psi}]$ is simply the sum of the squares of its jumps and since $(a+b)^2 \leq 2a^2 + 2b^2$ $[M^{\Phi \oplus \Psi}] \leq 2[M^{\Phi \oplus 0}] + 2[M^{0 \oplus \Psi}] $.  
Now $\PP^N$-almost surely
\begin{equation}
    \left[M^{\Phi \oplus 0}\right]_t = 
    \sum_{s \leq t} \left(\Phi\left(\tmu_s\right) - \Phi\left(\tmu_{s-}\right)\right)^2
    \leq
    \sum_{s \leq t} \lip{\Phi}^2\norm{\tmu_s -  \tmu_{s-}}_\mathrm{TV}^2 
    <
    \frac{9  \lip{\Phi}^2}{N}
\end{equation}
since there can be at most $N -1$ jumps.
The largest possible jump under $\PP^N$ for \tl is $2a_n / N\sqrt{2\gamma}$ of which at most $N\left<w,\tmu_0\right>\sqrt{2\gamma}/2a_N$ are possible
\begin{equation}
    \left[M^{0\oplus \Psi}\right]_t = 
    \sum_{s \leq t} \left(\Psi\left(\tl_s\right) - \Psi\left(\tl_{s-}\right)\right)^2
    \leq
    \sum_{s \leq t}\lip{\Psi}^2\norm{\tl_{s} - \tl_{s-}^2}
    \leq
    \lip{\Psi}^2 \frac{2a_N}{N\sqrt{2\gamma}}\left<w,\tmu_0\right>.
\end{equation}

By Jensen and the BDG inequality \cite[Thrm. 26.12]{K02} there exists a constant $c$ such that
\begin{equation*}
    \EE^N\left[\sup_{t}\abs{M^{\Phi \oplus \Psi}_t}\right]
    \leq \sqrt{\EE^N\left[\sup_{t}\left(M^{\Phi \oplus \Psi}_t\right)^2\right]}
    \leq c\EE^N\left[\sup_t \left[M^{\Phi \oplus \Psi}\right]_t\right] \rightarrow 0.
\end{equation*}
The same arguments apply (without the factor $\sqrt{2\gamma}$) to the case $\Phi \equiv 0$ and $\Psi = w$, so in particular the much weaker conclusion
\begin{multline}\label{e:expected-w-transfer}
    \EE^N\left[w\left(\tl_t\right) - w\left(\tl_0\right)\right]
    =
    \EE^N\left[\int_0^t \frac{w\left(y+z\right)}{2}
    K(y,z)\mathbbm{1}_{(a_N,\infty)}\left(w(y+z)\right)\tmu^{\oplus_N 2}_s(\dd y\dd z) \dd s\right]\\
    +
    \EE^N\left[\int_0^t w\left(y\right)
    K(y,\tl_s)\tmu_s(\dd y) \dd s\right]
\end{multline}
holds.

\begin{proposition}[Convergence of $\widetilde{\mathcal{L}}^N_1$] 
For $\Phi \in C^2_\mathrm{b}\left(\rca{\RR^{d+2}}\right)$ and any $\Psi$
\begin{equation}
    \lim_{n\rightarrow \infty} \EE^N\left[\int_0^T \left(\widetilde{\mathcal{L}}^N_1 (\Phi \oplus \Psi) \right)\left(\tmu_t, \tl_t\right)\dd t\right]
    =
    \EE^\ast\left[\int_0^T \Phi^\prime\left(\tmu_t\right)\left(\delta_{y+z}-\delta_y - \delta_z\right)K(y,z)\tmu_t(\dd y)\tmu_t(\dd z) \dd t\right].
\end{equation}
\end{proposition}
\begin{proof}
For any $R>0$ and $N$ large enough
\begin{multline}\label{eq:phi-estimate}
    \abs{\EE^N\left[\int_0^T \left[\Phi\left(\tmu_t + \frac1N\left(\delta_{y+z}-\delta_y - \delta_z\right)\right) - \Phi(\tmu_t)\right]
      NK(y,z)\mathbbm{1}_{(a_N,\infty)}\left(w(y+z)\right)\tmu^{\oplus_N 2}_t(\dd y\dd z) \dd t \right]} \\
    \leq
    \frac32 \lip{\Phi}\EE^N\left[\int_0^T \int w(y)w(z)\mathbbm{1}_{(a_N,\infty)}\left(w(y+z)\right)\tmu^{\oplus_N 2}_t(\dd y\dd z) \dd t \right]\\
    \leq
    \frac32 \lip{\Phi}\EE^N\left[\int_0^T \int w(y)w(z)\mathbbm{1}_{(R,\infty)}\left(w(y+z)\right)\tmu^{\oplus_N 2}_t(\dd y\dd z) \dd t \right]
\end{multline}
and the final expression can be seen to vanish as $R\rightarrow \infty$ by a couple of applications of the dominated convergence theorem.
\end{proof}

\begin{proposition}[$\widetilde{\mathcal{L}}^N_2$ vanishes]

\end{proposition}
\begin{equation*}
    \lim_{N\rightarrow \infty} \EE^N\left[\int_0^T \abs{\left(\widetilde{\mathcal{L}}^N_2 (\Phi \oplus \Psi) \right)\left(\tmu_t, \tl_t\right)}\dd t\right]
    = 0.
\end{equation*}
\begin{proof}
The estimate used in \eqref{eq:phi-estimate} can be used with $\Psi \equiv 0$. 
For $\Psi \in C_\mathrm{K}(\RR^{d+2})$
\begin{equation*}
    \EE^N\left[\int_0^T \abs{\left(\widetilde{\mathcal{L}}^N_2 (0 \oplus \Psi) \right)\left(\tmu_t, \tl_t\right)}\dd t\right]
    \leq
    \frac{a_N}{N}\lip{\Psi} \EE^N\left[\int_0^T \int NK(y,z)\mathbbm{1}_{(R,\infty)}\left(w(y+z)\right) \tmu_t^{\oplus_N 2}(\dd y \dd z)\dd t\right]
\end{equation*}
so it is sufficient to show that the expectation on the right is bounded and then use $a_n / N \rightarrow 0$.  The quantity to be bounded is loosely the rate at which clusters with $w$ smaller than $a_N$ merge to form clusters with $w$ larger than $a_N$ \rpcomment{Now need an argument, eg from uniqueness of giant cluster, that this cannot diverge.}
\end{proof}

\begin{proposition}[Convergence of $\widetilde{\mathcal{L}}^N_3$] 
For $\Phi \in C^2_\mathrm{b}\left(\rca{\RR^{d+2}}\right)$ and $\Psi$ ...

\end{proposition}
\begin{proof}
For $\Phi \in C^2_\mathrm{b}\left(\rca{\RR^{d+2}}\right)$ and $\Psi \in C^2_\mathrm{b}\left(\RR^{d+2}\right)$ 
\begin{multline*}
    \lim_{N\rightarrow \infty} \EE^N\left[\int_0^T \left(\widetilde{\mathcal{L}}^N_3 (\Phi \oplus \Psi) \right)\left(\tmu_t, \tl_t\right)\dd t\right]\\
    =
    \lim_{N\rightarrow \infty} \EE^N\left[\int_0^T -\Phi^\prime\left(\tmu_t\right)\left(\delta_y\right)K\left(y,\tl_t\right)\tmu_t(\dd y)\dd t\right]
    +\lim_{N\rightarrow \infty} \EE^N\left[\int_0^T \Psi^\prime\left(\tl_t\right)\left(y\right)K\left(y,\tl_t\right)\tmu_t(\dd y)\dd t\right]
\end{multline*}

The first term converges as one expects, the second term has a similar difficulty to the previous proposition.
\end{proof}


  We define $L(\mu^N)$ to be the drift \begin{equation*}L(\mu)=\int_{\mathbb{X}\times\mathbb{X}} \{\delta_{y+z}-\delta_y-\delta_z\}1_{y\neq z}K(y,z)\mu(dy)\mu(dz). \end{equation*} Therefore, \begin{equation*} \mathcal{L}^N \Phi_k(\mu)=\langle \phi_k, L(\mu^N)\rangle.\end{equation*} We therefore estimate \begin{multline}
    \mathbb{E}\left[\left.\int_0^t d\left(\mu^N_s, \mu^N_0+\int_0^s L(\mu^N_u) du\right)\right|\mathcal{F}_0\right]\leq \mathbb{E}\left[\left.\sum_{k=1}^\infty 2^{-k} \sup_{s\leq t}\left|M^{k, N}_s\right|^2\right| \mathcal{F}_0\right] \\ \leq \frac{12 t \left(\kappa \left<\pi_n, \mu^N_0 \right>^2
                 + 8 \gamma \left<\pi_n, \mu^N_0 \right> \left<\pi_e, \mu^N_0 \right>\right)}{N}.
\end{multline}

%Provided that $\langle \pi_n^2 + \pi_e^2, \mu^N_0\rangle<\infty$, there is a unique maximal process $(\mu_t)_{t<T}$ such that, for all $t<T$, 

%; \hspace{1cm} \langle \pi_n^2+\pi_e^2, \mu_t\rangle <\infty.


Under some additional assumptions we have for reasonable test functions $\phi$
\begin{equation}\label{eq:wk-kinetic}
    \langle \phi, \mu_t\rangle=\langle \phi, \mu^N_0\rangle +\int_0^t \langle \phi, L(\mu_s)\rangle \dd s.
\end{equation} 
