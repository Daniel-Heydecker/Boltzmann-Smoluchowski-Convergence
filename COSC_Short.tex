\section{Convergence of the Stochastic Coagulant}
\label{sec: csc}

We now turn to a preliminary version of Theorem \ref{thrm: convergence of stochastic coagulent}.
In this section, we will prove \emph{local uniform} convergence of the stochastic coagulant $\mu^N_t$ to a solution $\mu_t$ of (\ref{eq: E+G}); throughout, we fix $\mu_0$ satisfying (A1-4.). As mentioned in the introduction, we will consider a mild generalisation of the stochastic coagulant, which will be helpful for future reference: we allow initial data of the form 
\begin{equation}
    \mu^N_0 := \frac1N \sum_{i=1}^{l_N} \delta_{\left(1,v_i, \frac12 \abs{v_i}^2\right)}
\end{equation}
where the $(v_i)_{i=1}^{l_N}$ are the initial velocities for the underlying Kac process, $l_N\le N$, and we ask that the following conditions hold.
\\B1. As $N\rightarrow \infty$, the initial measures $\mu^N_0=\frac1N\sum_{i\le N}\delta_{(1,v_i,\frac{1}{2}|v_i|^2)}$ converge in probability to $\mu_0$ in the vague topology in probability: \begin{equation}d_0(\mu^N_0, \mu_0)\rightarrow 0 \hspace{1cm}\text{in probability}. \end{equation} 
\\B2. We also have the convergence \begin{equation} \langle x, \mu^N_0\rangle \rightarrow \langle x, \mu_0\rangle \hspace{1cm} \text{in probability}\end{equation} where we recall that $x$ is the identity function on $S$, and \begin{equation} \sup_{N\ge 1} \EE\langle \varphi^2, \mu^N_0\rangle <\infty. \end{equation}
We have the following corollary of the second point of B2. For $\eta>0$, let $X^N_\eta$ be the number of particles with kinetic energy exceeding $\eta:$ \begin{equation}
    X^N_\eta= \#\left\{i\le l_N: \frac{1}{2}|v_i|^2>\eta\right\}.
\end{equation}Then, as $\eta\rightarrow \infty$, we have the convergence \begin{equation}\label{eq: b3} \sup_{N\ge 1} \mathbb{E}\left[\frac{1}{N}X^N_\eta\right]\rightarrow 0. \end{equation} \medskip \\ For example, it is straightforward to verify that these hold in the case where $l_N=N$, and $v_1,....,v_N$ are independent samples from the distribution $m$ described in Section \ref{sec: results}. However, the more general hypotheses will be useful for a `duality' argument in Section \ref{sec: finiteness of second moment}. Our result is as follows:
\begin{lemma}\label{lemma: local uniform convergence of stochastic coagulent}
Suppose $\mu_0$ satisfies ({A1}-{4.}), and let $(\mu_t)_{t\ge 0}$ be the solution to (\ref{eq: E+G}) starting at $\mu_0$. Let $(\mu^N_t)_{t\ge 0}$ be stochastic coalescents with initial data $\mu^N_0$ satisfying (B1-2.). Then we have the local uniform convergence
\begin{equation*}
\forall t_\mathrm{f}\ge 0 \hspace{0.5cm} \sup_{t\le t_\mathrm{f}} d_0(\mu^N_t, \mu_t)\rightarrow 0 \text{ in probability} 
\end{equation*} where recall that $d_0$ is a complete metric inducing the vague topology.
\end{lemma} \begin{remark}
We will later upgrade the \emph{local} uniform convergence to \emph{full} uniform convergence in Lemma \ref{lemma: uniform convergence of coagulant}. We also remark that this does not immediately imply the convergence of the gel terms in Theorem \ref{thrm: convergence of stochastic coagulent}, as the test functions involved are  neither compactly supported nor even bounded. This will be dealt with in Sections \ref{sec: ECT}, \ref{sec: COG}, where the proofs build on this result.
\end{remark}
The proof proceeds as follows, based on the well known method of proving tightness and identifying possible limit paths. Firstly, we will argue that for any $t_\mathrm{f}\ge 0$, the processes $(\mu^N_t)_{0\le t\le t_\mathrm{f}}$ are tight in the Skorohod topology of $\mathbb{D}([0,t_\mathrm{f}],(\mathcal{M},d_0))$.  Then, we will argue that if $\overline{\mu}$ is any subsequential limit point, then for $\xi$ satisfying a certain regularity condition \hyperref[eq: reg]{(C1-2.)}, the restricted measures $\overline{\mu}_t^\xi(dx)=1_{S_\xi}\overline{\mu}_t$ solve the restricted dynamics (\ref{eq:rE1}, \ref{eq: rE2}) and that $\overline{\mu}^\xi_0=\mu^\xi_0$.
By uniqueness in Lemma \ref{lemma: restricted dynamics}, it follows that $\overline{\mu}^\xi$ coincides with the solution $\mu^\xi$ found in Lemma \ref{lemma: restricted dynamics}. For any subsequential limit $\xi$, the required regularity condition holds for sufficiently many cutoff $\xi$ to allow a limit $\xi \rightarrow \infty$, to conclude that $(\overline{\mu}_t)_{0\le t\le t_\mathrm{f}}=(\mu_t)_{0\le t\le t_\mathrm{f}}$. Writing $d_\mathrm{Sk}$ for a complete Skorohod metric on $\mathbb{D}([0,t_\mathrm{f}],(\mathcal{M},d_0))$, this implies \begin{equation} d_\mathrm{Sk}\left((\mu^N_t)_{0\le t\le t_\mathrm{f}}, (\mu_s)_{0\le t\le t_\mathrm{f}}\right) \rightarrow 0 \hspace{0.5cm} \text{in probability.} \end{equation} Since the limit process $(\mu_t)_{0\le t\le t_\mathrm{f}}$ is continuous in the vague topology $(\mathcal{M},d_0)$, it follows that we may upgrade to uniform convergence: \begin{equation} \sup_{0\le t\le t_\mathrm{f}}\hspace{0.1cm}d_0\left(\mu^N_t, \mu_t\right) \rightarrow 0 \hspace{0.5cm} \text{in probability}  \end{equation} as claimed. \medskip \\ We first prove some tightness results for the processes $(\mu^N_t)_{t\le t_\mathrm{f}}.$ 

\iffalse
\begin{lemma}\label{lemma: cpct contain}
Suppose conditions (B1-2.) hold. Then there exists a compact set $\mathcal{A} \subset \mathcal{M}$ \iffalse and $\widetilde{B} \subset \mathcal{M}^\ast$ \fi such that
\begin{equation*}
    \PP\left(\exists t \in [0,\infty) \colon \mu^N_t \notin \mathcal{A}  \right) \rightarrow 0
\end{equation*} as $N\rightarrow \infty.$
\fi
\iffalse and
\begin{equation*}
    \PP\left(\exists (t,\xi) \in [0,\infty)\times (0,\infty) \colon \left({\mu}^{N,\xi}_t, g^{N,\xi}_t\right) \notin \widetilde{B} \right) \rightarrow 0
\end{equation*}
\end{lemma}
\textcolor{red}{\textbf{Isn't $\mathcal{M}$ already compact?}}
\begin{proof}
Define compact subsets of $S$ by $A_r = \left\{x\in S \colon \pi_n(x) \leq 2r, \pi_e(x) \leq r \sigma_2(m)\right\}$ so that
\begin{equation*}
    \sup_{t\ge 0}\hspace{0.1cm} \mu_t^N(A_r^\mathrm{c}) \leq \frac{1}{2r} + \frac{\left<\pi_e, \mu_0^N\right>}{2r \sigma_2(m)}.
\end{equation*}
It is straightforward to verify that $\mathcal{A}:= \left\{\nu\in \mathcal{M} \colon \nu\left(A_r^\mathrm{c}\right) < 1/r\ \forall r\in\NN \right\}$ is a relatively compact subset of $\mathcal{M}$ in the vague topology.
By ({A3.}), we have $\mathbb{P}(\left<\pi_e, \mu_0^N\right> \ge \sigma_2(m))\rightarrow 0$, which implies that $ \PP\left(\exists t \colon \mu_t^N \notin \mathcal{A}\right) \rightarrow 0$.
\iffalse The same argument applies to $\mu^N _t1_{\left\{\pi_n \leq \xi\right\}}$, and noting that \begin{equation} \pi_n(g^{N,\xi}_t) \leq 1; \hspace{1cm}\pi_e(g^{N,\xi}_t) \leq \left<\pi_e, \mu_0^N \right> \rightarrow \frac12 \sigma_2(m)\end{equation} a similar argument applies for $g^{N,\xi}_t$. \fi 
\end{proof}
\fi
\begin{lemma}\label{lemma: bdd jump rate}
Suppose conditions ({B1}-{3.}) hold. Then
\begin{enumerate}[label=\roman{*}).]
    \item There process $
    \mu^N$ has at most  $N-1$ jumps in $[0,\infty)$;
    \item the jump rates of $\mu^N$ are bounded by
    \begin{equation*}
        \lambda^N := N\left\{\kappa  \left<\pi_n, \mu_0^N\right>^2
        +8 \gamma  \left<\pi_n, \mu_0^N\right> \left<\pi_e, \mu_0^N\right>
        \right\},
    \end{equation*}
    which is constant in time, and $\lambda^N/N \in \mathcal{O}_\mathrm{p}(1)$.
\end{enumerate}
\end{lemma}
\begin{proof}
For the first claim, we note that after $l_N-1 \le N-1$ jumps, precisely one cluster remains and so the system undergoes no further changes.  The second claim is a simple calculation using the conservation laws for $\pi_n$ and $\pi_e$. The claim that $\lambda^N/N \in \mathcal{O}_\mathrm{p}(1)$ follows from (B2.).
\end{proof}

\begin{lemma}\label{lemma: tight processes}
Suppose (B1-2.) hold, and let $t_\mathrm{f}>0$. Then
 the distributions of $\mu^N$ are tight in the space of probability measures on $\mathbb{D}\left([0,t_\mathrm{f}]);  \left(\mathcal{M}, d_0\right)\right)$.
\end{lemma}
\begin{proof}
First, we consider the real valued processes $\langle f, \mu^N_t\rangle$ for $f\in C_c(S).$ Compact containment is immediate, since $\abs{\left<f,\mu_t^N\right>} \leq \supnorm{f}$. For $\delta>0$, write $(I_i: i=1,...,\lceil t_\mathrm{f}/\delta\rceil)$ for a partition of $[0,t_\mathrm{f}]$ into intervals of length at most $\delta$. Recalling from Lemma \ref{lemma: bdd jump rate} that the rescaled jump rate $\lambda^N/N$ is $O_\mathrm{p}(1)$, it follows that, for all $\epsilon>0$, \begin{equation} \sup_{N\ge 1}\hspace{0.1cm} \mathbb{P}\left(\mu^N \text{ makes at least }\frac{\epsilon N}{3} \text{ jumps in any  } I_i: i\le \lceil t_\mathrm{f}/\delta\rceil\right) \rightarrow 0 \end{equation} as $\delta\rightarrow 0.$ Now, since the jumps are bounded by $\frac{3}{2}\supnorm{f}/N$, for any $\epsilon>0$ it follows that \begin{equation} \sup_{N\ge 1} \hspace{0.1cm} \mathbb{P}\left(\exists s, t \le t_\mathrm{f}: |s-t|<\delta, |\langle f, \mu^N_t-\mu^N_s\rangle|>\epsilon\right) \rightarrow 0 \end{equation} as $\delta \rightarrow 0.$ Therefore, for any fixed $f \in C_c(S),$ we may apply \cite[Chapter 3, Theorem 8.6c)]{EK86} to see that the processes $(\langle f, \mu^N_t\rangle)_{0\le t\le t_\mathrm{f}}$ are tight. \medskip \\ 
We now use a general result \cite[Theorem 4.6]{J86} on tightness in Skorohod spaces. Compact containment of $\mu^N_t$ in $(\mathcal{M}, d_0)$ is immediate, since $\mathcal{M}$ is itself compact, and it is straightforward that the maps $(\mu \mapsto \langle f, \mu\rangle: f\in C_c(S))$ on $(\mathcal{M},d_0)$ satisfy the requirements of the cited theorem. Therefore, the result of the previous paragraph shows that the processes $\mu^N$ are tight in $\mathbb{D}([0,t_\mathrm{f}],(\mathcal{M},d_0))$, as claimed.
\end{proof}

Having proven tightness, we now turn to the identification of possible limit paths. For $N\in\NN$ and $\xi \in \RR_+,$ we consider the processes $\mu^N, \mu^{N,\xi} \in \mathbb{D}\left([0,\infty); \left(\mathcal{M}, d_0\right)\right)$, where
$\mu^{N,\xi}_t = \mu^N_t 1_{S_\xi}$. We write $g^{N,\xi} \in \mathbb{D}\left([0,\infty); S_\mathrm{g}\right)$ for the data of the gel cutoff at this level:
\begin{equation}\label{eq: cutoff gel} g^{N,\xi}_t = \left<x, \mu_0^N -\mu^{N,\xi}_t\right>\end{equation} where we recall that $x$ denotes the identity function on $S$, so that
$\left(\mu^{N,\xi}, g^{N,\xi}\right) \in \mathbb{D}\left([0,\infty); \left(\mathcal{M}^\ast, d^\ast\right)\right). $  
We first identify the martingale parts of  the processes $\langle f, \mu^{N,\xi}_t\rangle$ and $g^{N,\xi}_t$.
For $\xi >0$ and $f\in C_c(S)$, let
\begin{equation}\label{eq:mg1} \begin{split}
\mathfrak{M}_t^{f,N,\xi} &:= \left<f, \mu_t^{N,\xi}\right> - \left<f, \mu_0^{N,\xi}\right> 
- \int_0^t \int_{S_\xi}f(x) \overline{K}\left(x,g^{N,\xi}_t\right)\mu^{N,\xi}_s(d x)d s\\
&\hspace{1cm}- \frac12 \int_0^t \int_{S_\xi^2}\left[f(x+y)1[\varphi(x+y) \leq \xi ] - f(x) - f(y)\right]
      \overline{K}(x,y)\mu^{N,\xi}_s(d x)\mu^{N,\xi}_s(d y) ds \\
     & \hspace{1cm} + \frac{1}{2N}\int_0^t \int_{S_\xi} \left[f(2x)1[\varphi(2x) \leq \xi ] - 2f(x)\right]\overline{K}(x,x)\mu^{N,\xi}_s(d x) d s
\end{split}\end{equation}
and
\begin{equation}\label{eq:mg2} \begin{split}
\widehat{\mathfrak{M}}_t^{N,\xi} &:= g_t^{N,\xi} - g_0^{N,\xi} 
- \int_0^t \int_{S_\xi}x \overline{K}\left(x,g^{N,\xi}_t\right)\mu^{N,\xi}_s(d x)d s\\& \hspace{1cm}
- \frac12 \int_0^t \int_{S_\xi^2}(x+y)1[\varphi(x+y) \leq \xi ]
      \overline{K}(x,y)\mu^{N,\xi}_s(d x)\mu^{N,\xi}_s(d y) ds \\ &\hspace{1cm}
      + \frac{1}{2N}\int_0^t \int_{S_\xi} 2x 1[\varphi(2x) \leq \xi ]\overline{K}(x,x)\mu^{N,\xi}_s(d x) d s.
\end{split}\end{equation}


\begin{lemma}\label{lemma: qvar}
Suppose conditions (B1-2.) hold, and write $(\mathcal{F}^{N}_t)_{t\ge 0}$ for the natural filtration of $(\mu^N_t)_{t\ge 0}$. Then, for all $\xi \in (0, \infty]$ and $f\in C_c(S)$, the processes $\mathfrak{M}_t^{f,N,\xi}, \widehat{\mathfrak{M}}_t^{N,\xi}$ are martingales with respect to $(\mathcal{F}^N_t)_{t\ge 0}.$
The quadratic variations are bounded by
\begin{equation*}
    \left[\left<f, \mu^{N,\xi}\right>\right]_t
    = \left[\mathfrak{M}^{f,N,\xi}\right]_t
    \leq \frac{9 \|f\|_\infty^2}{N}; \qquad
    \left[g^{N,\xi}\right]_t
    = \left[\widehat{\mathfrak{M}}_t^{N,\xi}\right]_t
    \leq \frac{2\xi}{N}\left<\varphi, \mu^N_0\right>  \qquad \forall t \in [0,\infty), \xi \in [0,\infty).
\end{equation*}
\end{lemma}
\begin{proof}
The martingale property is a standard; see, for example, \cite[Chapter 4, Proposition 1.7]{EK86} or \cite[Lemma 19.21]{K02}.
Since $\mathfrak{M}^{f,N,\xi}$ and $\widehat{\mathfrak{M}}^{N,\xi}$ are pure jump processes, the quadratic variation is the sum of the squares of the jumps, which can be bounded using Lemma~\ref{lemma: bdd jump rate} for the first claim and by straightforward maximisation arguments for the second.
\end{proof}

\iffalse Having now proven the tightness, we will argue that the only possible limit point is the deterministic path $(\mu_t)_{t\ge 0}$ given as the solution to (\ref{eq: E+G}). \fi

Next, we discuss a regularity condition for potential cutoff values $\xi$. This will allow us to deduce vauge convergence of the truncated measures $\mu^{N,\xi}_t=\mu^N_t1_{S_\xi}$, despite the discontinuity of the cutoff $1_{S_\xi}$.
\begin{lemma}[Regularity Condition]\label{lemma: regularity condition} Suppose that $\mu^N$ are stochastic coagulants satisfying (B1-2.), and that some subsequence $(\mu^{N_r})_{r\ge 1}$ of $(\mu^N)_{N\ge 1}$ converges in distribution to a random variable $\overline{\mu}$ in $\mathbb{D}([0,t_\mathrm{f}],(\mathcal{M},d_0))$. Then, for almost all $\xi>0$, the following hold: \begin{enumerate}[label=\roman{*}).]\item Almost surely, for  almost all $t\le t_\mathrm{f}$, \begin{equation} \label{eq: reg} \tag{{C1.}}
   \overline{\mu}_t\left(\left\{x\colon \varphi(x) = \xi\right\}\right) + 
    \overline{\mu}_t \otimes \overline{\mu}_t \left(\left\{(x,y) \colon \varphi(x+y) = \xi\right\}\right) = 0;
\end{equation} \item This also holds for $t=0$. That is, almost surely, \begin{equation} \label{eq: reg2} \tag{{C2.}}
   \overline{\mu}_0\left(\left\{x\colon \varphi(x) = \xi\right\}\right) + 
    \overline{\mu}_0 \otimes \overline{\mu}_0 \left(\left\{(x,y) \colon \varphi(x+y) = \xi\right\}\right) = 0. 
\end{equation} \end{enumerate} \end{lemma} \begin{proof} Since $\mathbb{D}([0,t_\mathrm{f}],(\mathcal{M},d_0))$ is separable, we may use the Skorohod representation theorem to realise all $\mu^N, \overline{\mu}$ on a common probability space, such that $\mu^{N_r}\rightarrow \overline{\mu}$ almost surely in the Skorohod topology of $\mathbb{D}([0,t_\mathrm{f}],(\mathcal{M},d_0))$, and such that $\langle \varphi, \mu^N_0\rangle \rightarrow \langle \varphi, \mu_0\rangle$ almost surely. In particular, $\langle \varphi, \mu^N_0\rangle$ are almost surely bounded. \medskip \\ We first consider the case of nonrandom $\nu^{N_r} \rightarrow \nu$ in $\mathbb{D}([0,t_\mathrm{f}],(\mathcal{M},d_0))$, and such that, for some constant $C$, \begin{equation} \sup_{N \ge 0}\hspace{0.1cm}\sup_{t\le t_\mathrm{f}} \langle \varphi, \nu^N_t\rangle \le C <\infty. \end{equation}  Let $\Psi$ be the measure on $[0,\infty)$ given by \begin{equation}\begin{split}\label{eq: Psi} \Psi(A)&=\nu_0(x: \varphi(x)\in A)+\nu_0\otimes\nu_0((x,y):\varphi(x+y)\in A)\\&\hspace{1cm}+\int_0^{t_\mathrm{f}} \left(\nu_t(x: \varphi(x)\in A)+\nu_t\otimes\nu_t((x,y): \varphi(x+y)\in A)\right) dt.\end{split} \end{equation} Since $\nu_t$ is a subprobability measure for all $t\le t_\mathrm{f}$, $\Psi$ has total mass at most $2(1+t_\mathrm{f})$, and so, for all but countably many $\xi\ge 0$, $\Psi(\{\xi\})=0.$ For such $\xi$, we have the desired properties that \begin{equation} \nu_0(x: \varphi(x)=\xi)+\nu_0\otimes\nu_0((x,y):\varphi(x+y)=\xi)=0; \end{equation}\begin{equation} \nu_t(x: \varphi(x)=\xi)+\nu_t\otimes\nu_t((x,y):\varphi(x+y)=\xi)=0 \text{ for almost all }t\le t_\mathrm{f}. \end{equation} We now show how this may be extended to the case of random $\mu^{N}, \overline{\mu}$ such that $\mu^{N_r}\rightarrow \overline{\mu}$ almost surely in $\mathbb{D}([0,t_\mathrm{f}],(\mathcal{M},d_0)).$ Let $\Psi$ be the random measure in (\ref{eq: Psi}) corresponding to $\overline{\mu}$, and let $A(\overline{\mu})$ be the random set \begin{equation} A(\overline{\mu})=\left\{\xi \ge 1: \hspace{0.1cm}\Psi(\{\xi\})>0\right\}. \end{equation} The argument above shows that, almost surely, $ \int_{A(\overline{\mu})} d\xi=0$; by Fubini, this implies that \begin{equation} \int_1^\infty \mathbb{P}(\xi \in A(\overline{\mu}))d\xi=0. \end{equation} Therefore, for almost all $\xi \ge 1$, $\mathbb{P}(\xi \in A(\overline{\mu}))=0$. For all such $\xi$, the following hold almost surely:  \begin{equation} \overline{\mu}_0(x: \varphi(x)=\xi)+\overline{\mu}_0\otimes\overline{\mu}_0((x,y):\varphi(x+y)=\xi)=0; \end{equation}\begin{equation} \overline{\mu}_t(x: \varphi(x)=\xi)+\overline{\mu}_t\otimes\overline{\mu}_t((x,y):\varphi(x+y)=\xi)=0 \text{ for almost all }t\le t_\mathrm{f}. \end{equation}  This corresponds precisely to the desired conditions (\hyperref[eq: reg]{C1-2.}).   \end{proof} 

This regularity condition now allows us to deduce weak convergence of the cutoff measures $\mu^{N_r,\xi}_t$.

\begin{lemma}\label{lemma: convergences in distribution}
Suppose that $\mu^N$ are stochastic coagulants satisfying (B1-2.), and that a subsequence  $\mu^{N_r}$ converges in distribution to a random variable $\overline{\mu}$ in $\mathbb{D}([0,t_\mathrm{f}],(\mathcal{M},d_0))$. If $\xi$ is such that (\hyperref[eq: reg]{C1-2.}) hold for $\overline{\mu}$, then writing $\overline{\mu}^\xi_t$ for the cutoff measures $\overline{\mu}^\xi_t=\overline{\mu}_t1_{S_\xi},$ we have the following convergences in distribution as $r\rightarrow \infty$. 
\begin{enumerate}[label=\roman{*}).] \item
For any $f\in C_c(S)$ and $0\le t \le t_\mathrm{f}$,
\begin{multline}
    \int_0^t \int_{S_\xi^2}\left[f(x+y)1[\varphi(x+y) \leq \xi ] - f(x) - f(y)\right]
      \overline{K}(x,y)\mu^{N_r,\xi}_s(d x)\mu^{N_r,\xi}_s(d y)d s
    \\\rightarrow
    \int_0^t \int_{S_\xi^2}\left[f(x+y)1[\varphi(x+y) \leq \xi ] - f(x) - f(y)\right]
      \overline{K}(x,y)\overline{\mu}^\xi_s(d x)\overline{\mu}^\xi_s(d y)d s;
\end{multline}
and
\begin{multline}
    \int_0^t \int_{S_\xi^2}(x+y)1[\varphi(x+y) > \xi ]
      \overline{K}(x,y)\mu^{N_r,\xi}_s(d x)\mu^{N_r,\xi}_s(d y)d s
    \\\rightarrow
    \int_0^t \int_{S_\xi^2}(x+y)1[\varphi(x+y) > \xi ]
      \overline{K}(x,y)\overline{\mu}^\xi_s(d x)\overline{\mu}^\xi_s(d y)d s.
\end{multline}
\item We write $\overline{g}^\xi$ for the data of the truncated gel corresponding to $\overline{\mu}$: \begin{equation}\label{eq: cutoff gel overline g xi} \overline{g}^{\xi}_t = \left<x, \mu_0 -\overline{\mu}^\xi_t\right>.\end{equation}   Then, for all $f\in C_c(S)$ and $t\le t_\mathrm{f}$, \begin{equation*}
    \int_0^t \int_{S_\xi}f(x) \overline{K}(x,g^{N_r,\xi}_s)\mu^{N_r,\xi}_s(d x)d s
    \rightarrow
    \int_0^t \int_{S_\xi}f(x) \overline{K}(x,\overline{g}^\xi_s)\overline{\mu}^\xi_s(d x)d s;
\end{equation*}
\begin{equation*}
    \int_0^t \int_{S_\xi}x \overline{K}(x,g^{N_r,\xi}_s)\mu^{N_r,\xi}_s(d x)d s \rightarrow \int_0^t \int_{S_\xi}x \overline{K}(x,\overline{g}^\xi_s)\overline{\mu}^\xi_s(d x)d s.
\end{equation*}
\item Let $L^\xi_\mathrm{g}, \widehat{L}^\xi_\mathrm{g}$ be the truncated drift operators in (\ref{eq: truncated drift 1}, \ref{eq: truncated drift 2}), and write $\mu^\xi_0, g^\xi_0$ for the cutoff measure and gel corresponding to the base measure $\mu_0$. Then, for all $t\le t_\mathrm{f}$ and $f\in C_c(S)$, \begin{equation}\begin{split} &\int_0^t \left[\langle f, \mu^{N_r,\xi}_s\rangle-\langle f,\mu^{N_r,\xi}_0\rangle -\int_0^s\left\langle f, L^\xi_\mathrm{g}\left(\mu^{N_r,\xi}_u,g^{N_r,\xi}_u\right)\right\rangle du\right]ds \\& \hspace{2cm} \rightarrow \int_0^t \left[\langle f, \overline{\mu}^{\xi}_s\rangle-\langle f,\mu^{\xi}_0\rangle -\int_0^s\left\langle f, L^\xi\left(\overline{\mu}^{\xi}_u,\overline{g}^\xi_u\right)\right\rangle du\right]ds \end{split}\end{equation} and \begin{equation} \int_0^t \left[g^{N_r,\xi}_s-g^{N_r,\xi}_0-\int_0^s \widehat{L}^\xi_\mathrm{g}(\mu^{N_r,\xi}_u,g^{N_r,\xi}_u)du\right]ds \rightarrow \int_0^t \left[\overline{g}^{\xi}_s-g^{\xi}_0-\int_0^s \widehat{L}^\xi_\mathrm{g}(\overline{\mu}^{\xi}_u,\overline{g}^{\xi}_u)du\right]ds. \end{equation}
\end{enumerate}
\end{lemma}
\begin{proof} As in the previous lemma, the Skorohod representation theorem shows that we may realise $\mu^N, \overline{\mu}$ on a common probability space, such that $\mu^{N_r}\rightarrow \mu$ almost surely in the Skorohod topology of $\mathbb{D}([0,t_\mathrm{f}],(\mathcal{M},d_0))$. Moreover, using (B1-2.) as above, $\langle \varphi, \mu^{N_r}_t\rangle$ is almost surely bounded, uniformly in $r\ge 1$ and $t\le t_\mathrm{f}$, and  \begin{equation}  d_0(\mu^{N_r}_0,\mu_0)\rightarrow 0; \hspace{1cm} \langle x, \mu^{N_r}_0\rangle \rightarrow \langle x, \mu_0\rangle \end{equation} almost surely. Since the map $(\mu_t)_{t\le t_\mathrm{f}}\mapsto \mu_0$ is continuous in the Skorohod topology of $\mathbb{D}([0,t_\mathrm{f}],(\mathcal{M},d_0))$, the first convergence displayed above implies that $\overline{\mu}_0=\mu_0$ almost surely. It is therefore sufficient to prove the results for the special case of nonrandom $\nu^N$, such that  $\nu^{N_r}\rightarrow \nu$ almost surely, and such that the preceding results hold with $\nu^N, \nu$ in place of $\mu^N, \overline{\mu}$. We will write $\nu_t^\xi, \nu^{N,\xi}_t$ for the cutoff measures, and $\overline{g}^\xi_t, g^{N,\xi}_t$ for the corresponding cutoff gel, defined as above with $\nu^N, \nu$ in place of $\mu^N, \overline{\mu}$.  \medskip \\ 
For part i)., Fix $f\in C_c(S)$ and $\xi\in [0, \infty)$. The functions \begin{equation}(x,y) \mapsto f(x)\overline{K}(x,y)1[\varphi(x) \le \xi,\varphi(y) \leq \xi]\end{equation} \begin{equation} (x,y)\mapsto f(x+y)\overline{K}(x,y)1[\varphi(x+y)\le \xi] \end{equation} are compactly supported, and continuous away from the exceptional set \begin{equation} E_\xi=\{(x,y)\in S^2: \varphi(x)=\xi \text{ or } \varphi(y)=\xi \text{ or } \varphi(x+y)=\xi\}. \end{equation} 

From condition (\ref{eq: reg}) it follows that, for almost all $t\le t_\mathrm{f}$, $(\nu_t\otimes \nu_t)(E_\xi)=0$. Since $d_0\left(\nu^N_t, \nu_t\right) \rightarrow 0$ for all but at most  countably many $t\le t_\mathrm{f}$, it follows that
\begin{equation}
    \int_{S_\xi^2}\left[f(x) + f(y)\right]\overline{K}(x,y)\nu^{N_r,\xi}_t(d x)\nu^{N_r,\xi}_t(d y)\rightarrow
\int_{S_\xi^2}\left[f(x) + f(y)\right]\overline{K}(x,y)\nu^\xi_t(d x)\nu^\xi_t(d y);
\end{equation}
\begin{equation}
     \int_{S_\xi^2}f(x+y)1[\varphi(x+y) \le \xi]\overline{K}(x,y)\nu^{N_r,\xi}_t(d x)\nu^{N_r,\xi}_t(d y)
    \rightarrow \int_{S^2} f(x+y)1[\varphi(x+y) \le \xi]\overline{K}(x,y)\nu^\xi_t(d x)\nu^\xi_t(d y)
\end{equation}
for almost all $t\le t_\mathrm{f}$.
Recalling that $\overline{K}(x,y)\le \Delta \varphi(x)\varphi(y)$ for some constant $\Delta=\Delta(\kappa,\gamma)$, we bound \begin{equation}\label{eq: bound on Lxigel 1}\begin{split}
    &\left|\int_{S_\xi^2} \left[f(x+y)1[\varphi(x+y)\le \xi]-f(x)-f(y)\right] \overline{K}(x,y)\nu^{N_r,\xi}_t(dx)\nu^{N_r,\xi}_t(dy)\right|   \le 3\Delta\|f\|_\infty\langle \varphi, \nu^{N_r}_0\rangle^2 \end{split}
\end{equation} and so, by bounded convergence, for all $t\le t_\mathrm{f}$, \begin{multline}
    \int_0^t \int_{S_\xi^2}\left[f(x+y)1[\varphi(x+y) \leq \xi] - f(x) - f(y)\right]
      \overline{K}(x,y)\nu^{N_r,\xi}_s(d x)\nu^{N_r,\xi}_s(d y)d s
    \\\rightarrow
    \int_0^t \int_{S_\xi^2}\left[f(x+y)1[\varphi(x+y) \leq \xi] - f(x) - f(y)\right]
      \overline{K}(x,y)\nu^\xi_s(d x)\nu^\xi_s(d y)d s
\end{multline}
which proves the first claim.
For the second claim, we note that $\nu^{N,\xi}_t, \nu^\xi_t$ are supported on $S_\xi$, and for all $x, y\in S_\xi$ we have the bound $\abs{x},\abs{y} \leq C\xi$, for some constant $C$; the second claim now follows. \medskip \\ For part ii), we first claim that for $t=0$ and almost all $t\le t_\mathrm{f}$, $g^{N_r,\xi}_t \rightarrow \overline{g}^\xi_t$. To see this, we recall that $\langle x, \nu^{N_r}_0\rangle \rightarrow \langle x, \nu_0\rangle = \langle x, \mu_0\rangle$ by construction, and using conditions (\hyperref[eq: reg]{C1-2.}), for $t=0$ and almost all $t\le t_\mathrm{f}$, \begin{equation} \langle x, \nu^{N_r,\xi}_t\rangle =\langle x1_{S_\xi},\nu^{N_r}_t\rangle \rightarrow \langle x1_{S_\xi},\nu_t\rangle =\langle x, \nu^\xi_t\rangle.  \end{equation} This follows because $x1_{S_\xi}$ is compactly supported, and except for exceptional times $t$, $x1_{S_\xi}$ is continuous except on a set of $\nu_t$-measure 0. Since $\nu_0=\mu_0$, this also implies that \begin{equation} g^{N_r,\xi}_0\rightarrow \langle x, \mu_0-\mu^\xi_0\rangle =g^\xi_0\end{equation} and we may therefore use $g^\xi_0$ in place of $\overline{g}^\xi_0$. \medskip \\  We also observe that $\varphi(g^{N_r,\xi}_t) \le \langle \varphi, \nu^{N_r}_0\rangle$ is bounded uniformly in $t\le t_\mathrm{f}$ and in $r$. The argument is now essentially identical to point i). above.  \medskip \\ For the first claim of item iii)., we first note that the the two parts above show that, for all $t\le t_\mathrm{f}$, \begin{equation}\int_0^t \left\langle f, L^\xi_\mathrm{g}\left(\nu^{N_r,\xi}_s,g^{N_r,\xi}_s\right)\right\rangle ds \rightarrow \int_0^t \left\langle f, L^\xi_\mathrm{g}\left(\overline{\mu}^\xi_s,\overline{g}^{\xi}_s\right)\right\rangle ds. \end{equation} Moreover, following (\ref{eq: bound on Lxigel 1}), we have the uniform bound, for all $u\le t_\mathrm{f},$ \begin{equation} \left|\left\langle f, L^\xi_\mathrm{g}\left(\nu^{N_r,\xi}_u,g^{N_r,\xi}_u\right)\right\rangle\right|\le 4\|f\|_\infty \langle \varphi, \nu^{N_r}_0\rangle^2. \end{equation} It therefore follows, by bounded convergence, that for all $t\le t_\mathrm{f}$, \begin{equation} \label{eq: part iii convergence 1} \int_0^t \int_0^s \left\langle f, L^\xi_\mathrm{g}\left(\nu^{N_r,\xi}_u,g^{N_r,\xi}_u\right)\right\rangle du ds \rightarrow \int_0^t \int_0^s \left\langle f, L^\xi_\mathrm{g}\left(\overline{\mu}^\xi_u,\overline{g}^{\xi}_u\right)\right\rangle du ds. \end{equation}  For the other terms, we note that $d_0(\nu^{N_r}_0, \nu_0)\rightarrow 0$, and that $d_0(\nu^{N_r}_t, \nu_t)\rightarrow 0$ for almost all $t\le t_\mathrm{f}$. Therefore, by (\hyperref[eq: reg]{C1-2.}) and using bounded convergence, \begin{equation} \int_0^t\langle f, \nu^{N_r,\xi}_s\rangle ds \rightarrow \int_0^t \langle f, \nu^\xi_s\rangle ds;\end{equation}  \begin{equation} \langle f, \nu^{N_r,\xi}_0\rangle \rightarrow \langle f, \nu^\xi_0\rangle = \langle f, \mu^\xi_0\rangle.  \end{equation} Combined with (\ref{eq: part iii convergence 1}), these prove the first claim. An identical argument proves the second claim.
\end{proof}

We will now show that for any subsequential limit point $\overline{\mu}$, the pair $(\overline{\mu}^\xi, \overline{g}^\xi)$ solves the restricted dynamics (\ref{eq:rE1}, \ref{eq: rE2}) whenever $\xi$ satisfies (\hyperref[eq: reg]{C1-2.}).

\begin{lemma}\label{lem: mild}
Suppose conditions (B1-2.) hold, and that some subsequence $(\mu^{N_r})_{r\ge 1}$ of $(\mu^N)_{N\ge 1}$ converges in distribution to a random variable $\overline{\mu}$ in $\mathbb{D}([0,t_\mathrm{f}],(\mathcal{M},d_0))$. Suppose also that $\xi$ is such that (\hyperref[eq: reg]{C1-2.}) hold for $\overline{\mu}$. Write $\overline{\mu}^\xi$ for the truncated measures and $\overline{g}^\xi$ for the truncated gel, as above, and similarly $\mu^\xi_0, g^\xi_0$. Then, almost surely, for all $f\in C_c(S)$ and all $t\le t_\mathrm{f}$,
\begin{equation}
   \left<f, \overline{\mu}^\xi_t\right> - \left<f, \mu^\xi_0\right> - \int_0^t \left<f, L^\xi_\mathrm{g}\left(\overline{\mu}^\xi_s, \overline{g}^\xi_s\right) \right> ds = 0;
\end{equation}
\begin{equation*} \overline{g}^\xi_t - {g}^\xi_0 - \int_0^t \widehat{L}^\xi_\mathrm{g}\left(\overline{\mu}^\xi_s, \overline{g}^\xi_s\right) d s= 0.
\end{equation*} It follows that, almost surely, $(\overline{\mu}^{\xi}_t,\overline{g}^\xi_t)_{0\le t\le t_\mathrm{f}}$ is the unique solution $(\mu^\xi_t, g^\xi_t)_{0\le t\le t_\mathrm{f}}$ to (\ref{eq:rE1}, \ref{eq: rE2}) started at $(\mu^\xi_0,g^\xi_0)$. In particular, $(\overline{\mu}^\xi_t, \overline{g}^\xi_t)_{0\le t\le t_\mathrm{f}}$ is deterministic.
\end{lemma}
\begin{proof}
Fix $f\in C_c(S)$. We first estimate the martingale terms $\mathfrak{M}^{f,N,\xi}$,  $\widehat{\mathfrak{M}}^{N,\xi}$. From \cite[Thrm. 26.12]{K02} and Lemma~\ref{lemma: qvar}, there exists a  constant $c=c(\kappa,\gamma)>0$ such that
\begin{equation*}
    \EE\left[\sup_{t\in[0,\infty)} \left(\mathfrak{M}_t^{f,N,\xi}\right)^2\right]
    \leq c \EE\left[\sup_{t\in[0,\infty)} \left[\mathfrak{M}^{f,N,\xi}\right]_t\right]
    \leq \frac{9c \|f\|_\infty^2}{N}.
\end{equation*} It follows that, for all $t\le t_\mathrm{f}$,\begin{equation} \int_0^{t} M^{f,N,\xi}_s d s \rightarrow 0\end{equation} in distribution.
We now estimate the self-interaction term: for some other constant $c$, we bound
\begin{equation*}
   \abs{\frac{1}{2N}\int_0^t \int_S \left[f(2x)1[\varphi(2x)\leq \xi] - 2f(x)\right]\overline{K}(x,x)\mu^{N,\xi}_s(d x) d s}
   \leq \frac{c}{N}\|f\varphi^2\|_\infty.
\end{equation*}
Therefore, from (\ref{eq:mg1}), it follows that for all $t \le t_\mathrm{f}$, \begin{equation} \int_0^t\left[\langle f,\mu^{N_r, \xi}_{s}\rangle -\langle f, \mu^{N_r,\xi}_0\rangle -\int_0^{s} \langle f, L^\xi_\mathrm{g}(\mu^{N_r,\xi}_u, g^{N_r,\xi}_u)\rangle du \right]ds\rightarrow 0 \end{equation} in distribution. Comparing this to item iii). of Lemma \ref{lemma: convergences in distribution}, we identify the two limits to conclude that almost surely, for all $t\le t_\mathrm{f}$, \begin{equation} \int_0^t\left[\langle f, \overline{\mu}^{\xi}_s\rangle -\langle f, {\mu}^{\xi}_0\rangle -\int_0^s \langle f, L^\xi_\mathrm{g}(\overline{\mu}^{\xi}_u, \overline{g}^{\xi}_u)\rangle \right]ds =0.\end{equation} Since $t\le t_\mathrm{f}$ is arbitrary and the integrand is right-continuous, this implies that, almost surely, for all $t\le t_\mathrm{f}$, \begin{equation} \langle f, \overline{\mu}^{\xi}_t\rangle -\langle f, {\mu}^{\xi}_0\rangle -\int_0^t \langle f, L^\xi_\mathrm{g}(\overline{\mu}^{\xi}_s, \overline{g}^{\xi}_s)\rangle  =0.\end{equation} Taking an intersection over $f$ belonging to a countable dense subset of $(C_c(S), \|\cdot\|_\infty)$ proves the first claim. The argument for the second claim is identical. It follows that $(\overline{\mu}^\xi_t, \overline{g}^\xi_t)_{0\le t\le t_\mathrm{f}}$ satisfies (\ref{eq:rE1}, \ref{eq: rE2}) almost surely, which implies that $(\overline{\mu}^\xi_t, \overline{g}^\xi_t)_{0\le t\le t_\mathrm{f}}=(\mu^\xi_t, g^\xi_t)_{0\le t\le t_\mathrm{f}}$ almost surely, by uniqueness in Lemma \ref{lemma: restricted dynamics}.
\end{proof}

We may now prove Lemma \ref{lemma: local uniform convergence of stochastic coagulent}, using the well-known combination of tightness and the identification of the limit. Tightness of the processes $\mu^N$ was proven in Lemma \ref{lemma: tight processes}, and so it is sufficient to characterise possible limit paths. Suppose a subsequence $(\mu^{N_r}_t)_{0\le t\le t_\mathrm{f}}$ converges in distribution to a limit $(\overline{\mu}_t)_{0\le t\le t_\mathrm{f}}$. From Lemma \ref{lemma: regularity condition}, there is an unbounded set $\mathcal{X}\subset [1,\infty)$ such that (\hyperref[eq: reg]{C1-2.}) hold for each $\xi\in\mathcal{X}$. By Lemma \ref{lem: mild}, it follows that the limit process $(\overline{\mu}_t1_{S_{\xi}})_{0\le t\le t_\mathrm{f}}$ is, almost surely, the measure part $(\mu^{\xi}_t)_{0\le t\le t_\mathrm{f}}$ of the unique solution to (\ref{eq:rE1}, \ref{eq: rE2}) which starts at $\mu_01_{S_\xi}$, for each such $\xi$. Moreover, we recall from Lemma \ref{lemma: E and U} that the full solution $(\mu_t)_{0\le t\le t_\mathrm{f}}$ to (\ref{eq: E+G}) is characterised by \begin{equation} \mu_t=\lim_{\xi\uparrow \infty} \mu^\xi_t \end{equation} in the sense of monotone limits. Therefore, almost surely, for all $t\le t_\mathrm{f}$, \begin{equation}  \overline{\mu}_t = \lim_{\xi \uparrow \infty; \hspace{0.1cm} \xi \in \mathcal{X}} \overline{\mu}_t1_{S_\xi} =  \lim_{\xi \uparrow \infty; \hspace{0.1cm} \xi \in \mathcal{X}} \mu^\xi_t = \mu_t. \end{equation} Therefore, the \emph{only} possible subsequential limit point is the deterministic path $(\mu_t)_{0\le t\le t_\mathrm{f}}$; together with tightness, this implies that $(\mu^N_t)_{0\le t\le t_\mathrm{f}} \rightarrow (\mu_t)_{0
\le t\le t_\mathrm{f}}$ in distribution in the Skorohod topology. Since the limit path is nonrandom, it follows that, for all $\epsilon>0$, \begin{equation}\label{eq: convergence in skorohod met} \mathbb{P}\left(d_\mathrm{Sk}((\mu^N_t)_{0\le t\le t_\mathrm{f}},(\mu_t)_{0\le t\le t_\mathrm{f}})>\epsilon\right) \rightarrow 0\end{equation} where we recall that $d_\mathrm{Sk}$ is a complete metric compatible with the Skorohod topology. \medskip \\ Returning to the cutoff dynamics in (\ref{eq:rE1}, \ref{eq: rE2}), one can see that each cutoff solution $(\mu^{\xi}_t)_{0\le t\le t_\mathrm{f}}=(\mu_t1_{S_\xi})_{0\le t\le t_\mathrm{f}}$ is continuous in total variation norm. Now, if $f\in C_c(S)$, one may choose $\xi$ such that $\text{supp}(f)\subset S_\xi$, so that for all $t\le t_\mathrm{f}$, \begin{equation} \langle f, \mu_t\rangle =\langle f, \mu^\xi_t\rangle.\end{equation} It follows that each process $\langle f, \mu_t\rangle$ is continuous for $f\in C_c(S)$, which implies that $\mu_t$ is continuous in the metric $d_0$. Therefore, every uniformly open neighbourhood of $(\mu_t)_{0\le t\le t_\mathrm{f}}$ contains a Skorohod-open neighbourhood of $(\mu_t)_{0\le t\le t_\mathrm{f}}$, and together with (\ref{eq: convergence in skorohod met}), this implies that, for all $\epsilon>0$, \begin{equation}  \mathbb{P}\left(\sup_{t\le t_\mathrm{f}} d_0(\mu^N_t, \mu_t)>\epsilon\right)\rightarrow 0\end{equation} as desired. 