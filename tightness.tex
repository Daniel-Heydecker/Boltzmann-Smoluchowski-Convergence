\section{Tightness of the processes}

For each $N\in\NN$ one has $\PP^N$-almost surely the conservation relations $\left<\pi, \mu_t \right> = \left<\pi, \mu_0 \right>$ for $\pi \in \left\{\pi_n, \pi_e, \pi_p \right\}$ and similarly $\left<\pi, \tmu_t \right> +\pi(\tl_t)= \left<\pi, \tmu_0 \right> + \pi(\tl_0)$ and in particular this conservation also holds for $\pi = w$.

\subsection{Point-wise compact containment}

\begin{proposition}
For all $N\in \NN$ $\mu$ and \tmu take $\PP^N$ almost surely for all $t$ values in a tight subset of $\rca{\RR^{d+2}}$, furthermore for every $\eta > 0$ one can find a tight set $A_\eta \subset \rca{\RR^{d+2}}$ such that $\PP^N\left(\exists t \colon \mu_t \notin A_\eta \text{ or } \tmu_t \notin A_\eta \right) \leq e^{-N \eta}$.
\end{proposition}
\begin{proof}
Let $A_{a,b} := \left\{x \in \RR^{d+2} \colon \pi_n(x) \leq a, \pi_e(x) \leq b \right\}$ and note that this closed and also $x\in A_{a,b}$ implies $\abs{\pi_p(x)} \leq \sqrt{2ab}$ so it is a compact subset of $\RR^{d+2}$.
Then using the conservation relations one has $\PP^N$-almost surely that
\begin{equation}
    \mu_t\left(A_{a,b}^c\right)
    \leq
    \mu_t\left(\left\{x\colon \pi_n(x) > a\right\}\right)
    +\mu_t\left(\left\{x\colon \pi_e(x) > b\right\}\right)
    \leq
    \frac{\left<\pi_n,\mu_0\right>}{a} + \frac{\left<\pi_e,\mu_0\right>}{b}
\end{equation}
and further for any $m \in \NN$
\begin{equation}
    \PP^N\left(\sup_t \mu_t\left(A_{a,b}^c\right) > \frac1m \right)
    \leq
    \PP^N\left( \frac{\left<\pi_n,\mu_0\right>}{a} > \frac{1}{2m}\right)
    + \PP^N\left( \frac{\left<\pi_e,\mu_0\right>}{b} > \frac{1}{2m}\right).
\end{equation}
Thus for $m\in\NN$ and $\eta> 0$ one can find $a=a(m,\eta)$ and $b=b(m,\eta)$ such that under appropriate assumptions on the initial velocity distribution $\nu_0$
\begin{equation}
    \PP^N\left(\sup_t \mu_t\left(A_{a(m,\eta),b(m,\eta)}^c\right) > \frac1m \right)
    \leq
    2^{-m}e^{-N \eta}
\end{equation}
Now one can construct a tight subset of $\rca{\RR^{d+2}}$:
\begin{equation*}
    \widetilde{A}_{\eta} := \bigcap_m \left\{\lambda \in \rca{\RR^{d+2}} \colon
    \lambda\left(K_{a(m,\eta),b(m,\eta)}^c\right) \leq \frac1m, \ 
    \lambda \geq 0, \ \lambda(\RR^{d+2}) \leq 1 \right\}
\end{equation*}
and
\begin{equation*}
    \PP^N\left(\exists t \colon \mu_t \notin \widetilde{A_\eta} \right)
    \leq e^{-N \eta}.
\end{equation*}
The same argument also shows that 
\begin{equation*}
    \PP^N\left(\exists t \colon \tmu_t \notin \widetilde{A_\eta} \right)
    \leq e^{-N \eta}.
\end{equation*}
\end{proof}

A corresponding result for $\tl$ is a simple consequence of the conservation relations and the initial distribution.
\begin{proposition}
Assuming $\nu_0$ has exponential moments, for every $\eta > 0$ there is a compact set $B_\eta \subset \RR^{d+2}$ such that
\begin{equation*}
    \PP^N\left(\exists t \colon \tl_t \notin B_\eta \right) \leq e^{-N\eta}.
\end{equation*}
\end{proposition}

Thus we have uniform in time compact containment with exponentially unlikely exceptions for $(\mu, 0)$ and $\left(\tmu, \tl\right)$ in
\sX in the product of the weak (and thus also weak*) topology on the measures with the Euclidean topology on $\RR^{d+2}$.



\subsection{Path space topologies}

As constructed we may take $(\mu, 0)$ and $(\tmu, \tl)$ to have c\`adl\`ag paths $[0,\infty) \rightarrow \sX$, recalling from \eqref{eq:state-space} that
$\sX =\left(\left\{\mu \in \rca{\RR^{d+2}} \colon \norm{\mu}_\mathrm{TV} \leq 1 \right\}, d\right)\times \RR^{d+2}$, which is a complete separable metric space.

The essential pointwise variation, $\epvar{f}$ of a c\`adl\`ag function $f$ from a time interval $[0,T)$ into a metric space $(E,r)$ is given by the supremum over partitions
$0=t_0 < t_1 < \dots < t_k < T$ of $\sum_{i=1}^k r\left(f(t_{i-1}),f(t_{i})\right)$.
Identifying $f(0)$ with $f(0+) = \lim_{t \searrow 0} f(t)$ the `hybrid topology' is introduced on spaces of c\`adl\`ag, $L^1$ and epvar bounded functions from $(0,T)$ into complete separable metric spaces in \cite{HPR16}.  This vector space is written $\mathrm{BV}\left(0,T; (E,r)\right) $ and will be used here with $E=\sX$.


After simplifying the second condition \cite[Theorem 3.18]{HPR16} states:
\begin{theorem}
A subset $\mathcal{F} \subset \mathrm{BV}\left(0,T; \left(E,r\right)\right)$ is topologically and sequentially compact in the hybrid topology if
\begin{enumerate}
    \item for any $x \in E$ the set is uniformly bounded in the sense that
    $\sup_{f\in\mathcal{F}} \int_0^T r\left(x, f(t)\right)\dd t + \epvar{f} < \infty$, and
    \item there is a compact $A \subset E$ and a countable dense
    $Q \subset (0,T)$ such that $\bigcup_{f\in\mathcal{F}} \bigcup_{t\in Q} \left\{f(t+) \right\} \subset A$.
\end{enumerate}
\end{theorem}

The forward part of Prohorov's theorem is also available:
\begin{proposition}[{\cite[Prop 4.7]{HPR16}}]
A tight collection of probability measures 
on $BV\left(0,T;(E,r)\right)$ with the hybrid topology is topologically and sequentially compact.
\end{proposition}


Note that $\PP^N$-almost surely
\begin{equation*}
    d\left(\tmu_t, \tmu_{t-}\right) \leq d\left(\mu_t, \mu_{t-}\right) \leq \norm{\mu_t - \mu_{t-}}_\mathrm{TV} \leq \frac3N
\end{equation*}
and since one starts with $N$ clusters, there can be at most $N-1$ merger events before all rates go to zero, thus $\PP^N$-almost surely $\epvar{\mu, 0} \leq 3$.
One can now prove tightness of the $\PP^N \circ (\mu, 0)^{-1}$ using the hybrid compact sets $\left\{f \colon f(t) \in \widetilde{A}_\eta\ \forall t \in (0,T) \text{ and } \epvar{f} \leq 3\right\}$.

From the definition of epvar one sees that 
\begin{multline*}
    \epvar{\tmu, \tl} = \epvar{\tmu} + \epvar{\pi_n\left(\tl\right)} + \epvar{\pi_e\left(\tl\right)} + \epvar{\pi_p\left(\tl\right)} \\
    \leq
    \epvar{\mu} + \left<\pi_e, \tmu_0\right> + \left<\pi_n, \tmu_0\right> + \left<\sqrt{2\pi_n \pi_e}, \tmu_0\right>.
\end{multline*}
Therefore under reasonable assumptions on the initial velocity distribution one can find $c(\eta) > 1$ such that
\begin{equation}
    \limsup_N \frac1N \log\PP^N\left(\max\left(\epvar{\mu, 0}, \epvar{\tmu, \tl}\right) > 3 c(\eta)\right) \leq -\eta.
\end{equation}
Defining
\begin{equation}
    \mathcal{F}_\eta
    := 
    \left\{f \colon f(t) \in \widetilde{A}_\eta\ \forall t \in (0,T) \text{ and } 
                    \epvar{f} \leq 3c(\eta)\right\}
\end{equation}
and (exponential) tightness follows:
\begin{equation*}
    \limsup_N \frac1N \log \PP^N\left((\mu, 0) \in \mathcal{F}_\eta, \left(\tmu, \tl\right) \in \mathcal{F}_\eta\right) \leq -\eta.
\end{equation*}
Using the Prohorov result mentioned above this implies
\begin{proposition}
There exists a probability measure $\PP^\ast$ on $\mathrm{BV}(0,T;\sX)$ and a sub-sequence of the $\PP^N$ which converge weakly to $\PP^\ast$.
\end{proposition}

\subsection{Skorohod J1-topology}
I claim without proof that the above proof can be extended to this setting.
Compact containment is already proved above, it only remains to control the modified variation.
The following will be needed:

For every $N$ the jump rate for $\left(\tmu, \tl\right)$ is, by construction, bounded by the jump rate for $(\mu, 0)$ for which one has the following estimate:
\begin{equation}\label{eq:rate-bound1}
    \sup_t \frac{N}{2}\int K(y,z)\mu_t(\dd y) \mu_t(\dd z)
    \leq
    \frac{N}{2} \left<w, \mu_t \right>^2
    \leq
    \frac{N}{2} \left<w, \mu_0 \right>^2.
\end{equation}

Thus taking some diverging positive sequences $c_N$, $\lambda_N$ with $c_N > T \lambda_N$
\begin{multline*}
    \PP^N\left(\text{more than } c_N \text{ jumps by time } T\right)\\
    \leq
    \PP^N\left(\text{more than } c_N \text{ jumps by time } T \text{ and } 
              \frac{\left<w, \mu_0 \right>^2}{2} \leq \frac{\lambda_N}{N}\right)
    + \PP^N\left(\frac{\left<w, \mu_0 \right>^2}{2} > \frac{\lambda_N}{N}\right)
\end{multline*}
one can show under reasonable assumptions on the initial distribution and choosing $c_N$ and $\lambda_N$ accordingly that for any $\eta > 0$
\begin{equation}\label{eq:rate-bound}
    \limsup_N \frac1N \log \PP^N\left(\text{more than } c_N \text{ jumps by time } T\right) \leq - \eta.
\end{equation}



\subsection{Locally uniform topology}
Here one has to take a continuous version of the process.
I think a linear interpolation between jump times is normal and helpful.
Compact containment for Arzela--Ascoli follows as in the hybrid topology.
Some additional estimates to show that the process cannot wiggle too much in a small interval would be required, but these should not be too hard since the jump times are just a thinning of a bounded rate Poisson process.