\section{Equality of the Critical Times} \label{sec: ECT}

In this section, we will prove that the critical time $t_\mathrm{c}$ or the graph process, introduced in Section \ref{sec: coupling_to_random_graph}, coincides with the gelation time for the limiting equation, defined in Section \ref{sec:SE} as the time at which mass and energy begin to escape to infinity. 
\begin{lemma}\label{lemma: connect critical times} Let $\mu_0$ be a measure on $S$ satisfying ({A1-4}.). Let $(\mu_t)_{t\ge 0}$ be the solution to (\ref{eq: E+G}) starting at $\mu_0$, with associated mass $M_t$ and energy $E_t$ of the gel; recall that $t_\mathrm{g}$ is defined by \begin{equation}
    t_\mathrm{g}:=\inf\{t\ge 0: \left<\varphi, \mu_t\right> < \left<\varphi, \mu_0\right>\} 
    = \inf\{t\ge 0: M_t+E_t>0\}.
\end{equation} Let $(G^N_t)$ be the random graph processes constructed above, and suppose that (B1-2.) hold for $G^N_t, \mu_0$. Then the critical time $t_\mathrm{c}$ for the graph transition process coincides with the gelation time $t_\mathrm{g}$: $ t_\mathrm{c}=t_\mathrm{g}. $ \end{lemma} The following is a straightforward corollary. \begin{corollary}\label{corr: actual expression for tg} Let $\mu_0$ be a sub-probability measure on $S$ satisfying (A1-4.), for a base measure $m$. Let $(\mu_t)_{t\ge 0}$ be the solution to (\ref{eq: E+G}) starting at $\mu_0$, and $t_\mathrm{g}$ the associated gelation time. Then $t_\mathrm{g}$ is given explicitly by the closed-form expression (\ref{eq: closed form for tc}). In the case where $m$ is a probability measure, this reduces to the expression (\ref{eq: closed form for tg}).   \end{corollary} \begin{proof} Let $l_N=\lceil N m(\mathbb{R}^d)\rceil$, and form $\mathbf{v}_N$ by sampling $l_N$ velocities independently from $m(\cdot)/m(\mathbb{R}^d).$ It is immediate that the resulting vertex space $\mathcal{V}=(\mathbb{R}^d, (\mathbf{v}_N)_{N\ge 1}, m)$ satisfies (B1-2.) for the measure $\mu_0$, and the critical time $t_\mathrm{c}$ of the associated graphs $G^N_t$ is given by the claimed expression (\ref{eq: closed form for tc}). From the previous lemma, it now follows that the gelation time $t_\mathrm{g}=t_\mathrm{c}$, which proves the claimed result. \end{proof} 
The proof of Lemma \ref{lemma: connect critical times} is based on the following weak version of the convergence of the gel in Theorem \ref{thrm: convergence of stochastic coagulent}, and may be taken as preliminary reading for the more detailed arguments in Section \ref{sec: COG}.
\begin{lemma} \label{lemma: WCOG} Let $(\mu_t)_{t\ge 0}, M_t, E_t$ and $G^N_t$ be as above. Fix $t>0$, and write $M^N_t, E^N_t$ for the scaled mass and energy of the largest component of $G^N_t$, as in Theorem \ref{thrm: convergence of stochastic coagulent}: \begin{equation} M^N_t:=\frac{1}{N}C_1(G^N_t); \hspace{1cm} E^N_t=\frac{1}{N}E(\mathcal{C}_1(G^N_t)). \end{equation} Then $M^N_t \rightarrow M_t$ and $E^N_t\rightarrow E_t$ in probability.
\end{lemma}
We first show that Lemma \ref{lemma: WCOG} implies Lemma \ref{lemma: connect critical times}; the remainder of this section is dedicated to the proof of Lemma \ref{lemma: WCOG}.
\begin{proof}[Proof of Lemma \ref{lemma: connect critical times}] Throughout, let $(v_i)_{i=1}^N$ be the velocities associated to the nodes of the random graph process, which we recall are independent of time. \medskip \\Firstly, suppose for a contradiction that $t_\mathrm{g}< t_\mathrm{c}$. Then $M_{t_\mathrm{c}}+E_{t_\mathrm{c}}>0$, but we bound \begin{equation}\label{eq: use of CS 0} M^N_{t_\mathrm{c}}+E^N_{t_\mathrm{c}} \le \left(\frac{1}{N}C_1(G^N_{t_\mathrm{c}})\right)^\frac{1}{2}\left(\frac{1}{N}\sum_{i=1}^N(1+\frac{1}{2}|v_i|^2)^2\right)^\frac{1}{2}.\end{equation} The first term converges to $0$ in probability, by definition of the phase transition, and the second term is bounded in $L^2$ by hypothesis (B2.). This implies that $M^N_{t_\mathrm{c}}+E^N_{t_\mathrm{c}}\rightarrow 0$ in probability, which contradicts Lemma \ref{lemma: WCOG}; we must therefore have that $t_\mathrm{g}\ge t_\mathrm{c}.$ \medskip \\ Conversely, if $t< t_\mathrm{g}$, then $M_t=0$ by definition. Now, the convergence \begin{equation} \frac{1}{N}C_1(G^N_t) \le M^N_t\rightarrow 0\end{equation} in probability implies that the largest cluster is of the order $o(N)$ with high probability, which is only possible if $t\le t_\mathrm{c}$. Since $t<t_\mathrm{g}$ was arbitrary, we must have $t_\mathrm{g}\le t_\mathrm{c}$, and together with the previous argument, we have shown that $t_\mathrm{g}=t_\mathrm{c}$ as claimed.  \end{proof} 
\subsection{Preparatory Lemmas} The proof of Lemma \ref{lemma: WCOG} is based on the following argument. We know, from Theorem \ref{thrm: RG2}, that any `mesoscopic' clusters contain negligable mass; thanks to the integrability assumption (A3.), the same is true for the energy. Therefore, almost all mass either belongs to the `microscopic' scale, whose convergence is quantified by Lemma \ref{lemma: local uniform convergence of stochastic coagulent}, or the giant component, whose convergence is the subject of interest here. Therefore, with a suitable approximation argument, the claimed convergence will follow from the quoted results.  \medskip\\ We begin with some preparatory lemmas; throughout, we will assume the notation of Lemma \ref{lemma: WCOG}. Firstly, we  bound the size of the largest cluster \emph{below}, even in the cases where there is no giant component. 
\begin{lemma}\label{lemma: lower bound on largest cluster} Fix $t>0$ and $r\in \mathbb{N}$, and let $G^N_t$ be the random graph process described above. Then \begin{equation}
    \mathbb{P}(C_1(G^N_t)\leq r)\rightarrow 0.
\end{equation} \end{lemma}
\begin{proof} We remark that, due to (A1, A4.), $m$ is not a point mass. Therefore, we can find open sets $V, W$ of positive $m$- measure such that $\inf_{v \in V, w \in W} |v-w|>0$; define \begin{equation} I_1=\{i \in \{1,2,...,N\}: v_i \in V\}; \hspace{1cm} N_1=\#I_1
\end{equation} and similarly, $I_2, N_2$, with $V$ replaced by $W$. By weak convergence (B1.) and openness of $V$ and $W$, $N_1\geq cN$ and $N_2\geq cN$ with high probability, for some constant $c>0$. Now, for all $i\in I_1, j\in I_2$ the edge $e=(ij)$ is present in $G^N_t$ with probability \begin{equation}
    \mathbb{P}(e \text{ present in }G^N_t) \geq \left(1-\exp\left(-\frac{\epsilon}{N}\right)\right) \geq \frac{\delta}{N}
\end{equation} for all $N$ large enough, for some positive $\epsilon(t, \kappa, \gamma, V, W)>0$ and $\delta=\delta(t, \kappa, \gamma, V, W)>0$. \medskip \\ Therefore, we can construct a random bipartite graph $H^N$, with vertex sets size $cN$ and independent edges occurring with probability $\frac{\delta}{N}$, such that $H^N \subset G^N_t$ with high probability. It is straightforward to see that the maximum degree $\Delta(H^N)\rightarrow \infty$ in probability, which implies the claim.  \end{proof}

For the proof of of Lemma \ref{lemma: connect critical times}, and later Theorem \ref{thrm: convergence of stochastic coagulent}, we will wish to study the convergence of integrals $\langle \varphi f, \mu^N_t\rangle$, for bounded continuous functions $f$ with non-compact support. However, the convergence result Lemma \ref{lemma: local uniform convergence of stochastic coagulent} only gives us information when the support of $f$ is compact. Our second preparatory lemma allows us to approximate the integrals $\langle \varphi f, \mu^N_t\rangle$ for functions \emph{whose support is bounded in the $\pi_n$-direction}.

\iffalse \begin{lemma}\label{lemma: etar} There exists a sequence $\eta_r \rightarrow \infty$ such that the sequence \begin{equation}
    \beta(r)=\sup_{N\geq 1} \hspace{0.1cm} \mathbb{E}\left[\sup_{t\geq 0}\hspace{0.1cm}\langle \varphi 1[\pi_e(x)>\eta_r, \pi_n(x)< r], \mu^N_t\rangle \right]
\end{equation} converges to $0$ as $r \rightarrow \infty.$ \end{lemma} \begin{proof} Let $\mu^N_t$ be a stochastic coagulant coupled to a random graphs process $G^N_t$. Using Cauchy-Schwarz, \begin{equation} \begin{split}
  & \langle \pi_e 1[\pi_e(x)>\eta_r, \pi_n(x)< r], \mu^N_t\rangle  = \frac{1}{N} \sum_{\substack{j\geq 1: C_j(G^N_t)\le r \\ E(\mathcal{C}_j(G^N_t))>\eta_r}} \hspace{0.2cm} E(\mathcal{C}_j(G^N_t))  %\\ & \leq \left(\frac{1}{N} \sum_{\substack{j\geq 1: C_j(G^N_t)\le r \\[1ex] E(\mathcal{C}_j(G^N_t))>\eta_r}} C_j(G^N_t)\right)^\frac{1}{2}\left(\frac{1}{N}\sum_{i=1}^N \frac{1}{4}|v_i|^4\right)^\frac{1}{2}  
   \\ & \hspace{3cm} \leq \langle \pi_n 1[\pi_e(x)>\eta_r, \pi_n(x)< r], \mu^N_t\rangle^\frac{1}{2}\left(\frac{1}{N}\sum_{i=1}^N \frac{1}{4}|v_i|^4\right)^\frac{1}{2}.
\end{split} \end{equation}
As remarked in Definition \ref{def: GNT}, the velocities $v_i$ associated with the graph nodes are constant in time, so the second factor is constant and bounded in $L^2$ by ({A3.}). Therefore, it is sufficient to prove the claim with $\varphi$ replaced by $\pi_n.$ \medskip \\ Choose $\eta_r\rightarrow \infty$ quickly enough that \begin{equation} \label{eq: choice of etar}
    r\hspace{0.1cm}m\left(\frac{1}{2}|v|^2>\frac{\eta_r}{r}\right)\rightarrow 0.
\end{equation}We observe that \begin{equation}
    \sup_{t\geq 0} \hspace{0.1cm} \langle \pi_n 1[\pi_e(x)>\eta_r, \pi_n(x)< r], \mu^N_t\rangle \leq \frac{1}{N}\sum_{i=1}^N 1_{A_i}  
\end{equation} where \begin{equation}
    A_i=\left\{\exists \hspace{0.1cm} t: C(i, G^N_t)< r, \hspace{0.2cm} E\left(\mathcal{C}(i, G^N_t)\right)> \eta_r \right\}.
\end{equation}
Let
\begin{equation}
    \mathcal{X}_{N,r}=\left\{i \in \{1,2,..,N\}: \hspace{0.1cm} \frac{1}{2}|v_i|^2>\frac{\eta_r}{r}\right\}; \hspace{1cm}X_{N,r}=\#\mathcal{X}_{N,r}.
\end{equation} For $i\in \mathcal{X}_{N,r}$, let $T_{i}$ be the first time that the cluster $\mathcal{C}(i, G^N_t)$ is of size $r$, and define \begin{equation}
    \mathcal{C}^*(i,r,N)=\mathcal{C}(i, G^N_{T_i});\hspace{1cm} \mathcal{C}^*(r,N)=\bigcup_{i\in \mathcal{X}_{N,r}} \mathcal{C}^*(i,r,N).
\end{equation} Then we bound $\#\mathcal{C}^*(r,N)\le r X_{N,R}$.  \medskip \\ With this notation, we can prove the claimed convergence. Fix $i$, and suppose that, for some time $t\geq 0$, $C(i, G^N_t)< r$ and $E(\mathcal{C}(i, G^N_t))> \eta_r.$ Then there exists some $j \in \mathcal{C}(i, G^N_t)$ with $\frac{1}{2}|v_j|^2 >\frac{\eta_r}{r}$, and so $j\in \mathcal{X}_{N,r}$. Since $C(j, C^N_t)=C(i, G^N_t)< r$, we must also have $t\le T_j$ and so $i\in\mathcal{C}(j,G^N_t) \subset \mathcal{C}^*(r,N).$ Therefore, \begin{equation}
    \sup_{t\geq 0} \hspace{0.1cm} \langle \pi_n 1[\pi_e(x)>\eta_r, \pi_n(x)\le r], \mu^N_t\rangle \le \frac{1}{N}\sum_{i=1}^N 1_{A_i} \le \frac{1}{N}\#\mathcal{C}(r,N) \le \frac{r}{N}X_{N,r}.
\end{equation}
Taking expectations, and using that the initial velocities are sampled independently from $m$,
\begin{equation} \begin{split}
    \mathbb{E}\left[\sup_{t\geq 0} \hspace{0.1cm} \langle \pi_n 1[\pi_e(x)>\eta_r, \pi_n(x)\le r], \mu^N_t\rangle\right] \le \frac{r}{N} \mathbb{E} X_{N,r}  = r\hspace{0.1cm}m\left(\frac{1}{2}|v|^2>\frac{\eta_r}{r}\right) \rightarrow 0. 
\end{split} \end{equation}
\end{proof}

\fi
\begin{lemma}[A step towards uniform integrability]\label{lemma: STUI}
Suppose that $\mu^N$ are stochastic coagulants satisfying (B1-2.). Then, for every positive $r \in \NN$,
\begin{equation}
    \beta(r,\eta):= \sup_{N\geq 1}
      \mathbb{E}\left[\sup_{t\geq 0}\bigg\langle \varphi 1[\pi_e(x)>\eta, \pi_n(x)\leq r], \mu^N_t\bigg\rangle \right]
    \rightarrow 0\hspace{1cm} \text{as }\eta\rightarrow \infty.
\end{equation}
\end{lemma}
\begin{proof}
Let $\mu^N_t$ be a stochastic coagulant coupled to a random graphs process $G^N_t$. Using Cauchy-Schwarz we note firstly that
\begin{equation} \begin{split}
 &\sup_{t\geq 0}\hspace{0.1cm}\bigg\langle \varphi 1[\pi_e(x)>\eta, \pi_n(x)\leq r], \mu^N_t\bigg\rangle
  \\[1ex] & \hspace{1cm}
\leq
  \left(\frac1N\sum_{j=1}^N \sum_{i \in \mathcal{C}_j(G_t^N)}\frac14 |v_i|^4 \right)^{\frac12}
  \left(\sup_{t\geq 0}\frac1N \sum_{j=1}^N\sum_{i \in \mathcal{C}_j(G_t^N)}
    1[E(\mathcal{C}_j(G_t^N)) > \eta, C_j(G_t^N)\leq r]\right)^{\frac12} \\[1ex] & \hspace{1cm}
=
  \left(\frac1N\sum_{i=1}^N \frac14 |v_i|^4 \right)^{\frac12}
  \left(\sup_{t\geq 0}\hspace{0.1cm}\bigg\langle \pi_n 1[\pi_e(x)>\eta, \pi_n(x)\leq r], \mu^N_t\bigg\rangle\right)^{\frac12}.
\end{split}\end{equation}
As remarked in Definition \ref{def: GNT}, the velocities $v_i$ associated with the graph nodes are constant in time, so the first factor is independent of $t\ge 0$, and is bounded in $L^2$ by the second assertion of (B2.).
Therefore, it is sufficient to prove the claim with $\varphi$ replaced by $\pi_n$.
\bigskip \\  Recall from the discussion below (B2.) that $X^N_{\eta/r}$ is given by \begin{equation} X^N_{\eta/r}=\#\left\{i\le N: \frac{1}{2}|v_i|^2>\frac{\eta}{r}\right\}.\end{equation} Fix $t\ge 0$. Now, if a cluster at time $t$ contains at most $r$ particles, but has total kinetic energy greater than $\eta$, then it must contain a particle of velocity $v$ with $\frac{1}{2}|v|^2>\frac{\eta}{r}$. Since each such cluster contains at most $r$ particles, the contribution of these clusters is at most \begin{equation}\bigg \langle \pi_n 1[\pi_e(x)>\eta, \pi_n(x)\le r], \hspace{0.005cm}\mu^N_t\bigg \rangle \le \frac{r}{N} \hspace{0.005cm}X^N_{\eta/r}.   \end{equation}  
Now, the right-hand side is independent of $t \ge 0$, and so is an upper bound when we maximise over $t.$ Taking expectations, we obtain
\begin{equation}
\EE\left[\sup_{t\ge 0} \bigg\langle \pi_n 1[\pi_e(x)>\eta, \pi_n(x)\leq r], \mu^N_t\bigg\rangle \right]
\leq
\frac{r}{N} \mathbb{E}\left[X^N_{\eta/r}\right].
\end{equation}
From (\ref{eq: b3}), this vanishes as $\eta\rightarrow \infty$, uniformly in $N$.
\end{proof}


\subsection{Proof of Lemma \ref{lemma: WCOG}} Using the two preparatory lemmas developed above, we now prove Lemma \ref{lemma: WCOG}. 
\begin{proof}[Proof of Lemma \ref{lemma: WCOG}]   Throughout, we let $(\mu^N_t)_{t\geq 0}$ be a stochastic coagulant coupled to a random graph process $(G^N_t)_{t\geq 0}$, as described in Section \ref{sec: coupling_to_random_graph}. We write $(v_i)_{i=1}^N$ for the velocities associated to the graph vertices. The case $t=0$ is trivial, and can be omitted. We deal first with the mass term $M^N_t$ and show later how this may be modified for the energy $E^N_t.$ \medskip \\ Fix $t> 0$, and let $\xi_N$ be a sequence, to be constructed later, such that \begin{equation}\label{eq: choice of xiN for WCOG}
       \xi_N\rightarrow \infty; \hspace{1cm} \frac{\xi_N}{N}\rightarrow 0; \hspace{1cm}\mathbb{P}(C_1(G^N_t)\geq \xi_N)\rightarrow 1.
   \end{equation}  We now construct `bump functions' as follows.  Let $\eta_r\rightarrow \infty$ be a sequence growing sufficiently fast that, in the notation of Lemma \ref{lemma: STUI}, $\beta(r, \eta_r)\rightarrow 0$, and let
 \begin{equation}
       S_{(r)} := \{x \in S: \pi_n(x)< r, |\pi_p(x)|\leq \sqrt{2r\eta_r}, \pi_e(x)\leq \eta_r\}.
 \end{equation}
 Let $\widetilde{g}_r$ be the indicator $\widetilde{g}_r=1[\pi_n(x)< r]$, and construct a continuous, compactly supported function $\widetilde{f}_r$ such that
 \begin{equation}
      0\leq \widetilde{f}_r\leq 1;\hspace{1cm} \widetilde{f}_r=1 \hspace{0.1cm} \text{ on } S_{(r)};\hspace{1cm} \widetilde{f}_r(x)=0 \hspace{0.1cm} \text{ if } \pi_n(x)\ge r.
 \end{equation}
 The final condition is compatible with continuity because $\pi_n:S\rightarrow \mathbb{N}$ is continuous and integer valued. We define $f_N=\widetilde{f}_{\xi_N}$ and $g_N=\widetilde{g}_{\xi_N}$.  We now decompose the difference $M^N_t-M_t:$ \begin{equation}\label{eq: decomposition of erorr in WCOG}\begin{split} M^N_t-M_t &= \underbrace{(\langle \pi_n, \mu_t\rangle -\langle \pi_n f_N, \mu_t\rangle)}_{:=\mathcal{T}^1_N} + \underbrace{\langle \pi_n f_N, \mu_t-\mu^N_t\rangle}_{:=\mathcal{T}^2_N} \\[1ex]&\hspace{2cm}+ \underbrace{\langle \pi_n (f_N-g_N), \mu^N_t\rangle}_{:=\mathcal{T}^3_N} +\underbrace{(
   \langle \pi_n g_N, \mu^N_t\rangle - (\langle \pi_n, \mu^N_0\rangle-M^N_t)}_{:=\mathcal{T}^4_N}
   \\[1ex]&\hspace{3cm}+ \underbrace{\langle \pi_n, \mu^N_0-\mu_0\rangle}_{:=\mathcal{T}^5_N} .\end{split} \end{equation} where we recall that $M_t=\langle \pi_n, \mu_0-\mu_t\rangle$. We now estimate the errors $\mathcal{T}^i_N$, $i=1,3,4,5;$ the remaining term $\mathcal{T}^2_N$ will be dealt with separately, and requires careful construction of the sequence $\xi_N$. \paragraph{1. Estimate on $\mathcal{T}^1_N$.} Let $h_N=1_{S_{(\xi_N)}}$, so that $h_N \le f_N \le 1$. As $N\rightarrow \infty$, $\pi_n h_N \uparrow \pi_n$, and so by monotone convergence, $
       \langle \pi_n h_N, \mu_t\rangle \uparrow \langle \pi_n, \mu_t\rangle$ This implies immediately that the (nonrandom) error $\mathcal{T}^1_N \rightarrow 0$.
\paragraph{2. Estimate on $\mathcal{T}^3_N$.} From the definitions of $f_N, g_N$, we observe that \begin{equation}
       |\mathcal{T}^3_N(t)|=\langle \pi_n(g_N-f_N), \mu^N_t\rangle \le  \langle \pi_n 1[\pi_n(x)<\xi_N, \pi_e(x)>\eta_{\xi_N}], \mu^N_t\rangle.
   \end{equation} Therefore, in the notation of Lemma \ref{lemma: STUI}, \begin{equation}
       \mathbb{E}\left[|\mathcal{T}^3_N(t)|\right] \leq \beta(\xi_N, \eta_{\xi_N}).
   \end{equation} By construction of $\eta_r$, and since $\xi_N \rightarrow \infty$, it follows that $\mathbb{E}[ |\mathcal{T}^3_N(t)|] \rightarrow 0,$ which implies convergence to $0$ in probability.
\paragraph{3. Estimate on $\mathcal{T}^4_N$.} By the choice (\ref{eq: choice of xiN for WCOG}) of $\xi_N$, we have that $\mathbb{P}( C_1(G^N_t)\geq \xi_N)\rightarrow 1.$ On this event, we have the equality \begin{equation}
           \begin{split}
               \langle \pi_n g_N, \mu^N_t\rangle &=\langle \pi_n, \mu^N_t\rangle - \langle \pi_n 1_{\pi_n\geq \xi_N}, \mu^N_t\rangle \\[2ex] & = \langle \pi_n, \mu^N_0\rangle-\frac{1}{N}\sum_{j\geq 1: C_j(G^N_t)\ge \xi_N} \hspace{0.1cm}\sum_{i \in C_j(G^N_t)} 1 \\[2ex] & = \langle \pi_n, \mu^N_0\rangle-M^N_t-\frac{1}{N}\sum_{j\ge 2:C_j(G^N_t)\ge \xi_N}\hspace{0.1cm}\sum_{i\in C_j(G^N_t)}1. 
           \end{split} 
       \end{equation} Therefore, with high probability, \begin{equation}\label{eq: bound on T4} \mathcal{T}^4_N(t) \le \frac{1}{N} \sum_{j\ge 2:C_j(G^N_t)\ge \xi_N} \hspace{0.1cm}\sum_{i\in C_j(G^N_t)} (1+\frac{1}{2}|v_i|^2) \end{equation} and we bound, on this event, \begin{equation}\mathcal{T}^4_N(t) \le \left(\frac{1}{N}\sum_{j\ge 2: C_j(G^N_t)\ge \xi_N} C_j(G^N_t)\right)^\frac{1}{2}\left(\frac{1}{N}\sum_{i=1}^N (1+\frac{1}{2}|v_i|^2)^2\right)^\frac{1}{2}. \end{equation} The first term is the mass of the mesoscopic clusters, which converges to $0$ in probability, by Theorem \ref{thrm: RG2}, and the second term is bounded in $L^2$ by the second part of (B2.). Together, these imply that $\mathcal{T}^4_N(t)\rightarrow 0$ in probability.
 \paragraph{4. Estimate on $\mathcal{T}^5_N$.} Using the first part of (B2.), we have the convergence in distribution \begin{equation} \langle \pi_n, \mu^N_0\rangle \rightarrow \langle \pi_n, \mu_0\rangle \end{equation} which implies that $\mathcal{T}^5_N\rightarrow 0$ in probability as desired. 
\paragraph{5. Construction of $\xi_N$, and convergence of $\mathcal{T}^2_N$.} It remains to show how a sequence $\xi_N$ can be constructed such that $\mathcal{T}^2_N \rightarrow 0$ in probability. Let $A^1_{r,N}, A^2_{r,N}$ be the events \begin{equation} \label{eq: definition of A1rn for WCOG}
       A^1_{r,N}=\left\{ |\langle \varphi \widetilde{f}_r, \mu^N_t-\mu_t\rangle|<\frac{1}{r}\right\}; \hspace{1cm}
       A^2_{r,N}=\left\{C_1(G^N_{t}) \geq r\right\}.
   \end{equation}
 Then, as $N\rightarrow \infty$ with $r$ fixed, both $\mathbb{P}(A^1_{r,N}), \mathbb{P}(A^2_{r,N}) \rightarrow 1$, by Lemma \ref{lemma: local uniform convergence of stochastic coagulent} and Lemma \ref{lemma: lower bound on largest cluster}. We now define $N_r$ inductively for $r\geq 1$ by setting $N_1=1$ and
 \begin{equation}
       \label{eq: recursive definition of Nr for WCOG} N_{r+1}=\min\left\{n \in \mathbb{N}: n\geq \max((r+1)^2, N_r+1),  \mathbb{P}(A^1_{r+1,N})>\frac{r}{r+1},  \mathbb{P}(A^2_{r+1,N})>\frac{r}{r+1} \hspace{0.1cm} \text{for all }N\geq n. \right\}
 \end{equation} 
 Now, we set $\xi_N=r$ for $N\in [N_r, N_{r+1})\cap\mathbb{N}.$ It follows that $\xi_N \rightarrow \infty$ and $\xi_N\leq \sqrt{N}\ll N$, and
 \begin{equation}
       \mathbb{P}\left(C_1(G^N_t))\geq \xi_N\right)\ge 1-\frac{1}{\xi_N} \rightarrow 1. 
 \end{equation} Therefore, $\xi_N$ satisfies the requirements (\ref{eq: choice of xiN for WCOG}) above. Moreover, \begin{equation}
       \mathbb{P}\left(|\mathcal{T}^2_N| <\frac{1}{\xi_N}\right) \ge \mathbb{P}\left(A^1_{\xi_N,N}\right) > 1-\frac{1}{\xi_N}\rightarrow 1
 \end{equation}
 and so, with this choice of $\xi_N$, $\mathcal{T}^2_N \rightarrow 0$ in probability. Since we have now dealt with every term appearing in the decomposition (\ref{eq: decomposition of erorr in WCOG}), it follows that $M^N_t\rightarrow M_t$ in probability, as claimed. \medskip \\ The arguments for the energy $E^N_t$ are identical to those above, using the same bound (\ref{eq: bound on T4}) on $\mathcal{T}^4_N$. \end{proof} We also note, for future use, an important corollary of this argument.
\begin{corollary}\label{corr: gel at tgel} At the instant of gelation, the gel is negligible: $ M_{t_\mathrm{g}}=E_{t_\mathrm{g}}=0.$  \end{corollary} \begin{proof} This follows from the critical case of Theorem~\ref{thrm: RG1} exactly as in (\ref{eq: use of CS 0}). \end{proof}