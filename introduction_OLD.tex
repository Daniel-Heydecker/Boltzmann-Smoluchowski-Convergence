\section{Introduction}

\subsection{Notation}
Clusters described by size, kinetic energy and momentum are considered are modelled as elements of $\RR^{2+d}$.
We write $\pi_n$, $\pi_e$ and $\pi_p$ respectively, for maps from a cluster to size (number of molecules), kinetic energy and momentum respectively.

Write $C_0(\RR^{2+d})$ for the Banach space that is the uniform closure of the compactly supported continuous functions on $\RR^{2+d}$.
Its Banach dual is $\rca{\RR^{d+2}}$, the space of signed measures on $\RR^{d+2}$ with the total variation norm.
The weak* topology on $\rca{\RR^{d+2}}$ generated by the base of semi-balls of the form
$\left\{\nu \colon \abs{\left<f,\nu\right> - \left<f,\nu\right>} < \epsilon \right\}$ is sequentially complete \cite[Corr 2.6.21]{Meg} and restricted to bounded balls, it is metrised by the following metric:  
Let $\{\phi_k, k\geq 1\}$ be a countable, dense, subset of the unit ball of $C_0(\RR^{2+d})$ and
\begin{equation} 
 d(\mu, \nu)=\sum_{k=1}^\infty 2^{-k} \min\left(1, |\langle \phi_k, \mu-\nu\rangle|\right)
 \leq \norm{\mu - \nu}_{\mathrm{TV}}
\end{equation}
where for $\phi\in C_0(\RR^{2+d})$ and $\mu \in \rca{M}(\RR^{2+d})$ the duality pairing is written $\left<f,\mu\right> := \int_{\RR^{2+d}} \phi(x) \mu(\dd x)$.
Thus one has a complete, separable metric space
\begin{equation}
 \left(\left\{\mu \in \rca{\RR^{d+2}} \colon \norm{\mu}_\mathrm{TV} \leq 1 \right\}, d\right)    
\end{equation}
for the empirical measures processes of the particle systems, which will now be introduced.




\subsection{Particle Systems}

These particle systems represent interaction clusters for molecules undergoing the Kac dynamics.
Each cluster is specified by momentum, size and kinetic energy as mentioned above.
Because of the underlying Kac model, for every cluster $Y$ $\pi_n(Y) \in \NN$ and there must exist $v_i \in \RR^d$ such that
\begin{equation}\label{eq:cluster-struct}
    \frac12 \sum_{i=1}^{\pi_n(Y)} \norm{v_i}^2 = \pi_e(Y)
    \qquad
    \sum_{i=1}^{\pi_n(Y)} v_i = \pi_p(Y).
\end{equation}
This will be guaranteed by the initial distributions and dynamics.

For each $N\in\NN$ construct a particle system
\begin{equation}\label{eq:particle-sys}
    X^N(t) = \left(X^N(t,i)\right)_{i=1}^{\abs{X^N(t)}}
\end{equation}
where each $X^N(t,i)$ is a cluster.
This particle system undergoes pure jump dynamics where an unordered pair of clusters $Y,Z$ merges at rate $\frac2N K(X,Y)$ with
\begin{equation} \label{eq: kernel}
  K(Y,Z):=
    \kappa \pi_n\left(Y\right)\pi_n\left(Z\right)
    + 2\gamma \left[\pi_n\left(Z\right)\pi_e\left(Y\right)
                  +\pi_n\left(Y\right)\pi_e\left(Z\right)
                  -\pi_p\left(Y\right)\cdot\pi_p\left(Z\right)\right]
\end{equation}
for some constants $\kappa \geq 0$, $\gamma > 0$ (the case $\gamma = 0$ can be treated with the same methods and is significantly simpler).
Under the cluster structure assumption \eqref{eq:cluster-struct}
\begin{equation}\label{eq:kernel-bound}
    K(Y,Z) \leq \kappa \pi_n\left(Y\right)\pi_n\left(Z\right)
               + 4 \gamma \left[\pi_n\left(Z\right)\pi_e\left(Y\right)
                  +\pi_n\left(Y\right)\pi_e\left(Z\right)\right]
            \leq w(Y) w(Z)
\end{equation}
where
\begin{equation}\label{eq:wdef}
  w(Y) = \sqrt{\kappa \vee 1}\pi_n(Y) + \frac{4\gamma}{\sqrt{\kappa \vee 1}} \pi_e(Y).    
\end{equation}
Cluster merging is understood as componentwise addition.
The initial condition is given by $\left(Y_i, 1, \frac12 \norm{Y_i}^2\right)_{i=1}^N$ where the $Y_i$ are iid samples from some distribution $\nu_0$ on $\RR^d$ with finite second moment. 

Let $a_N$ be a monotone increasing sequence such that $\lim_N a_n = +\infty$ and $\lim_n a_N / \sqrt{N} = 0$.
Two coupled processes on the Polish space
\begin{equation}\label{eq:state-space}
 \sX := \left(\left\{\mu \in \rca{\RR^{d+2}} \colon \norm{\mu}_\mathrm{TV} \leq 1 \right\}, d\right)
 \times
 \RR^{d+2}
\end{equation}
are now introduced:
\begin{equation}\label{eq:mu}
    (\mu ,0) = \left(\frac{1}{N} \sum_i \delta_{X^n(t,i)}, 0\right)
\end{equation}
and
\begin{equation}\label{eq:mut}
    \left(\widetilde{\mu} ,\widetilde{\lambda}\right) =
    \left(\frac{1}{N} \sum_i \delta_{X^n(t,i)} \mathbbm{1}_{[0, a_N]}\left(w\left(X^n(t,i)\right)\right),
          \frac{1}{N} \sum_i         X^n(t,i) \mathbbm{1}_{(a_N, \infty)}\left(w\left(X^n(t,i)\right)\right)\right)
\end{equation}

\begin{proposition}
Under $\PP^N$ \rpcomment{I need to introduce this and I am not quite sure how to say what I want} $(\mu, 0)$ is a Markov process with respect to its natural filtration with generator $\mathcal{L}^N$ given by
\begin{equation}\label{eq:mu-generator}
    \left(\mathcal{L}^N \Phi\right)(\nu, 0)
    =
    \int\left[\Phi\left(J^N_{y,z}\nu, 0\right) - \Phi(\nu, 0)\right]NK(y,z)\nu^{\otimes_N 2}(\dd y,\dd z)
\end{equation}
where
\begin{equation}
    J^N_{y,z}\nu = \nu + \frac1N\left(\delta_{y+z} - \delta_y - \delta_z\right)
\end{equation}
and $\nu^{\otimes_N 2}(A \times B) = \nu(A)\nu(B) - \nu(A\cap B)/N$.
\end{proposition}

\begin{proposition}
Less obviously under $\PP^N$ $(\tmu, \tl)$ is also a Markov process with respect to its natural filtration with generator
\begin{equation}\label{eq:tmu-generator}
    \left(\widetilde{\mathcal{L}}^N \Phi\right)(\nu, x)
    =
    \left(\widetilde{\mathcal{L}}^N_1 \Phi\right)(\nu, x)
    +\left(\widetilde{\mathcal{L}}^N_2 \Phi\right)(\nu, x)
    +\left(\widetilde{\mathcal{L}}^N_3 \Phi\right)(\nu, x)
\end{equation}
where
\begin{align*}
    \left(\widetilde{\mathcal{L}}^N_1 \Phi\right)(\nu, x) &=
    \frac12\int\left[\Phi\left(J^N_{y,z}\nu, x\right) - \Phi(\nu, x)\right]
         NK(y,z)\mathbbm{1}_{[0.a_N]}\left(w(y+z)\right)\nu^{\otimes_N 2}(\dd y \dd z)\\
    \left(\widetilde{\mathcal{L}}^N_2 \Phi\right)(\nu, x) &=
    \frac12\int\left[\Phi\left(\nu - \frac1N \left(\delta_y + \delta _z\right), x + \frac1N(y+z)\right) - \Phi(\nu, x)\right]
         NK(y,z)\mathbbm{1}_{(a_N,\infty)}\left(w(x+y)\right)\nu^{\otimes_N 2}(\dd y \dd z)\\
    \left(\widetilde{\mathcal{L}}^N_3 \Phi\right)(\nu, x) &=
     \int\left[\Phi\left(\nu - \frac1N \delta_y , x + \frac{y}{N}\right) - \Phi(\nu, x)\right]
         NK(y,\tl)\nu(\dd y).
\end{align*}
\end{proposition}

A kind of tightness of the jump rate:
\begin{proposition}
Let
\begin{equation}
    \zeta(R,N) := \frac{1}{2N} \int_0^\infty \sum_{i \neq j}K\left(X^N(t,i), X^N(t,j)\right) 
      \mathbbm{1}_{(R,\infty)}\left(w\left(X^N(t,i)\right) \right)
      \mathbbm{1}_{(R,\infty)}\left(w\left(X^N(t,j)\right) \right) \dd t,
\end{equation}
then for all $\epsilon > 0$, there exists $R_\epsilon > 0$ such that
\begin{equation}
    \lim_N \frac1N \log \PP^N\left(\zeta(R_\epsilon,N) \geq N\epsilon \right) < 0.
\end{equation}
\end{proposition}
\begin{proof}
The number of coagulations between pairs of clusters $\left\{Y,Z\right\}$ with $w(Y)> R$ and $w(Z) > R$ is Poisson with mean $\zeta(R,N)$ \rpcomment{this is not quite correct---I need to work with Poisson processes}, but only $\sum_i w\left(X^N(0,i)\right) / R$ such events are possible, thus
\begin{equation}
    \PP^N\left(\zeta(R_\epsilon,N) \geq N\epsilon \right)
    \leq
    \mathrm{poi}\left(N\epsilon,  \sum_i w\left(X^N(0,i)\right) / R\right)
\end{equation}
where $\mathrm{poi}(x, )$ is the cumulative distribution function for a Poisson random variable with mean $x$.
\end{proof}

\begin{corollary}[Convergence to 0 in probability]
For all $\epsilon > 0$
\begin{equation}
\limsup_N \frac1N \log\PP^N\left(
\int_0^\infty \int K(y,z)\mathbbm{1}_{(R,\infty)}\left(w\left(y\right) \right)
      \mathbbm{1}_{(R,\infty)}\left(w\left(z\right) \right) \mu_t^{\otimes_N 2}(\dd y \dd z) \dd t
      \geq \epsilon \right) < 0
\end{equation}
and the same is true when replacing $\mu^{\otimes_N 2}(\dd y \dd z)$ with $\tmu(\dd y) \tmu(\dd z)$.
\end{corollary}
The restriction to $\mu^{\otimes_N 2}$ and not the ordinary tensor product of $\mu$ with itself is important after the formation of a gel.